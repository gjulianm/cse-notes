\documentclass[palatino]{epflnotes}

\title{Numerical Analysis and Computational Mathematics}
\author{Guillermo Julián Moreno}
\date{16/17 - Fall semester}

% Additional packages

% --------------------

\begin{document}
\frontmatter
\pagestyle{plain}
\maketitle

\tableofcontents
\mainmatter
% Content

\chapter{Introduction}

For starters, we will define what we understand as a mathematical problem and when it is well-posed.

\begin{defn}[Mathematical problem] \label{def:MathProblem} A mathematical problem is the problem of finding solutions $x$ to the equation \[ F(x,d) = 0 \] for $x ∈ \mathcal{X}$ a solution, $d ∈ \mathcal{D}$ the given data; both in appropriate spaces $\mathcal{X}, \mathcal{D}$, and with $F$ some function.
\end{defn}

What we will want to find in these problems are continuous dependence on the data: we want small perturbations in the data to cause only small changes in the solution. Formally:

\begin{defn}[Continous dependence on the data] Given a \nref{def:MathProblem}, we say that a solution $x ∈ \mathcal{X}$ is continously dependent on the data $d ∈ \mathcal{D}$ if and only if for all $δd$ such that $d + δd ∈ \mathcal{D}$ something something.
\end{defn}

One example of ill-posed (not well-posed) problem is finding the number of real roots of a polynomial. For example, given the problem $F(x,d) = x^4 - x^2(2d-1)+d(d-1)$, the number of real roots (4 if $d ≥ 1$, 2 if $d ∈ (0,1)$, 0 if $d < 0$) is not a continuous function.

However, well-posed mathematical problems may exhibit large variations. We will introduce the condition number in order to measure these changes.

\begin{defn}[Conditioning number] Given a problem $F(x,d) = 0$ for data $d ∈ \mathcal{D}$, we define the conditioning number as \( K(d) ≝ \sup \set{ \frac{\sfrac{\norm{δx}}{\norm{x}}}{\sfrac{\norm{δd}}{\norm{d}}} ∀δd \tq d + δd ∈ \mathcal{D} } \)
\end{defn}

Copy from the slides here.

Analogous things for numerical problems.

\chapter{Numerical approximation of nonlinear equations}

\section{Bisection method}

\section{Newton method}

Include a method for systems.

\section{Fixed-point iterations}

Contractive application theorem.

\chapter{Interpolation, approximation of functions and data}

Least squares method.

\chapter{Numerical differentiation and integration}

\section{Numerical differentiation}

Try to approximate the value of $f'$ given $f ∈ C^1([a,b])$ for some $f ∈ (a,b) ⊆ ℝ$, or if only the set $\set{x_i, f(x_i)}_{i = 0}^n$ is provided. Finite difference, forward, backward, centered.

\begin{prop} Error associated to forward and backward differences: \begin{align*}
E_+f(x) &= -\frac{1}{2}hf''(ξ_+) \\
E_-f(x) &= \frac{1}{2}hf''(ξ_-)
\end{align*} for some $ξ_+, ξ_- ∈ (a,b)$.
\end{prop}

\begin{proof} Taylor
\end{proof}

For centered we need $f ∈ C^3$ and $E_cf(x) = -\frac{1}{2}h^2\left[f'''(ξ_+) + f'''(ξ_-)\right]$.

Boundary nodes: use forward/backward or forward/backward with increased accuracy.

\section{Numerical integration}

\chapter{Numerical linear algebra}

Something something

\section{Richardson method}

\subsection{Convergence study}

\begin{prop}
If $A, P ∈ ℝ^{n × n}$ are nonsingular, then the stationary preconditioned Richardson method converges to $\vx ∈ ℝ^n$ for all $\vx_0 ∈ ℝ^n$ if and only if \[ α \abs{λ_i}^2 < 2 \Re λ_i \] where $α ≠ 0$ and $λ_i$ are all the eigenvalues of $\inv{P}A$.
\end{prop}

\begin{corol} If $A, P ∈ ℝ^{n×n}$ are nonsingular with real eigenvalues, then the SPRM converges if and only if $0 < α λ_i < 2$ for all $i = 1, \dotsc, n$.
\end{corol}

For the election of the optimal value of α for the method, we will need to define an special norm.

\begin{defn}[Energy norm][Norm!energy] The energy norm of a vector $\vv ∈ ℝ^n$ with respect to a symmetric and positive definite matrix $A$ is defined as \( \norm{\vv}_A = \sqrt{\trans{\vv} A \vv}\)
\end{defn}

\begin{prop} If $A, P ∈ ℝ^{n×n}$ are symmetric and positive definite, then the STRM converges for all initial values if and only if \[ 0 < α < \frac{2}{λ_{\text{max}}} \] with $λ_\text{max}$ the maximum eigenvalue of $\inv{P}A$. Moreover, the spectral radius of the iteration matrix $B_α$ is minimum for $α = α_{\text{opt}}$ where \( α_{\text{opt}} = \frac{2}{λ_\text{min} + λ_{\text{max}}} \) with $λ_\text{min}, λ_\text{max}$ the minimum and maximum eigenvalues of $\inv{P}A$. In this case, for $α = α_\text{opt}$ we have that \[ \norm{\ve_k}_A ≤ d^k \norm{\ve_0}_A \qquad k = 0, 1, \dotsc \] with \[ d = \frac{K(\inv{P}A) - 1}{K(\inv{P}A) + 1} \qquad K(\inv{P}A) = \frac{λ_\text{max}(\inv{P}A)}{λ_\text{min}(\inv{P}A)} \]
\end{prop}

\begin{remark} When $A$ and $P$ are both spd, then the eigenvalues of $\inv{P}A$ are all real and positive. Note that they are solutions of $A\vx = λP\vx$ so it is possible to show that \[ \frac{λ_\text{min}(A)}{λ_\text{max}(P)} ≤ λ_i ≤ \frac{λ_\text{max}(A)}{λ_\text{min}(P)}\]
\end{remark}

\begin{remark} The closer is the preconditioning matrix $P$ to $A$, the closer is $K(\inv{P}A)$ to $1$ and the further is the convergence. In this case, however, $P\vz_k = \vr_k$ will be more expensive to solve.
\end{remark}

\section{Gradient methods}

With $A$ spd. This is equivalent to dynamic preconditions on the Richardson method with an optimal choice of $d_k$.

\begin{prop} If $A, P ∈ ℝ^{n×n}$ are spd, then the preconditioned gradient method converges to the solution for all the choices of initial value and we have the same result \[ \norm{\ve_k}_A ≤ d^k \norm{\ve_0}_A \qquad k = 0, 1, \dotsc \]
\end{prop}

Also if $K(\inv{P}A) \ll K(A)$, it is a good precondition.

\subsection{Interpretation of the gradient method}

Let us introduce the quadratic form \begin{align*}
\appl{Φ}{ℝ^n&}{ℝ} \\
\vx &\longmapsto\frac{1}{2} \trans{\vx} A \vx - \trans{\vx} \vb
\end{align*}

Since $A$ is symmetric, we have that $∇Φ(\vx) = A \vx - \vb$ so if $\vx$ solves the linear system, then $∇Φ(\vx) = 0$ - in particular, $\vx$ minimizes Φ. This means that solving the linear system is equivalent to finding a minimum for Φ.

So the shape of the Φ function will influence the behaviour of the method.


\chapter{Numerical approximation of eigenvalue problems}

\chapter{Numerical methods for ordinary differential equations}

\appendix

\chapter{Evaluation method}

Exam is written, and in the computer room. It covers all theoretical and practical arguments (3h/3.5h of duration). Examn is either free-form or multiple choice questions. Part of the questions should be solved numerically with MATLAB, as the exam includes the implementation and use of numerical methods.

In the exam, we can use the MATLAB tutorial, functions implemented in the exercise sessions and one page paper with formulae, definitions and theorems (only front face).

% \chapter{---}
% % -*- root: ../NumericalAnalysis.tex -*-



\backmatter

\nocite{scientificComputingMatlab}
\bibliography{../EPFLNotes.bib}

\printindex
\end{document}

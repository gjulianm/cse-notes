\documentclass[palatino]{epflnotes}

\title{Numerical Analysis and Computational Mathematics}
\author{Guillermo Julián Moreno}
\date{16/17 - Fall semester}

% Additional packages

% --------------------

\begin{document}
\frontmatter
\pagestyle{plain}
\maketitle

\tableofcontents
\mainmatter
% Content

\chapter{Introduction}

For starters, we will define what we understand as a mathematical problem and when it is well-posed.

\begin{defn}[Mathematical problem] \label{def:MathProblem} A mathematical problem is the problem of finding solutions $x$ to the equation \[ F(x,d) = 0 \] for $x ∈ \mathcal{X}$ a solution, $d ∈ \mathcal{D}$ the given data; both in appropriate spaces $\mathcal{X}, \mathcal{D}$, and with $F$ some function.
\end{defn}

What we will want to find in these problems are continuous dependence on the data: we want small perturbations in the data to cause only small changes in the solution. Formally:

\begin{defn}[Continous dependence on the data] Given a \nref{def:MathProblem}, we say that a solution $x ∈ \mathcal{X}$ is continously dependent on the data $d ∈ \mathcal{D}$ if and only if for all $δd$ such that $d + δd ∈ \mathcal{D}$ something something.
\end{defn}

One example of ill-posed (not well-posed) problem is finding the number of real roots of a polynomial. For example, given the problem $F(x,d) = x^4 - x^2(2d-1)+d(d-1)$, the number of real roots (4 if $d ≥ 1$, 2 if $d ∈ (0,1)$, 0 if $d < 0$) is not a continuous function.

However, well-posed mathematical problems may exhibit large variations. We will introduce the condition number in order to measure these changes.

\begin{defn}[Conditioning number] Given a problem $F(x,d) = 0$ for data $d ∈ \mathcal{D}$, we define the conditioning number as \( K(d) ≝ \sup \set{ \frac{\sfrac{\norm{δx}}{\norm{x}}}{\sfrac{\norm{δd}}{\norm{d}}} ∀δd \tq d + δd ∈ \mathcal{D} } \)
\end{defn}

Copy from the slides here.

Analogous things for numerical problems.

\chapter{Numerical approximation of nonlinear equations}

\section{Bisection method}

\section{Newton method}

Include a method for systems.

\section{Fixed-point iterations}

Contractive application theorem.

\chapter{Interpolation, approximation of functions and data}

\chapter{Numerical integration and derivation}

\chapter{Numerical linear algebra}

\chapter{Numerical approximation of eigenvalue problems}

\chapter{Numerical methods for ordinary differential equations}

\appendix

\chapter{Evaluation method}

Exam is written, and in the computer room. It covers all theoretical and practical arguments (3h/3.5h of duration). Examn is either free-form or multiple choice questions. Part of the questions should be solved numerically with MATLAB, as the exam includes the implementation and use of numerical methods.

In the exam, we can use the MATLAB tutorial, functions implemented in the exercise sessions and one page paper with formulae, definitions and theorems (only front face).

% \chapter{---}
% % -*- root: ../NumericalAnalysis.tex -*-



\backmatter

\nocite{scientificComputingMatlab}
\bibliography{../EPFLNotes.bib}

\printindex
\end{document}

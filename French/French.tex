\documentclass{epflnotes}

\title{French A1/A2 level notes}
\author{Guillermo Julián Moreno}
\date{16/17 - Fall semester}

% Additional packages
\usepackage{multicol}
% --------------------

\newcommand{\conjug}[6]{\begin{tabular}{rl}
je & #1 \\
tu & #2 \\
il & #3 \\
nous & #4 \\
vous & #5 \\
ils & #6 \\
\end{tabular}}

\begin{abstract}
Unordered notes from the french A1/A2 level intensive course at EPFL.
\end{abstract}

\begin{document}
\frontmatter
\maketitle

\tableofcontents
\mainmatter

% Content

\chapter{Verbs and grammar}

\section{Verbs}

\subsection{Passé composé}

General rule: \textit{avoir} + participe of the verb. Example:

\conjug
	{ai rêvè}
	{as rêvè}
	{a rêvè}
	{avons rêvè}
	{avez rêvè}
	{ont rêvè}

However, pronominal verbs and verbs meaning movement are conjugated with the verb \textit{être} instead of \textit{avoir}. The verbs meaning movement are \textit{arriver, partir, entrer, sortir, monter, descendre, tomber, aller, venir, naître, mourir, passer, retourner} and their derivatives. Examples:


\begin{minipage}{0.45\textwidth}
\centering
\conjug
	{suis arrivé(e)}
	{es arrivé(e)}
	{est arrivé(e)}
	{sommes arrivé(e)s}
	{êtes arrivé(e)s}
	{sont arrivé(e)s}
\end{minipage}
\begin{minipage}{0.45\textwidth}
\centering
\conjug
	{me suis levé(e)}
	{te es levé(e)}
	{se est levé(e)}
	{nous sommes levé(e)s}
	{vous êtes levé(e)s}
	{se sont levé(e)s}
\end{minipage}

When the passé composé is done with the verb \textit{être}, the participle corresponds in gender and number with the subject. For example, \textit{il est arrivé} but \textit{elle est arrivée}, and \textit{nous sommes arrivés}. The rule is that we add a final \textit{s} when the subject is plural and add a \textit{e} if the subject is femenine.

\subsection{Imperative}

The imperative is used to give orders or suggestions. It is a present without the subject, and only for \textit{tu, nous} and \textit{vous}. The only change with respect to that conjugation is that for verbs terminated in \textit{-er} and \textit{aller}, the final \textit{s} disappears.

If there are pronouns, they are placed after the verb (\textit{me} and \textit{te} change to \textit{moi} and \textit{toi}). In the case of negations, they're placed before the verb and unchanged.

Some examples:

\begin{multicols}{3}
\begin{itemize}
\item Attachez vos ceintures.
\item Mangez des fruites.
\item Va vite à l'école.
\item Téléphone-lui.
\item Ne te lève pas.
\item Ne mange pas ça.
\item Regarde la rue!
\item Écoute-le.
\item Prends-en.
\end{itemize}
\end{multicols}

\subsection{Futur proche}

Same as in spanish, conjugated with \textit{aller + infinitif verb}. Example: \textit{``Je vais acheter une voiture''} (I'm going to buy a car).

\subsection{Futur simple}

Again, same as in spanish. More used in writing. It is a regular tense most of the time, with the infinitive of the verb plus the termination:

\begin{center}
\conjug
	{manger\textbf{ai}}
	{manger\textbf{as}}
	{manger\textbf{a}}
	{manger\textbf{on}}
	{manger\textbf{ez}}
	{manger\textbf{ont}}
\end{center}

Some exceptions are presented in \fref{tab:FutureIrregular}.

\begin{table}
\centering
\begin{tabular}{cc|p{0.05cm}|cc}
\toprule
\textbf{Verb} & \textbf{1st person } & & \textbf{Verb} & \textbf{1st person }  \\
\midrule
\^etre & serai & & aller & irai \\
avoir & aurai & & venir & viendrai \\
faire & ferai & & voire & verrai \\
pouvoir & pourrai & & vouloir & vodrai \\
devroir & devrai & &  & \\ \bottomrule
\end{tabular}
\caption{Some irregular verbs in futur simple.}
\label{tab:FutureIrregular}
\end{table}

\subsection{Imparfait}

Same meaning as in spanish. The conjugation is made by getting the root of the verb conjugated in the first person plural, and then adding the corresponding terminations. For example, for the verb \textit{vouloir}, the first person plural is \textit{nous voulons} so the imparfait is the following:

\begin{center}
\conjug
	{voul\textbf{ais}}
	{voul\textbf{ais}}
	{voul\textbf{ait}}
	{voul\textbf{ions}}
	{voul\textbf{iez}}
	{voul\textbf{aient}}
\end{center}

\subsection{Conjugations of common verbs}

\subsubsection{Regular verbs}

All regular verbs (those ending in \textit{-er}) have the same conjugation. Pronominal/reflexive regular verbs are also the same.

\begin{minipage}{0.45\textwidth}
\centering
\conjug
	{parl\textbf{e}}
	{parl\textbf{es}}
	{parl\textbf{e}}
	{parl\textbf{ons}}
	{parl\textbf{ez}}
	{parl\textbf{ent}}
\end{minipage}
\begin{minipage}{0.45\textwidth}
\centering
\conjug
	{me lev\textbf{e}}
	{te lev\textbf{es}}
	{se lev\textbf{e}}
	{nous lev\textbf{ons}}
	{vous lev\textbf{ez}}
	{se lev\textbf{ent}}
\end{minipage}

\subsubsection{\^etre and avoir}


\begin{minipage}{0.45\textwidth}
\centering
\conjug
	{suis}
	{es}
	{est}
	{sommes}
	{\^etes}
	{sont}
\end{minipage}
\begin{minipage}{0.45\textwidth}
\centering
\conjug
	{ai}
	{as}
	{a}
	{avons}
	{avez}
	{ont}
\end{minipage}

\subsubsection{Other common verbs}

\begin{minipage}{0.23\textwidth}
\centering
\conjug
	{vais}
	{vas}
	{va}
	{allons}
	{allez}
	{vont}
\end{minipage}
\begin{minipage}{0.23\textwidth}
\centering
\conjug
	{peux}
	{peux}
	{peut}
	{pouvons}
	{pouvez}
	{peuvent}
\end{minipage}\begin{minipage}{0.23\textwidth}
\centering
\conjug
	{connais}
	{connais}
	{connaît}
	{connaissons}
	{connaissez}
	{connaissent}
\end{minipage}
\begin{minipage}{0.23\textwidth}
\centering
\conjug
	{sais}
	{sais}
	{sait}
	{savons}
	{savez}
	{savent}
\end{minipage}


\begin{minipage}{0.23\textwidth}
\centering
\conjug
	{prends}
	{prends}
	{prend}
	{prenons}
	{prenez}
	{prennent}
\end{minipage}
\begin{minipage}{0.23\textwidth}
\centering
\conjug
	{pars}
	{pars}
	{part}
	{partons}
	{partez}
	{partent}
\end{minipage}\begin{minipage}{0.23\textwidth}
\centering
\conjug
	{dois}
	{dois}
	{doit}
	{devons}
	{devez}
	{doivent}
\end{minipage}
\begin{minipage}{0.23\textwidth}
\centering
\conjug
	{veux}
	{veux}
	{veut}
	{voulons}
	{voulez}
	{veulent}
\end{minipage}


\begin{minipage}{0.3\textwidth}
\centering
\conjug
	{joins}
	{joins}
	{joint}
	{joignons}
	{joignez}
	{joignent}
\end{minipage}
\begin{minipage}{0.3\textwidth}
\centering
\conjug
	{me maquille}
	{te maquilles}
	{se maquille}
	{nous maquillons}
	{vous maquillez}
	{se maquillen}
\end{minipage}\begin{minipage}{0.3\textwidth}
\centering
\conjug
	{commence}
	{commences}
	{commence}
	{commençons}
	{commencez}
	{commencent}
\end{minipage}


\begin{minipage}{0.3\textwidth}
\centering
\conjug
	{réfléchis}
	{réfléchis}
	{réfléchit}
	{réfléchissons}
	{réfléchissez}
	{réfléchissent}
\end{minipage}
\begin{minipage}{0.3\textwidth}
\centering
\conjug
	{répands}
	{répands}
	{répand}
	{répandons}
	{répandez}
	{répandent}
\end{minipage}
\begin{minipage}{0.3\textwidth}
\centering
\conjug
	{produis}
	{produis}
	{produit}
	{produisons}
	{produisez}
	{produisent}
\end{minipage}

\section{Prepositions, pronouns and articles}

\subsection{\textit{L'article partitif} and quantities}

The partitive articles are used to specify an uncountable\footnote{Not mathematically speaking. This is not $\mathbb{R}$.} part of something. These articules are \textbf{du}\footnote{du = de + le} for the masculine (contracted to \textbf{de l'} when the following word starts in a vowel or muted h), \textbf{de la} for the femenine and \textbf{des} for the plural. Some examples:

\begin{itemize}
\item Je mange \textbf{du} pain.
\item Je mets \textbf{de l}'huile d'olive.
\item Il y a \textbf{des} verres.
\end{itemize}

One exception is that with the verbs to express liking or disliking (\textit{aimer, adorer, préférer, détester}) the articles \textbf{le, la, l', les} are used instead.

The quantity may be specified directly, in which case the article after the \textit{de} disappears. Same occurs when using quantifiers such as \textit{peu, trop, beaucoup, assez}, etc. Following the previous examples:

\begin{itemize}
\item Je mange du pain $\to$ Je mange \textbf{un kilo de} pain.
\item Je mets de l'huile d'olive $\to$ Je mets \textbf{un litre d'}huile d'olive.
\item Il y a des verres $\to$ Il y a \textbf{assez de} verres.
\end{itemize}

The article also disappears in the case of the negation, constructed as follows:

\begin{itemize}
\item Je mange du pain $\to$ Je \textbf{ne} mange \textbf{pas de} pain.
\item Je mets de l'huile d'olive $\to$ Je \textbf{ne} mets \textbf{pas d'}huile d'olive.
\item Il y a des verres $\to$ Il \textbf{n'}y a \textbf{pas de} verres.
\end{itemize}

Finally, the pronoun \textbf{en} can be used to replace the construction \textit{du/de la/des + nom}. Examples:

\begin{itemize}
\item Tu prends du sucre? Oui, j\textbf{'en} prends (\textit{$\equiv$ Oui, je prends du sucre}).
\item Tu prends du sucre? Non, j n\textbf{'en} prends pas (\textit{$\equiv$ Non, je ne prends pas du sucre}).
\end{itemize}

\subsection{Relative pronouns}

\textbf{qui} is used to replace the subject in a subordinated phrase, \textbf{que} to replace the direct complement. Two examples:

\begin{itemize}
\item \textit{Il est le professeur qui a un manteau rouge.}
\item \textit{Il est le professeur que nous avons en classe.}
\end{itemize}

\textbf{où} is used to replace a place or time.

\begin{itemize}
\item \textit{C'est le lieu où je suis tombé.}
\item \textit{Il est arrivé où elle est partie.}
\end{itemize}

Other two less important in these notes (not necessarily less important in french):

\begin{itemize}
\item \textbf{dont}: replaces \textit{de + object}. \textit{Le livre dont nous parlons est Les Misérables.}
\item \textbf{lequel}: similar to \textit{which}. \textit{Une langue est un prisme à travers lequel ses usagers sont condamnés à voir le monde.}
\end{itemize}

\subsection{Demonstrative and possesive articles}

The \textit{adjectif demonstratif} are found along the noun that one is talking about. They are \textbf{ce, cet, cette, ces}. Easy enough.

Posesive articles signal a possesion (obviously). They go in accordance with the gender and number of the thing being referred. \Fref{tab:Possesives} shows these adjectives.

\begin{table}[hbtp]
\centering
\begin{tabular}{c|cc|c}
 & \multicolumn{3}{c}{\textsc{What is possesed}} \\
\textsc{Who possesses}	& \multicolumn{2}{c}{Singular} & Plural \\
	& Masc. & Fem. & Both genders \\ \midrule
Je & Mon & Ma & Mes \\
Tu & Ton & Ta & Tes \\
Il/Elle & Son & Sa & Ses \\
Nous & \multicolumn{2}{c|}{Notre} & Nos \\
Vous & \multicolumn{2}{c|}{Votre} & Vos \\
Ils/Elles & \multicolumn{2}{c|}{Leur} & Leurs \\
\end{tabular}
\caption{Possesives in french.}
\label{tab:Possesives}
\end{table}

\section{Expressions and common constructions}

\subsection{Questions and answers}

There are three ways to ask questions in french. From less to more formal:

\begin{enumerate}
\item Normal phrase, different entonation. \textit{Tu viens demain?}
\item With \textit{Est-ce que}. \textit{Est-ce que tu viens demain?}
\item Inversion. \textit{Viens-tu demain?}
\end{enumerate}

In writing, only the last two forms should be used.

\subsection{Connaître and savoir}

While both verbs mean ``to know'', they're used in different occasions. Connaître goes with a \textbf{substanctive} and savoir with a \textbf{verb} or subordined phrase. For example:

\begin{itemize}
\item \textit{Je connais las mathematiques.}
\item \textit{Je sais faire mathematiques.}
\end{itemize}

\subsection{Locations}

There are different prepositions to use depending on the location you're using. Examples:

\begin{itemize}
\item With cities, \textit{à} is to be used. \textit{Je vais à Madrid}.
\item With countries, it's \textit{en} or \textit{au} depending on the genre of the country. \textit{Je vais en/au Espagne}.
\item Going to a building, it's \textit{dans}. \textit{Je vais dans l'école.}
\item Going to somebody's home or workplace, it's \textit{chez}. \textit{Je vais chez Luis}.
\end{itemize}

\subsection{\textit{C'est/Il est} - Identification and description}

There is a common confusion when using \textit{C'est} and \textit{Il est}. We use the former for identification (\textit{C'est un homme, C'est un sac}) and the latter for descrption (\textit{Il est grand, Il est noir}). The general rule is that \textit{C'est} goes with sustantives and \textit{Il/Elle est} with adjectives. Several examples:


\begin{itemize}
\item C'est un homme. Il est grand.
\item C'est une femme. Elle est grande.
\item Ce sont des sacs. Ils sont bleus.
\item Il est américain.
\item C'est un professeur. Il est chanteur.
\end{itemize}

\subsection{Talking about time}

There are several expression to talk about time. Examples:

\begin{itemize}
\item \textit{Je suis né \textbf{en} 1993}: I was born \textbf{in} 1993.
\item \textit{J'etudie français \textbf{depuis} 2014}: I study french \textbf{since} 2014.
\item \textit{J'etudie français \textbf{depuis} 3 ans}: I study french \textbf{since} 3 years ago.
\item \textit{Je suis dormis \textbf{depuis} 2 heures quand tu es arrivé:} I was asleep \textbf{for} 2 hours when you arrived.
\item \textit{J'ai étudié le français \textbf{pendant} 3 ans}: I studied French \textbf{for} 3 years.
\item \textit{J'ai vu un film \textbf{pendant} mon séjour}: I saw a film \textbf{during} my stay.
\item \textit{Le projet est suspendu \textbf{pour} un an}: The project is suspended \textbf{for} a year.
\item \textit{J'ai habité a Lausanne \textbf{de} 1900 \textbf{à} 1950}: I lived in lausanne \textbf{from} 1900 \textbf{to} 1950.
\end{itemize}

\chapter{Vocabulary}

\section{Work and employment}

\begin{multicols}{3}
\begin{itemize}
\item un demandeur d'emploi
\item un poste
\item un salarié
\item un salaire
\item le métier (occupation)
\item la profession
\item les collegues
\item le bureau
\item un employeur
\item un employé
\item le directeur
\item les horaires de travail
\item travailler a plein-temps/mi-temps
\item les vacances
\item recruter
\item licencier (fire)
\item retraiter (retirement)
\item chômage (paro)
\item un chômeur
\item stage (internship)
\item recruter
\item engager
\item expérience professionnelle
\item à plein temps
\item à temps partiel/à mi-temps
\item retraite
\item licenciée (fire)
\end{itemize}
\end{multicols}

\section{Personality}

\begin{multicols}{3}
\begin{itemize}
\item bon caractère
\item sympathique
\item mauvais caractère
\item timide
\item réservé
\item bavard (no se calla)
\item nerveux
\item calme
\item optimiste
\item pessimiste
\item intéressant
\item ennuyeux
\item super
\item adorable
\item merveilleux
\item bizarre
\item sérieux
\item traivailleur
\item chaleureux
\item sociable
\item tolérant
\item patient
\item honn\^ete
\item franc
\item b\^ete (stupid)
\item paresseux
\item froid
\item intolérant
\item malhonn\^ete
\item hypocrite
\item généreux
\item gentil
\item adorable
\item doux
\item sensible
\item cultivé
\item modeste
\item egoïste
\item méchant
\item agressif
\item impatient
\item prétentieux
\item arrogant
\end{itemize}
\end{multicols}

\section{V\^etements}

\begin{multicols}{3}
\begin{itemize}
\item des sous-v\^etements
\item une veste
\item un pantalon
\item une écharpe
\item un pull-over
\item des lunettes de soleil
\item un appareil photo
\item un portefeuille
\item des sandales
\item un maillot de bain
\item une trousse de toilette
\item des baskets
\item un bonnet
\item une robe
\item un t-shirt
\item un jean
\item un passeport
\item un short
\item des chaussettes
\item de la crème solaire
\item parapluie
\item cravate
\item bracelet
\item foulard
\end{itemize}
\end{multicols}

\section{Colors}

\begin{multicols}{3}
\begin{itemize}
\item noir
\item vert
\item gris
\item blanc
\item jaune
\item marron
\item rouge
\item violet
\item orange
\item bleu
\item rose
\end{itemize}
\end{multicols}

\section{Shopping}

\begin{multicols}{3}
\begin{itemize}
\item vitrine
\item article
\item taille
\item pointure
\item chaussures
\item essayer
\item cabine d'essayage
\item long
\item court
\item large
\item étroit
\item serré
\item grand
\item petit
\item coton
\item soie
\item laine
\item cuir
\item daim
\item sport
\item habillé
\item carte
\item liquide
\end{itemize}
\end{multicols}

Some phrases

\begin{multicols}{2}
\begin{itemize}
\item Comment ça vous va?
\item Ça me va très bien
\item C'est en quoi?
\item Ce modèle existe en quelles couleurs?
\item Il fait combien?
\item Ça co\^ute combien?
\item Quelle taille avez vous?
\item Je peux avoir la taille au-dessus/au-dessous
\end{itemize}
\end{multicols}

\section{Polite formulas}

\begin{itemize}
\item Je voudrais...
\end{itemize}

\chapter{Pronunciation}

\section{Contrapositions and similar-sounding letters}

\subsection{/s/ and /z/}

In french, an \textit{s} letter surrounded by vowels is pronounced like /z/. If there are two, like in \textit{ss}, it is pronounced as /s/.

/s/ and /z/ are pronounced in the same way, except for the fact that with the /z/ the sound comes also from the vocal chords. It is a similar contraposition than /v/ and /f/.

Example: \textit{poi\textbf{s}on} and \textit{poi\textbf{ss}on}.

\backmatter
\printindex
\end{document}

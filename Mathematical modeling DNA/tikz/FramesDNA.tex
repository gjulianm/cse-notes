\makeatletter
\pgfkeys{
	/framedpoint/.is family, /framedpoint,
	node style/.store in = \@framedpoint@nodestyle,
	node name/.store in = \@framedpoint@nodecoord,
	base style/.store in = \@framedpoint@basestyle,
	default/.style = {node style = {}, node name = C, base style = {}}
}

\pgfkeys{/framedpoint, default}

\newcommand{\framedpoint}[5][]{
	\pgfkeys{/framedpoint, #1}

	\coordinate  (\@framedpoint@nodecoord) at #2;
	\coordinate (Corner1) at ($(\@framedpoint@nodecoord) + #3 + #4$);
	\coordinate (Corner2) at ($(\@framedpoint@nodecoord) + #3 - #4$);
	\coordinate (Corner3) at ($(\@framedpoint@nodecoord) - #3 - #4$);
	\coordinate (Corner4) at ($(\@framedpoint@nodecoord) - #3 + #4$);

	\draw[thick, \@framedpoint@basestyle, fill = \@framedpoint@basestyle!20!white]
		($(\@framedpoint@nodecoord)!1.2!(Corner1)$) --
		($(\@framedpoint@nodecoord)!1.2!(Corner2)$) --
		($(\@framedpoint@nodecoord)!1.2!(Corner3)$) --
		($(\@framedpoint@nodecoord)!1.2!(Corner4)$) --
		cycle;

	\node[nodepoint, \@framedpoint@nodestyle] at (\@framedpoint@nodecoord) {};
	\draw[->] (\@framedpoint@nodecoord) -- ($(\@framedpoint@nodecoord) + #3$);
	\draw[->] (\@framedpoint@nodecoord) -- ($(\@framedpoint@nodecoord) + #4$);
	\draw[->] (\@framedpoint@nodecoord) -- ($(\@framedpoint@nodecoord) + #5$);

}
\makeatother

\begin{tikzpicture}

\framedpoint[base style = gray, node name = A1]{(0,0,0)}{(1,0,0)}{(0,0,1)}{(0,1,0)}
\framedpoint[base style = gray, node name = A2]{(4,0,0)}{(1,0,0)}{(0,0.1,1)}{(0,1,-0.1)}

\end{tikzpicture}

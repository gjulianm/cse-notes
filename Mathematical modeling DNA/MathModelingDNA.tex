\documentclass[palatino]{epflnotes}

\title{Mathematical modeling of DNA}
\author{Guillermo Julián Moreno}
\date{16/17 - Spring semester}

% Additional packages
\usepackage{siunitx}

% --------------------

\begin{document}
\frontmatter
\pagestyle{plain}
\maketitle

\tableofcontents
\mainmatter
% Content

\chapter{Introduction}

\section{Basics of dna}

The bases of DNA are denoted by $\mathcal{L} = \set{A,T,C,G}$ with a complementary  rule, where \begin{align*}
\conj{A} &= T &\conj{C} &= G \\
\conj{T} &= A &\conj{G} &= C
\end{align*}

An interesting? fact: the base of the DNA are antiparallel (orientation given by the orientation of the phosphate molecule).

The study of DNA can be separated in its three structures:
\begin{enumerate}
	\item The list of letters $x_i ∈ \mathcal{L} $ for $i = 1, \dotsc, 10^9$ (at least).
	\item The secondary structure is uniform: the double helix.
	\item The tertiary structure relates to the physical properties of the double helix, such as stiffness, which can vary greatly depending on the sequence (up to 25 \%), and its intrinsic shape (the shape of the central line).
\end{enumerate}

Molecular Dynamics simulations of the atoms in DNA and their surrounding medium (water) are considerably expensive, so the focus is in stochastic differential equations for which we can study output probability distribution.

The course-grained model implies simulating not all the atoms but only blocks (sugar, phosphate or whatever is a group for chemists), reducing thus the amount of degrees of freedom.

\section{Course-grained dna modeling}

We will be interested in models that predict the sequence ground-state structure and flexibility of a sequence. Each configuration space is a vector $w ∈ ℝ^N$ and the probability density function $ρ(w)$ depends on spome constants $Z,β$ and the free energy $U(w)$, so that \[ ρ(w) = \frac{1}{Z} e^{-βU(w)} \]

One special thing is four bases and not consider base pairs. ¿?

The set of parameters being modeled will by $6n$ intra-basepair and $6(n-1)$ inter-basepair degrees of freedom, so a totla of $N = 12n - 6$ degrees of freedom.

For the cgDNA model, the free energy is approximated as a quadratic form \[ U(w)= \frac{1}{2} (w - μ) · K(w -μ)\] with $μ = μ(S) ∈ ℝ^N$ being the ground state configuration and $K = K(S) ∈ ℝ^{N×N}$ being the ground-state stiffness, symmetric and positive-definite.

Of course, the question is how to get those ground states $μ, K$ from the sequence $S$ you are studying. The main assumption is junction and intra-basepair interaction energies, as shown in \fref{fig:MovementsDNA}.

\begin{figure}[tbp]
\centering
\includegraphics[width=0.8\textwidth]{img/movementsDNA.png}
\caption{Interpair and intrapair interactions}
\label{fig:MovementsDNA}
\end{figure}

\appendix

\chapter{Exercises}
% -*- root: ../MathModelingDNA.tex -*-
\section{Session 1 - Introduction}

\begin{problem} Every human cell constains twice the human genome ($\approx \num{3e9}$ base-pairs of DNA) for a total of about $\num{6e9}$ base-pairs of DNA. The diameter of a typical human nucleus cell is of the order of $\SI{10}{\micro\metre} = \SI{e5}{\angstrom} = \SI{e-5}{\metre}$.

\ppart By treating the DNA as a cylinder of length \SI{3.4}{\angstrom} per base-pair, calculate the total length of DNA in any cell. Compare this to the diameter of the cell.

\ppart By treating the DNA as a cylinder of diameter \SI{20}{\angstrom}, calculate the total volumne of DNA in any cell. Compare this to the total volume of the cell assuming it is spherical.

\solution

\spart

Multiply to have $\SI{2.04e10}{\angstrom} = \SI{2.04}{\metre}$, which is obviously a lot of times the diameter of the cell.

\spart



\end{problem}

\begin{problem} Let $β > 0$, $n ≥ 0$ and $\mK ∈ ℝ^{n×n}$ be a positive-definite symmetric matrix. Show that

\ppart The one-dimensional Gaussian integral is \[ \int_{ℝ} e^{-σ^2} \dif σ = \sqrt{π} \]

\ppart The $n$-dimensional Gaussian integral is \[ Z_n ≝ \int_{ℝ^n} e^{-β(\vx - \hat{\vx}) · \mK(\vx- \hat{\vx})} \dif \vx = \int_{ℝ^n} e^{-β\vx · \mK \vx} \dif \vx = \left(\frac{π}{β}\right)^{\sfrac{n}{2}} \sqrt{\det \inv{K}} \]

\ppart Prove that  \[ \frac{1}{Z_n} \int_{ℝ^n} x_i e^{-β(\vx -\hat{\vx} \mK (\vx - \hat{\vx})} \dif \vx = \hat{x}_i \] for $1 ≤ i ≤ n$.

\ppart Show that \[ \frac{1}{Z_n} \int_{ℝ^n} (x_i - \hat{x}_i)(x_j - \hat{x_j}) e^{-β(\vx - \hat{\vx}) · \mK (\vx - \hat{\vx})} \dif \vx = \frac{1}{2β}K_{ij}^{-1} \]

\solution

\spart

Let us calculate the square of the integral: \begin{align*}
I &= \int_{ℝ} e^{-x^2} \dif x \\
I^2 &= \left(\int_{ℝ} e^{-x^2} \dif x\right) \left(\int_{ℝ} e^{-y^2} \dif y\right) = \\
	&= \iint_{ℝ × ℝ} e^{-(x^2 + y^2)} \dif x \dif y
\end{align*}

Now we do a change of variable to radial coordinates $r = x^2 + y^2 ∈ ℝ^+$, $θ ∈ [0, 2π)$ where the jacobian is given by \[ \begin{cases} x = r \cos θ \\ y = r \sin θ \end{cases}
\implies \det \mJ = \begin{vmatrix} \cos θ & -r \sin θ \\ \sin θ & r \cos θ \end{vmatrix} = r \] so that the integral is rewritten and can be solved:
\begin{align*}
I^2 &=  \iint_{ℝ × ℝ} e^{-(x^2 + y^2)} \dif x \dif y  \\
	&= \int_0^{2π} \int_0^{∞} e^{-r^2} r \dif r \dif θ = - π \eval[1]{e^{-r^2}}_{r = 0}^{∞} = π \\
I &= \sqrt{π}
\end{align*}

\spart

As we are integrating in the whole $ℝ^n$ space it does not matter very much where are we putting the center $\hat{x}$ so we can ignore the first part. Knowing that $\mK$ is symmetric and positive-definite, we can diagonalize so say $\mD = \trans{\mM} \mK \mM$ with $\mD = \mathrm{diag}(λ_1, \dotsc, λ_n)$ diagonal, and the change of variable is $\vz = \trans{M} \vx$. The jacobian of this transformation is $\det \mM = 1$, so the integral changes right away: \[ \int_{ℝ^n} e^{-β\vx \mK \vx} \dif \vx = \int_{ℝ^n} e^{-β \vz \mD \vz} \dif \vz \] and now the exponent does something perfect, which is that $-β \vz \mD \vz = \sum_{i=1}^n β λ_i z_i^2$ and we end up with a product of exponentials of independent variables:
\[ \int_{ℝ^n} e^{-β \vz \mD \vz} \dif \vz = \prod_{i=1}^n \int_{ℝ} e^{-β λ_i z_i^2} \dif \vz = \prod_{i=1}^n \sqrt{\frac{π}{βλ_i}} = \left(\frac{π}{β}\right)^{\sfrac{n}{2}} \sqrt{\det \inv{K}} \]

\spart

Just notice that the integrand is an odd function around $\hat{x}_i$.

\spart

Note that $(x_i - \hat{x}_i) (x_j - \hat{x}_j) = x_i x_j - \hat{x}_i x_j - \hat{x}_j x_i - \hat{x}_j \hat{x}_i$. The two terms in the middle are going to get out because of the previous part.

\end{problem}

\backmatter
\printindex
\end{document}

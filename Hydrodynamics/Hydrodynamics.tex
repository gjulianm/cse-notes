\documentclass[palatino]{epflnotes}

\title{Hydrodynamics}
\author{Guillermo Julián}
\date{16/17 - Spring semester}

% Additional packages

% --------------------

\begin{document}
\frontmatter
\pagestyle{plain}
\maketitle

\tableofcontents
\mainmatter
% Content

\chapter{Introduction to hydrodynamics}

\section{Fundamental laws}

\subsection{Transport theorem and conservation of mass}

The transport theorem models the change of a certain quantity enclosed in a volume $Ω$. That change responds to changes in the quantity to measure and in the shape of Ω.

\begin{theorem}[Transport\IS theorem] Let $Ω(t) ∈ ℝ^3$ be a volume shape, with its shape change modeled by the velocity field $\appl{\vu}{ℝ^3 × ℝ}{ℝ}$ and $\appl{b}{ℝ^3×ℝ}{ℝ}$ the quantity being modeled. Then, the change over time of the total flux can be expressed as \( \od{}{t} \int_{Ω(t)} b(\vx, t) \dif V_t = \int_{Ω(t)} \dpd{b}{t} \dif V_t + \int_{∂Ω(t)} (\vu · \vn) b(\vx, t) \dif A_t \label{eq:TransportTheorem} \) where $\vn$ is the normal to $Ω(t)$, and $V_t, A_t$ are the volume and are elements of $Ω(t)$.
\end{theorem}

Basically, this theorem says that the change is accounted for by the change of the quantity itself plus the change of the flux exiting or entering the volume.

There are several expressions for conservation of mass based on \eqref{eq:TransportTheorem}:
\begin{align*}
\od{}{t}\iiint_Ω ρ \dif V + \iint_{∂Ω} ρ\vu \dif A &\qquad \text{Integral conservation form}\\
\frac{\Dif}{\Dif t} \iiint_V ρ \dif V = 0 &\qquad \text{Integral nonconservation form}\\
\dpd{ρ}{t} + \dv (ρ \vu) = 0 &\qquad \text{Differential conservation} \\
\frac{\Dif ρ}{\Dif t} + ρ \dv \vu = 0 &\qquad \text{Differential nonconservation}
\end{align*} where $ρ$ is the density of the fluid.

For an incompressible flow, we have $ρ$ constant and then we have $\dv \vu = 0$, which is usually called the \concept{Continuity\IS equation}.

\subsection{Stress modeling and momentum conservation}

Then some equation \[ ρ \od{b}{t} = Q + \dv A\] where $A$ are volume sources and $A$ surface fluxes.

For the surface forces, we can use a theorem due to Cauchy that shows that the surface force can be represented by a stress tensor\footnote{Also known in this context as a $3 × 3$ matrix.}, so the force is $\vt = σ \vn$.

We will divide the stresses in the fluid in two parts: one static, existing even in the abscene of motion (that is, pressure); and another one related to motion.

Pressure exerts a force normal to the surface, that is, $\vf(x) = -p(x) \vn(x)$ and the stress tensor is $σ = -p(x) I $ with $I$ the identity matrix.

For stresses in a moving fluid, we add forces tangential to the surface that are represented by a second part of the stress tensor, $τ(\vx)$, that is called the viscous stress tensor.

The viscous stress tensor has two group of components, $τ_{xx}, τ_{yy}, τ_{zz}$ which are the stretching terms; and the other ones (symmetric) which are the shear terms. Thus conservation of momentum is given by \[ ρ \frac{\Dif \vu}{\Dif t} = \dv σ + ρ \vf \]


\subsection{Navier-Stokes for incompressible fluids}

In our incompressible flow, we have that the viscous tensor is the symmetric part of the velocity so that $D = \frac{\trans{∇u} + ∇u}{2}$ and $ σ = - pI + 2μD$ and thus putting everything together the results is \begin{align*}
ρ \frac{\Dif \vu}{\Dif t} &= - ∇p + μΔ\vu + ρ \vf \\
\dv \vu &= 0
\end{align*}

In order to solve these equations, boundary conditions will be set at time $t = 0$ (initial conditions); and also at the interfaces.

For example, for liquid/solid interfaces, we will use either no-slip boundary conditions ($\vu_\text{liquid} = \vu_\text{solid}$ at the interface) or non-penetration ($\vu_\text{liquid} · \vn = \vu_\text{solid} · \vn$).

For liquid/liquid interfaces, we will have continuity of velocity $\vu_1 = \vu_2$, free-slip conditions ($\vu_{1} · \vn = \vu_{2} · \vn = \vv_\text{interface}$) or continuity of stress ($σ_1 · \vn = σ_2 · \vn$).

Finally, for liquid/gas interfaces (free surfaces), we have either no-slip or no-stress ($σ·\vn = 0$).

We will also non-dimensional equation in which we use gauges to measure the quantities of interest for the problem.

\seprule[Missing weeks 2 and 3]

\chapter{Stokes flow}

In this chapter we will consider Stokes flow, also called creeping flow. First, we will need to introduce the streamfunction, which greatly simplifies the study of this flow.

\section{Streamfunction}

\begin{defn}[Streamfunction] \label{def:Streamfunction} For a given 2-dimensional incompressible flow, the streamfunction $ψ(x,y,t)$ is defined as the volume flux through a curve from $A_0$ (a reference point) to the point $(x,y)$. This gives the following expressions for velocity $\vU = (u, v, 0)$ of the flow (in cartesian coordinates):
\( u = \dpd{ψ}{y} \qquad v = - \dpd{ψ}{x} \label{eq:StreamfunctionCartesian}\)
\end{defn}

The interesting thing about the streamfunction is that by its definition it can only applied to incompressible fluids. Indeed, if we calculate the divergence of the flow we have \[ \dv \vU = \dpd{u}{x} + \dpd{v}{y} = \mder{ψ}{2}{x}{}{y}{} -\mder{ψ}{2}{y}{}{x}{} = 0 \]

If we change to cylindrical coordinates with axisymmetry (remember that streamfunction model 2D flows), the equations change to \( \label{eq:StreamfunctionCylindricalAxisym} u = \frac{1}{r} \dpd{ψ}{r} \qquad v = - \frac{1}{r} \dpd{ψ}{z} \) and one can check that we still have conservation of mass.

\subsection{Relation to vorticity}

Remember that the vorticity of a flow is defined as $\vec{ω} = \rot \vU$. If we use the streamfunction formulation, we end up with the following short expression:
\[ \vec{ω} = - Δψ \ve_z\] which is incredibly convenient. Using this expressions in Navier-Stokes equations we would end up with advection-diffusion forms:
\[ \dpd{ω}{t} + u \dpd{ω}{x} + v \dpd{ω}{y} = ν Δω \]

\section{Stokes equation}

In order to derive the Stokes equation, we start off from Newton incompressible flow equations:
\[ ρ \frac{\Dif \vU}{\Dif t} = - ∇ p + μ Δ \vU + ρ\vf \]

Assuming no external forces ($\vf \equiv 0$) and steady flow, we have the following simplification:
\( \begin{aligned}
ηΔ\vU &= ∇p \\
\dv \vU &= 0
\end{aligned} \label{eq:StokesFlow} \)

We can derive that to other formulations:
\begin{gather*}
Δp = 0 \\
Δω = 0
\end{gather*}

The last one gives rise to the biharmonic equation: \[ ∇^4 ψ = Δ(Δψ) = 0 \] which is a fully determined equation once we put in boundary conditions.

An interesting fact from the linearity of the equations is that the flow is reversible: if we change the sign we have still a solution.

Additionally, the unique solution of the Stokes equations with suitable boundary conditions is the divergence free flow field that minimizes viscous dissipation, which is given by \[ φ = 2 μ \int e_{ij} e_{ij} \dif V \]

\subsection{Flow along a sphere}

We are going to study the flow along a sphere of radius $a$, in order to see why the falling. In this case, the Stokes streamfunction in spherical coordinates is given by
\( \label{eq:FlowSphereStokes}\begin{aligned}
U_r(r,θ) &= \frac{1}{r^2 \sin θ} \dpd{ψ}{θ} \\
U_θ(r,θ) &= -\frac{1}{r \sin θ} \dpd{ψ}{r}
\end{aligned} \)

We can also look at the rotational in spherical coordinates, which is \[ \vec{ω} = \rot \vU = \frac{1}{r} \left( \dpd{rU_θ}{r} - \dpd{U_r}{θ}\right)\ve_φ = - \frac{1}{r\sin θ} \left(\dpd[2]{ψ}{r} + \frac{1}{r^2} \dpd[2]{ψ}{θ} - \frac{\cot θ}{r^2} \dpd{ψ}{θ}\right) \ve_ψ \] and therefore we have the final equation of the bilaplacian as \( Δ^2_s ψ ≝ \left(\dpd[2]{}{r} + \frac{1}{r^2} \dpd[2]{}{θ} - \frac{\cot θ}{r^2} \dpd{}{θ}\right)^2 ψ = 0 \label{eq:BilaplacianSphereFlow} \)

For the boundary conditions, in the frame of the sphere we have no-slip and thus $u_r(a, θ) = u_θ(a, θ) = 0$. By plugging this into the expression of the flow from \eqref{eq:FlowSphereStokes}, we find the following restriction on the streamfunction: \[ ψ(a, θ) = \dpd{ψ}{r}(a, θ) = 0 \]

We also need far-field boundary conditions. When $r \to ∞$, we want $\vU(r, θ) \to U_∞ \ve_z$, so that we need \[ U_r \sim U_∞ \cos θ \qquad U_θ \sim -U_∞ \sin θ \quad\text{when } r \to ∞\]

Plugging that in \eqref{eq:FlowSphereStokes}, we have that \begin{align*}
\frac{1}{r^2 \sinθ} \dpd{ψ}{θ} &\convs[][r][∞] U_∞ \cos θ \\
\dpd{ψ}{θ} &\convs[][r][∞] U_∞ r^2 \cos θ \sin θ \\
ψ &\convs[][r][∞] \frac{U_∞r^2 \sin^2 θ}{2}
\end{align*} and doing that with $U_θ$ yields the same result. We are using, though, a very informal definition of convergence of functions.

We are going to suppose that the solution is of the form $ψ = f(r) \sin^2 θ$. We can compute the derivatives of this thing:
\begin{align*}
\dpd{ψ}{θ} &=  2 f(r) \sin θ \cos θ \\
\dpd[2]{ψ}{θ} &= 2 f(r) (\cos^2 θ - \sin^2 θ)
\end{align*}

Putting that in the equation of the bilaplacian \eqref{eq:BilaplacianSphereFlow}, we will have that \[Δ_s ψ = f'' \sin^2 θ + \frac{2f}{r^2} (\cos^2 θ  - \sin^2 θ) - \frac{\cos θ}{r^2 \sin θ} 2f(r) \sin θ \cos θ  \]

Computing the laplacian again we end up with \[ Δ_s^2 ψ = \left(\dod[2]{}{r} - \frac{2}{r^2}\right)^2 f \]

\seprule[Missing one lecture]

Now we are going to study parallel flows to see things. One example is Pouseille flows.

Another example is Hele-Shaw cell, which is the flow between two plates in a narrow gap. Even though it is a viscous-dominated flow, it will behave like potential flow which is traditionally for high reynolds numbers.

In this case we will use different length gauges for the $x$ and $y$ directions (in-plane) and for the $z$ direction (perpendicular to the plane). That induces different scalings on the velocities and that will change things in the continuity equation:
\begin{align*}
\dpd{x}{u} + \dpd{y}{v} + \dpd{z}{w} &= 0 \\
\frac{U_∞}{L} \dpd{\tilde{u}}{\tilde{x}} + \frac{U_∞}{L} \dpd{\tilde{v}}{\tilde{y}} + \frac{W}{h} \dpd{\tilde{w}}{\tilde{z}} &= 0 \\
\dpd{\tilde{u}}{\tilde{x}} + \dpd{\tilde{v}}{\tilde{y}} + \frac{WL}{hU_∞} \dpd{\tilde{w}}{\tilde{z}} &= 0
\end{align*}

Here, for the selection of the gauge $W$ we still cannot neglect anything so we will apply \concept{Dominant balance transport} or principle of least degeneracy: we want to keep things as open as possible. So, we will choose $W$ such that all the derivatives in the continuity equation have the same magnitude: \[ W = \frac{U_∞h}{L} \]

Now we go to the Navier-Stokes equations getting rid of the time derivatives (flow is steady)
\begin{align*}
u \dpd{u}{z}  + v \dpd{u}{y} + w \dpd{u}{z} &= - \frac{1}{ρ} \dpd{p}{x} + ν \left(\dpd[2]{u}{x} + \dpd[2]{u}{y} + \dpd[2]{u}{z}\right) \\
\frac{U_∞^2}{L} u \dpd{\tilde{u}}{\tilde{z}}  + \frac{U_∞^2}{L} v \dpd{\tilde{u}}{\tilde{y}} + \frac{WU_∞}{h} w \dpd{\tilde{u}}{\tilde{z}} &= - \frac{U_∞^2}{L}\frac{1}{ρ} \dpd{p}{x} + \frac{P}{ρL} ν \left(\frac{U_∞}{L^2} \dpd[2]{\tilde{u}}{\tilde{x}} + \frac{U_∞}{L^2} \dpd[2]{\tilde{u}}{\tilde{y}} + \frac{U_∞}{h^2} \dpd[2]{\tilde{u}}{\tilde{z}}\right)
\end{align*}

In the Laplacian, the term with $\frac{1}{h^2}$ is going to dominate so we can neglect the two other terms of that Laplacian, and that gives us the pressure gauge \[ P = \frac{νU_∞ ρL}{h^2} \] which in turn creates an effective Reynolds number which is considerably lower than the one we would measure with more naive gauges: \[ \rey = \frac{U_∞h^2}{νL} \]

We can do the same for the velocity in the Z direction:
\begin{align*}
u \dpd{w}{z}  + v \dpd{w}{y} + w \dpd{w}{z} &= - \frac{1}{ρ} \dpd{p}{z} + ν \left(\dpd[2]{w}{x} + \dpd[2]{w}{y} + \dpd[2]{w}{z}\right) \\
\frac{U_∞^2h}{L} \left(u \dpd{\tilde{w}}{\tilde{z}}  + v \dpd{\tilde{w}}{\tilde{y}} + w \dpd{\tilde{w}}{\tilde{z}}\right)
	&= - \frac{P}{ρh} \dpd{p}{z} + \frac{νU_∞}{Lh} \left(\cancelto{0}{\dpd[2]{\tilde{w}}{\tilde{x}} + \dpd[2]{\tilde{w}}{\tilde{y}}} + \dpd[2]{\tilde{w}}{\tilde{z}}\right) \\
\rey \frac{h^2}{L^2} \left(u \dpd{\tilde{w}}{\tilde{z}}  + v \dpd{\tilde{w}}{\tilde{y}} + w \dpd{\tilde{w}}{\tilde{z}}\right)
	&= - \frac{L^2}{h^2} \dpd{p}{z} + \dpd[2]{\tilde{w}}{\tilde{z}}
\end{align*}
neglecting as previously the $x, y$ derivatives of the laplacian and dividing in the last equation by $\frac{νU_∞}{Lh}$. There, the term that is going to dominate is the pressure one, so we can neglect the others and we have that \[ \dpd{p}{z} = 0\]

Note that had we started in the reverse order, we would have ended with a pressure gauge that would lead to the condition $ν ∂^2_z u = 0$, which gives a linear solution to $u$ that is not going to satisfy the boundary conditions.

That gives the equations that are on the slide and the solutions on the slide too. The solution is then potential flow. Furthermore, if we inject the solutions in the continuity equation we get
\begin{align*}
- \frac{1}{2μ}\left(\dpd[2]{p}{x} + \dpd[2]{p}{y}\right) z(z - h) &= - \dpd{w}{z} \\
\frac{z^3}{3} -  \frac{z^2h}{2} + C &= w
\end{align*}

To satisfy boundary conditions, we say $w(0) = 0$ so that $C = 0$, but then $w(h) ≠ 0$. The only option is then having \[ - \frac{1}{2μ}\underbracket{\left(\dpd[2]{p}{x} + \dpd[2]{p}{y}\right)}_{Δ_{\parallel} p} = 0 \implies Δ_\parallel p = 0 \] which in turn gives zero vorticity.

Of course all of this is extremely shady and non-mathematical. A validation is something. Intertial terms next slide this is weird. We can see that the flow in a rectangular duct is a good approximation of this. Then we can compute the flux and some weird approximation. Then we compute the flux in the previous flow and it is pretty simplifies \[ Q = \frac{h^3}{12μ} \dpd{p}{x} \] which is compatible with the inexact solution when $h \ll w$ (take $Q$ as the flux over width $w$).

\section{Lubrication}

Duct tape

% This is getting naughty

Slides, we get Pousielle with an equal sign instead of a minus (typo on the slide)

\section{Unsteady flows}

Self-similar diffusion.

Unsteady Couette, by separation of variables, linear profile at the end.

In Stokes first problem we have same thing but not boundary at $y = h$, so semi-infinite plane. There's no steady solution. Penetration layer of size $d \sim \sqrt{νt}$ where diffusion has already happened, which is very slow.

Funny thing about vortices (exercises) there are not many mechanisms that can destroy them, such at it happens with other hydrodynamic things (layers, things).

\subsection{Stokes second problem}

We will consider a semi-infinite plane of fluid, with the bottom plate oscillating with velocity $u_p \cos ωt$. We have unidirectional so $v = 0$ (no vertical motion) and then we have the diffusion equation \[ \dpd{u}{t} = ν \dpd[2]{u}{y} \] with some bizarre boundary conditions, $u(0,t) = u_p \cos ωt$ and $u(∞,t) \to 0$.

This is a kind of a harmonically forced problems, which in our case means that the functions are sinusoid oscillations.

We can solve this problem by the ansatz $u = \hat{u}(y) \cos (ωt + φ) = \hat{u}(y) e^{\imath ω t} + c.c.$ (complex conjugate) which is a way of taking into account the phase with $\hat{u}(y)$ being a complex function. Now the equation becomes \begin{align*}
\imath ω \hat{u} e^{iωt} &= ν \dod[2]{\hat{u}}{y} e^{\imath ω t} \\
\frac{\imath ω}{ν} \hat{u} &= \dod[2]{\hat{u}}{y}
\end{align*} so that the solution is \[ \hat{u}(y) = A e^{\sqrt{\frac{\imath ω}{ν}} y} + B e^{-\sqrt{\frac{\imath ω}{ν}} y} = B e^{\frac{1}{2}\sqrt{\frac{2ω}{ν}} y} e^{\frac{\imath}{2}\sqrt{\frac{2ω}{ν}} y} + A e^{- \frac{1}{2}\sqrt{\frac{2ω}{ν}} y} e^{- \frac{\imath}{2}\sqrt{\frac{2ω}{ν}} y} \]

To have $\hat{u}(∞) = 0$ we need $B = 0$, and to have $u(0) = U_p$ we need $A = U_p$. Therefore, the final solution will be \[ u(y,t) = U_p e^{-\sqrt{\frac{ω}{2ν}} y} \cos \left(ωt - \sqrt{\frac{ω}{2ν}} y\right)\]

So we find that we have oscillations with exponentially decaying amplitude and height-dependent phase. For fluid mechanics, the penetration length $δ$ scales like $\sqrt{\frac{ν}{ω}}$. This makes sense as it is simlar to what we had with the Stokes first problem.

% Draw the cone thing.

So, in general, diffusion layers stay very small when $ν$ is small. So, for high Reynolds numbers the moment does not have time to diffuse into the flow. Therefore, we will be able to consider any flow at high Reynolds numbers as flows with vorticity concentrated on special points (vortices), filaments and close to the boundary. Therefore, we will solve for potential flow everywhere except in those special parts.

\section{Boundary layers}

We know that inviscid flow theory does not provide a good prediction of the drag on a flat plane, as that approximation with high Reynolds number kills the second-order derivatives and we lose the possibility to have more boundary conditions. So we can impose no penetration but we cannot enforce no-slip, so we won't see drag.

We have to forget that model, and introduce a boundary layer in which viscous friction is not negligible and where we can impose no-slip condition.

In this layer, the streamwise velocity is much larger than the normal velocities, and also the gradient of the streamwise velocity in the normal direction is going to be really high.

First question is thickness of the boundary layer. Based on the other problems (1st stokes problem) we know that the diffusion goes like $δ \sim \sqrt{νt}$. How does that diffuse in space? We can do dimensional analysis (slides) and then $δ(x) \sim \sqrt{\frac{νx}{U}}$. In the slides $\rey_x = \frac{Ux}{ν}$.

How does the flow look in that boundary layer? For that we need the method of matched asymptotic expansion.

An example problem is Pouseille flow in a channel with a uniform flux through the channel walls. Looking for a streamwise indepedent solution $\vu(x,y,z) = U(y) \ve_x + V(y) \ve_y$. With continuity we have $\pd{v}{y} = 0$ so $V(y) = V_0$.

We have the $y$ momentum equaiton which simplifies to $0 = \pd{p}{y}$ an therefore $p(x) = -k x + P_0$. In the $x$ momentum we find \[ -v_0 \dpd{U}{y} = \frac{k}{ρ} + ν \dpd[2]{U}{y} \] and with the boundary conditions wehave that \[ U(y) = \frac{ka}{ρv_0} \left( - \frac{y}{a} + \frac{1 - e^{-\frac{v_0y}{ν}}}{1 - e^{-\frac{v_0a}{ν}}} \right) \]

I got lost. But solve for different orders of ε (1, ε, $ε^2$, ...) and turns out that it is the same that the asymptotic expansion of the exact solution.

In the other case $\rey \gg 1$ change pressure gauge to have everythin ok. That gives us a singular perturbation because ε is in from of the second derivative and we cannot enforce both boundary conditions. However, $1 - y$ almost the exact solution except for the region near to the lower wall.

We can rescale close to $y = 0$. That gives us a solution $\tilde{U}$. When $\tilde{y}$ goes too high, we are almost out of the boundary layer, so we match there the solutions. There's a typo and $\tilde{U} = A(1 - e^{-\tilde{y}})$.


We will apply this method to 2D flow:
\begin{align*}
∂_x u + ∂_y v &= 0 \\
(u∂_x + v ∂_y)u &= - \frac{1}{ρ} ∂_x p + ν Δu \\
(u∂_x + v ∂_y)v &= - \frac{1}{ρ} ∂_y p + ν Δv
\end{align*}

Now we make that non-dimensional. We say that $u = U_∞ \tilde{u}$, $u$ goes at the velocity of the free steam. If the boundary layer is small, there's a lubrication assumption so we set an unknown gauge $v = V \tilde{v}$.  Similary, $x = L \tilde{x}$, $y = δ \tilde{y}$ and $p = p_0 + P \tilde{p}$. The equations then are transformed. For the first, we have
\[ \frac{U}{L} ∂_{\tilde{x}} \tilde{u} + \frac{V}{δ} ∂_{\tilde{y}} \tilde{v} = 0 \] and applying dominant balance we set both terms to be of the same order, so the gauge $V$ is defined as $V = \frac{Uδ}{L}$.

For the momentum equations, we will have
\[ \frac{U^2}{L}\left( \tilde{u}∂_{\tilde{x}} \tilde{u} + \tilde{v} ∂_{\tilde{y}}\tilde{u} \right)= - \frac{P}{ρL} ∂_{\tilde{x}} \tilde{p} + ν\left( \frac{U}{L^2} ∂_{\tilde{x}}^2 \tilde{u} + \frac{U}{δ^2} ∂_{\tilde{y}}^2 \tilde{u} \right) \]

In the second derivatives, the $∂_x^2$ term is not going to be significant compared to the other one with $\frac{U}{δ^2}$, so that we find that \[ \frac{δ}{L} \sim \sqrt{\frac{ν}{UL}} = \sqrt{\rey} \] so that the boundary layer will be considerably thin. This is considerably bad for computational methods, as the mesh needs to account for a large $x$ scale but also modeling the behavior on the boundary layer of very small size in the perpendicular direction.

In the meantime, we got the pressure gauge as $P = ρU^2$. For the second momentum equation, we have
\[\frac{UV}{L}\left(\tilde{u}∂_{\tilde{x}} \tilde{v} + \tilde{v} ∂_{\tilde{y}} \tilde{v} \right) = - \frac{P}{ρδ} ∂_{\tilde{y}} \tilde{p} + ν \left( \frac{V}{L^2} ∂_{\tilde{x}}^2 \tilde{v} + \frac{V}{δ^2} ∂_{\tilde{y}}^2 \tilde{v}  \right)\]

In this case, we cet that the pressure gradient is dominant in the $y$ direction. The final equation system is
\begin{align*}
∂_{\tilde{x}} \tilde{u} + ∂_{\tilde{y}} \tilde{v} &= 0 \\
\tilde{u} ∂_{\tilde{x}} \tilde{u} + \tilde{v} ∂_{\tilde{y}} \tilde{u} &= - ∂_{\tilde{x}} \tilde{p} + ∂^2_{\tilde{y}} \tilde{u} \\
∂_{\tilde{y}} \tilde{p} &= 0
\end{align*}

So what we have is almost lubrication, but we also have advection. The important issue is the fact that the pressure does not vary in the boundary layer. For example, in aerodynamics, it allows to compute lift (which depends on pressure) just by using Euler equations: despite the fact that those not work in the boundary layer, the solution for the pressure they give is valid as it does not change along the boundary layer.

We need to impose boundary conditions now. We will focus on boundary layer in a flat plate. The Euler solution is $\vU = U_∞ \ve_x$, with a constant pressure. That kills the pressure gradient in the equation (asymptotic matching: $\tilde{p}(\tilde{y} \to ∞) = \bar{p}(\bar{y} \to 0) = p_\text{atm}$ where $\bar{p}$ is the pressure from the Euler equation).

We also have no-slip and no penetration conditions, so $\tilde{u} = \tilde{v} = 0$ on $\tilde{y} = 0$, and finally the asymptotic matching condition $\tilde{u}(\tilde{y} \to ∞) = \bar{u}(\bar{y} \to 0) = 1$ (we are using the appropriate gauge).

Now we use the self-similar approach to solve the equation. But before that we are going to use a trick for 2D, which is using a streamfunction: there exists a function $ψ$ such that $u = ∂_y ψ$ and $v = - ∂_xψ$. The continuity equation is therefore automatically satisfied and the momentum equation becomes (dropping tildes) \[ \dpd{ψ}{y} \frac{∂^2ψ}{∂x∂y} - \dpd{ψ}{x} \dpd[2]{ψ}{y} = \dpd[3]{ψ}{y} \] with boundary conditions \begin{align*}
∂_x ψ (y = 0) &= 0 &∀x \\
∂_y ψ (y = 0) &= 0 \\
∂_y ψ (y \to ∞) &= 1
\end{align*}

Integrating the first one, we have $ψ(y = 0) = 0$. With the third one, $ψ(y \to ∞) = y$ forgetting about that thing of having care when integrating limits.

For the self-similar solution, we search for a function $F(ψ,x,y)$. We define $ψ = ψ_* \hat{ψ}$ with $ψ_*$ a dilation coefficient. Similarly, we define $x = x_* \hat{x}$ and $y = y_* y$. Injecting this in the equations we have
\[ \frac{ψ_*^2}{x_* y_*^2} \dpd{\hat{ψ}}{\hat{y}} \frac{∂^2\hat{ψ}}{∂\hat{x} ∂\hat{y}} - \frac{ψ_*^2}{x_* y_*^2}\dpd{\hat{ψ}}{\hat{x}} \dpd[2]{\hat{ψ}}{\hat{y}} = \frac{ψ_*}{y_*^3} \dpd[3]{\hat{ψ}}{\hat{y}} \] and in the boundary conditions
\begin{align*}
ψ_* \hat{ψ} (y_* \hat{y} = 0) &= 0 \\
\frac{ψ_*}{y_*} \dpd{\hat{ψ}}{\hat{y}} (y_* \hat{y} = 0) &= 0 \\
ψ_* \hat{ψ}(y_* \hat{y} \to ∞) &= y_* \hat{y}
\end{align*}

For the hat problem to be the same as the original problem, we needto have $ψ_* = y_*$ and $\frac{ψ_*^2}{x_*y_*^2} = \frac{ψ_*}{y_*^3}$ so that $x_* = y_*^2$. So we will consider the equation $G(\sfrac{ψ}{\sqrt{x}}, x, \sfrac{y}{\sqrt{x}}) = 0$. I am hungry. Changing that we have that $G(\sfrac{\hat{ψ}}{\sqrt{\hat{x}}},x_* \hat{x}, \sfrac{\hat{y}}{\sqrt{\hat{x}}})$ and therefore $ψ = \sqrt{x} f(\sfrac{y}{\sqrt{x}})$. We will call $η ≝ \sfrac{y}{\sqrt{x}}$. This is more or less like a separation of variables but more general.

Now we need to rephrase that equation with our new set of variables. Please god no. All the coefficients disappear and we will drop the hats. First we solve the derivatives separately:
\begin{align*}
\dpd{η}{y} &= \frac{1}{\sqrt{x}} \\
\dpd{η}{x} &= - \frac{1}{2} \frac{y}{x \sqrt{x}} = - \frac{η}{2x} \\
\dpd{ψ}{y} &= \sqrt{x} f'(η) \dpd{η}{y} = f' \\
\frac{∂^2ψ}{∂x∂y} &= f''(η) \dpd{η}{x} = - \frac{ηf''}{2x} \\
\dpd{ψ}{x} &= \frac{1}{2\sqrt{x}} f + f'(η) \dpd{η}{x} = \frac{1}{2\sqrt{x}}( f -ηf') \\
\dpd[2]{ψ}{y} &= \frac{f''(η) }{\sqrt{x}} \\
\dpd[3]{ψ}{y} &= \frac{f'''(η)}{x}
\end{align*} and then the equation becomes
\begin{align*}
 - \frac{f'ηf''}{2x} - \frac{1}{2 x} (f - η f')f'' &= \frac{f'''}{x} \\
 -ff'' &= 2f'''
 \end{align*}

This is called the Blasius equation. We need boundary conditions, and when $y = 0$ $η = 0$, and same with $y = ∞ \implies η = ∞$, so that $f(0) = 0$, $f'(0) = 0$ and $f \to η$ when $y \to ∞$.

The solution is an exponential relaxation on $u = f'$.

On the practical level, we have to define the layer thickness. There's no unique definition.  Slides. Displacement thickness making the flux equal in both displaced and flat probdiles. Last one is momentum, but isntead of conserving mass flux you consider momentum flux.

\section{Inviscid fluid - Bernoulli potential}

We review this to have better understanding of boundary layers.

To solve we use Biot-Savart but numerically it is complicated. TWo vortexs with different sign travel with the same velocity, depending on their separation (closer imply faster). Two vortexes of same sign will rotate around the center and merge eventually.

When there's shear and vorticity, the vorticity is tilted towards the direction of the shear.Also, if there is convergence in a plane of flows, it goes upwards (stretching). There is also a term $\frac{1}{ρ} \od{ρ}{t}$ which is barocycaslkdas dependence and very important in metereology but not so much for us.

SPecial propertites on vorticity fields because they have $\dv \vec{ω} = 0$. Vortex lines needs to be either closed, infinite or touch the boundaries. In circulation, we integrate tarngential velocity along the contour, and that's equal to the flux of vorticity. But very difficult to use this.

NOw we look at circulation produced due to the viscous stress in the Stokes problem.

Kelving theorem. The variation of a circulation of  a flow around a countour that follows the i don't know what it is zero.

Then you get Lagrange that says that if the derivative is zero then $ω(x,0) = 0 \implies ω(x, t) = 0$.

Read Ryhming PPUR book for vorticity theorems.

Now inviscid flow. Reynods zero in NSE gives Stokes equations, Reynolds infinity gives Euler. Euler flow must be vorticity free, and that's the step to go to potential flow.

If we assume a conservative force (such a gravity), we have Euler equations slides.

1st bernoulli,from loi fondamentale we get that $∂_t = 0$ and $ω × \vU$ is perpendicular to $\vU$, so integrating along a streamline we get the enthalpy thing.

Fort the second assume unsteady flow but irrotational, replace and integrate.

Two counter examples: we cannot apply bernoulli on a field vorticity-free almost everywhere. The second example has the constant but we cannot apply  ernoulli.

The main idea of Bernoulli is that something.

Different here than in viscous dominated flows: there you flow from high to low pressure. Here there is no link between direction and pressure, only in velocity magnitude.

Potential flow. Vorticity almost everywhere zero. If we use streamfunction (for incompresisiblitlityasda0das) ψ we have $Δψ = 0$ and flow is irrotational. if we use potential φ we have $\rot \vU = 0$ and if if we have incompressible flow thereofre $Δφ = 0$.

For source/sink we conserve flux, for vortex we conserve circulation.

All solutions of potential flow are solutions of Stokes.

To descibe flow around a cylinder, you take dipole and uniform flow so that at $ψ = 0$ you have no flow penetration.

Look at θ in the other sense because things.

Secador.

Take care of branch of the square root with Joukowski transfomr.

However with potential flow we do not get drag. We now that drag decreases with \rey at first, and then before a crisis happens we get asymptotically constant drag coef. That's the region $F_z \propto ρU^2 R^2$ which is obstacle independent.

Drag will come from the shear force of the velocity profile. THe dominant component of viscous friction is $∂_y U_x$.



\appendix

\backmatter
\printindex
\end{document}

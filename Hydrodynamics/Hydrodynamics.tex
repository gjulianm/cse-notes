\documentclass[palatino]{epflnotes}

\title{Hydrodynamics}
\author{}
\date{16/17 - Spring semester}

% Additional packages

% --------------------

\begin{document}
\frontmatter
\pagestyle{plain}
\maketitle

\tableofcontents
\mainmatter
% Content

\chapter{Introduction to hydrodynamics}

\section{Fundamental laws}

\subsection{Transport theorem and conservation of mass}

The transport theorem models the change of a certain quantity enclosed in a volume $Ω$. That change responds to changes in the quantity to measure and in the shape of Ω.

\begin{theorem}[Transport\IS theorem] Let $Ω(t) ∈ ℝ^3$ be a volume shape, with its shape change modeled by the velocity field $\appl{\vu}{ℝ^3 × ℝ}{ℝ}$ and $\appl{b}{ℝ^3×ℝ}{ℝ}$ the quantity being modeled. Then, the change over time of the total flux can be expressed as \( \od{}{t} \int_{Ω(t)} b(\vx, t) \dif V_t = \int_{Ω(t)} \dpd{b}{t} \dif V_t + \int_{∂Ω(t)} (\vu · \vn) b(\vx, t) \dif A_t \label{eq:TransportTheorem} \) where $\vn$ is the normal to $Ω(t)$, and $V_t, A_t$ are the volume and are elements of $Ω(t)$.
\end{theorem}

Basically, this theorem says that the change is accounted for by the change of the quantity itself plus the change of the flux exiting or entering the volume.

There are several expressions for conservation of mass based on \eqref{eq:TransportTheorem}:
\begin{align*}
\od{}{t}\iiint_Ω ρ \dif V + \iint_{∂Ω} ρ\vu \dif A &\qquad \text{Integral conservation form}\\
\frac{\Dif}{\Dif t} \iiint_V ρ \dif V = 0 &\qquad \text{Integral nonconservation form}\\
\dpd{ρ}{t} + \dv (ρ \vu) = 0 &\qquad \text{Differential conservation} \\
\frac{\Dif ρ}{\Dif t} + ρ \dv \vu = 0 &\qquad \text{Differential nonconservation}
\end{align*} where $ρ$ is the density of the fluid.

For an incompressible flow, we have $ρ$ constant and then we have $\dv \vu = 0$, which is usually called the \concept{Continuity\IS equation}.

\subsection{Stress modeling and momentum conservation}

Then some equation \[ ρ \od{b}{t} = Q + \dv A\] where $A$ are volume sources and $A$ surface fluxes.

For the surface forces, we can use a theorem due to Cauchy that shows that the surface force can be represented by a stress tensor\footnote{Also known in this context as a $3 × 3$ matrix.}, so the force is $\vt = σ \vn$.

We will divide the stresses in the fluid in two parts: one static, existing even in the abscene of motion (that is, pressure); and another one related to motion.

Pressure exerts a force normal to the surface, that is, $\vf(x) = -p(x) \vn(x)$ and the stress tensor is $σ = -p(x) I $ with $I$ the identity matrix.

For stresses in a moving fluid, we add forces tangential to the surface that are represented by a second part of the stress tensor, $τ(\vx)$, that is called the viscous stress tensor.

The viscous stress tensor has two group of components, $τ_{xx}, τ_{yy}, τ_{zz}$ which are the stretching terms; and the other ones (symmetric) which are the shear terms. Thus conservation of momentum is given by \[ ρ \frac{\Dif \vu}{\Dif t} = \dv σ + ρ \vf \]


\subsection{Navier-Stokes for incompressible fluids}

In our incompressible flow, we have that the viscous tensor is the symmetric part of the velocity so that $D = \frac{\trans{∇u} + ∇u}{2}$ and $ σ = - pI + 2μD$ and thus putting everything together the results is \begin{align*}
ρ \frac{\Dif \vu}{\Dif t} &= - ∇p + μΔ\vu + ρ \vf \\
\dv \vu &= 0
\end{align*}

In order to solve these equations, boundary conditions will be set at time $t = 0$ (initial conditions); and also at the interfaces.

For example, for liquid/solid interfaces, we will use either no-slip boundary conditions ($\vu_\text{liquid} = \vu_\text{solid}$ at the interface) or non-penetration ($\vu_\text{liquid} · \vn = \vu_\text{solid} · \vn$).

For liquid/liquid interfaces, we will have continuity of velocity $\vu_1 = \vu_2$, free-slip conditions ($\vu_{1} · \vn = \vu_{2} · \vn = \vv_\text{interface}$) or continuity of stress ($σ_1 · \vn = σ_2 · \vn$).

Finally, for liquid/gas interfaces (free surfaces), we have either no-slip or no-stress ($σ·\vn = 0$).

We will also non-dimensional equation in which we use gauges to measure the quantities of interest for the problem.

\appendix

\backmatter
\printindex
\end{document}

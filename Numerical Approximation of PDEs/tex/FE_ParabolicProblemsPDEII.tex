% -*- root: ../NumericalApproximationofPDEs.tex -*-

\section{The heat equation}

The first equation that we will be studying will be the heat equation. We will search for a function in space and time $u$ \begin{align*}
\appl{u}{Ω×[0,T]&}{ℝ} \\
(x,t) &\longmapsto u(x,t)
\end{align*} that will model the temperature defined on a domain $Ω ⊂ ℝ^2$ (or $ℝ^3$ too) such that
\( \begin{cases}
\dpd{u}{t} - Δu = f & \text{in } Ω × (0,T) \\
u = 0 & \text{on } ∂Ω × (0,T) \\
u(x,0) = u_0(x) & ∀x ∈ Ω
\end{cases} \label{eq:PDE:HeatEquation} \)

This comes from an energy conservation principle of thermodynamics: for any domain $D ⊂ Ω$, the time variation of the thermal energy contained in $D$ must be equal to the heat that ``goes out'' through the boundary of the domain plus the energy due to the source term $f$. This translates to the following equation:
\( \od{}{t} \int_D ρ c_p u \dif x - \int_{∂D} k ∇u \vn \dif s = \int_D f \dif x \label{eq:PDE:ThermoConservation} \) which can be transformed into the differential equation \eqref{eq:PDE:HeatEquation} using the divergence theorem\footnote{$\int_D \dv \va = \int_{∂D} \va \vn$}, putting the derivative inside of the integral and getting rid of the constants and integrals.

The first remark is that we can get a first a priori error estimate.

\begin{prop}[A priori error\IS for the heat equation] \label{prop:PDE:APrioriHeatEq} If $u$ is a solution of the heat equation \eqref{eq:PDE:HeatEquation}, then the following a priori estimate holds:
\( \label{eq:PDE:APrioriHeatEq} \int_0^T \norm{u}_{H^1(Ω)}^2 \dif t ≤ \int_Ω u_0^2 \dif x + C_p^2 \int_0^T \norm{f}^2 \dif t \)
\end{prop}

\begin{proof} Multiplying the heat equation by $u$ and operating, we have it directly.
\begin{align*}
\int_Ω \dpd{u}{t} u - Δu u \dif x &= \int_Ω fu \dif x \\
\frac{1}{2} \dod{}{t} \int_Ω u^2 \dif x + \frac{1}{2} \int_Ω \abs{∇u}^2 \dif x - \cancelwhy{\int_{∂Ω} (∇u \vn)u \dif s}{\restr{u}{∂Ω} = 0} &= \int_Ω fu \dif x \\ &≤ C_p \norm{f}_{L^2(Ω)} \norm{∇u}_{L^2(Ω)} \\
\frac{1}{2} \od{}{t} \int_Ω u^2 + \abs{∇u}^2 &≤ \frac{C_p}{2} \norm{f}^2 + \frac{1}{2} \norm{∇u}^2 \\
\int_0^T \norm{u}_{H^1(Ω)}^2 \dif t &≤ \int_Ω u_0^2 \dif x + C_p^2 \int_0^T \norm{f}^2 \dif t
\end{align*}
\end{proof}

This estimate gives us uniqueness: if there are two solutions $u_1, u_2$, we can substract one from another and $u_1 - u_2$ satisfies the heat equation with source term $f = 0$ and initial condition $u_0 = 0$, and by the a priori estimate we have $\norm{u_1 - u_2 }= 0$. It also gives us a very natural consequence: the maximum principle.

\begin{prop}[Maximum\IS principle] If $f ≥ 0$ and $u_0 ≥ 0$, then the solution $u ≥ 0$.
\end{prop}

\begin{proof} We can decompose $u = u^+ - u_-$ with $u^+ = \sup (u, 0)$ and $u^- = \sup -u, 0$. Therefore, we can see that
\begin{align*}
\int_Ω \dpd{u}{t} u^- + ∇u ∇u^- &= \int_Ω f u^- \\
\int_Ω \dpd{u^-}{t} u^- + \abs{∇u^-}^2 &= \int_Ω - f u^- \\
\frac{1}{2} \od{}{t} \int_Ω (u^-)^2 + \int_Ω \abs{∇u^-}^2 &≤ 0
\end{align*}

Integrating between $0$ and $T$, we have
\[\frac{1}{2} \int_Ω u^-(x,T)^2 + \int_0^T \int_Ω \abs{∇u^-}^2 ≤ \frac{1}{2} (u_0^-)^2 \]

As $u_0 ≥ 0$, $u_0^- = 0$ so the left hand side must be less or equal than zero. The only possibility is $u^- = 0$ and therefore $u ≥ 0$.
\end{proof}

We also have an interesting property of the solution: for positive time, the solution is smooth even when the initial function $u_0$ is not.

Now we will see the existence of the solution.

\begin{prop}[Existence and uniqueness\IS for the heat equation solution] \label{prop:PDE:ExistUniqHeat} For all $T > 0$, for any source term $f ∈ L^2(T; L^2(Ω))$ and for any initial distribution $u_0 ∈ L^2(Ω)$, there exists a unique solution $u ∈ L^2(T; H_0^1(Ω))$ for the heat equation with $\pd{u}{t} ∈ L^2(T; H_0^1(Ω)')$ such that \( \dualp[H_0^1(Ω)'][H_0^1(Ω)]{\pd{u}{t}, v} + \int_Ω ∇u ∇v = \int_Ω fv \dif x \qquad ∀ v ∈ H_0^1(Ω) \label{eq:PDE:ExtUniqHeatBound} \) for $0 ≤ t ≤ T$ and with $u(x,0) = u_0(x)$ almost everywhere, and where $\dualp[H_0^1(Ω)'][H_0^1(Ω)]{·,·}$ is the duality product of $H_0^1(Ω)$. Therefore, we have the following estimate for the solution:
\[ \frac{1}{2} \od{}{t} \int_Ω (u(x,t))^2 \dif x + \int_Ω \abs{∇u(x,t)}^2 \dif x = \int_Ω f(x,t) u(x,t) \dif x \]
\end{prop}

Before going into the proof, a small remark on the spaces: we are using \nref{def:Fund:BochnerSpace} with the convention $L^p(T; X) \equiv L^p((0, T); X)$, that is, the ``time space'' is the interval $(0,T)$. Given that $L^2(Ω)$, $H^1_0(Ω)$ and $H_0^1(Ω)'$ are all Banach spaces, we can just use those as the image spaces of the function depending on time. For the dual $H_0^1(Ω)'$ (the space of $\pd{u}{t}$) recall the definition of the \nref{def:Fund:NormOperator}.

Some quick results on regularity: if $u ∈ L^2(T; H_0^1(Ω))$ and $∂_t u ∈ L^2(T; H_0^1(Ω)')$ then $u$ is continuous on time $u ∈ C^0([0, T]; L^2(Ω))$. We can see it as
\begin{align*}
\frac{1}{2} \od{}{t}  \int_Ω u^2 &= \dualp{∂_t u, u} \\
\frac{1}{2} \int_Ω u(x,T)^2 \dif x - \frac{1}{2} \int_Ω u(x,0)^2 \dif x &= \int_0^T \dualp{∂_t u, u} \dif t \\
&≤ \int_0^T \norm{∂_t u} \norm{∇u} \dif t \\
&≤ \left(\int_0^T \norm{∂_t u}^2\right)^{\sfrac{1}{2}} \left(\int_0^T \norm{∇u}^2\right)^{\sfrac{1}{2}}
\end{align*}

Additionally, if $∂Ω$ is $C^2$ (or $Ω$ is a convex polygon) and $u_0 ∈ H_0^1(Ω)$, then $u ∈ L^2(T; H^2(Ω))$ and $∂_t u ∈ L^2(T; L^2(Ω))$. This can be seen formally from taking $v =∂_t u$ as a test function.

\subsection{Space discretization: Finite element approximation}

For a step size $h > 0$ and the domain $Ω$ a polygon, we can define a regular mesh \mesh with the regularity assumption that $\frac{h_K}{ρ_K} ≤ C$. Consider the internal vertices $P_1, P_2, \dotsc, P_N$ (external vertices do not matter much, as the function is zero on the boundary), we will be interested in finding the shape function for those vertices.

As usual, we will consider the space $V_h = \spn \set{φ_1, \dotsc, φ_N}$ and will search for the solution $u_h ∈ V_h$ such that \( \int_Ω \dpd{u_h}{t} + ∇u_h ∇v_h = \int_Ω f v_h \qquad ∀v_h ∈ V_h \label{eq:PDE:HeatEqFEApprox} \)

A small remark: as the solution depends on tie the coefficients of the shape functions must depend on time.


If we take $v_h = φ_i$ for $i = 1, \dotsc, N$, we must have
\[ \sum_{j = 1} u_j'(t) \underbracket{\int_Ω φ_i φ_j}_{M_{ij}} + \sum_{j = 1}^N u_j(t) \underbracket{\int_Ω ∇φ_i ∇φ_j}_{A_{ij}} = \underbracket{\int_Ω f φ_i }_{f_i}\] which translates to the differential system \[ \mM \vu'(t) + \mA \vu(t) = \vf(t) \]

We also need to include the initial condition of $u(x,0) = u_0(x)$. For that, we will enforce a weak equality: $\int_Ω u_h(0) v_h = \int_Ω u_0 v_h \dif x $ for any $v_h ∈ V_h$ or in other words $u_h(0) = Π_h u_0$ where \[ Π_h w(x) = \sum_{j = 1}^N w_j φ_j(x) \] which can be solved in matrix form as $\mM \vw = \vec{W}$ where $W_i = \int_Ω w(x) φ_i(x) \dif x$.

\begin{prop}[A priori error\IS for the heat equation (space discretizacion)] \label{prop:PDE:APrioriHeatSpace} Let $Ω$ be a convex polygon, $u_0 ∈ H_0^1(Ω)$ and $f ∈ L^2(T; L^2(Ω))$. Then, there exists a constant $C > 0$ independendent of $h$ but dependent on the mesh aspect ratio such that \[ \int_0^T \int_Ω (∇(u - u_h))^2 ≤ Ch^2\]
\end{prop}

\begin{proof} With the assumptions of $Ω$ convex polygon, $u_0 ∈ H_0^1(Ω)$ and $f ∈ L^2(T; L^2(Ω))$, \fref{prop:PDE:ExistUniqHeat} tells us that $u ∈ L^2(T; H^2(Ω))$. We will consider the following norm: $\norm{v}^2 = \int_0^T \int_Ω \abs{∇v}^2$. By using the triangle equality, we can see that \[\norm{u - u_h} ≤ \norm{u - Π_h u} + \norm{Π_h u - u_h} \] where $\appl{Π_h}{L^2(Ω)}{V_h}$ is an interpolation operator that we will define later.

The first norm will come from an interpolation estimate, and it does not have anything to do with the heat equation. For the other term, we will use \eqref{eq:PDE:ExtUniqHeatBound} with $v_h$ and then taking the difference with the equation for the finite elements \eqref{eq:PDE:HeatEqFEApprox}, so that for any $v_h ∈ V_h$ we have \[
I = \int_Ω \dpd{}{t} (u - u_h) v_h + \int_Ω ∇(u - u_h)∇v_h \] and in particular for $v_h = Π_h u - u_h$ and using the orthogonality condition we have \[I = \int_Ω \dpd{}{t}(Π_h u - u + u - u_h)(Π_h u - u_h) + \int_Ω ∇(Π_h u - u + u - u_h) ∇(Π_h u - u_h)
\]

We have that $u - u_h$ cancels because I don't know why, and we will define $Π_h$ such that $Π_hu - u$ cancels too. Therefore, we will say that $Π_h w$ for $w ∈ L^2(Ω)$ satisfies that $\int_Ω Π_h w v_h = \int_Ω w v_h$ for any $v_h ∈ V_h$, which corresponds to solving a linear system and therefore is well defined.

We can prove that if $w ∈ H^2(Ω)$, then $\norm{w - Π_hw}_{L^2(Ω)} + h\norm{∇(w - Π_hw)} ≤Ch^2 \norm{w}_{H^2(Ω)}$. This was an exercise question, but with the definition of the interpolation operator we see that $\int (Π_h - w) Π_h w$ is zero and therefore we can replace $Π_h w$ with the Lagrange interpolant $r_h w$ because it will still be zero, and thefore it works because reasons.

We can permute derivative and $Π_h$ and $Π_hu - u$ will go away because reasons agains and this is the most shady proof of this course. Finally, \[ I = \int_Ω ∇(Π_hu - u) ∇(Π_hu-u_h)\] that is, \[ \frac{1}{2} \od{}{t} \norm{Π_hu - u_h}_{L^2(Ω)}^2  + \norm{∇(Π_h u -  u_h)}^2_{L^2(Ω)} ≤ \norm{∇(Π_h u -u_h)} \norm{∇(Π_hu - u_h)} \] and with \nref{prop:Fund:Young} and integrating over time we have \begin{align*}
\frac{1}{2} \od{}{t} \norm{Π_hu - u_h}_{L^2(Ω)}^2  + \frac{1}{2} \norm{∇(Π_h u -  u_h)}^2_{L^2(Ω)} &≤ \frac{1}{2} \norm{∇(Π_h u -u)}^2 \\
\int_Ω \left((Π_h -u_h)(x,T)\right)^2 \dif x + \int_0^T \int_Ω \abs{∇(Π_hu - u_h)}^2 \dif x \dif t &≤ \underbracket{\int_Ω \left((Π_h u - u_h)(x,0)\right)^2}_{= 0} + \\ &\quad + \int_0^T \int_Ω \abs{∇(Π_hu - u)}^2 \dif x \dif t
\end{align*} where the term is zero because it's the initial interpolant and that is zero because again reasons. And all of that is bounded by \textit{big underline} $C h^2 \int_0^T \norm{u}^2_{H^2(Ω)}$.

So the keys for the proof is to start from \[ I = \int_Ω \dpd{}{t} (Π_h u - u_h)(Π_hu - u_h) + \int_Ω ∇(Π_hu - u_h)∇(Π_hu - u_h)\] and then use an interpolant that kills the terms that we don't want.
\end{proof}

\begin{prop} If $u ∈ L^∞(T; H^2(Ω))$ and $\pd{u}{t} ∈ L^2(T; H^2(Ω))$ then there exists $C > 0$ independent of $h$ but depending on the mesh aspect ratio such that \[ \norm{(u - u_h)(T)}_{L^2(Ω)} ≤ Ch^2 \]
\end{prop}

\begin{proof} Exercise session. It's the same proof as the previous one but we kill the gradient term with the interpolant so that $\int_Ω ∇R_h w ∇v_h = \int_Ω ∇w ∇v_h$ for any $v_h ∈ V_h$, which is again perfectly computable.
\end{proof}

One of the exam questions will be heat equation, a priori estimate, finite elements, prove a posteriori estimate.

\begin{prop}[A posteriori error\IS for the heat equation] Assume that $u_0 ∈ H^2(Ω)$. Then there exists a constant $C> 0$ independent of $h, f, u_0$ and $u$ but dependent on the mesh aspect ratio such that \[ \int_0^T \int_Ω \abs{∇(u - u_h)}^2 ≤  C \int_0^T \sum_{K ∈ \mesh} η_K^2 + \text{higher order terms}  \] with \[ η_K^2 = \left(h^2_K\norm{f+ Δu_h - \pd{u_h}{t}} + \frac{1}{2} \sqrt{h_K} \norm{[∇u_h · \vn]}_{L^2(∂K)} \right)^2 \]
\end{prop}

\begin{proof} We start from the integral \[ I = \int_Ω \dpd{}{t} (u - u_h) (u - u_h) + \int_Ω ∇(u - u_h) ∇(u - u_h) \] and we want to relate it to the residual which in our case is defined as \[ \dualp{\mop{Residual}(v), u_h} = \int_Ω \dpd{u_h}{t} v + \int_Ω ∇u_h ∇v - \int_Ω f v \]

Then we do an extremely weird replacement $u$ by $f - u_h$ so that we have \begin{align*} I &= \int_Ω (f - \pd{u_h}{t}) (u - u_h) - \int_Ω ∇u_h ∇(u - u_h)\\
&= \int_Ω \left(f - \dpd{u_h}{t} \right)(u - u_h - v_h) - \int_Ω ∇u_h ∇(u - u_h - v_h) \quad ∀ v_ ∈ V_h \\
&= \sum_{K ∈ \mesh} \int_K \left(f - \dpd{u_h}{t} + Δu_h\right) (u - u_h - v_h) + \frac{1}{2} \int_{∂K} [∇u_h · \vn] (u - u_h - v_h)
\end{align*}

Choosing $v_h(x) = R_h \left((u - u_h)(x,t)\right)$ with $R_h$ the Clément interpolant almost everywhere for $0 ≤ t ≤ T$. Also \[ \norm{v}_{L^2(K)} + \sqrt{h_K} \norm{v}_{L^2(∂K)} ≤ Ch_K \norm{∇v}_{L^2(ΔK)} \] so that we can use Cauchy-Schwartz inequality \begin{align*}
I &≤ C\sum_{K ∈ \mesh}\left( h_K \norm{f - \dpd{u_h}{t} + Δu_h}_{L^2(K)} + \frac{1}{2} \sqrt{h_K} \norm{[∇u_h · \vn]}_{L^2(∂K)}\right) \norm{∇(u - u_h)}_{L^2(ΔK)} \\
&≤ C \left(\sum_{K ∈ \mesh} η_K^2\right)^{\sfrac{1}{2}} \left(\sum_{K ∈ \mesh} \norm{∇(u - u_h)}_{L^2(ΔK)}^2 \right)^{\sfrac{1}{2}} ≤ \tilde{C} \left(\sum_{K ∈ \mesh} η_K^2\right)^{\sfrac{1}{2}} \norm{∇(u - u_h)}_{L^2(Ω)} \end{align*}
so that finally we have \[ I = \frac{1}{2} \od{}{t} \norm{u - u_h}_{L^2(Ω)}^2  + \frac{1}{2} \norm{∇(u - u_h)}^2_{L^2(Ω)} ≤ \frac{1}{2} \tilde{C}\sum_{K ∈ \mesh} η_K^2
\] and now integrating in time \[ \underbracket{\norm{(u - u_h)(T)}^2_{L^2(Ω)}}_{\sim h^4} + \underbracket{\int_0^T \int_Ω \abs{∇(u - u_h)}^2}_{\sim h^2} ≤ \underbracket{\tilde{C} \int_0^T}_{\sim h^2} + \underbracket{\int_Ω (u -u_h)^2(0)}_{≤ \tilde{C}h^4} \] so we can discard the left term with $\sim h^4$.
\end{proof}

\subsection{Time and space discretization}

Now we will also discretize time with a time step $τ ≝ \frac{T}{M}$, with steps $t_n = τn$, $n = 0, \dotsc, M$. $u_h^n$ will the approximation of $u_h(t_n)$.

We will consider the implicit Euler scheme, so that for each step we have to solve the equation \( \int_Ω \frac{u_h^{n+1} - u_h^n}{τ} v_h + \int_Ω ∇u_h^{n+1} ∇v_h = \int_Ω f(t_{n+1}) v_h\qquad ∀v_h ∈ V_h \label{eq:PDE:ImplicitEulerHeat} \) and with $u_h^0 = Π_h u_0$ (the $L^2$ projection). But we could use the Lagrange interpolant too. We use the implicit scheme because it works with high order finite elements. The explicit scheme does not work well except for discontinuous finite elements.

\begin{prop}[Stability\IS of the Euler scheme for the heat equation] If $f = 0$, the implicit scheme is $L^2(Ω)$ is stable for all steps $h > 0$, $τ > 0$. That is, $\norm{u_h^{n+1}}_{L^2(Ω)} ≤ \norm{u_h^n}_{L^2(Ω)}$ for all $n = 0, M$.
\end{prop}

\begin{proof} Set $v_h = u_h^{n+1}$ in \eqref{eq:PDE:ImplicitEulerHeat} so that we have \[ \int_Ω \left(\frac{u_h^{n+1} - u_h^n}{τ} u_h^{n + 1} + \abs{∇u_h^{n+1}}^2 \right) = 0 \] and using the trick $(a - b)a = \frac{a^2}{2} - \frac{b^2}{2} + \frac{1}{2}(a-b)^2$ we have that the previous thing is equal to \[ \frac{1}{2} \int_Ω (u_h^{n+1})^2 - \frac{1}{2} \int_Ω (u_h^n)^2 + \frac{1}{2} \int_Ω (u_h^{n+1} - u_h^n)^2 + \int_Ω τ \abs{∇u_h^{n+1}}^2 = 0 \] and we have it.
\end{proof}

A remark: the explicit Euler scheme is $L^2(Ω)$ is stable provided that $τ ≤ C h^2$ where $C$ is the constant arising from the inverse inequality. Left as an exercise.

Also, \eqref{eq:PDE:ImplicitEulerHeat} is equivalent to solving a linear system as we saw in
\fref{sec:ODE:ParabolicProblemDiscrSpace}.

\begin{prop}[A priori error\IS for the heat equation (time-space discretization)] Assume that we have $u ∈ L^∞(T; H^2(Ω))$ and $\pd[2]{u}{t} ∈ L^2(T; H^2(Ω))$. Then there exists $C$ independent of both discretization steps $τ$ and $h$ such that \[ \sum_{m = 1}^M τ \norm{∇u(t_n) - ∇u_h^n}_{L^2(Ω)}^2 ≤ C (h^2 + τ^2) \] where $t_M = M τ = T$.
\end{prop}

Basically, this proposition is just the discrete counterpart of the integral of \fref{prop:PDE:APrioriHeatSpace} and accordingly the proof is very similar.

\begin{proof} We just have to plug the exact solution into the numerical scheme \eqref{eq:PDE:ImplicitEulerHeat}. We will write the heat equation and then use Taylor expansion to approximate the derivative: \begin{align*}
u(t_n) &= u(t_{n+1}) + \int_{t_{n}}^{t_{n+1}} \dpd{u}{t}(t) \dif t \\
\dpd{u}{t}(t) &= \dpd{u}{t}(t_{n+1}) + \int_{t_{n+1}}^t \dpd[2]{u}{t} (s) \dif s = \\
&= \frac{u(t_{n+1}) - u(t_n)}{τ} - \dpd{u}{t}(t_{n+1}) + \frac{1}{τ} \int_{t_n}^{t_{n+1}} \int_{t_{n+1}}^t \dpd[2]{u}{t} (s) \dif s \dif t
\end{align*} and therefore $u$ satisfies \begin{multline*}
\int_Ω \frac{u(t_{n+1}) - u(t_n)}{τ} v_h + \int_Ω ∇u(t_{n+1}) ∇v_h =\\ = \int_Ω f(t_{n+1}) v_h - \frac{1}{τ} \int_Ω \int_{t_{n+1}}^{t_n} \int_{t_{n+1}}^t \dpd[2]{u}{t}(s) v_h \dif s \dif t \qquad ∀v_h ∈ V_h
\end{multline*}

Now let $e^{n+1}(x) ≝ u(t_{n+1}) - u_h^{n+1}(x)$. Then we have \[ \int_Ω \frac{e^{n+1} - e^n}{τ} v_h + \int_Ω ∇e^{n+1} ∇v_h = - \frac{1}{τ} \int_Ω \int_{t_{n+1}}^{t_n} \int_{t_{n+1}}^t \dpd[2]{u}{t}(s) v_h \dif s \dif t  \qquad ∀v_h ∈ V_h\] (which is some kind of orthogonality property).

Adding and removing the $L^2$ projection of $u$, we have \[ e^{n+1} = \underbracket{u(t_{n+1}) - Π_hu(t_{n+1})}_{\text{Interpolation error}} + \underbracket{Π_hu(t_{n+1}) - u_h^{n+1}}_{e_h^{n+1}} \]

We know that the interpolation error is going to zero if $u$ is smooth enough. The question is what happens with $e_h^{n+1}$. We start from \begin{gather*}
\int_Ω\left( \frac{e^{n+1}_h - e_h^n}{τ} e_h^{n+1} + \abs{∇e_h^{n+1}}^2 \right) = \\
= \int_Ω \frac{Π_hu(t_{n+1}) - u(t_{n+1}) + \overbracket{u(t_{n+1}) - u_h^{n+1}}^{e^{n+1}} - (Π_hu(t_n) -u(t_n) + \overbracket{u(t_n) - u_h^n}^{e^n})}{τ} e_h^{n+1} \\
\quad + \int_Ω ∇(Π_h u(t_{n+1}) - u(t_{n+1}) + \overbracket{u(t_{n+1}) - u_h^{n+1}}^{e^{n+1}})∇e_h^n = \\
= \underbracket{\int_Ω \frac{Π_hu(t_{n+1}) - u(t_{n+1}) + Π_hu(t_n) - u(t_n)}{τ}e_h^{n+1}}_{\to 0 } + \int_Ω ∇(Π_hu(t_{n+1}) - u(t_{n+1}))∇e_h^{n+1} \\ \quad - \frac{1}{τ} \int_Ω \int_{t_{n+1}}^{t_n} \int_{t_{n+1}}^t \dpd[2]{u}{t}(s) e_h^{n+1} \dif s \dif t
\end{gather*}

As the first integral goes to zero because of the definition of the projection operator, so we are left with the error due to the space and time discretization. Now we have to conclude, using $(a-b)a = \frac{1}{2}a^2 - \frac{1}{2} b^2 + \frac{1}{2}(a-b)^2$ and Poincaré inequality to bound the norm of $e_h^{n+1}$ by that of its gradient, so therefore we have \begin{multline*}
\frac{1}{2τ} \int_Ω (e_h^{n+1})^2 - \frac{1}{2τ} \int_Ω (e_h^n)^2 + \frac{1}{2τ} (e_h^{n+1} - e_h^n)^2 + \int_Ω \abs{∇e_h^{n+1}}^2 ≤ \\ ≤ \norm{∇(u(t_{n+1}) - Π_h u(t_{n+1}))}_{L^2} \norm{∇e_h^{n+1}}_{L^2} + \norm{\frac{1}{τ}\int_Ω \int_{t_{n+1}}^{t_n} \int_{t_{n+1}}^t \dpd[2]{u}{t}(s)}_{L^2}C_p \norm{∇e_h^{n+1}}_{L^2}
\end{multline*}
and by using Young's inequality we know that \[ \norm{\frac{1}{τ}\int_Ω \int_{t_{n+1}}^{t_n} \int_{t_{n+1}}^t \dpd[2]{u}{t}(s)}_{L^2}C_p \norm{∇e_h^{n+1}}_{L^2} ≤ \frac{1}{4} \norm{\frac{1}{τ}\int_Ω \int_{t_{n+1}}^{t_n} \int_{t_{n+1}}^t \dpd[2]{u}{t}(s)}_{L^2}^2  + C_p^2 \norm{∇e_h^{n+1}}_{L^2}^2 \] and putting everything together it works. Trust me.

We have just to be sure that weird integral goes with $τ^2$, so \begin{multline*}
\int_Ω \left(\frac{1}{τ}\int_{t_{n+1}}^{t_n} \int_{t_{n+1}}^t \dpd[2]{u}{t}(s))\right)  ≤ \frac{1}{τ^2} \int_Ω \int^{t_{n+1}}_{t_n} \int_{t_{n+1}}^t τ^2 \left(\dpd[2]{u}{t}(s)\right)^2 ≤ \\  ≤ τ^2 \int_Ω \int_{t_n}^{t_{n+1}} \left(\dpd[2]{u}{t}\right)^2 ≤ τ \int_Ω \int_{t_n}^{t_{n+1}} \left(\dpd[2]{u}{t}\right)^2
\end{multline*}

This proof was doomed from the start. Fix it later.

Prof said that this is not going to be on the exam so ok.
\end{proof}

For the a posteriori error estimate, we are going to do the opposite: we will plug $u_h^n$ into the PDE.  In order to plug discrete solutions into a derivative, we will just consider $u_{hτ}(x,t)$ as the continuous piecewise linear interpolator of the discrete solutions, so that \[ u_{hτ}(x, t) = u_h^n(x) + (t - t_n) \frac{u_h^{n+1} - u_h^n}{τ} \] and we will have that $\pd{u_{hτ}}{t} = \frac{u_h^{n+1} - u_h^n}{τ}$. Therefore, $u_{hτ}$ satisfies that \( \label{eq:PDE:Killmenow}
\int_Ω \dpd{u_{hτ}}{t} v_h + \int_Ω ∇u_{hτ} ∇v_h = \int_Ω f(t) v_h
+ \int_Ω \left(f(t_{n+1}) - f(t)\right)v_h + \int_Ω ∇(u_{hτ} - u_h^{n+1}) ∇v_h
\)

Now we can word I can't recall the proposition.

\begin{prop}[A posteriori error\IS for the heat equation (time-space discretization)] There exists a constant $C > 0$ depending only on the mesh aspect ratio such that \[ \norm{(u - u_{hτ})(t_{n+1})}_{L^2}^2 + \int_{t_n}^{t_{n+1}} \norm{∇(u - u_{hτ})}^2_{L^2(Ω)} ≤ \norm{(u - u_{hτ})(t_n)}^2_{L^2} + C \int_{t_n}^{t_{n+1}} \sum_{K ∈ \mesh} η_K^2 \] with \[ η_K = h_K \norm{f - \dpd{u_{hτ}}{t} + Δu_{hτ}}_{L^2(K)} + \frac{1}{2} \sqrt{h_K} \norm{[u_{hτ} · \vn]}_{L^2(∂K)} + \norm{f(t) - f(t_{n+1})}_{L^2(K)} + \norm{∇(u_{hτ} - u_h^{n+1})}_{L^2(K)} \]
\end{prop}

\begin{proof} Set $e ≝ u - u_{hτ}$, and we start from \begin{align*}
I = \int_Ω \dpd{e}{t} e + \int_Ω \abs{∇e}^2 &= \int_Ω \left(\dpd{u}{t} - \dpd{u_{hτ}}{t}\right) e + \int_Ω ∇(u - u_{hτ}) ∇e = \\
&= \int_Ω \left(f - \dpd{u_{hτ}}{t} \right) e - \int_Ω ∇u_{hτ} ∇ e \\
&= \int_Ω \left(f - \dpd{u_{hτ}}{t}\right) ( e - v_h) - \int_Ω ∇u_{hτ} ∇(e - v_h) + \\
&\qquad + \int_Ω \left(f(t) - f(t_{n+1}) \right) v_h + \int_Ω ∇(u_{hτ} - u_{h}^{n+1})∇v_h
\end{align*} which is the residual and later we use \eqref{eq:PDE:Killmenow}.

The classical thing we do in space is to choose the Clèment interpolant for $v_h$ and then do things. But Here we can integrate by parts over all triangles so that is equal to
\begin{multline*}
\sum_{K ∈ \mesh} \left(\int_K \left(f - \dpd{u_{hτ}}{t} + Δu_{hτ}\right)(e - v_h) + \frac{1}{2} \int_{∂K} [∇u_{hτ}· \vn] (e - v_h) \right. + \\ \left. + \int_K (f(t) - f(t_{n+1})) v_h + \int_K ∇(u_{hτ} - u_h^{n+1}) ∇v_h \right)
\end{multline*}

Now selecting $v_h = R_h e$ the Clément interpolant in space for each $t$, so that we know that \[ \norm{v - R_h v}_{L^2(K)} + \sqrt{h_K} \norm{v - R_h v}_{L^2(∂K)} + h_K \norm{∇(v - R_hv)}_{L^2(K)} ≤ Ch_K\norm{∇v}_{L^2(ΔK)} \] and therefore \begin{align*}
I &≤ C \sum_{K ∈ \mesh} \left(h_K \norm{f - \dpd{u_{hτ}}{t} + Δu_{hτ}}_{L^2(K)} + \frac{1}{2} \sqrt{h_K} \norm{[∇u_{hτ} · \vn]}_{L^2(∂K)} \right. + \\
&\qquad \left. \norm{f(t) - f(t_{n+1})}_{L^2(K)} + \norm{∇(u_{hτ} - u_h^{n+1})}_{L^2(K)} \right)\norm{∇e}_{L^2(ΔK)} \\
&≤ C \left(\sum_{K ∈ \mesh} η_K^2 \right)^{\sfrac{1}{2}} \left(\sum_{K ∈ \mesh} \norm{∇e}_{L^2(ΔK)}^2 \right)^{\sfrac{1}{2}} ≤ \frac{\tilde{C}}{2} \left(\sum_{K ∈ \mesh} η_K^2 \right)^{\sfrac{1}{2}}  + \frac{1}{2} \norm{∇e}^2_{L^2(Ω)}
\end{align*} and then we have to integrate in time between $t_n$ and $t_{n+1}$.
\end{proof}



% -*- root: ../NumericalApproximationofPDEs.tex -*-

\section{The heat equation}

The first equation that we will be studying will be the heat equation. We will search for a function in space and time $u$ \begin{align*}
\appl{u}{Ω×[0,T]&}{ℝ} \\
(x,t) &\longmapsto u(x,t)
\end{align*} that will model the temperature defined on a domain $Ω ⊂ ℝ^2$ (or $ℝ^3$ too) such that
\( \begin{cases}
\dpd{u}{t} - Δu = f & \text{in } Ω × (0,T) \\
u = 0 & \text{on } ∂Ω × (0,T) \\
u(x,0) = u_0(x) & ∀x ∈ Ω
\end{cases} \label{eq:PDE:HeatEquation} \)

This comes from an energy conservation principle of thermodynamics: for any domain $D ⊂ Ω$, the time variation of the thermal energy contained in $D$ must be equal to the heat that ``goes out'' through the boundary of the domain plus the energy due to the source term $f$. This translates to the following equation:
\( \od{}{t} \int_D ρ c_p u \dif x - \int_{∂D} k ∇u \vn \dif s = \int_D f \dif x \label{eq:PDE:ThermoConservation} \) which can be transformed into the differential equation \eqref{eq:PDE:HeatEquation} using the divergence theorem\footnote{$\int_D \dv \va = \int_{∂D} \va \vn$}, putting the derivative inside of the integral and getting rid of the constants and integrals.

The first remark is that we can get a first a priori error estimate.

\begin{prop}[A priori estimate\IS for the heat equation] \label{prop:PDE:APrioriHeatEq} If $u$ is a solution of the heat equation \eqref{eq:PDE:HeatEquation}, then the following a priori estimate holds:
\( \label{eq:PDE:APrioriHeatEq} \int_0^T \norm{u}_{H^1(Ω)}^2 \dif t ≤ \int_Ω u_0^2 \dif x + C_p^2 \int_0^T \norm{f}^2 \dif t \)
\end{prop}

\begin{proof} Multiplying the heat equation by $u$ and operating, we have it directly.
\begin{align*}
\int_Ω \dpd{u}{t} u - Δu u \dif x &= \int_Ω fu \dif x \\
\frac{1}{2} \dod{}{t} \int_Ω u^2 \dif x + \frac{1}{2} \int_Ω \abs{∇u}^2 \dif x - \cancelwhy{\int_{∂Ω} (∇u \vn)u \dif s}{\restr{u}{∂Ω} = 0} &= \int_Ω fu \dif x \\ &≤ C_p \norm{f}_{L^2(Ω)} \norm{∇u}_{L^2(Ω)} \\
\frac{1}{2} \od{}{t} \int_Ω u^2 + \abs{∇u}^2 &≤ \frac{C_p}{2} \norm{f}^2 + \frac{1}{2} \norm{∇u}^2 \\
\int_0^T \norm{u}_{H^1(Ω)}^2 \dif t &≤ \int_Ω u_0^2 \dif x + C_p^2 \int_0^T \norm{f}^2 \dif t
\end{align*}
\end{proof}

This estimate gives us uniqueness: if there are two solutions $u_1, u_2$, we can substract one from another and $u_1 - u_2$ satisfies the heat equation with source term $f = 0$ and initial condition $u_0 = 0$, and by the a priori estimate we have $\norm{u_1 - u_2 }= 0$. It also gives us a very natural consequence: the maximum principle.

\begin{prop}[Maximum\IS principle] If $f ≥ 0$ and $u_0 ≥ 0$, then the solution $u ≥ 0$.
\end{prop}

\begin{proof} We can decompose $u = u^+ - u_-$ with $u^+ = \sup (u, 0)$ and $u^- = \sup -u, 0$. Therefore, we can see that
\begin{align*}
\int_Ω \dpd{u}{t} u^- + ∇u ∇u^- &= \int_Ω f u^- \\
\int_Ω \dpd{u^-}{t} u^- + \abs{∇u^-}^2 &= \int_Ω - f u^- \\
\frac{1}{2} \od{}{t} \int_Ω (u^-)^2 + \int_Ω \abs{∇u^-}^2 &≤ 0
\end{align*}

Integrating between $0$ and $T$, we have
\[\frac{1}{2} \int_Ω u^-(x,T)^2 + \int_0^T \int_Ω \abs{∇u^-}^2 ≤ \frac{1}{2} (u_0^-)^2 \]

As $u_0 ≥ 0$, $u_0^- = 0$ so the left hand side must be less or equal than zero. The only possibility is $u^- = 0$ and therefore $u ≥ 0$.
\end{proof}

We also have an interesting property of the solution: for positive time, the solution is smooth even when the initial function $u_0$ is not.

Now we will see the existence of the solution.

\begin{prop}[Existence and uniqueness\IS for the heat equation solution] For all $T > 0$, for any source term $f ∈ L^2(0,T; L^2(Ω))$ and for any initial distribution $u_0 ∈ L^2(Ω)$, there exists a unique solution $u ∈ L^2(0,T; H_0^1(Ω))$ for the heat equation with $\pd{u}{t} ∈ L^2(0,T; (H_0^1(Ω)')$ such that \[ \dualp[(H_0^1(Ω))'][H_0^1(Ω)]{\pd{u}{t}, v} + \int_Ω ∇u ∇v = \int_Ω fv \dif x \qquad ∀ v ∈ H_0^1(Ω)\] for $0 ≤ t ≤ T$ and with $u(x,0) = u_0(x)$ almost everywhere.
\end{prop}

Before going into the proof, we need to define those weird spaces. We will say that $v(x,t) ∈ L^2(0,T; X)$ with $X$ a Banach space if $v$ is strongly measurable\footnote{Basically, measurable in a way that we can integrate it over a Banach space.} and $\int_0^T \norm{v}_X < ∞$. This is valid for $X = L^2(Ω)$ (space of the source term $f$) and $X = H^1(Ω)$ (for the solution $u$). For the dual $X = (H_0^1(Ω))'$ (the space of $∂_t u$) recall the definition of the \nref{def:Fund:NormOperator}.
% TODO: Define better these spaces

Now, if $u ∈ L^2(0,T; H_0^1(Ω))$ and $∂_t u ∈ L^2(0,T; (H_0^1(Ω))')$, then it can be proved that $u ∈ C^0([0,T]; L^2(Ω))$: we start from
\begin{align*}
\frac{1}{2} \od{}{t}  \int_Ω u^2 &= \dualp{∂_t u, u} \\
\frac{1}{2} \int_Ω u(x,T)^2 \dif x - \frac{1}{2} \int_Ω u(x,0)^2 \dif x &= \int_0^T \dualp{∂_t u, u} \dif t \\
&≤ \int_0^T \norm{∂_t u} \norm{∇u} \dif t \\
&≤ \left(\int_0^T \norm{∂_t u}^2\right)^{\sfrac{1}{2}} \left(\int_0^T \norm{∇u}^2\right)^{\sfrac{1}{2}}
\end{align*}

Now regularity (¿?). If $∂Ω$ is $C^2$ or $Ω$ is a convex polygon and $u_0 ∈ H_0^1(Ω)$, then $u ∈ L^2(0,T; H^2(Ω))$ and $∂_t u ∈ L^2(0,T; L^2(Ω))$. This can be seen formally from taking $v =∂_t u$ as a test function.

\subsection{Finite element approximation}

For a step size $h > 0$ and the domain $Ω$ a polygon, we can define a regular mesh \mesh with the regularity assumption that $\frac{h_K}{ρ_K} ≤ C$. Consider the internal vertices $P_1, P_2, \dotsc, P_N$ (external vertices do not matter much, as the function is zero on the boundary), we will be interested in finding the shape function for those vertices.

As usual, we will consider the space $V_h = \spn \set{φ_1, \dotsc, φ_N}$ and will search for the solution $u_h ∈ V_h$ such that \[ \int_Ω \dpd{u_h}{t} + ∇u_h ∇v_h = \int_Ω f v_h \qquad ∀v_h ∈ V_h\]

A small remark: as the solution depends on tie the coefficients of the shape functions must depend on time.


If we take $v_h = φ_i$ for $i = 1, \dotsc, N$, we must have
\[ \sum_{j = 1} u_j'(t) \underbracket{\int_Ω φ_i φ_j}_{M_{ij}} + \sum_{j = 1}^N u_j(t) \underbracket{\int_Ω ∇φ_i ∇φ_j}_{A_{ij}} = \underbracket{\int_Ω f φ_i }_{f_i}\] which translates to the differential system \[ \mM \vu'(t) + \mA \vu(t) = \vf(t) \]

We also need to include the initial condition of $u(x,0) = u_0(x)$. For that, we will enforce a weak equality: $\int_Ω u_h(0) v_h = \int_Ω u_0 v_h \dif x $ for any $v_h ∈ V_h$ or in other words $u_h(0) = Π_h u_0$ where \[ Π_h w(x) = \sum_{j = 1}^N w_j φ_j(x) \] which can be solved in matrix form as $\mM \vw = \vec{W}$ where $W_i = \int_Ω w(x) φ_i(x) \dif x$.

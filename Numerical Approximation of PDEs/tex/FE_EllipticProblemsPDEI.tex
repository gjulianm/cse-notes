% -*- root: ../NumericalApproximationofPDEs.tex -*-
\section{Advection-diffusion-reaction}

A more complex form of elliptic problems are advection-diffusion-reaction, where the differential operator is \[ Lu = - \sum_{i,j=1}^d \dpd{}{x_i} \left(a_{ij}\dpd{u}{x_j}\right) + \sum_{i=1}^d b_i \dpd{u}{x_i} + c u = - \dv (A(x) \grad u) + \vb(x) \grad u + c u\] with $A(x) ∈ ℝ^{d×d}, \vb(x) ∈ ℝ^d, c ∈ ℝ$. Our problem, as usual, is $Lu = f$.

These three terms model respectively diffusion, with $A$ being the matrix of coefficients that depend on the axis (the material is not uniform), the transport term along the vector field, and the reaction (for example, a chemical that reacts and its concentration decreases)

In order for the problem to remain elliptic, we require $A(x)$ to be positive definite for all $x ∈ ℝ^d$.

As in previous cases, we can have Dirichlet boundary conditions $\restr{u}{Γ_D} = g$ and Neumann boundary conditions (which are a little bit more complicated): \[ A \grad u · \vn - (b \vn) u = h \quad\text{ on } Γ_N \]

We will want to do a weak formulation where the boundary terms of the problem appear naturally. As always, we multiply by a test function and integrate by parts the divergence and possibly the $b \grad u$ depending on the boundary conditions. Without doing the computations, the weak formulation will end up being \( \int_Ω A \grad u \grad v + b \grad u v + c u v = \int_Ω fv + \int_{Γ_N} h v \label{eq:Elliptic:ADRProblemWeak} \) with $u ∈ H^1$, $\restr{u}{Γ_D} = g$.

This has the same structure as in previous cases, so we are in the conditions of the \nref{thm:Theory:LaxMilgram} and there exists a unique solution. Coercivity would need a little bit of work, and may present problems with mixed boundary conditions. Enforcing Neumann conditions on boundaries with incoming flow may cause problems with coercivity.

However, regularity and the results discussed in \fref{sec:PDE:PoissonRegularity} still hold without problems.

\section{Linear elasticity}

\begin{figure}[hbtp]
\inputtikz{ElasticDeformation}
\caption{Elastic deformation of a something.}
\label{fig:Elliptic:ElasticDeformation}
\end{figure}

In this problem, we start with an undeformed configuration $Ω ⊂ ℝ^d$, and we want to study the displacement $\appl{\vu}{Ω}{ℝ^d}$. The involved terms are the strain measure \[ ε (\vu) = \frac{\grad \vu + \trans{(\grad \vu)}}{2}\], the stress tensor $σ = σ(ε)$ given by \[ σ_{ij} = \sum_{k,l=1}^d c_{ijkl} ε_{kl} \], which usually can be expressed as \[ σ(ε) = 2με + λ\tr(ε) I \] with $μ,λ$ the Lamé constants.

With all of this, our balance equation is \( \begin{cases} - \dv σ(ε(\vu)) = \vec{f} & \text{in } Ω \\
\vu = \vec{g} & \text{on } Γ_D \\
σ(u) · \vn = \vd & \text{on } Γ_N \end{cases} \)

We may want to write now our weak formulation and integrate by parts, caring a little bit about what is a tensor and what is a vector
\begin{align*}
0
	&= \int_Ω \left[- \dv (σ(ε(\vu))) - \vec{f} \right] · \vv = \\
	&= \int_Ω σ(ε(\vu)) \grad \vv - \int_{∂Ω} (σ · \vn) · \vv - \int_Ω \vec{f} \vv \\
\int_Ω σ(ε(\vu)) \colon \vv &= \int_Ω \vec{f}\vv + \int_{Γ_N} \vd \vv
\end{align*}

I'm a little bit lost but \[ \int_Ω - \dv (σ) \vv = \int_Ω - \sum_i \sum_j ∂_j σ_{ij}v_i = \sum_{ij} \int_Ω σ_{ij}∂_jv_i - \int_{∂Ω}σ_{ij}n_j v_i \]

We can rewrite the first term because \begin{align*}
\int_Ω σ (ε(\vu)) \colon \vv
	&= \int_Ω σ(ε(\vu))\colon ε(\vv) = \\
	&= \int_Ω(2με(\vu) + λ\tr(ε(\vu))I) \colon ε(\vv) = \\
	&= \int_Ω 2με(\vu) \colon ε(\vv) + λ\tr(ε(\vu))I \colon ε(\vv) = \\
	&= \int_Ω 2μ \frac{\grad \vu + \trans{(\grad \vu)}}{2} \colon \frac{\grad \vv + \trans{(\grad \vv)}}{2} + λ\dv \vu \dv \vv = \\
	&= a(\vu, \vv)
\end{align*}

So we have a bilinear form again and whatever.

\section{Error estimates}

Along this section, we will see examples of elliptic problems and applying to them the error estimates devised in \fref{sec:Theory:ApproxResults}.

\subsection{Error estimates for finite element approximations}

Again, we discuss the estimates for the model problem and then we will see that the procedures generalize. This problem will be the Poisson equations with boundary conditions:
\[ \begin{cases}
-Δu =f & \text{in } Ω \\
∂_\vn u = d & \text{on } Γ_N \\
u= g & \text{on } Γ_D
\end{cases}\]

Its abstract weak form is the problem of finding a function $u ∈ V_g$ such that $a(u,v) = F(v)$ for any $v ∈ V_0$, with
\begin{align*}
V_g &= \set{v ∈ H^1(Ω) \tq \restr{v}{Γ_D} = g} \\
a(u,v) &= \int_Ω ∇u∇v \\
F(v) &= \int_Ω fv + \int_{Γ_N} d v
\end{align*} and assuming that $a$ is continuous ($a(u,v) ≤ M \norm{u}_V \norm{v}_V$) and coercive with coercivity coefficient $α = \frac{1}{1 + c_p^2}$, and $F$ a bounded linear operator. For the spaces $V_0, V_g$ we pick the $H^1$ norm.

Our finite element space will be $X_h^r$, the space of $P_r$ finite elements on the mesh $\mesh$. We can construct also the finite element space vanishing on the boundary as $V_{0,h} = X_h^r ∩ H_{Γ_D}^1$. The finite element space for $V_g$ is a little bit more tricky, but we can simply try to approximate the boundary value with the interpolation operator so $V_{h,g} = \set{ v_h ∈ X_h^r \tq \restr{v_h}{Γ_D} = I_h^r g}$.

If he have homogeneous Dirichlet boundary conditions (that is, $g \equiv 0$) we have the \nref{lem:Theory:Cea} which says that \[ \norm{u - u_h}_V ≤ \frac{M}{α} \inf_{v_h ∈ V_h} \norm{u - v_h}_V \]

With that, we can apply the error estimates for interpolation so that $\inf_{v_h ∈ V_{h,0}} \norm{u -v_h}_{H^1} ≤ \norm{u - I_h^r u}_{H^1} ≤ \abs{u}_{H^η}$ and our estimate is \[ \norm{u-u_h}_{H^1} ≤ C h^{η - 1} \abs{u}_{H^η} \] with $η = \min \set{r + 1, s}$ and $u ∈ H^s(Ω)$.

\subsubsection{Error estimate in $L^2$}

We will use the Aubin-Nitsche trick and we will use duality. In order to estimate the error, we define $e_h = u - u_h$ and the adjoint problem of searching for a $φ ∈ V$ such that \[ a(v,φ) = \int_{Ω} e_h v \qquad ∀v ∈ V\]

Thus, in order to estimate the error in $L^2$ we can do the following: \[ \norm{u-u_h}^2_{L^2} = \norm{e_h}_{L^2}^2 = \int_Ω e_h e_h = a(e_h, φ) = a(u - u_h, φ) \]

By the Galerkin orthogonality condition, we know that $a(u - u_h, v_h) = 0$ for any $v_h ∈ V_h$. That allows to substract any $w_h ∈ V_h$ to φ, and then \[ a(u-u_h, φ) = a(u - u_h, φ - w_h) ≤ M \norm{u - u_h}_V \inf_{w_h ∈ V_h} \norm{φ-w_h}_V \]

We must know now which is the regularity of this function φ. To do that, we must notice that φ is the solution to the porblem \[
\begin{cases} - Δφ = e_h & \text{in } Ω \\
∂_\vn φ = 0 & \text{on } Γ_N \\
φ = 0 & \text{on } Γ_D
\end{cases} \] which in turn depends on the smoothness of $e_h$. We will use however only that $e_h ∈ L^2$, if Ω is a convex polygon. So, assuming $φ ∈ H^2$ we can bound the error in $L^2$, getting one more order of convergence: \[ \norm{u - u_h}_{L^2} ≤ C h \norm{u - u_h}_{H^1} \]

% -*- root: ../NumericalApproximationofPDEs.tex -*-
\section{Stokes problem}

Once we know the theory, we are going to apply it to the study of the Stokes problem and its error estimates. Recall that the Stokes problem is the following:
\(
\begin{cases}
-Δ \vu + \grad p = \vf & \text{in } Ω \\
\dv \vu = 0 & \text{in } Ω \\
\vu = 0 & \text{on } ∂Ω
\end{cases} \label{eq:PDE:StokesProblem} \)

The weak formulation of this is finding $u ∈ H_0^1(Ω)^2$, $p ∈ L_0^2(Ω)$ such that
\( \int_Ω ∇\vu : ∇\vv - \int_Ω p \dv \vu - \int_Ω \dv \vu q  = \int_Ω \vf \vv \) for all functions $\vv ∈ H_0^1(Ω)^2$ and $q ∈ L_0^2(Ω)$ and we can prove the inf-sup condition \eqref{eq:PDE:InfSup} using the trick of that for any $p ∈ L^2(Ω)$ there is a $\vw ∈ H_0^1(Ω)^2$ such that $\norm{p}^{2}_{L^2(Ω)} = \int_Ω p \dv \vw $ with $\norm{∇w} ≤ C \norm{p}$.

\subsection{Finite element approximation}

As discussed in \fref{sec:PDE:NumericalApproximationMixedProblems}, we need to tweak a little bit the numerical approximation by finite elements to get a stable method, adding the term \( - \sum_{K ∈ \mesh} α h_k^2 \int_K (- Δ\vu_h + ∇p_h -f )(-Δ\vv_h + ∇q_h) \) with $α > 0$. We use piecewise linear gradient element so the gradient is not continuous. We can prove that the scheme is stable by setting $v_h = u_h$, $q_h = - p_h$ so that \[ \norm{∇u_h}_{L^2(Ω)}^2 + \sum_{K ∈ \mesh} α h_K^2 \norm{∇p_h}^2_{L^2(K)} = 0 \] when $f = 0$ so that we need $∇u_h = ∇p_h = 0$. Then they are constant and because they are zero on the boundary they are zero everywhere, and therefore the kernel of the matrix is zero. If we hadn't added the last term, we wouldn't have restrictions on $p_h$ and the matrix would not be invertible.

\subsection{A priori error estimates}

The a priori estimate is \[ \norm{u,p} ≤ C \sup \frac{a(u,p; v,q)}{\norm{v,q}} \]

\subsection{A posteriori error estimates}

\begin{prop} \label{eq:PDE:APosterioriErrorStokes} For the Stokes problem \eqref{eq:PDE:StokesProblem}, there exists a constant $C > 0$ independent of $f, \vu, p, h$ but dependendent on the mesh aspect ratio such that \( \norm{∇(u - u_h)}^2_{L^2(Ω)} + \norm{p - p_h}_{L^2(Ω)} ≤ C \sum_{K ∈ \mesh} η_K^2 \) for the solutions $u_h, p_h$ of the Galerkin problem, with \[ η_K^2 = \left(h_k \norm{f + Δu_h - ∇p_h}_{L^2(K)} + \frac{1}{2} h_K^{\sfrac{1}{2}} · \norm{[∇\vu_h · \vn]}_{L^2(∂K)}\right)^2 + \norm{\dv u_h}^2_{L^2(K)}\]
\end{prop}

\begin{proof} We know from a proposition that I did not copy from the previous lecture, that there exists a constant $C$ such that \[ \norm{\vu - \vu_h, p - p_h} ≤ C \sup_{(v,q) ∈ H} \frac{a(u - u_h, p - p_h, v, q)}{\norm{v, q}} \] with $H = V × Q$

We want to prove that $∀(v,q) ∈ H$, $a(u - u_h, p - p_h, v, q) ≤ C_2 \norm{v,q} \left(\sum_{K ∈ \mesh} η_K^2\right)^{\sfrac{1}{2}}$. We will try to do it by relation to the residual.

We have that \begin{align*}
a(u - u_h, p - p_h, v, q)
	&= F(v,q) - a(u_h, p_h, v, q) = \\
	&= F(v,q) - a(u_h, p_h, v, q) + \\
	&\quad + (F - F_h)(v_h, q_h) - (a - a_h)(u_h, p_h, v, q) = \\
	&= \int_Ω \left[\vf (\vv - \vv_h) - ∇\vu : ∇(\vv - \vv_h) \right. \\
	&\qquad \left. + p_h \dv (\vv - \vv_h) + \dv \vu_h (q - q_h)\right] + \\
	&\quad + \sum_{K ∈ \mesh} α h_k^2 (f + Δ\vu_h - ∇p_h) (-∇\vv_h + ∇q_h)
\end{align*}

What we have done is that we have used the inf-sup condition to relate the error to the residual, and now we will evaluate it to get to the strong form of the equation by integrating by parts over all elements of the mesh.
\begin{align*}
a(u - u_h, p - p_h, v, q) &= \sum_{K ∈ \mesh} \int_K \left[(f + Δ\vu_h - ∇p_h)(v - v_h) + \dv \vu_h (q - q_h)\right] + \\
	&\quad + \frac{1}{2} \int_{∂K} [∇\vu_h · \vn] (\vv - \vv_h) + \underbracket{[p_h (\vv - \vv_h) · \vn]}_{ = 0} + \\
	&\quad+ αh_K^2(f + Δ\vu_h - ∇p_h)∇q_h
\end{align*}

Now we choose $\vv_h = R_h \vv$, the Clément interpolant. For $q_h$ we have to choose $q_h = 0$, and to justify this we would need to prove that the error is bounded above and below by the error estimator we are proving. The proof of that fact is lengthy and technical so we will not do it.

Once the functions are chosen, we use Cauchy-Schwartz so that \begin{align*}
a(u - u_h, p - p_h, v, q)
	&≤ C_3 \sum_{K ∈ \mesh} \left(h_K \norm{\vf + Δ\vu_h - ∇p_h}_{L^2(K)} \norm{∇\vv}_{L^2(ΔK)} + \right. \\
	&\quad+ \left. \frac{1}{2} h_K^{\sfrac{1}{2}} \norm{[∇\vu_h · \vn]}_{L^2(∂K)} + \norm{\dv \vu_h}_{L^2(K)}\norm{q}_{L^2(K)} \right) \\
	&≤ C_3 \left(\sum_{K ∈ \mesh} \left(h_K \norm{\vf + Δ\vu_h - ∇p_h} + \frac{1}{2}h_K^{\sfrac{1}{2}} \norm{[∇\vu_h · \vn]}\right)^2 \right)^{\sfrac{1}{2}} + \\
	&\quad
	+ \left(\sum_{K ∈ \mesh} \norm{∇\vv}_{L^2(ΔK)}^2\right)^{\sfrac{1}{2}}
	+ \left(\sum_{ K ∈ \mesh} \norm{\dv \vu_h}^2\right)^{\sfrac{1}{2}}
	· \left(\sum_{K ∈ \mesh} \norm{q}_{L^2(K)}^2\right)^{\sfrac{1}{2}}
\end{align*} where $ΔK$ is the neighbouring triangles of $K$ and $\dv \vu_h \to 0$ as $h \to 0$.

So it works by using Cauchy-Schwartz once again.
\end{proof}

\section{Optimal control}

\begin{wrapfigure}{R}{0.4\textwidth}
\centering
\inputtikz{Glacier}
\caption{Model of the movement of a glacier with three different boundaries.}
\label{fig:PDE:Glacier}
\end{wrapfigure}

One example of a problem on optimal control is the movement $u$ of a glacier, so that it follows the equation $-Δu = f$. We have two sections of the bedrock: $Γ_1$ where we have no slip (so $u = 0$) and a section $Γ_2$ with a certain sliding $∂_\vn u = q$, with $q$ unknown.

We also have the border $Γ_3$ of the ice with the air, with observations $u_0$ of the displacement and with no force of the air onto the ice (so $∂_\vn u = 0$ there).

We will want to minimize the integral \[ \int_{Γ_3} (u - u_0)^2 \] (that is, we want our model to approximate the observations we have) constrained to $u = 0$ on $Γ_1$ and being a solution of the weak problem \[ \int_Ω ∇u ∇v - fv = \int_Ω q v \qquad ∀v \st \restr{v}{Γ_1} = 0\]

This problem is, however, ill-posed. We will have to actually minimize \[ \frac{1}{2} \int_{Γ_3} (u - u_0)^2 + \frac{α}{2} \int_{Γ_2} q^2\] to have well-posedness.

We will study first a simpler model problem: we have a primal variable $u$ and a control variable $q$ such that $-Δu = f + q$ on $Ω ⊂ ℝ^n$, with observations of $u$ in $Ω_0 ⊂ Ω$ and with $u = 0$ on the boundary $∂Ω$.

The problem will be to find $(u,q) ∈ H_0^1(Ω) × L_2(Ω)$ such that the quantity \[ \frac{1}{2}\int_{Ω_0} (u - u_0)^2 + \frac{α}{2} \int_Ω q^2 \] is minimized under the constraint of the weak formulation of the problem, which is \[ \int_Ω ∇u ∇v = \int_Ω (f + q)v \qquad ∀v ∈ H_0^1(Ω)\]

A small parenthesis: if we were solving this in $ℝ^n$ instead of in a functional environment, we would have to solve \begin{align*} \mA \vu &= \vf \\ \mM_0 \vu &= \mM_0 \vu_0 \end{align*} with $\vu_0$ the vector of observations and $\mM_0$ a matrix selecting the coordinates on which the observations are defined.

In this case, we would define the functional $J(\vv) = \frac{α}{2} \norm{\mA \vv - \vf} + \frac{1}{2} \norm{\mM_0(\vv - \vu_0)}$ (that is, discrepancy with the model and with the observations) and then try to find a $\vv ∈ ℝ^n$ that minimizes it. The solution is given by $J'(\vu) = \vec{0}$, which computed would give \[ J'(\vu) = \trans{\mM_0} \mM_0 (\vu - \vu_0) + α \mA^T (\mA \vu - \vf) = \vec{0} \] which can be written as a system if we define $\vq = \mA \vu - \vf$.

Now we have to translate that to the PDE setting. We introduce the functional to minimize, the Lagrangian \( \linop (u, λ, q) = \frac{1}{2} \int_{Ω_0} (u - u_0)^2 + \frac{α}{2} \int_Ω q^2 + \int_Ω \left(∇u ∇λ - (f + q)λ\right) \label{eq:PDE:Lagrangian} \) with $λ ∈ H_0^1(Ω)$ the Lagrange multiplier corresponding to the constraint, $u ∈ H_0^1(Ω)$ and $q ∈ L^2(Ω)$.

The solution satisfies \[ \dpd{\linop(u,λ,q)}{u} = \dpd{\linop(u,λ,q)}{λ} = \dpd{\linop(u,λ,q)}{q} = 0\] with the derivative understood in the sense of the Gateaux derivative.

\begin{defn}[Gateaux derivative][Derivative!Gateaux] Given a function $\appl{F}{X}{Y}$ between two Banach spaces, we define the Gateaux derivative at $v ∈ X$ in the direction $u ∈ X$ as \( \dualp{\dif F(v), u} = \lim_{ε \to 0} \frac{F(v + εu) - F(v)}{ε} \) where we consider $\dif F(v)  ∈ X'$ as an operator in the dual space.
\end{defn}

Thus, in this case we can define the derivative of the Lagrangian as \[ \dualp{∂_u \linop(u, λ, q), v} = \lim_{ε \to 0} \frac{\linop(u + εv, λ, q) - \linop(u, λ,q)}{ε} \] and we claim that actually \[ \dualp{∂_u \linop(u, λ, q), v} = \int_{Ω_0} (u - u_0) v + \int_Ω ∇v ∇λ\quad ∀v ∈ H_0^1(Ω) \]

Indeed, computing we have that \begin{align*}
\linop(u + εv, λ, q) - \linop(u, λ, q)
	&= \frac{1}{2} \int_{Ω_0} \left((u + εv - u_0) ^ 2 - (u - u_0)^2\right) + \int_Ω ∇(u + εv - u) ∇λ = \\
	&= \int_{Ω_0} εv(u - u_0) + ε^2 v^2 + \int_Ω ε∇v∇λ
\end{align*} which when divided by $ε$ and with $ε \to 0$ works as we said. Similarly, we will have that \begin{align*}
\dualp{∂_λ \linop(u, λ, q), μ} &= \int_Ω ∇u ∇μ - (f+q)μ = 0  & ∀μ ∈ H_0^1(Ω) \\
\dualp{∂_q \linop(u, λ, q), r} &= α \int_Ω qr - \int_Ω rλ  = 0 & ∀r ∈ L^2(Ω)
\end{align*}

We can write strong forms of those equations, which gives in order
\begin{align*}
-Δλ + \ind_{Ω_0}(u - u_0) &= 0 &\text{(dual problem)}\\
-Δu = (f + q) &= 0 & \text{(primal problem)} \\
αq - λ &= 0
\end{align*}

Sometimes λ is called the dual variable.

With this system we can eliminate $q$ so we get our final problem to solve. We will need to find $(u, λ) ∈ H_0^1(Ω) × H_0^1(Ω)$ such that \( \begin{cases} \int_Ω ∇u ∇μ - \left(f + \frac{1}{α} λ \right)μ = 0 & ∀μ ∈ H_0^1(Ω) \\
\int_Ω ∇λ ∇v + \int_{Ω_0} (u - u_0) v = 0 & ∀v ∈ H_0^1(Ω) \end{cases} \)

We can formulate this in weak form by calling \begin{align*}
a(u,λ; μ, v) &= \int_Ω \left(∇u ∇μ - \frac{1}{α}λμ\right) + \int_Ω ∇λ ∇v + \int_{Ω_0} uv \\
F(μ, v) &=
\end{align*}

Well-posedness of this problem is a consequence of the inf-sup condition with the space $H ≝ H_0^1(Ω) × H_0^1(Ω)$ and $\norm{u, λ}_H^2 = \norm{∇u}_{L^2} + \norm{∇λ}_{L^2}$.

\begin{prop} \label{prop:PDE:OptimalControlInfsup} For the optimal control problem, for any $α >0$ there exists a constant $C > 0$ such that \[\norm{u, λ}_H ≤ C \sup_{\substack{(μ, v) ∈ H \\ (μ, v) ≠ 0}} \frac{a(u,λ;μ,v)}{\norm{μ, v}_H} \]
\end{prop}

\begin{proof} As we did for the Stokes problem it will suffice to prove that there exists two constants $C_1, C_2$ such that for any $(u, λ) ∈ H$ there will exist $(μ, v) ∈ H$ such that \begin{align*}
a(u,λ;μ,v) &≥C_1\norm{u,λ}^2 \\
\norm{μ, v} &≤ C_2\norm{u, λ}
\end{align*}

\proofpart{Lower bound}

We will start selecting $(μ, v) = (u, λ)$ to have some ``good terms'' with the squares of the gradients. Thus, we would have
\[
a(u,λ; u, λ) = \int_Ω \abs{∇u}^2 + \abs{∇λ}^2 - \frac{1}{α}λu + \ind_{Ω_0} u λ
\]

We still have some ``bad terms'' that we cannot bound, so we will compute \[ a(u,λ; -λ, u) = \int_Ω \frac{1}{α} λ^2 + \ind_{Ω_0} u^2 \]
and then try to substract the second equation weighted by some coefficent $β$ from the first one. Thus, we are choosing $(μ, v) = (u - βλ, λ + βu)$ and
\[
a(u,λ; u - βλ, λ + βu) = \int_Ω \abs{∇u}^2 + \abs{∇λ}^2 + \frac{β}{α} λ^2 + β \ind_{Ω_0} u^2- \frac{1}{α} λu + \ind_{Ω_0} uλ
\]

Now we can start bounding different terms. For $\frac{1}{α} λu$ we use Cauchy-Scharwz so that \[ \int_Ω \frac{1}{α} λu ≤ \frac{1}{α} \norm{λ}_2 \norm{u}_u ≤ \frac{β}{2α} \int_Ω λ^2 + \frac{1}{2αβ} \int_Ω u^2 \] and similarly \[ \int_{Ω_0} uλ ≤ \frac{β}{2} \int_{Ω_0} u^2 + \frac{1}{2β} \int_Ω λ^2 \]

Introducing those in the equation, we have
\[
a(u,λ; u - βλ, λ + βu) = \int_Ω \left(\abs{∇u}^2 + \abs{∇λ}^2\right) + \underbracket{\frac{β}{2α} \int_Ω λ^2 + \frac{β}{2} \int_{Ω_0} u^2}_{≥ 0} - \frac{1}{2β} \int_Ω λ^2 - \frac{1}{2αβ} \int_Ω u^2
\]

We will throw out the positive terms (we want to bound $a$ from below), and using \nref{thm:Fund:PoincareInequality}, we can bound the negative terms:
\begin{align*}
-\int_Ω λ^2 &≥ - C_p^2 \int_Ω \abs{∇λ}^2 \\
-\int_Ω u^2 &≥ - C_p^2 \int_Ω \abs{∇u}^2 \\
\end{align*} and so
\[ a(u,λ; u - βλ, λ + βu) ≥ \left(1 - \frac{C_p^2}{2αβ}\right) \int_Ω \abs{∇u}^2 + \left(1 - \frac{C_p^2}{2β}\right) \int_Ω \abs{∇λ}^2 \]

That quantity is greater than $\frac{1}{2} \norm{u,λ}^2$ if we have
\[ 1 - \frac{C_p^2}{2αβ} ≥ \frac{1}{2} \quad \text{and} \quad 1 - \frac{C_p^2}{2β} ≥ \frac{1}{2} \] so that our $β$ is defined as \[ β = C_p^2 \max \set{1, \frac{1}{α}} \]

\proofpart{Lower bound}

With the election of $(μ, v)$ from the previous part of the proof we have directly the lower bound $\norm{μ, v} ≤ C_2\norm{u, λ}$.
\end{proof}

This proposition gives us existence and unicity of the solution for the problem.

\subsection{Finite element approximation}

For a given mesh size $h > 0$, we will construct the mesh \mesh of triangles or tetrahedrons $K$ with $\mathop{diam}\, h_K ≤ h$, with $P_1, \dotsc, P_N$ the internal vertices and $φ_1, \dotsc, φ_N$ the shape functions associated to those vertices.

Thus, our space of finite elements is $V_h = \spn \set{φ_1, \dotsc, φ_n} ⊂ H_0^1(Ω)$ and we will search for $(u_h, λ_h) ∈ V_h^2$ such that \[ a(u_h, v_h; μ_h, v_h) = F(μ_h, v_h) \quad ∀ (μ_h, v_h) ∈ V_h^2\]

We could say that existence and uniqueness of the finite element approximation comes from the existence and uniqueness of the original problem, but we will see it using linear algebra and trying to invert the matrix of the system. Our system will be \[ \begin{pmatrix} \mM_0 & \mA \\ \mA & - \frac{1}{α} \mM \end{pmatrix} \begin{pmatrix} \vu \\ \vec{λ} \end{pmatrix} = \begin{pmatrix} \mM_0 \vu_0 \\ \vf \end{pmatrix} \]

We can try to eliminate $\vec{λ}$ from that equation, so that $\vec{λ = α \inv{\mM} (\vf - \mA \vu} $ and then \[ (\mM_0 + α \mA \inv{\mM} \mA) \vu = \]

That matrix is symmetric and positive definite and so invertible. Thus, we can start with the error estimates.

\subsection{A priori error estimates}

\begin{prop} There exists a constant $C > 0$ indepedent of $f, u_0, h$ but dependent on the mesh aspect ratio such that \[ \norm{u - u_h, λ - λ_h}^2 ≤ C \sum_{K ∈ \mesh} η_K^2 \] with
\begin{multline*}
η_K^2 =
	\left(
		h_K \norm{f + Δu_h + \frac{1}{α} λ_h}_{L^2(K)}
		+ \frac{1}{2} h_K^{\sfrac{1}{2}} \norm{[∇u_h · \vn]}_{L^2(∂K)}
	\right)^2 + \\
	+ \left(
		h_K \norm{\ind_{Ω_0} (u_0 - u_h) + Δλ_h}_{L^2(K)}
		+ \frac{1}{2} h_k^{\sfrac{1}{2}} \norm{[λ_h · \vn]}_{L^2(∂K)}
	\right)^2
\end{multline*}
\end{prop}

\begin{proof} By using the inf-sup condition from \fref{prop:PDE:OptimalControlInfsup}, we have a constant $C_1 > 0$ independent of $h$ such that \[ \norm{u - u_h, λ - λ_h} ≤ C_1 \sup _{(μ,v) ∈ H} \frac{a(u - u_h, λ - λ_h; μ,v)}{\norm{μ,v}}\]

In non-mixed problems our strategy to prove these bounds was to use coercivity and then relate it to the residual. Here we will to the same but with the inf-sup condition, using that
\[ \norm{u - u_h, λ - λ_h} ≤ C_1  \sup _{(μ,v) ∈ H} \frac{F(μ,v) - a(u_h, λ_h; μ,v)}{\norm{μ,v}} \]

We want to prove that for any $(μ,v) ∈ H$ there exists another constant $C_2 > 0$ independent of $f, u_0$ and $h$  such that \[ F(u,v) - a(u_h, λ_h; μ,v) ≤ C_2 \left(\sum_{K ∈ \mesh} η_K\right)^{\sfrac{1}{2}} \norm{μ,v} \]

Indeed,
\begin{align*}
& F(μ, v) - a(u_h, λ_h; μ,v)= \\
	&= \int_Ω \left(\left(f + \frac{1}{α}λ_h\right) μ - ∇u_h ∇μ\right)
	+ \int_Ω\left(\ind_{Ω_0}(u - u_0) v - ∇λ_h ∇v \right) = \\
	&=\int_Ω \left(f + \frac{1}{α}λ_h\right) (μ - μ_h) - ∇u_h ∇(μ - μ_h) + \\
	&\qquad + \int_Ω \ind_{Ω_0} (u_0 - u_h)(v - v_h) - ∇λ_h ∇(v - v_h) = \\
	&= \sum_{K ∈ \mesh} \int_K (f + \frac{1}{α}λ_h + Δu_h)(μ - μ_h) + \frac{1}{2} \int_{∂K} [∇u_h · \vn](μ - μh) + \\
	&\qquad + \int_K \ind_{Ω_0} (u_0 - u_h) + Δλ_h)(v - v_h) + \frac{1}{2} \int_{∂K} [∇λ_h · \vn](v - v_h)
\end{align*} and then we use Cauchy-Schwarz inequality to boud by the norms of everything.

We choose the Clément interpolant for the finite element approximations so that $μ_h = R_h μ$ and $v_h = R_h v$ and we have the bounds \begin{align*}
\norm{μ - R_h μ}_{L^2(K)} &≤ C_3 h_k \norm{∇v}_{L^2(ΔK)} \\
\norm{μ - R_h μ}_{L^2(∂K)} &≤ C_3 h_k^{\sfrac{1}{2}} \norm{∇v}_{L^2(ΔK)}
\end{align*} so that we can continue:
\begin{gather*}
F(μ, v) - a(u_h, λ_h; μ,v) ≤ \\
	≤ C_3 \sum_{K ∈ \mesh} \left(h_K \norm{f +  \frac{1}{α}λ_h + Δ_h}_{L^2(K)} + \frac{1}{2} h_K^{\sfrac{1}{2}} \norm{[∇u_h · \vn]}_{L^2(∂K)}\right) \norm{∇μ}_{L^2(ΔK)} + \\
	+\quad \sum_{K ∈ \mesh} \left(h_K \norm{\ind_{Ω_0} (u_0 - u_h) + Δλ_h}_{L^2(K)} + \frac{1}{2} h_K^{\sfrac{1}{2}} \norm{[∇λ_h · \vn]}_{L^2(∂K)}\right) \norm{∇λ}_{L^2(ΔK)}
\end{gather*}

We get the final bound by using the discrete Cauchy-Schwarz inequality\footnote{For any vector $\pesc{\va, \vb} ≤ \norm{\va} + \norm{\vb}$ so in particular $a_1 b_1 + a_2 b_2 ≤ \sqrt{a_1^2 +a_2^2} + \sqrt{b_1^2 + b_2^2}$}, that will give us a third constant that depends on the angle of the mesh elements and in some way ``measures'' how many times are we counting some terms over the same boundary.

\end{proof}

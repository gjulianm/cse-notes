% -*- root: ../NumericalApproximationofPDEs.tex -*-
\section{The Poisson problem / Laplace equation}

\begin{problem}
Let $Ω ⊂ ℝ^2$ be an open bounded set with $C^1$ boundary $∂Ω$ and $f ∈ L^2(Ω)$. Given $\vb ∈ ℝ^2$, we define $Γ_- = \set{x ∈ ∂Ω \tq \vb · \vn < 0}$ where $\vn$ is the outwards unit normal vector to $∂Ω$. We assume that $Γ_- ≠ ∅$ and we consider the problem of finding a function $\appl{u}{Ω}{ℝ}$ such that \[ \begin{cases}
- \dv (\mB ∇u) + \vb · ∇u + cu = f & \text{in }Ω \\
u = 0 & \text{on }Γ_- \\
\mB ∇u · \vn = 0 & \text{on } Γ_+ ≝ ∂Ω \setminus Γ_-
\end{cases}\] where $\mB ∈ ℝ^{2×2}$ is a symmetric positive definite matrix and $c ∈ ℝ^+$.

\ppart Write the weak form of the problem.
\ppart Show that the problem is well-posed.
\ppart Consider the finite element method presented in the course and the resulting linear system $\mA \vu = \vf$. Prove that the matrix $\mA$ is invertible.
\solution

\spart

As always, we multiply by a test function $v ∈ H_{Γ_-}^1 ≝ \set{v ∈ H^1(Ω) \tq \restr{v}{Γ_-} = 0}$  and integrate on the whole domain:
\[ -\int_Ω \dv (\mB ∇u)v + (\vb · ∇u) v + cuv = \int_Ω fv  \]

We have to integrate by parts in the first term to get rid of the second derivative:
\begin{align*}
-\int_Ω \dv (\mB∇u) v
	&= -\int_{∂Ω}(\mB∇u) v + \int_Ω  ∇v (\mB ∇u) = \\
	&= - \cancelwhy{\int_{Γ_+} (\mB∇u) v}{\restr{B∇u · \vn}{Γ_+} = 0} - \cancelwhy{\int_{Γ_-} (\mB∇u) v}{\restr{v}{Γ_-} = 0} + \int_{Ω} ∇v (\mB ∇u)  = \\
	&= \int_Ω \trans{∇v} {\mB} ∇u
\end{align*}

Thus, the weak form is given by the two following forms:
\begin{align*}
a(u,v) &= \int_{Ω} \trans{∇v} \mB ∇u + \int_Ω (\vb · ∇u)v + \int_Ω cuv \\
F(v) &= \int_Ω fv
\end{align*}

\spart

To show well-posedness we need to prove the hypothesis of the \nref{thm:Theory:LaxMilgram}. Continuity of $F$ is trivial as $f ∈ L^2$. For continuity of $a$, we bound each integrand separately. For the first, as $\mB$ is symmetric and positive definite, it has a finite number of eigenvalues and all positive, so we consider $λ_\text{max}$ its maximum one and then\footnote{Ignoring the absolute value because the integrand is always positive.} \[ \int_Ω \trans{∇v} \mB ∇u ≤ \int_Ω λ_\text{max} \abs{∇v} \abs{∇u} ≤ λ_\text{max} \abs{v}_{H^1} \abs{u}_{H^1} ≤ λ_\text{max} \norm{v}_{H^1} \norm{u}_{H^1} \]

For the second one, we have that \[ \int_Ω \abs{(\vb · ∇u) v } ≤ \int_Ω \abs{\vb} \abs{∇u} \abs{v} ≤ \abs{\vb} \norm{v}_{H^1} \norm{u}_{H^1} \]

The third one is trivial, and so we have continuity.

We study coercivity again by each integrand. For the first one, we consider $λ_\text{min} > 0$  the minimum eigenvalue of $\mB$ and so \[ \int_Ω \trans{∇v} \mB ∇v ≥ \int_Ω λ_\text{min} \abs{∇v}^2 = λ_\text{min} \abs{v}_{H^1}^2 ≥ \frac{λ_\text{min}}{1 + C_p} \norm{v}_{H^2}^2 \] using also \nref{thm:Fund:PoincareInequality}\footnote{More specifically, its \fref{crl:Fund:PoincareReverse}.}. For the second, we see that $∇(v^2) = 2 v ∇v$ so we can replace and use integration by parts \begin{align*}
\int_Ω (\vb ∇v)v
	&= \frac{1}{2} \int_Ω \vb · ∇(v^2)
	= - \frac{1}{2} \int_Ω v^2 \underbracket{\dv \vb}_{= 0} + \frac{1}{2} \int_{∂Ω} \vb v^2 = \\
	&= \frac{1}{2} \underbracket{\int_{Γ_+} \vb v^2}_{>0\; (\restr{\vb · \vn}{Γ_+} > 0)} + \frac{1}{2} \underbracket{\int_{Γ_-} \vb v^2}_{= 0\; (\restr{v}{ΦΓ_-} = 0)} ≥ 0
\end{align*}

The third one is again trivially greater than zero, so we have coercivity of $a$ and the form is well-posed.

\spart

The finite elements method implies searching in a finite space $V_h$, so that $a(u_h, v_h)$ can be expressed as $\trans{\vv} \mA \vv = \trans{\vv} \vf$. As $a$ is coercive, we have that for any non-null $\vu ∈ V_h$ we have $\mA \vu ≠ 0$ so that $\ker \mA = 0$ and $\mA$ is invertible.

\end{problem}


\begin{problem}

\solution

\end{problem}


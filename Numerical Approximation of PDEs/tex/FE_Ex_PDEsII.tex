% -*- root: ../NumericalApproximationofPDEs.tex -*-
\section{Session 1 - The Poisson problem / Laplace equation}

\begin{problem}
Let $Ω ⊂ ℝ^2$ be an open bounded set with $C^1$ boundary $∂Ω$ and $f ∈ L^2(Ω)$. Given $\vb ∈ ℝ^2$, we define $Γ_- = \set{x ∈ ∂Ω \tq \vb · \vn < 0}$ where $\vn$ is the outwards unit normal vector to $∂Ω$. We assume that $Γ_- ≠ ∅$ and we consider the problem of finding a function $\appl{u}{Ω}{ℝ}$ such that \[ \begin{cases}
- \dv (\mB ∇u) + \vb · ∇u + cu = f & \text{in }Ω \\
u = 0 & \text{on }Γ_- \\
\mB ∇u · \vn = 0 & \text{on } Γ_+ ≝ ∂Ω \setminus Γ_-
\end{cases}\] where $\mB ∈ ℝ^{2×2}$ is a symmetric positive definite matrix and $c ∈ ℝ^+$.

\ppart Write the weak form of the problem.
\ppart Show that the problem is well-posed.
\ppart Consider the finite element method presented in the course and the resulting linear system $\mA \vu = \vf$. Prove that the matrix $\mA$ is invertible.
\solution

\spart

As always, we multiply by a test function $v ∈ H_{Γ_-}^1 ≝ \set{v ∈ H^1(Ω) \tq \restr{v}{Γ_-} = 0}$  and integrate on the whole domain:
\[ -\int_Ω \dv (\mB ∇u)v + (\vb · ∇u) v + cuv = \int_Ω fv  \]

We have to integrate by parts in the first term to get rid of the second derivative:
\begin{align*}
-\int_Ω \dv (\mB∇u) v
	&= -\int_{∂Ω}(\mB∇u) v + \int_Ω  ∇v (\mB ∇u) = \\
	&= - \cancelwhy{\int_{Γ_+} (\mB∇u) v}{\restr{B∇u · \vn}{Γ_+} = 0} - \cancelwhy{\int_{Γ_-} (\mB∇u) v}{\restr{v}{Γ_-} = 0} + \int_{Ω} ∇v (\mB ∇u)  = \\
	&= \int_Ω \trans{∇v} {\mB} ∇u
\end{align*}

Thus, the weak form is given by the two following forms:
\begin{align*}
a(u,v) &= \int_{Ω} \trans{∇v} \mB ∇u + \int_Ω (\vb · ∇u)v + \int_Ω cuv \\
F(v) &= \int_Ω fv
\end{align*}

\spart

To show well-posedness we need to prove the hypothesis of the \nref{thm:Theory:LaxMilgram}. Continuity of $F$ is trivial as $f ∈ L^2$. For continuity of $a$, we bound each integrand separately. For the first, as $\mB$ is symmetric and positive definite, it has a finite number of eigenvalues and all positive, so we consider $λ_\text{max}$ its maximum one and then\footnote{Ignoring the absolute value because the integrand is always positive.} \[ \int_Ω \trans{∇v} \mB ∇u ≤ \int_Ω λ_\text{max} \abs{∇v} \abs{∇u} ≤ λ_\text{max} \abs{v}_{H^1} \abs{u}_{H^1} ≤ λ_\text{max} \norm{v}_{H^1} \norm{u}_{H^1} \]

For the second one, we have that \[ \int_Ω \abs{(\vb · ∇u) v } ≤ \int_Ω \abs{\vb} \abs{∇u} \abs{v} ≤ \abs{\vb} \norm{v}_{H^1} \norm{u}_{H^1} \]

The third one is trivial, and so we have continuity.

We study coercivity again by each integrand. For the first one, we consider $λ_\text{min} > 0$  the minimum eigenvalue of $\mB$ and so \[ \int_Ω \trans{∇v} \mB ∇v ≥ \int_Ω λ_\text{min} \abs{∇v}^2 = λ_\text{min} \abs{v}_{H^1}^2 ≥ \frac{λ_\text{min}}{1 + C_p} \norm{v}_{H^2}^2 \] using also \nref{thm:Fund:PoincareInequality}\footnote{More specifically, its \fref{crl:Fund:PoincareReverse}.}. For the second, we see that $∇(v^2) = 2 v ∇v$ so we can replace and use integration by parts \begin{align*}
\int_Ω (\vb ∇v)v
	&= \frac{1}{2} \int_Ω \vb · ∇(v^2)
	= - \frac{1}{2} \int_Ω v^2 \underbracket{\dv \vb}_{= 0} + \frac{1}{2} \int_{∂Ω} \vb v^2 = \\
	&= \frac{1}{2} \underbracket{\int_{Γ_+} \vb v^2}_{>0\; (\restr{\vb · \vn}{Γ_+} > 0)} + \frac{1}{2} \underbracket{\int_{Γ_-} \vb v^2}_{= 0\; (\restr{v}{ΦΓ_-} = 0)} ≥ 0
\end{align*}

The third one is again trivially greater than zero, so we have coercivity of $a$ and the form is well-posed.

\spart

The finite elements method implies searching in a finite space $V_h$, so that $a(u_h, v_h)$ can be expressed as $\trans{\vv} \mA \vv = \trans{\vv} \vf$. As $a$ is coercive, we have that for any non-null $\vu ∈ V_h$ we have $\mA \vu ≠ 0$ so that $\ker \mA = 0$ and $\mA$ is invertible.

\end{problem}


\begin{problem} Let $Ω = (0,1)$ and consider the nodes $x_i = ih$ for $i = 0, \dotsc, N + 1$ with $h = \frac{1}{N+1}$. Let $V_h$ be the finite dimensional space $V_h = \spn\set{φ_i}_{i=1}^N$ spanned by the $N$ affine shape functions defined by \( φ_i(x) ≝ \begin{cases}
\frac{x - x_{i-1}}{x_i - x_{i-1}} & x ∈ [x_{i-1}, x_i] \\
\frac{x_{i+1} - x}{x_{i+1} - x_{i}} & x ∈ [x_{i}, x_{i + 1}] \\
0 & \text{otherwise}
\end{cases} \label{eq:ExPDEII:ShapeFuncs}\)

\ppart Show that there exists a constant $C > 0$ such that for all $h > 0$, for all $v_h ∈ V_h$ and for all $i = 0, \dotsc, N$, we have \( \int_{x_i}^{x_{i+1}} (v_h'(x))^2 \dif x ≤ Ch^{-2} \int_{x_i}^{x_{i+1}} v_h^2(x) \dif x \)

\ppart Show that there exists two constants $C_1, C_2 > 0$ such that for all $h > 0$ and for all $v ∈ H^2(0,1)$, $i = 0, \dotsc, N$ we have \begin{align*}
\int_{x_i}^{x_{i+1}} (v'(x) - (r_hv)' (x))^2 \dif x &≤ C_1 h^2 \int_{x_i}^{x_{i+1}} (v''(x))^2 \dif x \\
\int_{x_i}^{x_{i+1}} (v(x) - r_hv (x))^2 \dif x &≤ C_2 h^2 \int_{x_i}^{x_{i+1}} (v''(x))^2 \dif x
\end{align*} where $r_h v ≝ \sum_{i = 1}^N v(x_i) φ_i$ is the Lagrange interpolant of $v$.

\solution

\spart

First, we see that the derivative of $φ_i$ is defined by \[ φ_i'(x) =\begin{cases}
\frac{1}{x_i - x_{i-1}} & x ∈ [x_{i-1}, x_i] \\
\frac{-1}{x_{i+1} - x_{i}} & x ∈ [x_{i}, x_{i + 1}] \\
0 & \text{otherwise} \end{cases} \] and then we can compute both integrals:
\begin{align*}
\int_{x_i}^{x_{i+1}} (v_h'(x))^2 \dif x &= \int_{x_i}^{x_{i+1}} (v_i φ_i' + v_{i+1}φ_{i+1}')^2 \dif x = \\
&= \frac{(v_{i+1} - v_i)^2}{h} \\
\int_{x_i}^{x_{i+1}} v_h^2(x) \dif x &= \int_{x_i}^{x_{i+1}} (v_i φ_i + v_{i+1}φ_{i+1})^2 \dif x = \\
&=\frac{h}{6}(v_i^2 + (v_{i} + v_{i+1})^2 + v_{i+1}^2)
\end{align*}

Then we apply Cauchy-Schwarz and we have it.

\spart

It suffices to prove the results for $v ∈ C^2([0,1])$. Let $w = v - r_h v$. As for any vertex $x_i$ we have $w(x_i) = 0$, then by Rolle's theorem there exists some $ξ_i ∈ (x_i, x_{i+1})$ such that $w'(ξ_i) = 0$ so that we can write $w'(x) = \int_{ξ_i}^x w''(s) \dif s$. Additionally, as $r_hv$ is a polynomial of degree one, we have that \[ w'(x) = \int_{ξ_i}^{x} w''(s) \dif s = \int_{ξ_i}^x v''(s) \dif s \]

Taking absolute values and applying the Cauchy-Schwarz inequality\footnote{Dirty trick: $\int f = \pesc{1, f} ≤ \norm{1} \norm{f}$.} we have \[ \abs{w'(x)} ≤ \int_{x_i}^{x_{i+1}} \abs{v''(s)} \dif s ≤ h^{\sfrac{1}{2}} \left(\int_{x_i}^{x_{i+1}} \abs{v''(s)}^2 \dif s \right)^{\sfrac{1}{2}} \]

Integrating over the interval $[x_i, x_{i+1}]$ yields the first inequality. For the other one, simply say that $\abs{w} ≤ \int_{x_i}^{x_{i+1}} \abs{w'}$ and integrate.

\end{problem}

\section{Session 2 - Interpolation error}

\begin{problem} Let $Ω = (0,1)$, let $N ∈ ℕ^+$ and consider the nodes $x_i = ih$ for $i = , \dotsc, N + 1$. Let $V_h$ be the finite dimensional space generated by the usual basis functions $\set{φ_i}_{i=1}^N$ (see \eqref{eq:ExPDEII:ShapeFuncs}). Show that there exists a constant $C > 0$ such that for all $h >0$ and for all $v ∈ H^1(0,1)$ and for all $i = 0, \dotsc, N$ we have \[ \norm{v - r_hv}_{L^2(x_i, x_{i+1}} ≤ Ch \norm{v'}_{L^2(x_i, x_{i+1})} \] where $r_h v$ is the Lagrange interpolant of $v$ defined by $r_h v ≝ \sum_{i = 1}^N v(x_i) φ_i$.

\solution

By density, it suffices to prove this for $v ∈ C^1([0,1])$. Let us define $w = v - r_h v$, it is clear that \[ w(x) = \int_{x_i}^x w'(s) \dif s = \int_{x_i}^x v'(s) - (r_hv)'(s) \dif s\]

Now, as the Lagrange interpolant is linear we have that $(r_hv)'(s) = \frac{v(x_{i+1}) - v(x_i)}{h}$ when restricted to $[x_i, x_{i+1}]$. Using that $v(x_{i+1}) = v(x_i) + \int_{x_i}^{x_{i+1}} v' $ we have \begin{multline*} w(x) = \int_{x_i}^x v'(s) - \frac{v(x_{i+1}) - v(x_i)}{h} \dif s = \int_{x_i}^x \left(v'(s) - \frac{1}{h} \int_{x_i}^{x_{i+1}} v'(t) \dif t \right) = \\ = \int_{x_i}^x v'(s) \dif s - \frac{x - x_i}{h} \int_{x_i}^{x_{i+1}} v'(t) \dif t \end{multline*}

Now we can take the absolute value of $w$ and by using Cauchy-Schwarz we get \[ \abs{w(x)} ≤ \underbracket{\int_{x_i}^x \abs{v'(s)} \dif s}_{≤ \int_{x_i}^{x_{i+1}} \dotsb} + \underbracket{\frac{x - x_i}{h}}_{≤ 1} \int_{x_i}{x_{i+1}} \abs{v'(s)} \dif s ≤ 2 \int_{x_i}{x_{i+1}} \abs{v'(s)} \dif s ≤ \]

\end{problem}

\begin{problem} Let $f ∈ L^2(0,1)$ and let $ε > 0$. We consider the problem of finding $\appl{u}{(0,1)}{ℝ}$ such that \[ \begin{cases} - εu'' + u' + u = f & 0 < x < 1 \\ u(0) = u(1) = 0 \end{cases}\]

\ppart Write the weak form of the problem.
\ppart (A priori error estimates) Let $0 = x_0 < x_1 < \dotsb < x_N < x_{N+1} = 1$ be a discretization of $[0,1]$ and let $h_i = x_{i+1} - x_i$, $h ≝ \max_i h_i$. Show that if $u ∈ H^2(0,1)$ then there exists a constant $C > 0$ independent of $h$ such that \[ \norm{u - u_h}_{L^2(0,1)} ≤ Ch \]

\ppart (A posteriori error estimates) Show that \[ \norm{u' - u_h'}_{L^2(0,1)} ≤ C\left(\sum_{i=0}^N η_i^2\right)^{\sfrac{1}{2}}\] with \[ η_i^2 = h_i^2 \norm{\frac{1}{ε} f + u_h'' - \frac{1}{ε}u_h' - \frac{1}{ε}u_h}^2_{L^2(x_i, x_{i+1})}\] where $C$ is the constant previously derived. Use the Lagrange interpolant and the result of the previous exercise.

\solution

\spart

As usual, multiply by test function, integrate and apply divergence theorems
\begin{align*}
-ε \int_Ω u''v + \int_Ω u'v + \int_Ω uv &= \int_Ω fv \\
-ε \left(\eval[1]{u'v}_{∂Ω} - \int_Ω u'v'\right) + \int_Ω u'v + \int_Ω uv &= \int_Ω fv \\
\underbracket{\int_Ω εu'v' + \int_Ω u'v + \int_Ω uv}_{a(u,v)} &= \underbracket{\int_Ω fv}_{F(v)}
\end{align*}


\spart

We will, in order apply Galerkin orthogonality for some $v_h ∈ V_h$, coercivity of $a$ and continuity
\begin{align*}
a(u - u_h, u - u_h) &= a(u - u_h, u - v_h + \underbracket{v_h - u_h}_{∈ V_h}) \\
ε \norm{u - u_h}^2_V &≤ M \norm{u - u_h}_V \norm{u - v_h}_V
\end{align*}

Now choose $v_h = r_hv$ so we have the estimation of the previous exercise $\norm{v - r_hv}_V ≤ Ch \norm{v'}_V$ and we have it.

\spart

For any $u_h ∈ V_h$ and $v ∈ V$, we define the residual as \[ R_{u_h} (v) = F(v) - a(u_h, v)\] (we might omit the $u_h$ subscript) which corresponds to the error made when we insert $u_h$ instead of $u$. Note that $R(v_h) = 0$ for any $v_h ∈ V_h$. Thanks to coercivity of $a$ we have that \begin{align*}
ε\norm{u' - u_h'}^2_{L^2(0,1)} & ≤ a(u - u_h, u - u_h) \\
	&= a(u, u - u_h) - a(u_h, u - u_h) \\
	&= R(u - u_h)
\end{align*}

Then we proceed to estimate the residual. For any $v ∈V$ we have that
\begin{multline*}
R(v) = R(v - v_h) = F(v - v_h) - a(u_h, v - v_h)
\end{multline*}

\end{problem}

\section{Session 3 - Error bounds and numerical tests}

\begin{problem} Let $V$ be a Hilbert space with norm $\norm{·}$, $\appl{a}{V×V}{ℝ}$ a bilinear form and $\appl{F}{V}{ℝ}$ a bilinear form. Assume that $a$ is coercive with constant $α$ and continuous with constant $β$.

Let $u ∈ V$ be such that $a(u,v) = F(v)$ for any $v ∈ V$. Let $V_h ⊂ V$ be a finite dimensional space and $u_h ∈ V_h$ such that $a(u_h, v_h) = F(v_h)$ for any $v_h ∈ V_h$. Prove the following estimates:

\ppart A priori: $α \norm{u - u_h} ≤ β \norm{u - v_h}$ for any $v_h ∈ V_h$.
\ppart A posteriori: $α \norm{u - u_h} ≤ \sup_{v ∈ V\minuszero} \frac{F(v) - a(u_h, v)}{\norm{v}}$.

\solution

\spart

By coercivity, $α\norm{u - u_h}^2 ≤ a(u - u_h, u - u_h)$. Now we add and substract $v_h ∈ V_h$ in the second argument and using bilinearity we have \begin{align*}
α\norm{u - u_h}^2 &≤ a(u - u_h, u - u_h) = a(u - u_h, u - v_h + v_h - u_h) \\
&= a(u - u_h, u - v_h) + a(u - u_h, v_h - u_h) = \\
&= a(u - u_h, u - v_h) + \underbracket{a(u, v_h - u_h)}_{= 0} - \underbracket{a(u_h, v_h - u_h)}_{= 0}  \\
&= a(u - u_h, u - v_h) \\
α\norm{u - u_h}^2 &= \abs{a(u - u_h, u - v_h)} ≤ β \norm{u - u_h} \norm{ u - v_h} \\
α\norm{u - u_h} &≤ β \norm{u - v_h}
\end{align*}

\spart

Again applying coercivity
\begin{align*}
α \norm{u - u_h}^2 &≤ a(u - u_h, u - u_h) \\
	&= a(u - u_h, u - v + v - u_h) = a(u - u_h, v) + a(u - u_h, u - v - u_h) \\
	&= F(u - u_h) - a(u_h, u - u_h) = \resid[u_h](u - u_h)
\end{align*}

Maybe. IDK.

\end{problem}

\section{Session 4 - Goal oriented errors, superconvergence and adaptive mesh algorithm}

\begin{problem} Let $ε > 0$. Consider the problem of finding $\appl{u}{(0,1)}{ℝ}$ such that \[ \begin{cases}
-ε u'' + u' = 0 & 0 < x < 1 \\
u(0) = u(1) = 1
\end{cases}\]

Let $ρ ∈ L^2(0,1)$ be given. Consider a discretization $0 = x_0 < x_1 < \dotsb < x_N < x_{N+1} = 1$ of $[0,1]$ with $h_i = x_{i+1} - x_i$ and $I_i = [x_{i}, x_{i+1}]$. Show that there exists a constant $C$ independent of $h$ and $u$ such that \[ \int_0^1 ρ · (u - u_h) \dif x ≤ C\sum_{i = 1}^N h_i \norm{1 + εu_h'' - u_h'}_{L^2(I_i)} \norm{φ' - φ_h'}_{L^2(I_i)}\] where $u_h$ is the finite element approximation of $u$ and φ and $φ_h$ are the solution and its approximation of the dual problem \[
\begin{cases}
- ε φ'' - φ' = ρ & 0 < x < 1\\
φ(0) = φ(1)  = 0
\end{cases}\]

\solution

\end{problem}

\section{Session 5 - Stokes problem}

\begin{problem} \label{ex:PDE2:Stokes} Let $Ω ⊂ ℝ^2$ be aconvex polygonal domain. We consider the Stokes problems of finding $\appl{\vu}{Ω}{ℝ^2}$ and $\appl{p}{Ω}{ℝ}$ such that
\[ \begin{cases}
- Δ\vu + ∇p = \vf & \text{in } Ω \\
\dv \vu = 0 & \text{in } Ω \\
\vu = 0 & \text{on } Ω
\end{cases}\]

\ppart Show that the weak form of the problem is finding $(\vu, p) ∈ V× Q$ such that \[ a(\vu, p; \vv, q) = F(\vv, q)\qquad ∀(\vv, q) ∈ V×Q\] with \begin{align*}
a(\vu, p; \vv, q) &= \int_Ω ∇\vu : ∇\vv - \int_Ω p \dv \vv - \int_Ω q \dv \vu \\
F(\vv, q) &= \int_Ω \vf · \vv
\end{align*} where $V, Q$ are spaces to be specified.

\ppart For any $h > 0$ let \mesh be a conformal mesh of triangles $K$ with diameter $h_K ≤ h$ and consider $V_h ⊂ V$, $Q_h ⊂ Q$ two finite dimensional spaces consisting of affine elements $\pspace[1]$. The solution of the weak form is approximated by solving the associated Galerkin problem with stabilization term:
\begin{align*}
a_h(\vu_h, p_h;\vv_h, q_h) &≝ a(\vu_h, p_h; \vv_h, q_h) - α \sum_{K ∈ \mesh} h_K^2 \int_K ( - Δ\vu_h + ∇p_h) · (- Δ\vv_h + ∇q_h) \\
F_h(\vv_h, q_h) &≝ F(\vv_h, q_h) - α \sum_{K ∈ \mesh} h_K^2 \int_K \vf · (- Δ\vv_h + ∇q_h) \\
\end{align*} with $α ≥ 0$ a stability parameter. Show that there exists a constant $C > 0$ independent of $h, \vu$ and $p$ such that \[ \norm{\vu - \vu_h, p - p_h}^2 ≤ Ch^2\left(\abs{\vu}_{H^2(Ω)}^2 + \abs{p}_{H^1(Ω)}^2\right)\] where \[ \norm{\vv, q}^2 ≝ \norm{∇\vv}^2_{L^2(Ω)} + α \sum_{K ∈ \mesh} h_K^2 \norm{∇q}^2_{L^2(K)}\]

\solution

\spart

As usual, multiply by test function $\vv$ and integrate:
\begin{align*}
- Δ\vu + ∇p &= \vf \\
- \int_Ω Δ\vu · \vv + \int_Ω ∇p · \vv &= \int_Ω \vf \vv
\end{align*}

Now we operate with details on the vector laplacian, and using integration by parts:
\begin{align*}
\int_Ω Δ\vu · \vv &= \int_Ω \sum_{i = 1}^N Δu_i v_i = \int_Ω \sum_{i,j = 1}^N \dpd[2]{u_i}{x_j} v_i \\
&= \sum_{i,j=1}^N \left(\int_{∂Ω} v_i \dpd{u_i}{x_j} - \int_Ω \dpd{u_i}{x_j} \dpd{v_i}{x_j} \right) ≝ - \int_Ω ∇\vu : ∇\vv
\end{align*}

For $∇p · \vv$, recall the divergence of the product: $\dv (p \vv) = p \dv \vv + ∇p · \vv$. By Gauss theorem, $\int_Ω \dv \vv = \int_{∂Ω} \vv$, so forcing $\vv$ to be zero on the boundary we have \[ \int_Ω ∇p · \vv = - \int_Ω p \dv \vv\]

Now we only have to multiply by $q$ in the continuity equation:
\[
\dv \vu = 0 \implies
\int_Ω q \dv \vu = 0
\]
and as it is zero, we can sum it in the previous one to have
\[ \underbracket{\int_Ω ∇\vu:∇\vv - \int_Ω p \dv \vv - \int_Ω q \dv \vv}_{a(u, p; v, q)} = \underbracket{\int_Ω \vf \vv}_{F(v)} \]

Regarding the spaces, we need $Q = L^2(Ω)$ and $V = [H_0^1(Ω)]^N$.

\spart

For this proof we are going to do the interpolation: we first go from $\vu$ to the interpolant (which we know how to bound) and then from there to the finite element approximation. By the triangular inequality, we have that
\[ \norm{\vu - \vu_h, p - p_h}^2 ≤ \norm{\vu - r_h \vu, p - Π_h p}^2 + \norm{r_h \vu - \vu_h, Π_h p - p_h} \] where $r_h$ is the Clément interpolant of $p$ and $Π_h$ the Lagrange interpolant of $\vu$, so that we directly have the following estimations
\begin{align*}
\norm{\vu - r_h \vu} &≤ Ch^2\abs{\vu}_{H^2(Ω)} \\
\norm{p - Π_hp} &≤ Ch \abs{p}_{H^1(Ω)}
\end{align*}


\end{problem}

\section{Session 6 - Convection-diffusion problem with stabilization: Well-posedness and a posteriori errors}

\begin{problem} Let $Ω ⊂ ℝ^2$ be a polygonal domain. We consider the convection-diffusion problem of finding a $\appl{u}{Ω}{ℝ}$ such that \[ \begin{cases}
-εΔu + \vb ∇u = f & \text{in } Ω \\
u = 0 & \text{on } ∂Ω
\end{cases}\] where $ε > 0$ is a constant and $\vb ∈ ℝ^2$. For any $h > 0$, let \mesh be a conformal mesh of Ω into triangles $K$ with diameter $h_K ≤ h$ and consider the finite dimensional space
\[ V_h = \set{v_h ∈ C^0(\adh{Ω}) \st \restr{v_h}{K} ∈ \pspace[1] \; ∀K ∈ \mesh} ∩ H_0^1(Ω) \]

We seek an $u_h ∈ V_h$ that verifies \[ \int_Ω \left(ε ∇u_h ∇v_h + (\vb ∇u_h)v_h - fv_h\right) + \sum_{K ∈ \mesh} α_K \int_K (-εΔu_h + \vb ∇\vu_h - f)(-εΔv_h + \vb∇v_h) = 0\] for all $v_h ∈ V_h$, where $α_K = \frac{h_K}{\norm{\vb}_{L^∞(K)}}$ if $\frac{h_K\norm{\vb}_{L^∞(K)}}{ε} ≥ 1$ and $α_K = \frac{h_K^2}{ε}$ otherwise.

\ppart Prove tha

\solution

\end{problem}

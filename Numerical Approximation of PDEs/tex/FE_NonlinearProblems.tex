% -*- root: ../NumericalApproximationofPDEs.tex -*-
\section{Non linear problems}

Our model non-linear problem will the the following: find $\appl{u}{Ω ⊂ ℝ^2}{ℝ}$ such that \(
\begin{cases}
- Δu + u^3 = f & \text{in } Ω \\
u = 0 & \text{on } ∂Ω
\end{cases} \label{eq:PDE:NonlinearProblem}
\)

The cube does not have any specific physical significance, but it will be interesting for the mathematics.

The weak formulation of this problem is finding $u ∈ H_0^1(Ω)$ such that \( \int_Ω ∇u ∇v + u^3 v = \int_Ω fv \qquad ∀ v ∈ H_0^1(Ω) \label{eq:PDE:WeakNonlinearProblem} \)

We could have some problems with that term $u^3 v$: we need to ensure that it is bounded. Indeed, it is. We have that $\int_Ω u^3 v ≤ \norm[0]{u^3}_2 \norm{v}_2$. We know that $\norm{v} < ∞$ as $v ∈ H_0^1(Ω)$. For the other norm, we can see that $\norm[0]{u^3}_2 = \norm[0]{u}_6^3$ and by the \nref{thm:SobolevEmbedding}, as we are working in $Ω ⊂ ℝ^2$ and $u ∈ H^1(Ω)$, there is an embedding $H^1(Ω) ⊂ L^q(Ω)$ for $2 ≤ q < ∞$ and in particular $\norm{u}_6 < ∞$.

Thus we can talk about uniqueness and existence.

\begin{prop}[Existence and uniqueness\IS for a non-linear elliptic PDE] \label{prop:NonLinearProblemExistenceUniqueness} For the non-linear problem, if we assume $f ∈ L^2(Ω)$ and $∂Ω ∈ C^2$ (or $Ω$ is a convex polygon), then there exists a unique $u ∈ H^1_0(Ω)$ solution of \eqref{eq:PDE:WeakNonlinearProblem} and $C > 0 $ depending on the domain such that \( \norm{∇u}_{L^2(Ω)} ≤ C\norm{f}_{L^2(Ω)} \label{eq:PDE:NonLinearBound} \)
\end{prop}

\begin{proof} Uniqueness and the bound are not difficult. For the existence problem we will need an extra problem.

\proofpart{Bound \eqref{eq:PDE:NonLinearBound}}

Assume that there exists a solution $u$ of the problem. Then \[ \int_Ω \abs{∇u}^2 ≤ \int_Ω \abs{∇u}^2 + \underbracket{\int_Ω u^4}_{ > 0} = \int_Ω fu ≤ C_p \norm{f}_{L^2(Ω)} \norm{∇u}_{L^2(Ω)} \]

\proofpart{Uniqueness}

Let $u_1, u_2 ∈ H_0^1$ be solutions of the problem. Then,
\[ \int_Ω ∇(u_1 - u_2) ∇v + \int_Ω(u_1^3 - u_2^3) v = 0\]

By using the mean value theorem, we know that $φ(a) - φ(b) = φ'(θa + (1 - θ)b) (a - b)$ for some $θ ∈ (0,1)$. Choosing $φ(u) = u^3$, we can get rid of that substraction $u_1^3 - u_2^3$ and then \[
\int_Ω ∇(u_1 - u_2) ∇v + \int_Ω 3(θu_1 + (1 - θ)u_2)^2 (u_1 - u_2) v = 0
\]

Now we take $v = u_1 - u_2$ and we have
\[ \int_Ω \abs{∇(u_1 - u_2)}^2 + \underbracket{\int_Ω 3(θu_1 + (1 - θ) u_2)^2 (u_1 - u_2)^2}_{ ≥ 0} = 0\] so by some reason I'm not really sure of turns out $u_1 - u_2 = 0$.

\proofpart{Existence}

For this part of the proof we will use the \nref{thm:Fund:SchoeferFixedPoint}, with our operator \begin{align*}
\appl{G}{H_0^1(Ω)&}{H_0^1(Ω)} \\
u &\longmapsto w \st \int_{Ω} ∇w ∇v = \int_Ω (f - u^3)v \quad ∀v ∈ H_0^1(Ω)
\end{align*}

That is, $G$ maps $u$ to the weak solution of the problem $- Δw = (f - u^3)$, which is a well posed problem. In particular, if $u$ is a solution we have $G(u) = u$, so the fixed point is a solution of the problem.

We have to prove that $G$ is continuous and compact to apply the theorem. For continuity, let $w_1 = G(u_1)$ and $w_2 = G(u_2)$. We need to prove that $\norm{w_1 - w_2} \to 0$ when $\norm{u_1 - u_2} \to 0$. We can see that \[
\int_Ω ∇(w_1 - w_2) ∇v = \int_Ω - (u_1^3 -u_2^3)v \\
	= - \int_Ω (u_1 - u_2)(u_1^2 + u_1 u_2 + u_2^2)v
\]

Now we can bound the product of three elements by the product of the norms in $L^2$ for one factor and $L^4$ for the others\footnote{If $u,v,w ∈ H_0^1(Ω)$, then $\int uvw ≤ \norm{u}_2 \norm{vw}_2 ≤ \norm{u}_2 \norm{v}_4 \norm{w}_4$}. We apply that to the previous equation and assuming $v = w_1 - w_2$ we haves
\begin{align*}
\norm{∇(w_1 - w_2)}^2 &≤ \norm{u_1^2 + u_1u_2 + u_2^2}_{L^2} \norm{u_1 - u_2}_{L^4} \norm{w_1 - w_2}_{L^4} \\
\norm{∇(w_1 - w_2)} &≤ C_1 \norm{u_1^2 + u_1u_2 + u_2^2}_{L^2} \norm{∇(u_1 - u_2)}_{L^2} \\
	&≤ C_2 \left(\norm{∇u_1}^2_{L^2} + \norm{∇u_2}^2_{L^2}\right) \norm{∇(u_1 - u_2)}_{L^2}
\end{align*} so all the factors on the right-hand side are bounded and $\norm{∇(u_1 - u_2)} \to 0$ and $G$ is continuous.

Now we need to prove that $G$ is a \nref{def:Fund:CompactAppSubseq}. We take then $\set{u_n}_{n = 1}^∞  ⊂ H_0^1(Ω)$ a bounded sequence, and we want to prove that for a certain $u_{n_j}$ subsequence then $\set{w_{n_j}}_{j = 1}^∞$ converges, with $w_{n_j} = G(u_{n_j})$.

Since $∂Ω ∈ C^2$, then $w ∈ H^2(Ω) ∩ H_0^1(Ω)$ and indeed $f - u^3 ∈ L^2(Ω)$. We need to use the fact that the immersion from $H^2(Ω)$ to $H^1(Ω)$ is compact (this is also true for $H^1$ to $L^2$). So, if we take a bounded sequence $\set{w_n}$ in $H^2(Ω)$, then there exists a subsequence $\set[0]{w_{n_j}}$ convergent in $H^1(Ω)$. As $G$ is continuous, if $\set{u_n}$ is bounded then $\set{w_n}$ will be bounded too.

Finally, we need the ``a priori estimate'' of the fixed point theorem. But that part is easy: let $u ∈ H_0^1(Ω)$ be such that $u = λG(u) = λw$ with $0 ≤ λ ≤ 1$. In that case $u$, is solution of the problem $- Δu = λ(f - u^3)$ which is a well posed problem and $u$ is bounded.

Therefore, we are in the conditions of the \nref{thm:Fund:SchoeferFixedPoint}, there is a fixed point $u = G(u)$ and our problem has a solution.

\end{proof}

\subsection{Finite element approximation}

As usual, for $h > 0$ we will have our mesh \mesh of $Ω$, which will be a convex polygon domain. Then, we consider our finite element space $V_h = \spn \set{φ_1, \dotsc, φ_N}$ with $φ_i$ the shape functions corresponding to the internal vertices (we want the function to be zero in the boundary).

The Galerkin approximation given by finding $u_h ∈ V_h$ such that \[ \int_Ω (∇u_h ∇v_h + u_h^3 v_h) = \int_Ω f v_h \qquad ∀v_h ∈ V_h\]

Existence and unicity comes directly from the \nref{prop:NonLinearProblemExistenceUniqueness}, so we will not worry about that.

Now the question is how to compute $u_h$. The answer will be to use the Newton method. We want to find the solution $u_h ∈V_h$ such that \[ \dualp{F(u_h), v_h} ≝ \int_Ω ∇u_h ∇v_h + u_h^3 v_h - fv_h = 0\] so we can try to use the Newton method for solving nonlinear equations. Thus, for each step we will get the next iteration of the solution by solving \[ ∇F(u_h^n) (u_h^n - u_h^{n+1}) F(u_h^n)\] for $u_h^{n+1}$ which seems easy enough except for the fact that we do not know exactly what is the gradient of a functional.

For that, we will introduce the Fréchet derivative.

\begin{defn}[Fréchet derivative][Derivative\IS Fréchet] Let $\appl{F}{X}{X^*}$ be an application from a Banach space to its dual. It will be Fréchet differentiable at $u ∈ X$ if there exists a bounded linear operator $\appl{l_u}{X}{X^*}$ such that \[ \lim_{w \to 0} \frac{\norm{F(u + w) - F(u) - l_u(w)}_{X^*}}{\norm{w}_X} = 0\]

We then write $\Dif F(u) (w) = l_u (w)$ as the Fréchet derivative..
\end{defn}

We can now compute the Fréchet differential of $F$. Let $u,v,w ∈ H_0^1(Ω)$, so we can compute the duality product to see what's linear on $w$: \begin{align*}
\dualp{F(u+w) - F(u), v}
	&≝ \int_Ω ∇(u+w) ∇v + (u+w)^3 v - ∇u ∇v - u^3 v \\
	&= \underbracket{\int_Ω ∇w∇v + 3u^2wv}_{\dualp{l_u(w), v}} + \int_Ω 3uw^2v + w^3v
\end{align*} and we have a candidate for our differential. We have to check that the limit works.

First, recall that the norm of the dual is defined as $\norm{T}_{X^*} = \sup_{v ∈ X \setminus\set{0}} \frac{\dualp{X,v}}{\norm{v}} $ so that \begin{align*}
\norm{F(u + w) - F(u) - l_u(w)}_{(H_0^1(Ω))^*}
	&= \sup_{\substack{v ∈ H_0^1(Ω) \\ v ≠ 0}} \frac{\dualp{F(u+w) - F(u) - l_u(w), w}}{\norm{∇v}_{L^2(Ω)}} = \\
	&= \sup_{\substack{v ∈ H_0^1(Ω) \\ v ≠ 0}} \frac{\int_Ω 3uw^2v + w^3 v}{\norm{∇v}_{L^2(Ω)}}  \\
	&≤ \sup_{\substack{v ∈ H_0^1(Ω) \\ v ≠ 0}} \frac{C(3\norm{∇u} \norm{∇w}^2 + \norm{∇w}^3) \norm{∇v}}{\norm{∇v}} \\
	&≤ C(3\norm{∇u} \norm{∇w}^2 + \norm{∇w}^3) \convs[][w][0] 0
\end{align*} and therefore our Fréchet derivative is given by $\dualp{l_u(w), v} = \dualp{\Dif F(u)w, v} = \int_Ω ∇w ∇v + 3u^2 w v $.

Back to Newton now, for each step we will solve \[ \dualp{\Dif F(u_h^n) w_h, v_h} = \dualp{F(u_h^n), v_h} \] for $w_h^n$ and then update $u_h^{n+1} = u_h^n - w_h$. Practically, this means computing the linear problem \[ \int_Ω ∇w_h ∇u_h + 3u_h^2 w_h v_h = \int_Ω  ∇u_h^n ∇v_h (u_h^n)^3 v_h - f v_h\]

It can be seen that the left hand side is a coercive bilinear form and therfore the underlying matrix of coefficients of shape functions will be symmetric and positive definite, and this linear problem will be solved.

In practice, this Newton method is quadratic and the error at each stage is bounded by $e_{n+1} ≤ Ce_n^2$. The disadvantage is that the method only converges if the initial guess $u_h^0$is sufficiently close to $u_h$, which is an unknown solution. For this non linear problem it will usually work, but if we were to solve other, more complex nonlinear problems (such as Navier-Stokes with high Reynolds number) we could find that the method diverges or oscillates around a non-solution.

\subsection{A posteriori error estimates}

\begin{prop} Consider the non-linear problem \eqref{eq:PDE:NonlinearProblem} with the domain $Ω$ a convex polygon, $f ∈ L^2(Ω)$ and a regular mesh $\mesh$ (bounded aspect ratio, $\sfrac{h_K}{ρ_K} ≤ c$). Then, there exists a constant $C >0$ independent  of $h$, $f$ and $u$ but depending on the mesh aspect ratio such that \[ \norm{∇(u - u_h)}^2_{L^2(Ω)} ≤ C \sum_{K ∈ \mesh} η_K^2 \] with \[ η_K = h_K \norm{f + Δu_h - u_h^3}_{L^2(K)} + \frac{1}{2} \sqrt{h_K} \norm{[∇u_h · \vn]}_{L^2(∂K)} \]
\end{prop}

\begin{proof} As always, we start from the norm of the error and then add any $v_h ∈ V_h$ by the Galerkin orthogonality \begin{align*}
\int_Ω \abs{∇(u - u_h)}^2
	&= \int_Ω f(u - u_h) - u^3(u - u_h) - ∇u_h ∇(u - u_h) \\
	&= \int_Ω (f - u_h^3)(u - u_h) - ∇u_h ∇(u - u_h) - (u^3 - u_h^3) (u - u_h) \\
	&= \int_Ω (f - u_h^3)(u - u_h - v_h) - ∇u_h ∇(u - u_h - v_h) - \underbracket{\int_Ω (u^3 - u_h^3) (u - u_h)}_{≥ 0} \\
	&≤ \sum_{K ∈ \mesh} \left( \int_K (f - u_h^3)(u - u_h -v_h) + \frac{1}{2} \int_{∂K} [∇u_h · \vn] (u - u_h - v_h)\right)
\end{align*}
where we already saw the positiveness of $\int_Ω (u^3 - u_h^3)(u - u_h)$ in the uniqueness proof of \fref{prop:NonLinearProblemExistenceUniqueness}, so we can get rid of it and continue as usual.

Now we choose $v_h = R_h(u - u_h)$ the Clément interpolant and we get as always the result.
\end{proof}


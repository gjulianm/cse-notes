% -*- root: ../NumericalApproximationofPDEs.tex -*-
\chapter{Introduction to the finite elements method}

Along this part, we will center on the problem of solving numerically differential equations on 1D such as \( \label{eq:FE:BasicOde} \begin{cases} Du = f & \text{ in } [a,b] ⊂ ℝ \\ u(a) = u_a \\ u(b) = u_b \end{cases} \) with $D$ some kind of differential operator.

Usually, the first approach to a numerical solution of these equations imply the use of finite differences, using an approximation of the derivatives such as \[ f'(x) \approx \frac{f(x + h) - f(x)}{h} \] which then converts the problem \eqref{eq:FE:BasicOde} to one of a linear system $Au = f$ of $N$ equations, one for each of the intervals of length $h$ in which $[a,b]$ gets divided. This, however, poses a problem in terms of the size of the system $\sfrac{b - a}{h}$ equations, and is also ill-conditioned when $h$ is too small.

We will instead search for another approach. We will construct a space $V_h$ of finite dimension, and search there for a solution $u_h$ that approximates $u$. Usually, $V_h$ will be a space of piecewise polynomial functions. These polynomials will be defined on a mesh \mesh defined by a set of $N$ points $\set{a = x_1 < x_2 < \dotsb < x_{N} = b}$, that we will call vertices.

Thus, the generic space will be \( X_h^r = \set{ f ∈ C^0([a,b]) \tq \restr{f}{I_j} ∈ \mathbb{P}_r(I_j) \quad ∀ I_j ∈ \mesh } \label{eq:FE:FiniteElementSpace} \), being $I_j$ each of the intervals of the mesh and $\mathbb{P}_r(I_j)$ the space of polynomials of degree up to $r$ defined in the interval $I_j$. Usually, we will work in a subset of this space to use only functions compatible with the boundary contitions.

Something that the attentive reader will have noticed by now is the fact that we didn't make functions in $X_h^r$ derivable. We just forced continuity, which is a weird thing given the fact that we are trying to solve differential equations, whose solutions by definition should have some degree of smoothness.

However, we will actually be interested in ``non-continuous'' ``solutions'' to the PDE problems, because those actually exist in physical settings. For example, a vibrating string may have a point force as initial condition, and that is not continuous. This is the motivation to introduce the weak formulation of a differential equation.

\section{Weak formulation}

A physical motivation for the weak formulation is the introduction of ``virtual displacements'' but I have to admit that it is only intuitive if you are a physicist. The mathematical idea is to move to a setting where we can forgive some non-smoothness of the function. This framework is the integration of a product: multiplying by a test functions with minimal regularity and using integration by parts we will be able to move the derivatives from our solution to the test function and then require less regularity on the solution.

For a practical example, let our equation be $-u'' = f$. We will multiply by a function $v$ in some space that we will decide later, integrate in the domain $Ω$ of the solution and then apply integration by parts:
\begin{align*}
- u'' &= f \\
\int_Ω - u'' v &= \int_W fv \\
\int_Ω u' v' - \restr{u'v}{∂Ω} &=\int_Ω fv
\end{align*}

If we choose $v$ to be $0$ on the boundary $∂Ω$, we will eliminate the second term and will have ended up with \[ \int_Ω u' v' = \int_Ω fv\] which indeed admits solutions of class $C^1$, one less degree than we originally needed.

Usually, we will use a more abstract form, generic for any problem.

\begin{defn}[Weak\IS abstract formulation] \label{def:FE:WeakAbstract} A weak abstract formulation of a differential equation is the problem of finding $u ∈ V$ with $V$ a certain Hilbert space such that \[ a(u,v) = F(v) \quad ∀v ∈ V\], with $\appl{a}{V × V}{ℝ}$ a bilinear form and $\appl{F}{V}{ℝ}$ a linear functional on $V$.
\end{defn}

This will allow us to tackle theoretical problems and theorems in a generic setting without worrying about the actual differential operators.

\section{Galerkin approximation and finite element spaces}

Once we have the \nref{def:FE:WeakAbstract}, we can define an approximate problem. We will construct a family of finite dimensional spaces $V_h ⊂ V$ with a proper approximation property (specifically, that $\inf_{v_h ∈ V_h} \norm{v_h - v} \convs[][h][0] 0$ for any $v ∈ V$) and search for a solution there. Thus, we will find a solution $v_h ∈ V_h$ such that \( a(u_h, v_h) = F(v_h) \quad ∀v_h ∈V_h \label{eq:FE:GalerkinApprox} \)

Recall that previously we defined these finite element spaces in \eqref{eq:FE:FiniteElementSpace}. The interesting thing of that kind of finite element spaces is the fact that we can define a very convenient basis for them. That is a problem that reduces to that of finding a basis of functions for the polynomials defined on one interval, and for that we can use the Lagrange polynomials.

\begin{defn}[Lagrangian basis\IS of a polynomial function space] Let $\mathbb{P}_r(I)$ be the space of polynomials of degree $r$ or less in an interval $I = [a, b] ⊂ ℝ$. Each polynomial is completely defined by its values at $r + 1$ points, so we define a set of nodes $\set{x_0 = a < x_1 < \dotsb < x_r = b}$. The basis $\set{φ_i}_{i = 0}^r$ of $\mathbb{P}_r(I)$ is then defined by the polynomials $φ_i$ of degree $r$ such that $φ_i(x_j) = δ_{ij}$ with $δ_{ij}$ being the Kronecker's delta.
\end{defn}

\begin{figure}[tp]
\centering
\inputtikz{LagrangianBasis}
\caption{Several examples of Lagrangian basis polynomials for degrees $1, 2$ and $3$ respectively.}
\label{fig:FE:LagrangianBasis}
\end{figure}

Why is this basis useful? Let's look again at the Galerkin approximation \eqref{eq:FE:GalerkinApprox}. Given that both $a$ and $F$ and linear, for the approximation to hold for any $v_h ∈ X_h^r$ we only need to check it for all the elements $\set{φ_i}_{i = 0}^n$ of the base of $X_h^r$ (any $v_h$ is a linear combination of these elements). That is, we need to ensure that \[ a(u_h, φ_i) = F(φ_i) \quad ∀i = 1, \dotsc, n \]

Again, using linearity of $a$ we can reformulate that as \[ \sum_{j = 0}^n u_j a(φ_j,φ_i) = F(φ_i) \quad ∀i = \dotsc, n\] where $u_j$ are the unknown coefficients of $u_j$ in the base $\set{φ_j}$. The problem is thus reduced to a linear system \[ \mA \vu = \vec{f} \] where the \concept{Stiffness\IS matrix} $\mA$ is defined by \[ \mA_{ij} = a(φ_i, φ_j)\] and analogously $\vec{f}_i = F(φ_i)$.

And although this approach can always be constructed for any finite element space, the advantage with the Lagrangian basis and polynimal finite spaces is that the coefficients are actually the values of $u$ at the nodes, which will make computations far easier.

% TODO: Things here
\chapter{Theoretical analysis of the weak formulation of a differential problem}


\chapter{Stabilized methods}

\section{Motivation: Diffusion-transport problems with dominant transport}

In previous sections we have discussed the theoretical framework for numerical approximation of PDEs, but there are practical issues that need addressing. We will explore them using as a motivation diffusion-transport problems with dominant transport.

These problems model the behaviour of the concentration $u(x,t)$ of a pollutant at time $t$ in the position $x$ in a narrow channel of fluid. To formulate the equations, we must enforce conservation of mass: the change of pollutant in an interval along a period of time must be equal to the pollutant entering that interval minus the one exiting it. This translates to the equation \[ \od{}{t} \int_{x}^{x + Δx} u(x,t) \dif x = q(x,t) - q(x + Δx, t)\] with $q(x,t)$ being the quantity of pollutant moving right at point $x$ and time $t$. Divide both sides by $Δx$ and make $Δx \to 0$ and we have that our equation is \[ u_t = q_x \]

Now, we must decide how to model the movement of the pollutant $q$. There are two phenomenons that we will want to model here:

\begin{itemize}
	\item \textbf{Convection}  or transport: the pollutant moves with velocity $b$. Thus, $q(x,t) = b · u(x,t)$.
	\item \textbf{Diffusion}: the pollutant dissolves itself moving from regions with high concentration to regions with low concentration. A similar model than in the heat equation, so $q(x,t) = - μ u_x(x,t)$ with μ the diffusion coefficient.
\end{itemize}

Combining both apportations we arrive to the general formulation of these kind of problems:
\( \label{eq:Stabilized:DiffusionTransport} u_t - μu_{xx} + bu_x = 0\)

We will consider only the static problem (this part regards only ODEs) so $u_t = 0$. From that, we derive the weak formulation:
\begin{align*}
-μu'' + bu' &= 0 \\
\int_Ω -μu'' v \dif x  + \int_Ω bu' v\dif x &= 0 \\
\int_Ω μu' v' \dif x + \int_Ω bu' v \dif x &= 0
\end{align*} forcing $v = 0$ on the boundary. Thus, our bilinear form is \[ a(u,v) = \int_Ω μu' v' \dif x + \int bu' v \dif x \]

We will now check the conditions of the \nref{lem:TheorWeak:LaxMilgram}. For continuity, it is easy to see that \[  \norm{a(u,v)} ≤ (μ + b) \norm{u} \norm{v} \]

For coervicity, we check and \[ \norm{a(v,v)} = \norm{\int_{Ω} v'(μv' + bv) \dif x } ≥ \dotsc > μ \norm{v}^2 \]

This gives us the following error estimation (from where?) of the approximation: \[ \norm{u - u_h} ≤ C \frac{M}{α} h^r \abs{u}_{H^{r+1}(Ω)} \] with $M = μ + b$ and $α = μ$ the continuity and coercivity constants. And now the magic questions: what happens if we are in a setting were transport is dominant and $\sfrac{b}{μ} \gg 1$?

It is easy to notice then that $\sfrac{M}{α} \gg 1$ and thus we can find that we need even bigger meshes (smaller $h$) to achieve a good approximation. In practice, the method still works but the computational cost is considerably higher.

What we will try to do in this section is to see how can we make the problem computationally easier, while maintaining its solution. Thus the name of ``stability'' methods.

\section{A framework for stabilized methods}

We will consider our usual Galerkin approximation problem of finding a $u_h ∈ V_h$ such that \[ a(u_h, v_h) = F(v_h) \quad ∀v_h ∈ V_h\]

From this, we will introduce the generalized Galerkin problem.

\begin{defn}[Generalized Galerkin problem][Galerkin problem!generalized] \label{def:Stabilized:GeneralizedGalerkin} Let $\appl{a, \mathcal{L}_h}{V_h × V_h}{ℝ}$ be two bilinear forms and $\appl{F,Ψ_h}{V_h}{ℝ}$ two linear functionals on $V_h$. The generalized Galerkin problem is then the problem of finding $u_h ∈ V_h$ such that \[ a(u_h, v_h) + \mathcal{L}_h(u_h, v_h) = F(v_h) + Ψ_h(v_h) \quad ∀v_h ∈ V_h \]

Usually, we will denote $a_h = a + \mathcal{L}_h$ and $F_h = F + Ψ_h$. $\mathcal{L}_h$ and $Ψ_h$ will be called \textbf{stabilization terms}.
\end{defn}

Which kind of stabilization terms would we want to use? Well, we don't want to change the solution of the system but only stabilize the approximations. Thus, we will search for strongly consistent stabilization terms.

\begin{defn}[Strongly consistent stabilization terms][Stabilization term!strongly consistent] In a \nref{def:Stabilized:GeneralizedGalerkin}, we will say that $\mathcal{L}_h, Ψ_h$ are strongly consistent if \[ \mathcal{L}_h(u, v_h) = Ψ(v_h) \quad ∀v_h ∈ V_h\] with $u$ the exact solution to the problem.
\end{defn}


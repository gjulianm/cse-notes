% -*- root: ../NumericalApproximationofPDEs.tex -*-
We start by considering a simple Poisson equation, given $Ω ⊂ ℝ^d$ a bounded open domain with Lipschitz boundary $∂Ω$. There, we set the problem \[ \begin{cases}
-Δu = f & \text{ in } Ω \\
u =  0 & \text{ in } ∂Ω
\end{cases} \]

This problem can be seen as solving the behaviour of an elastic membrane of shape $Ω$, with $u$ being the height of the membrane, or as a heat diffusion equation with $u$ the temperature and $f$ the sources/sinks of heat.

In \fref{sec:Theory:WeakFormulationPDE} we saw that the weak abstract form was  \[ \int_Ω \grad u · \grad v = \int_Ω f v \]

It is easy to see that we can apply the \nref{thm:Theory:LaxMilgram} to our problem: if $v ∈ H^1_0(Ω)$, then its $L^2$-norm is bounded so \[ \abs{\int_Ω fv} ≤ \norm{f}_{L^2} \norm{v}_{L^2} ≤ \norm{f}_{L^2} \norm{v}_{H^1} < ∞ \]

Same happens with $a$: the boundedness is directly obtained from the fact that $u, v ∈ H^1_0(Ω)$. Coercivity can be a little bit more complicated. For that, we must use the \nref{thm:Fund:PoincareInequality} which tells us that there is a constant $C_p > 0$ such that $\int v^2 ≤ C_p \int \abs{\grad u}^2$ for all $u ∈ H_0^1$. So, in this case we have that \[
\norm{u}^2_{H^1} = \int v^2 + \int \abs{\grad v}^2 ≤ (1+C_p^2) \int \abs{\grad u}^2
\] so our coercivity constant is $\frac{1}{1 + C_p^2}$.

This means that, applying the \nref{thm:Theory:LaxMilgram}, we have a unique solution $u ∈ V$ such that \[ \norm{u}_{H^1_0} ≤ (1+C_p^2) \norm{F}_{H^1_0} \] with $C_p$ depending on the shape of the problem domain, even when it is a convex polygon.

\subsection{Poisson problems with mixed boundary conditions}

We will want to know what happens when we have mixed boundary conditions, that is, \nref{def:Theory:DirichletBoundary} and \nref{def:Theory:NeumannBoundary}. We will study the problem \( \begin{cases}
-Δu =f & \text{in }Ω \\
u = g & \text{on }Γ_D \\
∂_\vn u = h & \text{on }Γ_N \end{cases} \) with $Γ_D ∪ Γ_N = ∂Ω$.

We can start with the weak formulation as previously \[ \int_Ω fv = \int_Ω -Δu v = \int_Ω \grad u · \grad v - \int_{∂Ω} ∂_\vn u v = \int_Ω \grad u \grad v - \int_{Γ_N} ∂_\vn v - \int_{Γ_D} ∂_n u v\]

The problem here is how to deal with the boundary term on $Γ_D$. As we did in the previous section, we can select $\restr{v}{Γ_D} = 0$. Thus, our weak formulation is \( \int_Ω \grad u · \grad v = \int_Ω f v + \int_{Γ_N} h v \label{eq:Elliptic:WeakFormulationMixedBoundary} \) with $v ∈ H^1_{Γ_D}$, where we can define \( H^1_{Γ_D} = \set{ v ∈ H^1 \tq \restr{v}{Γ_D} = 0 } \)

Our solution should however live in $V_g = \set{v ∈ H^1 \tq \restr{v}{Γ_D} = g}$, which is not linear but only an affine subspace.

If we have $g = 0$, we still can use a \nlref{def:Theory:WeakAbstractFormulation} with $F(v) = \int fv + \int_{Γ_N} hv$ to find a solution $u ∈ H^1_{Γ_D} = V_0$. With $g ≠ 0$ things become a little bit more difficult.

\subsubsection{Lifting technique}

If the datum $g$ is not null, we need to find a solution $u ∈ V_g$, and as we said before this is not a linear subspace and, even worse, it is different to the test space $V_0$ so we cannot apply Lax-Milgram.


However, if $g ∈ H^{\sfrac{1}{2}}(∂Ω)$, we can apply the \nref{thm:Fund:Trace} and show that it is the trace of some function $G ∈ H^1(Ω)$ such that $\restr{G}{∂Ω} = g$ and $\norm{G}_{H^1(Ω)} ≤ γ\norm{γ}_{H^{\frac{1}{2}}(∂Ω)}$. We can then write $u = u_0 + G$ with $u_0 ∈ V_0$, and as $a$ is a linear form the weak abstract form changes: \begin{align*}
a(u, v) &= F(v) \\
a(u_0 + G, v) &= F(v) \\
a(u_0, v) &= \underbracket{F(v) - a(G, v)}_{\tilde{F}(v)}
\end{align*} which is a problem for which we can apply the \nref{thm:Theory:LaxMilgram} and solve for $u_0$, and then reconstruct the solution as $u = u_0 + G$.

\subsection{Derivation of the weak formulation based on calculus of variations}

We can try to study other approach to this problem with an example, which is the deformation $u$ of a membrane $Ω$ under a certain force $f$. In that case, we try to find a solution that minimizes the elastic energy $E = \int_Ω \frac{1}{2} κ \norm{\grad u}^2$. Supposing $κ = 1$, our energy functional to minimize is \( J(u) = \frac{1}{2} \int_Ω \norm{\grad u}^2 - \int_Ω f u - \int_{Γ_N} h u \label{eq:Elliptic:EnergyFunctional} \) so our solution should be \[ u = \argmin_{\substack{v ∈ H^1(Ω) \\ v = \restr{g}{Γ_D}}}  J(v) \] where we search for the function in $H^1(Ω)$ because we need for the gradient to be square integrable. We need also $f ∈ L^2$, and $h ∈ H^{-\sfrac{1}{2}}$ (the topological dual space of $H^{\sfrac{1}{2}}$) to be able to integrate that last term.

So, how do we do this? If the argument was real, we could find the point with gradient $0$ (all directional derivatives are null). But the functional $J$ from \eqref{eq:Elliptic:EnergyFunctional} is an application $\appl{J}{H^1}{ℝ}$, so we need something different. However, we can still translate the concept of ``gradient $0$''. If $J$ were a real variable function, we could do \[ \grad J (u) = 0 \iff \grad J(u) · \vA = 0\; ∀\vA ∈ ℝ^N \iff \lim_{ε \to 0} \frac{J(u + ε\vA) - J(u)}{ε} = 0 \]

That last notion is the one we can translate to the functional case. If we consider $u$ to be the equilibrum, we can study any variation of $u ∈ V_g$\footnote{Remember from the previous section that $V_g = \set{v ∈ H^1 \tq \restr{v}{Γ_D} = g}$.} such that $u + ε v ∈ V_g$ (that is, respecting the boundary conditions), with $v ∈ V_g$ forcibly.

Knowing this, we can try to calculate the ``derivative'', where some linear terms will be canceled but we will have to deal with the quadratic ones:
\begin{align*}
\Dif_v J(u) &= \lim_{ε \to 0} \frac{J(u + εv) - J(u)}{ε} = \\
	&= \lim_{ε \to 0} \frac{1}{ε} \left[ \frac{1}{2} \int_Ω \grad(u+εv)· \grad(u + εv) - \int_Ωf(u + εv) - \int_{Γ_N} h · (u + εv)\right. \\
	&\qquad \left.- \frac{1}{2}\int_Ω \grad u \grad u + \int_Ω fu + \int_{Γ_N} h u \right] = \\
	&= \lim_{ε \to 0}\frac{1}{ε} \left[ \frac{1}{2}\left( \int_Ω \grad(u+εv) \grad (u + εv) - \grad u \grad u \right) - ε \int_Ω fv - ε \int_{Γ_N} h v \right] = \\
	&= \lim_{ε \to 0}\frac{1}{ε} \left[ \frac{1}{2} \left( \grad u \grad u + ε \grad u \grad v + ε \grad v \grad u + ε^2 \grad v \grad u - \grad u \grad u\right) - ε \int_Ω fv - ε \int_{Γ_N} h v \right] = \\
	&= \int_Ω \grad u \grad v - \int_Ω fv - \int_{Γ_N} h v
\end{align*} which is the same weak formulation of the problem we had previously in \eqref{eq:Elliptic:WeakFormulationMixedBoundary}.

\subsection{Regularity of the solution}
\label{sec:PDE:PoissonRegularity}

We may have proved that the solution exists, but we will also be interested in knowing the regularity of the function in order to be able to prove rates of convergence, for example. That will be given in the following theorem.

\begin{theorem}[Shift theorem][Theorem!shift] \label{thm:PDE:Shift} Consider the PDE problem \[
-Δu =f \qquad \text{in }Ω \] with $f ∈ H^m(Ω)$, $Ω$ a smooth domain ($∂Ω ∈ C^{m+2}$, that is, we can parametrize the boundary as a $C^{m+2}$ manifold) and with the following restrictions depending on the boundary conditions:
\begin{itemize}
	\item Full Dirichlet conditions $\restr{u}{Γ_D} = g ∈ H^{m + \sfrac{3}{2}}(Ω)$.
	\item Full Neumann conditions $∂_\vn u = h ∈ H^{m + \sfrac{1}{2}}(Ω)$.
\end{itemize}

Under those conditions, $u ∈ H^{m+2}(Ω)$.
\end{theorem}

Thus, for example, having only $∂Ω ∈ C^2$, we would have $u∈H^2(Ω)$ and $\norm{u}_{H^2(Ω)} ≤ C_Ω \norm{f}_{L^2(Ω)}$.

However, if we don't have that regularity on the border we have weird things. Although as discussed previously we still have existence of the solution, the regularity depends on the angles of the polygon, with $u ∈ H^s(Ω)$ and $1≤s<2$. For a convex polygon we have the same results as above. If the polygon is not convex, such as an ``L'' shape, we have $s = \sfrac{3}{2}$. If we have something like a ``crack'' in the domain, we would only have $s = 1$.

It is also interesting to see that if $f ∈ C^∞(Ω)$ then $u ∈ C^∞(Ω)$, but not necessarily smooth on the closure (that is, we can't guarantee that $u ∈ C^∞(\adh{Ω})$).

\subsubsection{Corner singularities}

\begin{wrapfigure}{L}{0.3\textwidth}
\centering
\inputtikz{CornerSingularity}
\caption{Corner singularity in a domain.}
\label{fig:Elliptic:CornerSingularity}
\end{wrapfigure}

One could try to see what happens if we have corner singularities, for example, in a square domain. In a set inside of the square we will have perfect smoothness, but the corners may present problems.

Suppose we want to solve $- Δ u = 0$ in a corner such as \fref{fig:Elliptic:CornerSingularity}. In that case, we will work in polar coordinates and the basis of our solution will be \[φ_k(r,θ) = r^{\frac{kπ}{ω}} \sin \frac{kπθ}{ω} \]

The worst case would be $k = 1$, which would leave us in a case of $u ∈ H^s$ with $s < 1 + \sfrac{π}{ω}$. If $ω > π$ (the case of the square), we have $s < 2$, that is, we only have $H^1$ regularity.

We would have the same situation with Neumann conditions. However, mixed boundary conditions (Neumann on one side, Dirichlet on the other) are more problematic: the solutions are \[ φ_k(r,θ) = r^\frac{(k + \sfrac{1}{2})π}{ω} \sin \frac{(k + \sfrac{1}{2})πθ}{ω} \] and, in the worst case ($k = 0$) we have singularities even in the flat case ($ω = π$) which was not a problem in full Neumann or Dirichlet conditions.


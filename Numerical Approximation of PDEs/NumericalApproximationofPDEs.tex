\documentclass[palatino,twoside]{epflnotes}

\title{Finite elements: Numerical approximation of Differential Equations}
\author{Guillermo Julián Moreno}
\date{July 2017}

% Additional packages
\usepackage{tikztools}
\usepackage{fastbuild}

\precompileTikz
% --------------------

\newif\ifincludefirstsemester
\includefirstsemestertrue
% \includefirstsemesterfalse

\begin{abstract}
This document aggregates the notes of several courses on the finite elements method for numerical approximation of differential equations, imparted at EPFL during the academic year 2016/2017.
\end{abstract}

\begin{document}
\frontmatter
\pagestyle{plain}
\maketitle

\tableofcontents
\mainmatter
% Content

\ifincludefirstsemester
\part{Prelude: Fundamentals on analysis}
% -*- root: ../NumericalApproximationofPDEs.tex -*-
\chapter{Analysis fundamentals}

\section{Multivariable calculus}

Just as a reference, some basic theorems of analysis that will be useful.

\begin{prop}[Integration\IS by parts] \label{prop:Fund:IntegParts} Let $Ω ⊂ ℝ^d$ be an open subset with Lipschitz boundary $∂Ω$, and suppose $\appl{u}{ℝ^d}{ℝ^d}$ and $\appl{v}{ℝ^d}{ℝ}$ are two derivable functions\footnote{Actually, the requirement is just $u,v ∈ H^1(Ω)$.}. Then, the following equality holds:
\( \int_Ω v \dv u = \int_{∂Ω} u v - \int_Ω ∇v · u \)
\end{prop}

Also, the Stokes theorem which is incredibly useful for an elegant generalization of Green and Gauss theorems.

\begin{theorem}[Stokes\IS theorem] \label{thm:Fund:Stokes} Let $M$ be a compact orientable manifold of dimension $d$ with border $∂M$, and let $ω$ be a $(d-1)$ differential form. Then, \( \int_{∂M^+} ω = \int_M \dif ω\) where $∂M^+$ is the border in the positive orientation. For this orientation, a thumb rule: the manifold should be at the left of the border.
\end{theorem}

\subsection{Useful inequalities}

\begin{prop}[Cauchy-Schwarz inequality][Inequality!Cauchy-Schwarz] \label{prop:Fund:CauchySchwarz} Let $V$ be a vector space with inner product $\pesc{·,·}$, and let $a, b ∈ V$. Then, the following holds: \[ \abs{\pesc{a,b}} ≤ \norm{a} \norm{b} \]
\end{prop}

\begin{prop}[Young's inequality][Inequality!Young] \label{prop:Fund:Young} Let $a,b ∈ ℝ^+$. Then, if $p, q ∈ ℝ^+$ are such that $\sfrac{1}{p} + \sfrac{1}{q}$, then \[ ab ≤ \frac{a^p}{p} + \frac{b^q}{q} \]

Another version: if $ε > 0$, then \[ ab ≤ \frac{a^2}{2ε} + \frac{εb^2}{2} \]
\end{prop}

\chapter{Functional analysis fundamentals}

\section{Banach and Hilbert spaces}

The first fundamental of functional analysis is, as usual, to define the basic spaces in which we are going to work. These are going to be, obviously, spaces of functions. The next step is what do we want to study in these spaces. The first notion is that of convergence: knowing when a sequence of functions converges to another and in which sense. Instead of going like in other courses defining special types of convergence, we are actually going to rely on topology. We will treat functions like points in a topological space, and saying that they converge to something if they start getting closer to that something.

So, what we need is a distance or, actually, a norm\footnote{A norm $\norm{·}$ induces a distance directly by $\dst(x,y) = \norm{x-y}$.}. Let's go with the definition:

\begin{defn}[Norm] Given a space $V$ over a field $K$, we say that an application $\appl{\norm{·}}{V}{ℝ}$ is a norm if and only if
\begin{enumerate}
	\item Given $λ ∈ K$ and $x ∈ V$, then $\norm{λx} = \abs{λ}\norm{x}$.
	\item Verifies the triangle equality: given $x,y ∈ V$, then $\norm{x + y} ≤ \norm{x} + \norm{y}$.
	\item It separates points, that is, $\norm{x} = 0 \iff x = 0$.
\end{enumerate}

If the last property is not verified, we say we only have a \concept{Seminorm}.
\end{defn}

We will then work in spaces equipped with a norm. We will ask that these spaces have a nice property with respect to this norm: completeness. We say that a space $V$ is \concept[Completeness]{complete} if and only if for every Cauchy sequence $(x_n) ⊂ V$, we have convergence in the space $x_n \to x ∈ V$. In other words, we don't want to worry whether we are going to be out of the space when working with convergent sequences.

With this, we can go on to the notion of a Banach space:

\begin{defn}[Banach space][Space!Banach] A vector space $V$ is called Banach if it has a norm and is complete with respect to that norm. Formally, the Banach space is the space with the norm, that is, $X = (V, \norm{·}_V)$.
\end{defn}

Banach spaces are the ones we do topology in. Notions of compactness and convergence\footnote{Mainly the two topological aspects we will be interested in: compactness gives us finiteness and convergence, duh, is convergence.} will be studied on these spaces.

The next step is to be able to do geometry, measuring angles and specially talking about ortogonality. This is accomplished with the scalar product:

\begin{defn}[Scalar product][Product!scalar] Given a vector space $V$, we say that an application $\appl{\pesc{·,·}}{V×V}{ℝ}$ is a scalar product if and only if it is antilinear\footnote{Beware with complex numbers: $\pesc{λx,y} = λ\pesc{x,y}$ but $\pesc{x,λy} = \conj{λ}\pesc{x,y}$.} , positive definite and symmetric\footnote{Again, take into account that with complex numbers we have $\pesc{x,y} = \conj{\pesc{y,x}}$.}.
\end{defn}

Again, we have a name for spaces with a scalar product:

\begin{defn}[Hilbert space][Space!Hilbert] We say that a vector space with a scalar product $(V, \pesc{·,·}_V)$ if and only if it is Banach with respect to the induced norm $\norm{x}_V = \sqrt{\pesc{x,x}_V}$.

As it was the case with Banach space, the subjacent space and the tuple space-scalar product is often used interchangeably, but the good mathematician should know what is what.
\end{defn}

\subsection{$L^p$ spaces}

Now on to a specific example of Banach and Hilbert spaces: $L^p$ spaces. These will be our basic workbench, so it pays off to define them carefully.

The first question we have to make is which kind of functions do we want to study. One could think that, given that this is a subject on differential equations, we should be studying $C^k$ spaces for some $k ∈ ℕ$. However, this turns out to be too restrictive, and the reason is one of completeness: these spaces are too small. One can see that we can approximate absolute monster functions (and even things that are not even functions) with infinitely smooth functions. It would be certainly difficult to work in these kind of spaces where one can step out of them at the slightest oversight.

So, if we are dropping smoothness assumptions, we can work with the related operation: integration. Despite this horribly weak argument, this is the best choice although it requires a little bit of care and measure theory. Using integration yields nice complete spaces, a direct definition of a norm and a very good tolerance for monsters.

The first base is then to define the basic space. Given $Ω ∈ ℝ^n$ and a measure $μ$ on $ℝ$, we define the $\mathcal{L}^p$ space as \( \mathcal{L}^p(Ω, μ) ≝ \set{\appl{f}{Ω}{ℝ} \tq \int_Ω\abs{f}^p \dif μ < ∞ } \label{eq:Fundamentals:LCalSpace} \) with the corresponding norm \( \norm{f}_{\mathcal{L}^p(Ω,μ)} = \left(\int_Ω \abs{f}^p \dif μ \right)^{\sfrac{1}{p}} \label{eq:Fundamentals:LpNorm} \)

As the measure used is usually the Lebesgue measure, the `μ' is usually dropped off the notation.

What is important --- and the reason why we need measure theory --- is that the norm defined in \eqref{eq:Fundamentals:LpNorm} is not actually a norm for the space $\mathcal{L}^p$: it does not separate points. The function $f \equiv 0$ has obviously norm 0, but a function that doesn't vanish on a set of measure zero has norm 0 too.

The solution is a little bit `stupid', but it works: we define those two functions as to be the same one, and in fact we will do that with every two functions that are different only on a set of measure zero. With a little bit more formality, we define the following equivalence relation\footnote{Checking the properties is left as an exercise to the distrustful reader.}:
\( ∀f,g ∈ \mathcal{L}^p(Ω), \quad f \sim g \iff \int_Ω \abs{f-g} \dif μ = 0 \label{eq:Fundamentals:EquivRel} \)

With this we can define what is a $L^p$ space:

\begin{defn}[{$L^p$} space][Space!{$L^p$}] Let $Ω ⊂ ℝ^n$ be a domain, and $μ$ a measure on $ℝ$. Then, we define the $L^p(Ω,μ)$ space as the quotient $\quot{\mathcal{L}^p(Ω,μ)}{\sim}$, with the space defined as in \eqref{eq:Fundamentals:LCalSpace} and the equivalence relation of \eqref{eq:Fundamentals:EquivRel}.

In other words, $L^p$ space is the space of functions $p$-integrable on $Ω$, identifying those which only differ on a set of measure zero.

One special case: when $p = ∞$, the norm is given as \[ \norm{f}_{∞} = \sup_{x ∈ Ω} \abs{f(x)} \] and thus the $L^{∞}(Ω)$ space is the one of bounded functions in Ω.
\end{defn}

The $L^p$ spaces are always Banach for $0 < p ≤ ∞$. In the case $p = 2$, it is a Hilbert space with the scalar product being \[ \pesc{f,g}_{L^2(Ω)} = \int_Ω f g \dif μ \]

Proofs of these facts are left to the reader.

\subsection{Linear operators}
\label{sec:Fund:Linops}

We will usually be interested in considering linear operators between Banach and/or Hilbert spaces.

\begin{defn}[Linear operator][Operator!linear] An application $\appl{\linop}{X}{Y}$ between vector spaces $X$, $Y$ is said to be linear if $\linop(a + b) = \linop(a) + \linop(b)$ for all $a,b ∈ X$ and $\linop(λa) = λ\linop(a)$ for any $λ ∈ ℂ$.
\end{defn}

In infinite dimensional spaces, linear operators are not necessarily continuous. We will be interested in those, so we give the formal definition.

\begin{defn}[Continous operator][Continuity!of an operator] \index{Operator!continuous} \label{def:Fund:ContinuousOp} Given a Banach space $V$ and a linear operator $T$ on $V$, we say that it is continuous if and only if for every Cauchy sequence $(φ_n)_{n ≥ 1} ⊂ V$ with $φ_n \to φ$ we have that \[ \lim_{n\to ∞} \pesc{T,φ_n} = \pesc{T, φ}\]
\end{defn}

Some other properties of these operators:

\begin{defn}[Symmetric operator][Operator!symmetric] \label{def:Fund:SymmetricOp} A linear operator $\appl{\linop}{V}{V'}$ is said to be symmetric if \[ (\linop u)(v) = (\linop v)(u) \] where $V'$ is the topological dual space of $V$.
\end{defn}

\begin{defn}[Antisymmetric operator][Operator!antisymmetric] \index{Operator!skew-symmetric} \label{def:Fund:AntisymmetricOp} A linear operator $\appl{\linop}{V}{V'}$ is said to be antisymmetric or skew-symmetric if \[ (\linop u)(v) = - (\linop v)(u) \]
\end{defn}

Note that if $V$ is a Hilbert space, then $V' = V$ and evaluating an element $\linop u ∈ V'$ at $v ∈ V$ is the same that doing the inner product: $(\linop v)(u) = \pesc{\linop v, u}_V$. Not that it matters but sometimes notation gets confusing between the duality product and inner product.

As it happens with linear algebra matrices, linear operators can always be decomposed in a symmetric and an antisymmetric part. For example, if we take $V = H_0^1(Ω)$ and the general  elliptic operator \[ \linop u = - (αu')' + (βu)' + γu \] with $α, β, γ$ smooth functions, we have the symmetric part as \[ \linop_S = -(αu')' + γu \] and the antisymmetric as \[ \linop_{SS} = (bu)' \]

\section{Distributions}

In the previous section we explained how our space is going to be the one of integrable functions. However, we have not acknowledged something that is really important for PDEs: how can we differentiate general integrable functions?

This requires us to bring in the theory of distributions, which are actually useful not only because they will allow us to define ``derivatives'' of general functions, but also because they are a useful tool when solving PDEs.

A good motivation for distributions is given in \cite{DistributionsFourierTransform}, questioning how do we actually measure things. For example, when we have a thermometer, we don't measure the temperature in just one point in space: we do some kind of ``average'' between all the points of the mercury. An interesting thought experiment is to see what would happen if we measured the temperature of a zone in which there is one single point with infinite temperature. Could we measure that infinite? Or maybe we wouldn't see it because the other points would smooth it out?

Distributions can be seen as a model of these kind of things. We couldn't model a point with infinite value with a function, but maybe we could think of that as some kind of generalized function.

On to the mathematics, we will first define what our ``thermometers'' are going to be. The space of test functions will usually be defined as $\mathcal{D}(Ω) = C_0^∞(Ω)$; infinitely smooth functions with compact support on some open set $Ω ∈ ℝ^n$. This is one example of tradeoffs in functional analysis: by requiring the test functions to be infinitely smooth, we will be able to work with distributions without any kind of regularity property. If we relaxed the requirements on test functions, we would have a smaller class of distributions.

Now, what is a distribution? We will first have the formal definition.

\begin{defn}[Distribution] Let $Ω ⊂ ℝ^n$ be a domain, and $\mathcal{D}(Ω) = C_0^∞(Ω)$ our space of test functions. A distribution is then a continuous linear operator $\appl{T}{\mathcal{D}(Ω)}{ℝ}$.

The space of distributions on $\mathcal{D}(Ω)$ is denoted by $\mathcal{D}'(Ω)$.
\end{defn}

This is actually a direct translation of the motivation we were talking of before. A distribution is just something that, when measured with a test functions, gives us a number. This is usually denoted by $T(φ)$ or $\pesc{T, φ}$.

We require it to be linear because non-linear things are just awful. Continuity is also very important: if we are using a sequence of test functions converging to some other function, we want the result to be the same for the two possible paths of first measuring, then doing the limit and viceversa. That's the formal definition given in \fref{def:Fund:ContinuousOp}.

In the specific case of test functions $\dstr(Ω)$ with $Ω ∈ ℝ^d$, we have to be careful with the definition of convergence: given a sequence $(φ_n)_{n≥1} ⊂ \dstr(Ω)$, we say that it converges to $φ ∈ \dstr(Ω)$ if and only if $\sop φ_n ⊂ K$ for some compact set $K ⊂ Ω$, and if $∂^\vA φ_n \to ∂^\vA φ$ with respect to the supremum norm for every multiindex\footnote{Reminder: a multiindex $\vA = (α_1, α_2, \dotsc, α_d)$ is a notation for the mixed partial derivatives $\frac{∂^{\abs{\vA}}}{∂x_1^{α_1}∂x_2^{α_2} \dotsb ∂x_d^{α_d}}$.} $\vA ∈ ℕ^d$.

We have to see two examples of distributions. One will convince us that distributions are really generalized functions by seeing that functions are always distributions. Indeed, given $f ∈ \espLloc[1][Ω]$\footnote{Locally integrable functions, that is, funcitons integrable over every compact subset of Ω. This is a bigger class than $L^p$ (trivially, every integrable function in Ω is also integrable on any subset of it).}, we can assign it a distribution $T_f$ by means of \[ T_f(φ) = \int_Ω f φ \]

The second example convinces us that this space is actually bigger and possibly useful: we can define the \concept{Dirac's delta} centered in $x ∈ Ω$ as a distribution \[ \pesc{δ_x, φ} = φ(x)\] that gives us the value of the test function at $x$. It is easy to check that it is a distribution, but we couldn't define it as a function\footnote{We could, however, define it as a measure: given a set $E$, $δ_x(E) = 1$ if $x ∈ E$, and 0 if $x ∉ E$.} (it is zero everywhere but with an ``infinite impulse'' at $x$).

\subsection{Distributional derivatives}

After studying distributions, we can define derivatives on them. The motivation is to make this derivative coincide with the usual derivative when the distribution is associated to a function. So, let $f ∈ C^1(Ω)$, $T_f ∈ \dstr'(Ω)$ the associated distribution and $\dstr(Ω)$ our space of test functions. We would like that $T_f'(φ) = T_{f'}(φ)$ for all $φ ∈ \dstr(Ω)$. Writing it down and integrating by parts: \[ T_{f'}(φ) = \int_Ω f' φ \dif μ = \underbracket{\restr{f φ}{∂Ω}}_{=0} - \int_Ω fφ' \dif μ = - \int_Ω fφ' \dif μ \] where φ vanishes on the border of Ω as it must have compact support. This is enough to define the distributional derivative.

\begin{defn}[Distributional derivative][Derivative!distributional] Let $Ω ⊂ ℝ^d$ be an open set and $T ∈ \dstr'(Ω)$ a distribution. Then, we say that the partial distributional derivative (with multiindex) is $∂^\vA T$ if for every test function $φ ∈ \dstr(Ω)$ \[ \pesc{∂^\vA T, φ} = (-1)^{\abs{\vA}} \pesc{T, ∂^\vA φ} \]
\end{defn}

Other differential operators can be defined in an analogous manner.

\begin{defn}[Distributional gradient][Gradient!distributional] \label{def:Fund:DistrGradient} Let $Ω ⊂ ℝ^d$ be an open set and $T ∈ \dstr'(Ω)$ a distribution. Then, we say that the distributional gradient of $T$ is $∇T$ if for every test function $φ ∈ \dstr(Ω)$ we have \[ \pesc{∇T,φ} = \pesc{T, ∇φ} \]
\end{defn}

\begin{defn}[Distributional divergence][Divergence!distributional] \label{def:Fund:DistrDiver} Let $Ω ⊂ ℝ^d$ be an open set and $T ∈ \dstr'(Ω)$ a distribution. Then, we say that the distributional divergence of $T$ is $\dv T$ if for every test function $φ ∈ \dstr(Ω)$ we have \[ \pesc{\dv T,φ} = \pesc{T, \dv φ} \]
\end{defn}

\section{Sobolev spaces}

Even though we have been able to define derivatives for arbitrary functions, we have the problem that not every $L^p$ function has a $L^2$ distributional derivative. For example, the function\footnote{Reminder: given a set $A$, we define the indicator or characteristic function as $\ind_A(x) = \begin{cases} 1 & x ∈ A \\ 0 & x ∉ A \end{cases}$.} $\ind_{ℝ^+}$ has distributional derivative $δ_0$, which is not even a function.

This is the motivation to define Sobolev spaces, spaces in which we will be able to do distributional derivatives without problems.

\begin{defn}[Sobolev space][Space!Sobolev] Let $Ω ⊆ ℝ^n$ be an open set. We define the Sobolev space of order $k ∈ ℕ$ and $1 ≤ p < ∞$ as the space of $L^p$ functions whose distributional derivatives are again $L^p$ functions. That is, \( W^{k,p} (Ω) ≝ \set{ u ∈ L^p(Ω) \tq ∂^\vA u ∈ L^p(Ω) \; ∀\abs{\vA} ≤ k} \)

Sobolev spaces are Banach spaces given the norm \( \norm{f}_{W^{k,p}} = \left(\sum_{\abs{\vA} ≤ k} \norm{∂^\vA u}_{L^p}^p\right)^{\sfrac{1}{p}} \label{eq:Fundamentals:NormSobolev} \) with the special case for $p = ∞$ \( \norm{f}_{W^{k,∞}} = \sup_{\abs{\vA} ≤ h} \norm{∂^\vA u}_{L^∞} \label{eq:Fundamentals:NormSobolevInfty} \)
\end{defn}

The special case is with $p = 2$, named as $H^k(Ω) = W^{k,2}(Ω)$, which is a Hilbert space with the inner product given by \[ \pesc{f,g}_{H^k(Ω)} ≝ \sum_{\abs{\vA} ≤ k} \int_Ω ∂^\vA f · ∂^\vA g \dif x \] which clearly induces the norm defined previously in \eqref{eq:Fundamentals:NormSobolev}.

For notational convenience, we can define the \concept{Hilbert\IS seminorm} $\abs{·}_{H^k}$ as simply by \( \label{eq:HkSeminorm} \abs{f}_{H^k(Ω)}^2 = \sum_{\abs{\vA} = h} \norm{∂^\vA f}_{L^2(Ω)}^2 \), and thus \[ \norm{f}_{H^k(Ω)}^2 = \sum_{m=0}^k \abs{f}^2_{H^m(Ω)} \]

\subsection{Embeddings of Sobolev spaces and regularity}

One important aspect of Sobolev spaces is ``where do they live'' or, in other words, in which spaces can we embed a given Sobolev space and what does that tell us about the regularity of the functions in it. This is solved by the Sobolev embedding theorem.

\begin{theorem}[Sobolev embedding theorem][Sobolev!embedding theorem] \label{thm:SobolevEmbedding} Let $Ω$ be an open subset of $ℝ^d$ with a Lipschitz continuous boundary\footnote{A boundary is Lipschitz continuous if it is locally the graph of a function with \nref{def:Fund:LipschitzCont}.}. Then, the following embedding properties hold:
\begin{enumerate}
	\item If $0 ≤ 2k < d$, then $H^k(Ω) ⊂ L^q(Ω)$ with $q = \sfrac{2d}{(d-k)}$.
	\item If $2k = d$, then $H^k(Ω) ⊂ L^q$ for $2 ≤ q < ∞$.
	\item If $2(k-m) > d$, then $H^k(Ω) ⊂ C^m(\adh{Ω})$.
\end{enumerate}
\end{theorem}

The last property is the most interesting, as it tells us that regularity of a given function depends on the number of integrable derivatives and on the dimension in the space.

There are also two embeddings directly from the definition, say $H^{k+1}(Ω) ⊂ H^k(Ω)$ and $H^0(Ω) = L^2(Ω)$.

\subsection{Fractional Sobolev spaces and continuity}

Until now, we have defined Sobolev spaces with integer degree $k$. However, for the next section we will need what we call fractional Sobolev spaces. These will give us ``degrees'' of continuity: a Sobolev space of degree $4.5$ will be the space of functions that have 4 distributional derivatives in $L^p$ and for which the last derivative is ``somewhat continuous'', with that somewhat being graded by $0.5$. This will be later useful for defining trace spaces and calculating the value of a function in the boundary with certain regularity guarantees.

This ``somewhat continuity'' actually has a formal definition: Hölder continuity.

\begin{defn}[Hölder continuity][Continuity!Hölder] \label{def:Fund:HolderContinuity} Let $X,Y$ be two metric spaces and $\appl{f}{X}{Y}$ be a function between them. We say that $f$ is Hölder continuous of degree $α$ if there exists constants $C,α ∈ ℝ^+$ such that for every $x,y ∈ ℝ^d$ the following ``Hölder condition'' holds: \( \norm{f(x) - f(y)}_Y ≤ C\norm{x - y}_X^α \label{eq:Fundamentals:HolderCondition} \)

A function will be locally Hölder continuous if we can find those constants $C,α$ for a neighborhood of every point $x ∈ ℝ^d$.
\end{defn}

\subsubsection{Unrelated but interesting rambling on Hölder continuity and derivability}

Although for the definition we let the degree be $α ∈ ℝ^+$, we are only interested in the case $α ∈ (0,1)$. With $α = 0$, we only have bounded variation, which is not necessarily continuous at all. With any $α > 0$, we have that $f$ is continuous. Magic happens with $α = 1$, a case for which we will need an additional definition.

\begin{defn}[Lipschitz continuity][Continuity!Lipschitz] \label{def:Fund:LipschitzCont} Let $X,Y$ be two metric spaces and  $\appl{f}{X}{Y}$ a function between them. We say that $f$ is Lipschitz continuous in $X$ if there exists a constant $K ∈ ℝ$ such that, for every $x,y ∈ X$, we have \( \frac{\norm{f(x) - f(y)}_Y}{\norm{x-y}_X} ≤ K \label{eq:Fundamentals:LipschitzCondition}  \)

As it was the case with Hölder continuity, a function will be locally Lipschitz continuous if we can find the constant $K$ for a neighborhood of every point $x ∈ X$.
\end{defn}


It is straightforward to see that Hölder continuity of degree $1$ is equivalent to Lipschitz continuity. What is more interesting is the relationship of Lipschitz continuity and derivation.

\begin{theorem}[Rademacher's\IS Theorem] Let $\appl{f}{ℝ^m}{ℝ^n}$ be a Lipschitz continuous function. Then, it is differentiable almost everywhere (that is, the only points where it isn't differentiable have Lebesgue measure zero).
\end{theorem}

The proof is a little bit complex but uses measure theory so I'm writing it down only the part for one dimension. First of all, an important part which is the Riesz Representation Theorem, which gives us an equivalence between functionals and functions of the same Hilbert space.

\begin{theorem}[Riesz Representation Theorem] \label{thm:RieszRepresentation} Let $H$ be a Hilbert space, and let $H'$ be its topological dual space\footnote{Beware: the topological dual space is the space of all \textbf{continuous} linear functionals on $H$.}. Given $x ∈ H$ an element of the Hilbert space, we can define a continuous linear functional \[ φ_x(y) = \pesc{y,x} \;∀y ∈ H\]


The theorem states that \textbf{every} continuous linear functional has this representation as an scalar product. More broadly, the mapping $\appl{Φ}{H}{H'}$ defined as above is an isometric anti-isomorphism:
\begin{enumerate}
	\item Φ is bijective.
	\item Φ conserves norms: $\norm{Φ(x)} = \norm{x}$.
	\item Φ is antilinear: $\norm{Φ(x + y) } = Φ(x) + Φ(y)$ and $Φ(λx) = \conj{λ}Φ(x)$.
\end{enumerate}
\end{theorem}

We are also going to need the Lebesgue differentiation theorem. Just as an improved notation, we define the ``averaging integral'' by $\fint_A f \dif μ = \frac{1}{μ(A)}\int f \dif μ$ where $μ(A)$ is the measure of the set $A$.

\begin{theorem}[Lebesgue\IS differentiation theorem] Let $\appl{f}{ℝ^m}{R^n}$ be a Lebesgue integrable function. Then, for almost every $x ∈ ℝ^m$ we have that \( \lim_{r \to 0} \frac{\int_{\bola_r(x)} f \dif μ}{μ(\bola_r(x))} = \fint_{\bola_r(x)} f \dif μ  = f(x) \)
\end{theorem}

This theorem gives us a nice corollary, which is the fundamental theorem of calculus for Lebesgue integrals.

\begin{corol} \label{crl:LebesgueDifferentiation} Let $f$ be a Lebesgue integrable function, and let $F(x) = \int_{-∞}^x f \dif μ$. Then $F$ is differentiable almost everywhere with $F'(x) = f(x)$.
\end{corol}

With this, we can do the proof of Rademacher's theorem. We will only need one inequality:

\begin{prop}[Hölder inequality][Inequality!Hölder] \label{prop:HolderInequality} Let $f ∈ L^p$ and $g ∈ L^q$ with $p,q$ conjugate exponents (that is, $\frac{1}{p} + \frac{1}{q} = 1$). Then, $fg ∈ L^1$ and \[ \norm{fg}_1 ≤ \norm{f}_p \norm{g}_q \]
\end{prop}

\begin{proof}[Rademacher's Theorem] What we are going to prove is that $\appl{f}{[a,b]}{ℝ}$ can be expressed as $f(x) = \int_a^b g(x) \dif x$ with $g$ differentiable. To construct this function, we first define a continuous linear functional and will then use the \nref{thm:RieszRepresentation}. For simplicity, we can assume $[a,b] = [0,1]$.

First, recall the definition of simple functions. These are functions of the form $s(x) ≝ \sum α_i \ind_{[a_i, a_{i+1}]}$. We can define a linear functional $I$ on these functions given by \[ I(s) ≝ \sum α_i \left(f(a_{i+1}) - f(a_{i})\right) \]

Now we will prove that this functional is bounded and thus continuous (linearity is trivial):
\begin{align*}
\abs{I(s)} &=\abs{\sum α_i \left(f(a_{i+1}) - f(a_{1})\right)} \\
	&≤ \sum \abs{α_i} \abs{f(a_{i+1}) - f(a_{i})} \\
	&≤ K \sum \abs{α_i} \abs{a_{i+1} - a_{i}} \quad\text{ (Lipschitz continuity)} \\
	&= K \norm{s}_1 \\
	&≤ K \norm{\ind_{(0,1]}}_2 \norm{s}_2 \quad\text{ (Hölder inequality)} \\
	&= K \norm{s}_2
\end{align*}

As the simple functions are a dense subset of $L^2$, this means that $I$ is bounded for every $L^2$ function and thus it is a continuous linear functional on $L^2([0,1])$, and the same with $L^1$. Using the \nref{thm:RieszRepresentation}, there exists a function $φ ∈ L^2([0,1])$ such that $I(ψ) = \pesc{ψ, φ}$ for every $ψ ∈ L^2([0,1])$. As $I$ is also a linear functional on $L^1$, φ must be in $L^1$ too and thus it is Lebesgue integrable.

Finally, we just choose $ψ_x = \ind_{[0,x]}$ so we have \[ I(ψ_x) = f(x) - f(0) = \int_0^x φ \dif μ \], which is the assumption of \fref{crl:LebesgueDifferentiation} and thus we have differentiability almost everywhere.
\end{proof}

\subsubsection{Definition of fractional Sobolev spaces}

So, back on track. Previously, we have defined \nref{def:Fund:HolderContinuity} with $α ∈ (0,1)$. We will want to expand that to $L^p$ functions, and that is done via the Slobodecki seminorm.

\begin{defn}[Slobodecki seminorm][Seminorm!Slobodecki] Let $1 ≤ p < ∞$, $α ∈ (0,1)$ and $Ω ∈ ℝ^d$ an open subset. Then, for a given $f ∈ L^p(Ω)$ we can define its Slobodecki seminorm as
\(\label{eq:SlobodeckiSeminorm} [f]_{α,p,Ω} = \left(\iint\limits_{Ω×Ω} \frac{\abs{f(x) - f(y)}^p}{\abs{x-y}^{αp+d}} \dif x \dif y \right)^\frac{1}{p} \)
\end{defn}

This allows us to generalize the notion of ``stronger continuity'' to $L^p$ functions: in this case, we create a notion of ``stronger than $L^p$ integrability''\footnote{I think. I'm not sure about that}. With that, we can define a fractional Sobolev space.

\begin{defn}[Sobolev\IS fractional space] Given $1 ≤ p < ∞$, $s ∈ ℝ^+$ and $Ω ⊂ ℝ^d$ an open set, we define $θ ≝ s - \floor{s}$ and the corresponding Sobolev fractional space as \[ W^{s,p}(Ω) = \set{f ∈ W^{\floor{s}, p} \tq \sup_{\abs{\vA} = \floor{s}} [∂^\vA f]_{θ,p,Ω} < ∞}\] which is a Banach space when equipped with the norm \[ \norm{f}_{W^{s,p}(Ω)} = \norm{f}_{W^{\floor{s}, p}} + \sup_{\abs{\vA} = \floor{s}} [∂^\vA f]_{θ,p,Ω}\]
\end{defn}

\section{Trace spaces}

To finish with this chapter on fundamentals, we need to visit the trace spaces. Until now, we have studied spaces with not many restrictions on their functions, and in which we can take derivatives at liberty (Sobolev spaces) not worrying too much about regularity. This will allow us to study partial differential equations in a very general setting leaving regularity aside.

However, one aspect of PDEs that we haven't touched here is boundary conditions. Which is the regularity of functions defined in the boundary of a domain? What requirements should we impose on that boundary? These questions are answered by the trace operator, which is an easy way to deal with the restriction on the boundary.

\begin{theorem}[Trace\IS theorem] \label{thm:Fund:Trace} Let $Ω ⊂ ℝ^d$ be an open domain with a Lipschitz continuous boundary $∂Ω$, and $k ≥ 1$. Then, there exists an unique linear map (\textbf{trace operator}) \begin{align*}
\appl{γ_0}{H^k(Ω)&}{H^{k-\sfrac{1}{2}}(∂Ω)}
\end{align*} such that $γ_0(v) = \restr{v}{∂Ω}$ if $v ∈ H^k(Ω) ∩ C^0(\adh{Ω})$ (that is, the trace operator is the restriction on the boundary when the function is continuous in the closure).

Moreover, there is a constant $C_T > 0$ such that \[ \norm{γ_0(v)}_{L^2(∂Ω)} ≤ C_T \norm{v}_{H^k(Ω)} \]

This results holds if instead of $∂Ω$ we take a subset $Γ ⊂ ∂Ω$ with nonzero measure.
\end{theorem}

If the function does not happen to be continuous in the closure, the trace is assigned as a limit: $C^∞(Ω)$ is dense in $H^k(Ω)$, so for any $v ∈ H^k(Ω)$ we can find a sequence $\set{v_n} ⊂ C^∞(Ω)$ such that $v_n \to v$, and then we compute $γ_0(v) = \lim_{n \to ∞} γ_0(v_n)$.

Note that $H^{k-\sfrac{1}{2}}(Ω) ⊂ L^2(Ω)$.

A space we will usually be interested in will be $H^1_0(Ω)$, which is the set of $H^1$ functions that are null on the boundary: \[ H^1_0(Ω) = \set{ f ∈ H^1(Ω) \tq γ_0(f) = 0 }\]

Similarly, we can define $H^1_Γ(Ω)$ for partial boundaries $Γ ⊂ ∂Ω$ (with nonzero measure) as \[ H^1_Γ(Ω) = \set{f ∈ H^1(Ω) \tq γ_Γ(f) = 0} \]

\section{Poincaré inequalities}

Now, just for some inequalities that can be considerably useful in the future relating $L^2$ and $H^1$ seminorms.

\begin{theorem}[Poincaré inequality][Inequality!Poincaré] \label{thm:Fund:PoincareInequality} Let $Ω ⊂ ℝ^d$ be an open domain. Then there exists a positive constant $C_Ω > 0$ such that, for every $v ∈ H^1_0(Ω)$ the following holds: \[ \norm{v}_{L^2} ≤ C_Ω \abs{v}_{H^1}\]
\end{theorem}

The inequality still holds if the function vanishes only on part of the boundary.

\begin{prop}[Friedrichs inequality][Inequality!Friedrichs] The \nref{thm:Fund:PoincareInequality} still holds for $v ∈ H^1_Γ$, with $Γ ⊂ ∂Ω$ of nonzero measure.
\end{prop}

A corollary of the Poincaré inequality is the following reverse inequality

\begin{corol} \label{crl:Fund:PoincareReverse} Let $Ω ⊂ ℝ^d$ be an open domain. Then there exists a positive constant $C_p > 0$ such that, for every $v ∈ H^1_0(Ω)$ the following holds: \[ \norm{v}_{H^1} ≤ C_p \abs{v}_{H^1} \]
\end{corol}

\begin{proof} By simple computations, \[ \norm{v}_{H^1} = \norm{v}_{L^2} + \abs{v}_{H^1} ≤ (C_Ω + 1) \abs{v}_{H^1} \]
\end{proof}

And, by extension, equivalence of the full norm and seminorm in the Hilbert space.

\begin{corol} \label{crl:Fund:FullSeminormEquivalent} Let $Ω ⊂ ℝ^d$ be an open domain. Then the seminorm $\abs{·}_{H^1(Ω)}$ and the norm $\norm{·}_{H^1(Ω)}$ are equivalent.
\end{corol}

\begin{proof} By definition, $\abs{v}_{H^1(Ω)} < \norm{v}_{H^1(Ω)}$ and by \fref{crl:Fund:PoincareReverse} $\norm{v}_{H^1(Ω)} ≤ C_p \abs{v}_{H^1(Ω)}$.
\end{proof}

\section{Functional analysis theorems}

Some useful, abstract functional analysis theorems, without proofs.

\begin{theorem}[Closed range\IS theorem] \label{thm:Fund:ClosedRange} Let $X, Y$ be Banach spaces and let $D ⊂ X$ be dense in $X$. Let $\appl{T}{D}{Y}$ be a closed linear operator with transpose\footnote{$\appl{T'}{Y'}{D'}$ is the transpose of $T$ if and only if $\pesc{Tx, y'} = \pesc{x, T'y'}$ for all $x ∈ D$ and $y' ∈ Y'$.} $T'$ Then, the following claims are equivalent:

\begin{enumerate}
	\item The image of $T$ is closed in $Y$.
	\item The image of $T'$ is closed in $X'$.
	\item $\img T = \left(\ker T'\right)^{\perp}$.
	\item $\img T' = \left(\ker T\right)^{\perp}$.
\end{enumerate}
\end{theorem}

\begin{theorem}[Hahn-Banach\IS theorem] \label{thm:Fund:HahnBanach} Let $\appl{T}{V⊂X}{Y}$ a bounded linear operator with $X$ normed space and $Y$ Banach, such that $V$ is dense in $X$. Then, there is a unique bounded linear extension $\appl{\tilde{T}}{X}{Y}$ with $\norm{\tilde{T}} = \norm{T}$.
\end{theorem}



\part{Theory of finite elements}
% -*- root: ../NumericalApproximationofPDEs.tex -*-
\chapter{Introduction to the finite elements method}

For motivation, we will study the problem of solving numerically differential equations  such as \( \label{eq:FE:BasicOde} \begin{cases} Du = f & \text{ in } [a,b] ⊂ ℝ \\ u(a) = u_a \\ u(b) = u_b \end{cases} \) with $D$ some kind of differential operator. For example, with $D(u) = u' + u$.

Usually, the first approach to a numerical solution of these equations imply the use of finite differences, using an approximation of the derivatives such as \[ f'(x) \approx \frac{f(x + h) - f(x)}{h} \] which then converts the problem \eqref{eq:FE:BasicOde} to one of a linear system $Au = f$ of $N$ equations, one for each of the intervals of length $h$ in which $[a,b]$ gets divided. This, however, poses a problem in terms of the size of the system $\sfrac{b - a}{h}$ equations, and is also ill-conditioned when $h$ is too small.

We will instead search for another approach. We will construct a space $V_h$ of finite dimension, and search there for a solution $u_h$ that approximates $u$. Usually, $V_h$ will be a space of piecewise polynomial functions. These polynomials will be defined on a mesh \mesh defined by a set of $N$ points $\set{a = x_1 < x_2 < \dotsb < x_{N} = b}$, that we will call vertices.

Thus, the generic space will be \( X_h^r = \set{ f ∈ C^0([a,b]) \tq \restr{f}{I_j} ∈ \mathbb{P}_r(I_j) \quad ∀ I_j ∈ \mesh } \label{eq:FE:FiniteElementSpace} \), being $I_j$ each of the intervals of the mesh and $\mathbb{P}_r(I_j)$ the space of polynomials of degree up to $r$ defined in the interval $I_j$. Usually, we will work in a subset of this space to use only functions compatible with the boundary contitions.

Something that the attentive reader will have noticed by now is the fact that we didn't make functions in $X_h^r$ derivable. We just forced continuity, which is a weird thing given the fact that we are trying to solve differential equations, whose solutions by definition should have some degree of smoothness.

However, we will actually be interested in ``non-continuous'' ``solutions'' to the PDE problems, because those actually exist in physical settings. For example, a vibrating string may have a point force as initial condition, and that is not continuous. This is the motivation to introduce the weak formulation of a differential equation.

\section{Weak formulation}
\label{sec:Theory:WeakFormulation}

A physical motivation for the weak formulation is the introduction of ``virtual displacements'' but I have to admit that it is only intuitive if you are a physicist. The mathematical idea is to move to a setting where we can forgive some non-smoothness of the function. This framework is the integration of a product: multiplying by a test functions with minimal regularity and using integration by parts we will be able to move the derivatives from our solution to the test function and then require less regularity on the solution.

For a practical example, let our equation be $-u'' = f$. We will multiply by a function $v$ in some space that we will decide later, integrate in the domain $Ω$ of the solution and then apply integration by parts:
\begin{align*}
- u'' &= f \\
\int_Ω - u'' v &= \int_W fv \\
\int_Ω u' v' - \restr{u'v}{∂Ω} &=\int_Ω fv
\end{align*}

If we choose $v$ to be $0$ on the boundary $∂Ω$, we will eliminate the second term and will have ended up with \[ \int_Ω u' v' = \int_Ω fv\] which indeed admits solutions of class $C^1$, one less degree than we originally needed. This is a new formulation, that we will call the weak formulation.

\begin{defn}[Weak\IS formulation of an ODE problem] \label{def:Theory:WeakFormODE} Given a domain $Ω = [a,b] ⊂ ℝ$ and a function $f ∈ C^2(Ω)$, find $u ∈ H_0^1(Ω)$ such that \( \int_Ω u' v' = \int_Ω f v \label{eq:Theory:WeakFormODE} \) for every $v ∈ H_0^1(Ω)$.
\end{defn}

\subsection{Weak formulation on $ℝ^3$}
\label{sec:Theory:WeakFormulationPDE}

If we wanted to move to partial differential equations, things do not usually change much. For example, say we have an elliptic PDE of the form \(
\begin{cases}
-Δu = f & \text{ in } Ω \\
u =  0 & \text{ in } ∂Ω
\end{cases} \label{eq:PoissonProblem} \) for some open domain $Ω ∈ ℝ^d$ and some function $f ∈ C^2(Ω)$.

This equation models, for example, the distribution of heat in a domain, the concentration of a chemical in a fluid at rest, an electric potential in presence of distributed charges, the deformation of a membrane...

This problem can be solved in the strong form, where we approximate the second derivative using Taylor series and then try to find the solution (finite differences method). But we can also use the weak formulation, which is used for the finite elements method.

A physical interpretation is to think of the solutions as solutions that will zero-out the forces for every possible small displacement, that is, solving \[ \int_Ω (-Δu -f) v = 0 \] for all ``virtual displacement'' $\appl{v}{Ω}{ℝ}$ that is sufficiently smooth.

In order to have an improved form of that, we use integration by parts: \begin{multline*} \int_Ω - Δu v = -\int \dv (\grad u) v = \\ = - \int_Ω \dv(\grad u · v) + \int_Ω \grad u · \grad v = - \int_{∂Ω} \underbracket{(\grad u · \vn)}_{∂_\vn u} · v + \int_Ω \grad u · \grad v \end{multline*}

We have a small problem with the integration on the boundary. But if we think of the previous physical argument, if $v$ is a virtual displacement, we should have it constrained in the boundary, so $\restr{v}{∂Ω} = 0$, the boundary term disappears and our weak formulation is \[ \int_Ω \grad u · \grad v = \int_Ω f v \]

Another way to think of this is that is $v$ is any kind of test function, it must have compact support contained in $Ω$ and thus it must be zero on the boundary.

Once we have the formulation, we can study the regularity requirements for $v$. We would like to have bounded integrals, and by applying the \nref{prop:HolderInequality} we see that we need $\norm{\grad v}_2$ and $\norm{v}_2$ to be bounded. That, together with the restriction on the boundary, means that we need $v ∈ H_0^1(Ω)$ (see the introduction for the definition of this space). Thus, we can reformulate again the problem as follows:

\begin{defn}[Weak\IS formulation of a PDE problem] \label{def:Theory:WeakFormPDE} Given a domain $Ω⊂ℝ^d$ and a function $f ∈ C^2(Ω)$, find $u ∈ H_0^1(Ω)$ such that \( \int_Ω \grad u · \grad v = \int_Ω f v \label{eq:Theory:WeakFormPDE} \) for every $v ∈ H_0^1(Ω)$.
\end{defn}

\section{Galerkin approximation and finite element spaces}

Once we have the weak formulation by finding functions in some Hilbert space $V$ (in the previous examples, $V = H^1_0(Ω)$) we can define an approximate problem. We will construct a family of finite dimensional spaces $V_h ⊂ V$ with a proper approximation property (specifically, that $\inf_{v_h ∈ V_h} \norm{v_h - v} \convs[][h][0] 0$ for any $v ∈ V$) and search for a solution there. Thus, we will find a solution $v_h ∈ V_h$ such that \( a(u_h, v_h) = F(v_h) \quad ∀v_h ∈V_h \label{eq:FE:GalerkinApprox} \)

Recall that previously we defined these finite element spaces in \eqref{eq:FE:FiniteElementSpace}. The interesting thing of that kind of finite element spaces is the fact that we can define a very convenient basis for them. That is a problem that reduces to that of finding a basis of functions for the polynomials defined on one interval, and for that we can use the Lagrange polynomials.

\begin{defn}[Lagrangian basis\IS of a polynomial function space] Let $\mathbb{P}_r(I)$ be the space of polynomials of degree $r$ or less in an interval $I = [a, b] ⊂ ℝ$. Each polynomial is completely defined by its values at $r + 1$ points, so we define a set of nodes $\set{x_0 = a < x_1 < \dotsb < x_r = b}$. The basis $\set{φ_i}_{i = 0}^r$ of $\mathbb{P}_r(I)$ is then defined by the polynomials $φ_i$ of degree $r$ such that $φ_i(x_j) = δ_{ij}$ with $δ_{ij}$ being the Kronecker's delta.
\end{defn}

\begin{figure}[tp]
\centering
\inputtikz{LagrangianBasis}
\caption{Several examples of Lagrangian basis polynomials for degrees $1, 2$ and $3$ respectively.}
\label{fig:FE:LagrangianBasis}
\end{figure}

Why is this basis useful? Let's look again at the Galerkin approximation \eqref{eq:FE:GalerkinApprox}. Given that both $a$ and $F$ and linear, for the approximation to hold for any $v_h ∈ X_h^r$ we only need to check it for all the elements $\set{φ_i}_{i = 0}^n$ of the base of $X_h^r$ (any $v_h$ is a linear combination of these elements). That is, we need to ensure that \[ a(u_h, φ_i) = F(φ_i) \quad ∀i = 1, \dotsc, n \]

Again, using linearity of $a$ we can reformulate that as \[ \sum_{j = 0}^n u_j a(φ_j,φ_i) = F(φ_i) \quad ∀i = \dotsc, n\] where $u_j$ are the unknown coefficients of $u_j$ in the base $\set{φ_j}$. The problem is thus reduced to a linear system \[ \mA \vu = \vec{f} \] where the \concept{Stiffness\IS matrix} $\mA$ is defined by \[ \mA_{ij} = a(φ_i, φ_j)\] and analogously $\vec{f}_i = F(φ_i)$.

And although this approach can always be constructed for any finite element space, the advantage with the Lagrangian basis and polynimal finite spaces is that the coefficients are actually the values of $u$ at the nodes, which will make computations far easier.

\chapter{Theoretical analysis of the weak formulation}

\section{Definition of the problem and weak formulation}

Now on to the hard theory. We will start off a differential problem of finding a function $\appl{u}{Ω}{ℝ}$ for a given domain $Ω ⊂ ℝ^d$ such that \( \begin{cases} Du = f & \text{in Ω} \\ \text{boundary conditions} \end{cases} \) with $D$ an appropriate differential operator. The boundary conditions can be of three types:

\begin{defn}[Dirichlet boundary condition][Boundary condition!Dirichlet] \label{def:Theory:DirichletBoundary} A condition that fixes the value of $u$ in the boundary $∂Ω$.
\end{defn}

\begin{defn}[Neumann boundary condition][Boundary condition!Neumann] \label{def:Theory:NeumannBoundary} A restriction on the normal derivative of $u$. That is, assuming $\vn$ is normal to $∂Ω$, fixing the value $∂_\vn u = h$.
\end{defn}

\begin{defn}[Robin boundary condition][Boundary condition!Robin] \label{def:Theory:RobinBoundary} A weighted combination of Dirichlet and Neumann boundary conditions. That is, restricting $au + b∂_\vn u = g$ on $∂Ω$.
\end{defn}

Starting from this point, we can usually integrate by parts and obtain a bilinear form as described in \fref{sec:Theory:WeakFormulation}. To tackle these problems in a generic way, we can unify the definitions we saw of the \nlref{def:Theory:WeakFormODE} and the \nlref{def:Theory:WeakFormPDE} in one single weak abstract formulation, which will also allow us to work with different expressions other than $\int_Ω ∇u ∇v$, for example.

The idea is that in both cases, the integral of two functions is a bilinear operator from appropriate Hilbert spaces, and the right-hand side is a linear operator. Thus, our generic formulation is the following:

\begin{defn}[Weak\IS abstract formulation of a PDE problem] \label{def:Theory:WeakAbstractFormulation} Given a Hilbert space $V$, a bilinear form $\appl{a}{V×V}{ℝ}$ and a linear form $\appl{F}{V}{ℝ}$, find $u ∈ V$ such that \( a(u,v) = F(v)\quad ∀ v ∈ V \)
\end{defn}

Bounded operators are already defined (check \cite{ApuntesAnalisisFunc}). For bilinear forms, two definitions that we will use for the existence theorems.

\begin{defn}[Continous\IS bilinear form] Given $V$ a Hilbert space and $\appl{a}{V×V}{ℝ}$ a bilinear form, we say that it is continous (or bounded\footnote{As with linear forms, both notions are equivalent}) if and only if exists $M > 0$ such that \[ \abs{a(u,v)} ≤ M \norm{u}_V \norm{v}_V\] for all $u,v ∈ V$.
\end{defn}

\begin{defn}[Coercive\IS bilinear form][Coercivity] Given $V$ a Hilbert space and $\appl{a}{V×V}{ℝ}$ a bilinear form, we say that it is coercive if an inly if exists a constant $α > 0$ such that \[ a(u,u) ≥ α \norm{u}^2\]

As notation, α is called the coercivity coefficient
\end{defn}

\section{Lax-Milgram theorem}

With these conditions we can formulate the existence and uniqueness theorem, also called the Lax-Milgram theorem.

\begin{theorem}[Lax-Milgram\IS theorem] \label{thm:Theory:LaxMilgram} Given a Hilbert space $V$, a bilinear, continuous and coercive form $\appl{a}{V×V}{ℝ}$ and a linear, bounded operator $\appl{F}{V}{ℝ}$; the \nlref{def:Theory:WeakAbstractFormulation} has a unique solution $u ∈V$ such that \( \norm{u}_V ≤ \frac{1}{α} \norm{F}_{V^*} \label{eq:Theory:LaxMilgramBound} \) with α the coercivity constant of $a$.
\end{theorem}

\begin{proof} \citep[Theorem A.3]{larsson2008partial} For each $u ∈ V$, we can define the functional $\appl{a_u}{V}{ℝ}$ such that $a_u(v) = a(u,v)$. It is bounded by definition, so we can apply the \nref{thm:RieszRepresentation}: there exists an unique element $w ∈ V$ such that $a(u,v) = \pesc{w, v}$. In that case, let us write $a(u,v) = \pesc{Au, v}$ for $\appl{A}{V}{V}$ a certain operator on the Hilbert space. On the other hand, again by the Riesz representation theorem we have that there exists an unique $b ∈ V$ such that $F(v) = \pesc{b, v}$ for all $v ∈ V$. Thus, these are equivalent equations: \begin{align*}
a(u,v) &= F(v) \quad ∀v ∈ V\\
\pesc{Au, v} &= \pesc{b, v} \quad ∀v ∈ V\\
u &= \inv{A}b
\end{align*}

The problem of finding a solution is then reduced to that of showing that the inverse of the operator $A$ is well defined.

It is easy to see that $A$ must be be linear, and also bounded: \[ \norm{Au}^2 = \pesc{Au, Au} = a(u, Au) ≤ M \norm{u} \norm{Au} \implies \norm{Au} ≤ M \norm{u}\] by continuity of the bilinear form $a$. Also, using coercivity, we can see that \( α\norm{v}^2 ≤ a(v,v) = \pesc{Av, v} ≤ \norm{Av} \norm{v} \implies α \norm{v} ≤ \norm{Av} \label{eq:Theory:LaxMilgramPr:1} \) so that $Av = 0$ implies $v = 0$ and thus $A$ is injective. We only have to see that $A$ is surjective to prove existence and uniqueness of the solution $u$.

First, we will show that $R(A)$, the range of $A$, is a closed linear subspace of $V$. Let $\set{Av_j}_{j = 1}^∞ ⊂ R(A)$ be a sequence of elements converging to $w ∈ V$. Then, for any $i, j > 1$ and using \eqref{eq:Theory:LaxMilgramPr:1} we have that \[ \norm{v_j - v_i} ≤ \frac{1}{α} \norm{Av_j - Av_i} \convs[][i,j] 0 \] so the sequence $\set{v_j}$ converges to a certain element $v ∈ V$. By continuity, we then have that $A v_j \to Av = w$ and so $R(A)$ is closed.

Finally, assume that $R(A) ≠ V$. In that case, there must exist a non-null $w ∈ V$ such that $w$ is orthogonal to $R(A)$. But then, using orthogonality and coercivity, we would have that \[ α \norm{w}^2 ≤ a(w,w) = \pesc{Aw, w} \eqreasonup{$Aw ∈ R(A), R(A) \perp w$} 0\] which is a contradiction because we said $w ≠ 0$. Thus, $R(A) = V$, the operator $A$ has an inverse and the solution $u$ exists and is unique.

For the bound estimation \eqref{eq:Theory:LaxMilgramBound}, using coercivity we have that \[ α \norm{u}^2 ≤ a(u,u) = F(u) ≤ \norm{F} \norm{u} \implies \norm{u} ≤ \frac{1}{α} \norm{F}_{V^*} \]
\end{proof}

\chapter{Approximation of differential problems - Galerkin method}

During this chapter, we will assume that we have a \nlref{def:Theory:WeakAbstractFormulation} given by \[ a(u,v) = F(v) \quad ∀v ∈ V\] with $V$ a Hilbert space and $a, F$ under the conditions of the \nref{thm:Theory:LaxMilgram}, and we will study how to approximate that problem. The issue is obviously that $V$ tends to be an infinite-dimensional space, and that does not go well with computations and such. We will bypass that by using finite-dimensional approximate spaces.

\section{Galerkin approximation}

The idea of the Galerkin approximation is, as we said, limit the problem to a finite-dimensional Hilbert space on which we can solve computationally the problem.

Formally, we introduce a sequence of finite dimensional subspaces $V_h ⊂ V$ with $\dim V_h = N_h$ and with a certain approximability property: that for every $v ∈ V$ we can approximate it as well as we want: \[ \lim_{h\to 0} \inf_{v_h ∈ V_h} \norm{v-v_h}_V = 0 \]

\begin{defn}[Generalized Galerkin Formulation][Galerkin formulation!Generalized] \label{def:GalerkinFormulationGen} Given a \nlref{def:Theory:WeakAbstractFormulation}, we can reformulate it as finding $u_h ∈ V_h$ such that \[ a_h(u_h, v_h) = F_h(v_h) \quad ∀v_h ∈ V_h \] with the requirement that $a_h \convs[][h] 0$, $F_h \convs[][h][0] 0$ for some definition of convergence of functionals that we will see latters.
\end{defn}

However, the restriction that $V_h ⊂ V$ can be sometimes too strict. A \concept[Galerkin formulation!Non-conforming]{Non-conforming Galerkin Formulation} is one with $V_h\nsubseteq V$. That takes it to the Petrov-Galerkin approximation in which we allow different spaces for the solution and for the test functions.

\begin{defn}[Petrov-Galerkin approximation] Find $u_h ∈V_h$ such that \[ a_h(u_h, v_h) = F(v_h) \quad ∀v_h ∈ W_h \]
\end{defn}

Now to the interesting part: are these problems well-posed? Do they have unique solutions? And, more importantly, do these solutions converge to the actual solution? We will discuss it now.

\begin{prop} The formulation of a PDE problem as a \nref{def:GalerkinFormulationGen} has an unique solution.
\end{prop}

\begin{proof}
The idea here will be to apply the \nref{thm:Theory:LaxMilgram}. We know that $(V_h, \norm{·}_V)$ is a Hilbert space, $a(·,·)$ is continuous in $V_h$ because it is continuous in $V$, so $\abs{a(u,v)} ≤ M \norm{u}_V \norm{v}_V\; ∀v∈V_h ⊂ V$, and is coercive with at least the same coefficient because of the same reason. Same happens for $F$, so we are in the conditions of the lemma and thus the problem has an unique solution.
\end{proof}

Knowing that there are optimal and unique solutions, we can go on to an algebraic formulation. Let $u_h = \sum_j u_j φ_j$, $v_h = \sum_i v_i φ_i$ with $\set{φ_i}_{i=1}^{N_h}$ a basis of $V_h$. Then, the problem becomes
\begin{align*}
a(u_h,v_h) &= F(v_h) \\
a\left(\sum_j u_j φ_j, \sum_i v_i φ_i\right) &= F\left(\sum_i v_i φ_i\right) \\
\sum_{j,i} u_j v_i \underbracket{a(φ_j, φ_i)}_{A_{ij}} &= \sum v_i \underbracket{F(φ_i)}_{F_i}
\end{align*}

We can write this in a matrix form. If $\vu = (u_1, \dotsc, u_{N_h})$ and $\vv = (v_1, \dotsc, v_{N_h})$ then the equation becomes \begin{align*}
\trans{\vv}A \vu &= \trans{\vv} \vf \quad ∀\vv ∈ ℝ^{N_h} \\
A \vu &= \vf
\end{align*} which is a system of linear equations that can be solved in a purely algebraic manner. There are some nice properties of the matrix $A$ that come from the problem statement. It is positive definite because $a$ is positive.

\section{Convergence analysis of the Galerkin method}
\label{sec:Theory:ConvergenceGalerkin}

Now, on to the quality of the approximation. We will want to ensure that the approximate solution converges in some manner. If our bilinear form is symmetric, we can use Galerkin orthogonality:

\begin{defn}[Galerkin orthogonality][Orthogonality!Galerkin] Given a bilinear form $a$, we say that a solution $u$ and its approximation $u_h$ fulfill the Galerkin orthogonality condition if and only if \[ a(u - u_h, v_h)= 0 \quad v_h ∈ V_h \]

Given that a coercive symmetric bilinear form defines an inner product, Galerkin orthogonality is equivalent to saying that $u - u_h \perp V_h$.
\end{defn}

The usefulness of this property is the fact that, if $u - u_h$ is orthogonal to $V_h$, then we have an \concept{Optimality property} given by \( \norm{u - u_h}_a ≤ \inf_{v_h ∈ V_h} \norm{u - v_h}_a \label{eq:Elliptic:Optimality} \) with $\norm{·}_a = \sqrt{a(·,·)}$ the norm induced by the inner product $a(·,·)$.

For the general case, we have the following lemma for the optimality result, which will tell us that the approximation to the solution is bounded by how well can we approximate any function with our finite spaces.

\begin{lemma}[Ceà's\IS lemma] \label{lem:Theory:Cea} Given a \nref{def:GalerkinFormulationGen}, the following optimality result holds: \[ \norm{u - u_h}_V ≤ \frac{M}{α} \inf_{v_h ∈ V_h} \norm{u - v_h}_V \] for $M ∈ ℝ^+$ and $α$ the coercivity coefficient of $a$.
\end{lemma}

\begin{proof} Using the coercivity property, we know that \[ \norm{u - u_h}^2_V ≤ \frac{1}{α} a(u - u_h, u -u_h)\]

Adding and substracting $v_h ∈ V_h$, we can continue so \begin{align*}
\norm{u - u_h}^2_V &≤ \frac{1}{α} a(u - u_h, u -u_h + v_h - v_h) \\
	&= \frac{1}{α} a(u - u_h) + \frac{1}{α} \underbracket{a(u - u_h, v_h - u_h)}_{=0\;\text{(Galerkin Orthogonality)}} \\
\norm{u - u_h}^2_V &≤ \frac{M}{α} \norm{u - u_h}_V\norm{u - v_h}_V \\
\norm{u - u_h}_V &≤ \frac{M}{α} \inf_{v_h ∈ V_h} \norm{u - v_h}_V
\end{align*}
\end{proof}

\section{Finite element spaces}

Several properties of the approximation will depend on how good our finite element spaces are, and so we will define and construct these spaces along these sections.

\begin{defn}[Finite elements space] A finite element space is a space of piecewise polynomial functions over a partition of a domain $Ω ∈ ℝ^N$ into non-overlapping polyhedra, called a mesh.
\end{defn}

\begin{defn}[Polyhedral mesh] A polyhedral mesh $\mesh$ on $Ω ∈ ℝ^N$ is the union of a finite number of polyhedra $K_j$, $j = 1, \dotsc, N_k$ such that $\bigcup_{j=1}^{N_k} \adh{K_j} = \adh{Ω}$ and $\intr{K}_j ∩ \intr{K}_i = ∅$ if $i ≠ j$.
\end{defn}

Usually, in 2D the polyhedrae used are either triangles or squares. In 3D, we have tetrahedron, cubes or rectangular pyramids.

\begin{defn}[Geometrical conformal mesh] A geometrical conformal mesh is a mesh for which if $\adh{K_i} ∩ \adh{K_j} ≠ ∅$ then the intersection is either a common vertex or a common edge or face. That means that half-edge intersections are not allowed in this case.
\end{defn}

A mesh can be defined by certain parameters.

\begin{itemize}
	\item \concept[Mesh!diameter]{Element diameter} of an element $K ∈ \mesh$ as $h_K = \max_{x,y∈K} \abs{x-y}$.
	\item \concept[Mesh!inner diameter]{Element inner diameter} $ρ_K$  as the diameter of the largest ball in $K$.
	\item \concept{Mesh\IS size} $h = \max_{K∈\mesh} h_K$ which gives an idea of the size of the largest element of the mesh.
	\item \concept[Mesh!aspect ratio]{Aspect ratio} $γ_K = \sfrac{h_K}{ρ_K}$. A high aspect ratio indicates elongated elements, while a ratio near to one indicates more regular elements.
	\item \concept[Mesh!minimum size]{Mesh minimum size} $h_{min} = \min_{K ∈ τ_k} h_K$.
\end{itemize}

We will be interested in a regular family of meshes, that is, some family with a bounded aspect ratio.

\begin{defn}[Regular family of meshes][Mesh!regular] \label{def:Theory:RegularFamilyMeshes} A family of meshes $\set{\mesh}_{h \to 0}$ is said to be regular if the maximum aspect ratio is bounded by some constant $γ$ for all $h$: $\max_{K ∈ \mesh} γ_K ≤ γ ∈ ℝ$. In other words, force that for all $h > 0$ and $∀K ∈ \mesh$ $h_K ≤ γρ_K$.
\end{defn}

\begin{defn}[Quasi-uniform family of meshes][Mesh!quasi-uniform] A family of meshes is quasi-uniform if it is regular and $\sfrac{h_{min}}{h} ≥ δ$.
\end{defn}

In order to be able to do calculations, we will be interested specifically on affine meshes with a reference element.

\begin{defn}[Affine mesh][Mesh!affine] \label{def:Theory:AffineMesh} An affine mesh is a mesh for which each element $K$ can be mapped onto a certain reference element $\hat{K}$ by an affine transformation
\begin{align*}
\appl{F_K}{\hat{K}&}{K} \\
\hat{x} & \longmapsto \mB_K \hat{x} + \vb_K
\end{align*} where $\mB_K ∈ ℝ^{d×d}$ is some matrix and $\vb_K ∈ ℝ^d$ a vector, both such that $K = F_K(\hat{K})$.
\end{defn}

The reference element will be the ``base'' of this mesh. If we are using triangular meshes, $\hat{K}$ will be the triangle of vertices $(0,0), (1,0), (0,1)$, for quadrilateral meshes we have $\hat{K} = [0,1]^2$ and so on.

\subsection{Affine triangular meshes and mapping to the reference element}
\label{sec:Theory:ReferenceElement}

The easiest mesh is an affine triangular mesh, which is usually enough for the applications we will see on these courses. These are meshes where the reference element is a simplex\footnote{If I'm not mistaken and remember well that a simplex in 2D is a triangle.}

The construction of the mapping is considerably easy. Given $K ⊂ ℝ^d$, a simplex defined by $d + 1$ vectors $\set{\va_i}_{i = 1}^{d + 1}$, the mapping is given by the following vector and matrix: \( \vb_K = \va_1 \qquad \mB_K = \begin{pmatrix} \va_{2} & \cdots & \va_{d + 1} \end{pmatrix} \)

A special thing about this mapping is that we have several properties that will help us in later proofs.

\begin{lemma}[Properties of the mapping to the reference element][Reference element!mapping properties] \label{lem:Theory:PropertiesRefMapping} Let $h_K$ and $ρ_K$ be the outer and inner diameters of $K$ and $\hat{h}, \hat{ρ}$ the diameters of $\hat{K}$. Then \( \norm{\mB_K} ≤ \frac{h_K}{\hat{ρ}} \quad \norm{\inv{\mB_K}} ≤ \frac{\hat{h}}{ρ_K} \quad \det \mB_K = \frac{\abs{K}}{\abs{\hat{K}}} \label{eq:Theory:PropertiesRefMapping} \) where $\norm{\mB_K} = \sup_{ξ ∈ ℝ^d, ξ ≠ 0} \frac{\norm{\mB_Kξ}}{\norm{ξ}}$ is the spectral norm of $\mB_K$.
\end{lemma}

The proof is linear algebra and I don't want to.

\subsection{Base functions for finite element spaces}

\section{Approximation results for finite element spaces}
\label{sec:Theory:ApproxResults}

Once we have the spaces defined, we will be interested in bounding the approximation errors made, more specifically with continuous triangular finite elements on affine meshes: \[ x_h^r = \set{v ∈ C^0(Ω) \tq \restr{v}{K} ∈ \mathbb{P}_r(K) \; ∀K ∈ \mesh } \]

That is, given a function $u ∈ H^s(Ω)$ we would like to quantify the best approximation error in $H^1$ and $L^2$ norms, given by \[ \inf_{v_h ∈ X_h^r} \norm{u - v_h} \] with the norm of the corresponding space. The typical procedure for obtaining these estimates will be the building of a particular function $v_h ∈ X_h^r$ starting from $u$, and this particular function will usually be the interpolant operator $I_h^r$.

In order to prove the interpolation theorems, we will first develop local approximation estimates, that will tell us how well can we approximate any given function by a polynomial in a certain element of the mesh (\fref{sec:Theory:LocalApproxEstimates}). Then, we will see how well the interpolation operator approximates any given function locally (\fref{sec:Theory:LocalInterpEstimates}), and finally we will move to global interpolation estimates (\fref{sec:Theory:GlobalInterpEstimates}) and the theorems (\fref{sec:Theory:InterpTheorems}).

\subsection{Local approximation estimates}
\label{sec:Theory:LocalApproxEstimates}

As discussed, we will first understand the local approximation properties of the finite element space on a single element of the mesh $K ∈ \mesh$; that is, proving bounds on the approximation of a given function $v ∈ H^s(K)$ by polynomials of degree $r$.

\begin{lemma}[Deny-Lions\IS lemma] \label{lem:Theory:DenyLions} Given any bounded convex Lipschitz domain $K ⊂ ℝ^d$ and $s ≥ 0$, let $η = \min \set{s, r + 1}$. Then there exists a constant $C_{DL} > 0$ such that \( ∀v ∈ H^s(K) \quad \inf_{p ∈ \pspace (K)} \norm{v - p}_{H^m(K)} ≤ C_{DL} \abs{v}_{H^η}\; m = 0, 1, \dotsc, η \label{eq:Theory:DenyLions} \) with the constant depending on $h_K$, $ρ_K$, $m$, $s$ and $d$.

Moreover, asymptotically as $h_K \to 0$, the constant scales\footnote{Remember that $γ_K = \sfrac{h_K}{ρ_K}$} as \( \label{eq:Theory:DenyLionsScaling} C_{DL} \sim h_K^{n-m} γ_K^m \)
\end{lemma}

\begin{proof} There's a boring proof (constructive in 1D using Taylor expansion\footnote{I am seriously starting to hate a lot Taylor expansions.}) and an interesting one % http://math.stackexchange.com/questions/1374822/deny-lions-lemma
\end{proof}

The key point of this lemma is that the best approximation error is related\footnote{Again, a recall that the seminorm is defined as $\abs{v}^2_{H^{r}} = \sum_{\abs{\vA} = r} \norm{\Dif^\vA v}^2_{L^2}$.} to the derivatives of $v$ of order $r + 1$ if $v$ is smooth enough, or else it's related to the highest possible derivatives of $v$.

\subsection{Local interpolation estimates}
\label{sec:Theory:LocalInterpEstimates}

Once we know bounds for the approximation of any given function, we are interested in setting bounds for our interpolation operator $I_h^r$ which, given a function $u$, returns a polynomial of degree $r$. Let us denote by $v_K$ and $I_{h,K}^rv$ the restrictions of $v$ and $I_h^rv$ on $K$, respectively. Our aim is estimating $\abs{v_K - I_{h,K}^rv}_{H^m(K)}$.

However, we do not exactly know how to construct the interpolation polynomial in an arbitrary mesh element $K$, so we will work on understanding how things change when we map the element $K$ onto the reference element $\hat{K}$ via the mapping $F_K$ as defined in \fref{sec:Theory:ReferenceElement}.

Let $\hat{v}_K = v_K ○ F_K$ and $\hat{I}_{h,K}^r = I_{h,K}^rv ○ F_K$. Note that
\[ \hat{I}_{h,K}^r  = I_{h,K}^rv ○ F_K = \sum_{i=1}^{N_r} v(\va_{i,K}) · \left(\restr{φ_i}{K} ○ F_K\right) = \sum_{i=1}^{N_r} \hat{v}_K(\hat{\va}_i) \hat{φ}_i = I_{\hat{K}}^r \hat{v}_K \]
where $\set{\va_{i,K}}_{i = 1}^{N_r}$ is the set of nodes defining the degrees of freedom on $K$ and $\restr{φ_i}{K}$ the corresponding Lagrangian basis functions restricted to $K$, and the same for $\hat{\va}_i$ and $\hat{φ}_i$.

First result we want is to see how does the seminorm change when it's mapped.

\begin{lemma}[Seminorm transformation][] \label{lem:Theory:SeminormTrans} For any $v ∈ H^m(K)$ with $M ≥ 0$, let $\hat{v} = v ○ F_k$. Then, $\hat{v} ∈ H^m(\hat{K})$ and there exists a constant $C_{sn} > 0$ depending on $m$ such that \begin{align}
\abs{v}_{H^m(K)} &≤ C_{sn} \norm{\mB_K^{-1}}^m · \abs{\det \mB_K}^{\sfrac{1}{2}} · \abs{\hat{v}}_{H^m(\hat{K})} \\
\abs{\hat{v}}_{H^m(K)} &≤ \hat{C}_{sn} \norm{\mB_K}^m · \abs{\det \mB_K}^{-\sfrac{1}{2}} · \abs{v}_{H^m(K)}
\end{align}

Moreover, $C_{sn} = \hat{C}_{sn} = 1$ for $m = 0,1$.
\end{lemma}

\begin{proof} Proof given only for $m = 0,1$. For $m = 0$ we have that with just a change of variable\[ \norm{v}^2_{L^2(K)} = \int_{K} v^2 = \int_{\hat{K}} \hat{v}^2 \abs{\det \mB_K} = \abs{\det \mB_K} \norm{\hat{v}}^2_{L^2(\hat{K})} \] and the other equality can be proved analogously.

For $m = 1$ it is easy to see that $∇\hat{v} = \trans{\mB_K} ∇v$ so again we can integrate, do a change of variable and have it.
\end{proof}

Our next lemma is proving that the interpolant is a continuous operator, that is, small changes on the function being interpolated give small changes on the result.

\begin{lemma}[Continuity of interpolant operator][Interpolant operator!continuity] \label{lem:Theory:ContInterpolant} Let $\appl{I_{\hat{K}}^r}{C^0(\hat{K})}{\pspace (\hat{K})}$ be the finite element interpolant operator on the reference element $\hat{K} ⊂ ℝ^d$. Then, for $d ≤ 3$, $I_{\hat{K}}^4$ is a linear bounded operator from $H^2(\hat{K})$ to any $H^m(\hat{K})$ with $0 ≤ m ≤ r + 1$. That is, there exists $C_{I,m} > 0$ such that \( \norm{I_{\hat{K}}^r \hat{v}}_{H^m(\hat{K})} ≤ C_{I,m} \norm{v}_{H^2(\hat{K})} \qquad ∀\hat{v} ∈ H^2(\hat{K}) \)
\end{lemma}

\begin{proof} Just do operations with the left hand side and it works because the basis elements have bounded norm, and as the embedding $H^2 \hookrightarrow C^0$ is continuous for $d ≤ 3$, everything works.
\end{proof}

Finally, the proof that the interpolant is exact on polynomials.

\begin{lemma}[Exactness of the interpolant operator][Interpolant operator!exactness] \label{lem:Theory:ExactnessInterpolant} The interpolant operator $\appl{I_{\hat{K}}^r}{C^0(\hat{K})}{\pspace (\hat{K})}$ is exact on $\pspace (\hat{K})$, that is, $I_{\hat{K}}^r p = p$ for any $p ∈ \pspace (\hat{K})$.
\end{lemma}

\begin{proof} Yes.
\end{proof}

With this, we can finally prove the local error estimate.

\begin{lemma}[Local error estimate][Interpolant operator!global estimates] \label{lem:Theory:LocalError} Let $K ∈ \mesh$ be an element of the mesh with outer diameter $h_K$ and inner diameter $ρ_K$. Then for $s ≥ 2$ and any $0 ≤ m ≤ η ≝ \min\set{s, r+1}$ there exists a constant $C_l > 0$ depending on $s,r,m$ and the reference element $\hat{K}$ such that \( \abs{v - I_{h,K}^r v}_{H^m(K)} ≤ C_l \left(\frac{h_K}{ρ_K}\right)^m h_{K}^{η - m} \norm{v}_{H^η(K)} \label{eq:Theory:LocalErrorEstimate} \)
\end{lemma}

\begin{proof} For a first estimate, we use in sequence \nlref{lem:Theory:SeminormTrans}, \nlref{lem:Theory:ExactnessInterpolant} and the \nlref{lem:Theory:ContInterpolant}. Also, we set $\hat{p} ∈ \pspace (\hat{K})$ an arbitrary polynomial. Operating, we have
\begin{align*}
\abs{v - I_{h,K}^r v}_{H^m(K)}
	&≤ C_{sn} \norm{\mB_K^{-1}}^m · \abs{\det \mB_K}^{\sfrac{1}{2}} · \abs{\hat{v} - I_{\hat{K}}^r \hat{v}}_{H^m(\hat{K})} \\
	&≤ C_{sn} \norm{\mB_K^{-1}}^m · \abs{\det \mB_K}^{\sfrac{1}{2}} · \left(\abs{\hat{v} - \hat{p}}_{H^m(\hat{K})} +\abs{\hat{p} - I_{\hat{K}}^r \hat{v}}_{H^m(\hat{K})}\right) \\
	&≤ C_{sn} \norm{\mB_K^{-1}}^m · \abs{\det \mB_K}^{\sfrac{1}{2}} · \left(\abs{\hat{v} - \hat{p}}_{H^m(\hat{K})} +\abs{I_{\hat{K}}^r(\hat{p} - \hat{v})}_{H^m(\hat{K})}\right) \\
	&≤ C_{sn} \norm{\mB_K^{-1}}^m · \abs{\det \mB_K}^{\sfrac{1}{2}} · \left(\abs{\hat{v} - \hat{p}}_{H^m(\hat{K})} + C_{I,m} \norm{\hat{p} - \hat{v}}_{H^2(\hat{K})} \right) \\
	&≤ C_{sn} (1 + C_{I, m}) \norm{\mB_K^{-1}}^m · \abs{\det \mB_K}^{\sfrac{1}{2}} · \norm{\hat{v} - \hat{p}}_{H^{\max\set{m, 2}} (\hat{K})}
\end{align*}

And since $\hat{p}$ is arbitrary, the equality holds for any $\hat{p} ∈ \pspace(\hat{K})$ so that \[ \abs{v - I_{h,K}^r v}_{H^m(K)} ≤ C_{sn} (1 + C_{I, m}) \norm{\mB_K^{-1}}^m · \abs{\det \mB_K}^{\sfrac{1}{2}} · \inf_{\hat{p} ∈ \pspace(\hat{K})} \norm{\hat{v} - \hat{p}}_{H^{\max\set{m, 2}} (\hat{K})} \]

Now we can use the local approximation estimate from \nlref{lem:Theory:DenyLions} and again the \nlref{lem:Theory:SeminormTrans} to obtain \begin{align*}
\abs{v - I_{h,K}^r v}_{H^m(K)}
	&≤ C_{sn}(1 + C_{I,m}) C_{DL} \norm{\mB_K^{-1}}^m · \abs{\det \mB_K}^{\sfrac{1}{2}} · \abs{\hat{v}}_{H^η(\hat{K})} \\
	&≤ C_{sn}\hat{C}_{sn} (1 + C_{I,m}) C_{DL} \norm{\mB_K^{-1}}^m · \norm{\mB_K}^η· \abs{v}_{H^η(\hat{K})} \\
	&≤ C_l \left(\frac{h_K}{ρ_K}\right)^m h_{K}^{η - m} \abs{v}_{H^η(K)}
\end{align*}
\end{proof}

\subsection{Global interpolation estimates}
\label{sec:Theory:GlobalInterpEstimates}

Almost last section: estimates for the global error.

\begin{theorem}[Global interpolation estimates][Interpolant operator!global estimates] \label{thm:Theory:GlobalInterpEstimates} Given a family of regular triangulations $\set{\mesh}_{h > 0}$ of a polygonal domain $Ω ⊂ ℝ^d$ with $d ≤ 3$ and the space $X_h^r$ of continuous finite elements of degree $r$, for $s ≥ 2$ and $0 ≤ m ≤ η ≝ \min \set{s, r+1}$, it holds the following estimate \( \norm{v - I_h^rv}_{H^m(Ω)} ≤ C_l γ^m \left(\sum_{K ∈ \mesh} h^{2(η - m)}_K \abs{v}^2_{H^η(K)}\right)^{\sfrac{1}{2}} \) for any $v ∈ H^s(Ω)$, where $γ = \max_{K ∈ \mesh}γ_K$.
\end{theorem}

\begin{proof}
Calculations based on \fref{lem:Theory:LocalError}.
\end{proof}

\subsection{Interpolation error theorems}
\label{sec:Theory:InterpTheorems}

We finish them with the theorems of the interpolation error. These theorems are actually weaker than \fref{thm:Theory:GlobalInterpEstimates} but they are more convenient. However, the advantage of this stronger theorem is the representation of the interpolation error as the sum of local contributions for each element of the mesh, which can lead to results with mesh adaptivity.

\begin{theorem}[Interpolation error\IS for smooth functions] Given a \nlref{def:Theory:RegularFamilyMeshes} $\set{\mesh}_{h > 0}$ of a polygonal domain $Ω ⊂ ℝ^d$, with $d ≤ 3$, and a space $X_h^r$ of continuous finite elements of degree $r ≥ 1$, there exist $C_m > 0$ and $m = 0, \dotsc, r$ such that for any function $u ∈ H^s$ with $s ≥ r + 1$ we have the following estimates:
\begin{align}
\norm{u - I_h^r u}_{L^2(Ω)} &≤ C_0 h^{r + 1} \abs{u}_{H^{r+1}(Ω)} \label{eq:Theory:InterpSmoothL2} \\
\norm{u - I_h^ru}_{H^m(Ω)} &≤ C_m h^{r + 1 - m} \abs{u}_{H^{r+1}(Ω)} \label{eq:Theory:InterpSmoothHm}
\end{align} with $C_m$ that depends on the aspect ratio $γ$ of the mesh family, $r$ and $m$, but otherwise independent of $h$.
\end{theorem}

\begin{theorem}[Interpolation error\IS for possibly non-smooth functions] \label{thm:Theory:InterpErrorNonSmooth} Given a \nlref{def:Theory:RegularFamilyMeshes} $\set{\mesh}_{h > 0}$ of a polygonal domain $Ω ⊂ ℝ^d$, with $d ≤ 3$, and a space $X_h^r$ of continuous finite elements of degree $r ≥ 1$, for $s ≥ 2$ and with $η ≝ \min \set{s, r + 1}$ there exists $C_m > 0$ such that for any function $u ∈ H^s$ with $s ≥ r + 1$ we have the following estimates:
\begin{align}
\norm{u - I_h^ru}_{H^m(Ω)} &≤ C_m h^{η -m} \abs{u}_{H^{η}(Ω)} \label{eq:Theory:InterpSmoothHm}
\end{align} for $0 ≤ m ≤ η$ and with $C_m$ that depends on the aspect ratio $γ$ of the mesh family, $r$ and $m$, but otherwise independent of $h$.
\end{theorem}

\section{Non-conforming Galerkin method}

Some times we will need a generalized Galerkin method in which something different happens. This is the Strang lemma, for which we need an additional definition.

\begin{defn}[Uniform coercivity][Coercivity\IS uniform] Let $\set{V_h}$ be a family of subspaces of $V$, dense in $V$. Then, a bilinear form $a_h$ is uniformly coercive on $\set{V_h}$ if there exists a constant $α^* > 0$ independent of $h$ such that \[ a_h(v_h, v_h) ≥ α^* \norm{v_h}^2_{V}\quad ∀h > 0,\, v_h ∈ V_h \]
\end{defn}

Then the lemma commes.

\begin{lemma}[Strang's lemma][Lemma!Strang] Consider the problem of finding $u ∈V$ such that \[ a(u,v) = F(v) \quad ∀ v ∈ V\] where $V$ is a Hilbert space, $F ∈ V'$ is a linear and continuous functional, and $\appl{a}{V×V}{ℝ}$ is a continuous (constant $M$) and coercive bilinear form. Consider its associated generalized Galerkin problem given by finding $u_h ∈V_h$ such that \[ a_h(u_h, v_h) = F_h(v_h) \quad ∀v_h ∈ V_h\] where $\set{V_h}_{h > 0}$ is a family of subspaces of $V$ and dense in $V$.

If $a_h$ is uniformly coercive on $V_h$ with constant $α^*$, we have the following results:
\begin{enumerate}
	\item The generalized Galerkin problem is well posed with unique solution $u_h$.
	\item There is an a-priori estimation of the solution given by \[ \norm{u_h}_V ≤ \frac{1}{α^*} \norm{F_h}_{V_h'} \]
	\item The approximation error is bound by \(
	\label{eq:Theory:StrangBound} \begin{aligned}
	\norm{u -u_h}_V ≤& \inf_{w_h ∈ V_h} \left[\left(1 + \frac{M}{α^*} \right)\norm{u - w_h}_V \right. \\
	& \qquad \left.+ \frac{1}{α^*} \sup_{\substack{v_h ∈ V_h \\ v_h ≠ 0}} \frac{\abs{a(w_h, v_h) - a(w_h, v_h)}}{\norm{v_h}_V}\right] \\
	& + \frac{1}{α^*} \sup_{\substack{v_h ∈ V_h \\ v_h ≠ 0}} \frac{\abs{F(v_h) - F_h(v_h)}}{\norm{v_h}_V}
	\end{aligned}\)
\end{enumerate}
\end{lemma}


\part{Numerical approximation of ODEs}
% -*- root: ../NumericalApproximationofPDEs.tex -*-

\fi

\part{Numerical approximation of PDEs}

\chapter{Elliptic problems}

\section{Poisson problem (or Laplace equation)}
\ifincludefirstsemester% -*- root: ../NumericalApproximationofPDEs.tex -*-
We start by considering a simple Poisson equation, given $Ω ⊂ ℝ^d$ a bounded open domain with Lipschitz boundary $∂Ω$. There, we set the problem \( \begin{cases}
-Δu = f & \text{ in } Ω \\
u =  0 & \text{ in } ∂Ω
\end{cases} \label{eq:PDE:PoissonProblem} \)

This problem can be seen as solving the behaviour of an elastic membrane of shape $Ω$, with $u$ being the height of the membrane, or as a heat diffusion equation with $u$ the temperature and $f$ the sources/sinks of heat.

In \fref{sec:Theory:WeakFormulationPDE} we saw that the weak abstract form was  \[ \int_Ω \grad u · \grad v = \int_Ω f v \]

It is easy to see that we can apply the \nref{thm:Theory:LaxMilgram} to our problem: if $v ∈ H^1_0(Ω)$, then its $L^2$-norm is bounded so \[ \abs{\int_Ω fv} ≤ \norm{f}_{L^2} \norm{v}_{L^2} ≤ \norm{f}_{L^2} \norm{v}_{H^1} < ∞ \]

Same happens with $a$: the boundedness is directly obtained from the fact that $u, v ∈ H^1_0(Ω)$. Coercivity can be a little bit more complicated. For that, we must use the \nref{thm:Fund:PoincareInequality} which tells us that there is a constant $C_p > 0$ such that $\int v^2 ≤ C_p \int \abs{\grad u}^2$ for all $u ∈ H_0^1$. So, in this case we have that \[
\norm{u}^2_{H^1} = \int v^2 + \int \abs{\grad v}^2 ≤ (1+C_p^2) \int \abs{\grad u}^2
\] so our coercivity constant is $\frac{1}{1 + C_p^2}$.

This means that, applying the \nref{thm:Theory:LaxMilgram}, we have a unique solution $u ∈ V$ such that \[ \norm{u}_{H^1_0} ≤ (1+C_p^2) \norm{F}_{H^1_0} \] with $C_p$ depending on the shape of the problem domain, even when it is a convex polygon.

\subsection{Poisson problems with mixed boundary conditions}

We will want to know what happens when we have mixed boundary conditions, that is, \nref{def:Theory:DirichletBoundary} and \nref{def:Theory:NeumannBoundary}. We will study the problem \( \begin{cases}
-Δu =f & \text{in }Ω \\
u = g & \text{on }Γ_D \\
∂_\vn u = h & \text{on }Γ_N \end{cases} \) with $Γ_D ∪ Γ_N = ∂Ω$.

We can start with the weak formulation as previously \[ \int_Ω fv = \int_Ω -Δu v = \int_Ω \grad u · \grad v - \int_{∂Ω} ∂_\vn u v = \int_Ω \grad u \grad v - \int_{Γ_N} ∂_\vn v - \int_{Γ_D} ∂_n u v\]

The problem here is how to deal with the boundary term on $Γ_D$. As we did in the previous section, we can select $\restr{v}{Γ_D} = 0$. Thus, our weak formulation is \( \int_Ω \grad u · \grad v = \int_Ω f v + \int_{Γ_N} h v \label{eq:Elliptic:WeakFormulationMixedBoundary} \) with $v ∈ H^1_{Γ_D}$, where we can define \( H^1_{Γ_D} = \set{ v ∈ H^1 \tq \restr{v}{Γ_D} = 0 } \)

Our solution should however live in $V_g = \set{v ∈ H^1 \tq \restr{v}{Γ_D} = g}$, which is not linear but only an affine subspace.

If we have $g = 0$, we still can use a \nlref{def:Theory:WeakAbstractFormulation} with $F(v) = \int fv + \int_{Γ_N} hv$ to find a solution $u ∈ H^1_{Γ_D} = V_0$. With $g ≠ 0$ things become a little bit more difficult.

\subsubsection{Lifting technique}

If the datum $g$ is not null, we need to find a solution $u ∈ V_g$, and as we said before this is not a linear subspace and, even worse, it is different to the test space $V_0$ so we cannot apply Lax-Milgram.


However, if $g ∈ H^{\sfrac{1}{2}}(∂Ω)$, we can apply the \nref{thm:Fund:Trace} and show that it is the trace of some function $G ∈ H^1(Ω)$ such that $\restr{G}{∂Ω} = g$ and $\norm{G}_{H^1(Ω)} ≤ γ\norm{γ}_{H^{\frac{1}{2}}(∂Ω)}$. We can then write $u = u_0 + G$ with $u_0 ∈ V_0$, and as $a$ is a linear form the weak abstract form changes: \begin{align*}
a(u, v) &= F(v) \\
a(u_0 + G, v) &= F(v) \\
a(u_0, v) &= \underbracket{F(v) - a(G, v)}_{\tilde{F}(v)}
\end{align*} which is a problem for which we can apply the \nref{thm:Theory:LaxMilgram} and solve for $u_0$, and then reconstruct the solution as $u = u_0 + G$.

\subsection{Derivation of the weak formulation based on calculus of variations}

We can try to study other approach to this problem with an example, which is the deformation $u$ of a membrane $Ω$ under a certain force $f$. In that case, we try to find a solution that minimizes the elastic energy $E = \int_Ω \frac{1}{2} κ \norm{\grad u}^2$. Supposing $κ = 1$, our energy functional to minimize is \( J(u) = \frac{1}{2} \int_Ω \norm{\grad u}^2 - \int_Ω f u - \int_{Γ_N} h u \label{eq:Elliptic:EnergyFunctional} \) so our solution should be \[ u = \argmin_{\substack{v ∈ H^1(Ω) \\ v = \restr{g}{Γ_D}}}  J(v) \] where we search for the function in $H^1(Ω)$ because we need for the gradient to be square integrable. We need also $f ∈ L^2$, and $h ∈ H^{-\sfrac{1}{2}}$ (the topological dual space of $H^{\sfrac{1}{2}}$) to be able to integrate that last term.

So, how do we do this? If the argument was real, we could find the point with gradient $0$ (all directional derivatives are null). But the functional $J$ from \eqref{eq:Elliptic:EnergyFunctional} is an application $\appl{J}{H^1}{ℝ}$, so we need something different. However, we can still translate the concept of ``gradient $0$''. If $J$ were a real variable function, we could do \[ \grad J (u) = 0 \iff \grad J(u) · \vA = 0\; ∀\vA ∈ ℝ^N \iff \lim_{ε \to 0} \frac{J(u + ε\vA) - J(u)}{ε} = 0 \]

That last notion is the one we can translate to the functional case. If we consider $u$ to be the equilibrum, we can study any variation of $u ∈ V_g$\footnote{Remember from the previous section that $V_g = \set{v ∈ H^1 \tq \restr{v}{Γ_D} = g}$.} such that $u + ε v ∈ V_g$ (that is, respecting the boundary conditions), with $v ∈ V_g$ forcibly.

Knowing this, we can try to calculate the ``derivative'', where some linear terms will be canceled but we will have to deal with the quadratic ones:
\begin{align*}
\Dif_v J(u) &= \lim_{ε \to 0} \frac{J(u + εv) - J(u)}{ε} = \\
	&= \lim_{ε \to 0} \frac{1}{ε} \left[ \frac{1}{2} \int_Ω \grad(u+εv)· \grad(u + εv) - \int_Ωf(u + εv) - \int_{Γ_N} h · (u + εv)\right. \\
	&\qquad \left.- \frac{1}{2}\int_Ω \grad u \grad u + \int_Ω fu + \int_{Γ_N} h u \right] = \\
	&= \lim_{ε \to 0}\frac{1}{ε} \left[ \frac{1}{2}\left( \int_Ω \grad(u+εv) \grad (u + εv) - \grad u \grad u \right) - ε \int_Ω fv - ε \int_{Γ_N} h v \right] = \\
	&= \lim_{ε \to 0}\frac{1}{ε} \left[ \frac{1}{2} \left( \grad u \grad u + ε \grad u \grad v + ε \grad v \grad u + ε^2 \grad v \grad u - \grad u \grad u\right) - ε \int_Ω fv - ε \int_{Γ_N} h v \right] = \\
	&= \int_Ω \grad u \grad v - \int_Ω fv - \int_{Γ_N} h v
\end{align*} which is the same weak formulation of the problem we had previously in \eqref{eq:Elliptic:WeakFormulationMixedBoundary}.

\subsection{Regularity of the solution}
\label{sec:PDE:PoissonRegularity}

We may have proved that the solution exists, but we will also be interested in knowing the regularity of the function in order to be able to prove rates of convergence, for example. That will be given in the following theorem.

\begin{theorem}[Shift\IS theorem] \label{thm:PDE:Shift} Consider the PDE problem \[
-Δu =f \qquad \text{in }Ω \] with $f ∈ H^m(Ω)$, $Ω$ a smooth domain ($∂Ω ∈ C^{m+2}$, that is, we can parametrize the boundary as a $C^{m+2}$ manifold) and with the following restrictions depending on the boundary conditions:
\begin{itemize}
	\item Full Dirichlet conditions $\restr{u}{Γ_D} = g ∈ H^{m + \sfrac{3}{2}}(Ω)$.
	\item Full Neumann conditions $∂_\vn u = h ∈ H^{m + \sfrac{1}{2}}(Ω)$.
\end{itemize}

Under those conditions, $u ∈ H^{m+2}(Ω)$.
\end{theorem}

Thus, for example, having only $∂Ω ∈ C^2$, we would have $u∈H^2(Ω)$ and $\norm{u}_{H^2(Ω)} ≤ C_Ω \norm{f}_{L^2(Ω)}$.

However, if we don't have that regularity on the border we have weird things. Although as discussed previously we still have existence of the solution, the regularity depends on the angles of the polygon, with $u ∈ H^s(Ω)$ and $1≤s<2$. For a convex polygon we have the same results as above. If the polygon is not convex, such as an ``L'' shape, we have $s = \sfrac{3}{2}$. If we have something like a ``crack'' in the domain, we would only have $s = 1$.

It is also interesting to see that if $f ∈ C^∞(Ω)$ then $u ∈ C^∞(Ω)$, but not necessarily smooth on the closure (that is, we can't guarantee that $u ∈ C^∞(\adh{Ω})$).

\subsubsection{Corner singularities}

\begin{wrapfigure}{L}{0.3\textwidth}
\centering
\inputtikz{CornerSingularity}
\caption{Corner singularity in a domain.}
\label{fig:Elliptic:CornerSingularity}
\end{wrapfigure}

One could try to see what happens if we have corner singularities, for example, in a square domain. In a set inside of the square we will have perfect smoothness, but the corners may present problems.

Suppose we want to solve $- Δ u = 0$ in a corner such as \fref{fig:Elliptic:CornerSingularity}. In that case, we will work in polar coordinates and the basis of our solution will be \[φ_k(r,θ) = r^{\frac{kπ}{ω}} \sin \frac{kπθ}{ω} \]

The worst case would be $k = 1$, which would leave us in a case of $u ∈ H^s$ with $s < 1 + \sfrac{π}{ω}$. If $ω > π$ (the case of the square), we have $s < 2$, that is, we only have $H^1$ regularity.

We would have the same situation with Neumann conditions. However, mixed boundary conditions (Neumann on one side, Dirichlet on the other) are more problematic: the solutions are \[ φ_k(r,θ) = r^\frac{(k + \sfrac{1}{2})π}{ω} \sin \frac{(k + \sfrac{1}{2})πθ}{ω} \] and, in the worst case ($k = 0$) we have singularities even in the flat case ($ω = π$) which was not a problem in full Neumann or Dirichlet conditions.

\fi
% -*- root: ../NumericalApproximationofPDEs.tex -*-

\subsection{Numerical approximation - Galerkin method}

In this section we will study the numerical approximation of the Poisson problem by the Galerkin problem we saw in \fref{sec:Theory:GalerkinApprox}, centering too in the computational complexity of solving this problem.

As always, we can consider a finite element space $V_h ∈ H_0^1(Ω)$ spanned by a finite number of basis functions $\set{φ_i}_{i = 1}^N$ and we try to solve the problem there. As we have a finite element space, we can consider the coefficients of the functions as a vector $\vu ∈ ℝ^N$ so that the bilinear form can be expressed in matrix form: \[ a(u_h, v_h) = \sum_{i,j = 1}^N u_i · v_i \underbracket{a(φ_i, φ_j)}_{A_{ij}} = \trans{\vu} \mA \vv \]

This allows us to see that $\mA$ is positive definite and invertible. For the first, we can see that coercivity gives it directly: $\vv^T \mA \vv = a(v_h, v_h) ≥ α \norm{v_h}^2$. For invertibility, if $\trans{\vv} \mA \vv = 0$, then $a(v_h, v_h) = 0$ but by coercivity $a(v_h, v_h) ≥ \norm{v_h}^2$ so we need $v_h = 0$. Thus, $\ker \mA = \set{0}$ and it is invertible.

This allows us to formulate the following proposition for the bounding of the eigenvalues\footnote{Which is just a problem-specific version of \fref{thm:Theory:ConditioningMatrix}.}, which in turn will let us give bounds for the time required to solve this problem.

\begin{prop} \label{prop:PDE:BoundsMatrix} Let $\mA ∈ ℝ^{N×N}$ be the stiffness matrix of the Poisson problem \eqref{eq:PDE:PoissonProblem} defined by $A_{ij} = \int_Ω ∇φ_i ∇φ_j$. Then, there exists two constants $C_1, C_2 > 0$ such that \[ C_1 \vv^T \mM \vv ≤ \vv^T \mA ≤ \frac{C_2}{h^2} \vv^T \mM \vv \] with $\mM$ the mass matrix given by $M_{ij} = \int_Ω φ_i φ_j$.
\end{prop}

For the proof of this proposition we will need the inverse inequality.

\begin{prop}[Inverse inequality][Inequality!inverse] \label{prop:PDE:InverseInequality} Let $V_h ∈ H_0^1(Ω)$ be a finite element space defined on a \nref{def:Theory:RegularFamilyMeshes} \mesh. Then, for all $h > 0$ and for any $K ∈ \mesh$ the following holds: \[ \int_K \abs{∇v_h}^2 ≤ \frac{C}{h_K^2} \int_K v_h^2 \]
\end{prop}

\begin{proof}
\end{proof}

\begin{proof}[\Fref{prop:PDE:BoundsMatrix}] First, by using \nref{thm:Fund:PoincareInequality}, we know that there exists a constant $C_p > 0$ such that $\norm{v}_{L^2} ≤ C_p \norm{∇v}_{L^2}$ which gives us the bound \[ \vv^T \mM \vv ≤ C_p^2 \vv^T \mA \vv \]

For the other bound, we use the \nref{prop:PDE:InverseInequality} summing over all mesh elements $K ∈ \mesh$ and using the inverse assumption that $Ch ≤ h_k ≤ h$, so that \[ \int_Ω \abs{∇v_h}^2 ≤ \frac{C}{h^2} \int_Ω v_h^2 \] which gives directly the matrix bound.
\end{proof}

As discussed in \fref{sec:Theory:ConditioningNumber}, the numerical method to solve these problems will be preconditioned conjugate gradient to avoid memory storage problems. This method has the following property:

\begin{prop} Let $\vu^k$ be the approximation obtained by the preconditioned conjugate gradient method of the porblem $\mA \vu = \vf$ after $k$ steps, with $\mM$ as a preconditioner. Then, the following error estimate holds: \[ \norm{\vu - \vu^k}_{CG} ≤ 2 \norm{\vu - \vu^0}_{CG} \left(\frac{\sqrt{K(\inv{\mM}\mA)} - 1}{\sqrt{K(\inv{\mM}\mA)} + 1}\right)^k\] with $\norm{·}_{CG}$ defined as follows: \[ \norm{\vv}_{CG} = \]
\end{prop}

With the bounds explained in \fref{prop:PDE:BoundsMatrix} we can bound the condition number: \[ C_1 ≤ \inf_{\vv ∈ ℝ^N \setminus\set{0}} \frac{\vv^T \mA \vv}{\vv \mM \vv} ≤ λ_j = \frac{\vx_j^T \mA \vx_j}{\vx_j^T \mM \vx_j} ≤ \sup_{\vv ∈ ℝ^N \setminus\set{0}} ≤ \frac{C_2}{h^2}\] so that $K(\inv{\mM} \mA) ≤ \frac{C_2}{C_1h^2}$.

Thus, in order to reduce the initial error by a factor of $ε$, we see that \[ ε\norm{\vu - \vu^0}_{CG} = \norm{\vu - \vu^k}_{CG} ≤ 2 \norm{\vu - \vu^0}_{CG} \left(\frac{\sqrt{K(\inv{\mM}\mA)} - 1}{\sqrt{K(\inv{\mM}\mA)} + 1}\right)^k = 2 \norm{\vu - \vu^0}_{CG} ( 1 - \mathcal{O}(kh))\] so that the number $k$ of iterations is $k = \mathcal{O}(\sfrac{1}{h})$.

\subsection{A priori error estimates}

This section is nothing specially new, as it is just applying the results of \fref{sec:Theory:ApproxResults}.

\begin{prop} Given a \nref{def:Theory:RegularFamilyMeshes} with aspect ratio bounded by $γ ∈ ℝ^+$ for a convex polygon, then there exists a constant $C >0$ independent of $h$ and $u$ such that \[ \norm{u - u_h}_{L^2} + h \norm{∇(u - u_h)}_{L^2} ≤ Ch^2 \norm{∇^2 u}_{L^2} \]
\end{prop}

\begin{proof} We will prove separately the two estimates.

\proofpart{Bound for $\norm{∇u - ∇u_h}_{L^2}$}

By definition, we know that $\norm{∇u - ∇u_h}_{L^2} = \int_Ω ∇(u - u_h) · ∇(u - u_h)$. By \nref{def:Theory:GalerkinOrthog}, we know that, for any $v_h ∈ V_h$,  $\int_{Ω} ∇(u - u_h) · ∇v_h = a(u - u_h, v_h) = 0$ so that \[ \norm{∇u - ∇u_h}_{L^2} = \int_Ω ∇(u - u_h) · ∇(u - u_h) ≤ \norm{∇(u - u_h)}_{L^2} \norm{∇(u - v_h)}_{L^2}\]

The trick is now choosing the correct $v_h$, and we will select the Lagrange interpolant given by \[ v_h = I_h u = \sum_{j=1}^N u(p_j) φ_j \]

Since $Ω$ is a convex polygon, $u ∈ H^2(Ω)$  and therefore for $u ∈ C^0(\adh{Ω})$ we choose $v_h = I_h u$. Then we can use the \nref{thm:Theory:InterpErrorSmooth} and finally $\norm{∇(u - u_h)}_{L^2} ≤ \norm{∇(u - I_hu)}_{L^2} ≤ Ch \norm{∇^2 u}_{L^2}$.
\end{proof}

\subsection{A posteriori error estimates}

The a posteriori error estimates will give us error estimates after the approximate solution has been computed.

\begin{prop}[A posteriori error\IS for the Poisson problem] Given a shape regular mesh then there exists a constant $C_1 > 0$ independent of $h,u,f$ but depending on the mesh aspect ratio such that \( \norm{∇(u - u_h)}_{L^2(Ω)}^2 ≤ C_1 \sum_{K ∈ \mesh} η_K^2 \label{eq:PDE:APostGradError} \) where $η_K = h_K \norm{Δu_h + f}_{L^2(K)} + h_K^{\sfrac{1}{2}} \norm{[∇u_h]}_{L^2(∂K)}$ and $[∇u_h]$ is the jump of $∇u_h$ in the boundary between two mesh elements $K$ and $K'$ given by $[∇u_h] = \restr{∇u_h}{K} - \restr{∇u_h}{K'}$

Moreover, if $Ω$ is a convex polygon then there exists another constant $C_2$ independent of $h,u,f$ but depending on the mesh aspect ratio such that \( \norm{u - u_h}_{L^2(Ω)^2} ≤ C_2 \sum_{K ∈ \mesh} h_K^2 η_K^2 \label{eq:PDE:APostError} \)
\end{prop}

\begin{proof}

\proofpart{\eqref{eq:PDE:APostGradError}}
% TODO: Fix this crappy proof.

So something something \[ \norm{∇(u - u_h)}_{L^2}^2 = \int_Ω ∇(u - u_h) ∇(u - u_h) = \int_Ω f(u - u_h) - ∇u_h ∇(u - u_h)\] where the last part is $\dualp{\mop{Residual}\, u_h, u - u_h}$. We can substract any test function $v_h ∈ V_h$ so that everything is still the same in \[ \sum_{K ∈ \mesh} \int_K f(u - u_h - v_h) - ∇u_h ∇(u - u_h - v_h)\]

Integrating by parts, we are left with \[ \int_K f(u - u_h - v_h) - ∇u_h ∇(u - u_h - v_h) = \int_K (f + Δu_h)(u - u_h - v_h) - \int_{∂K} ∇u_h · \vn (u - u_h - v_h) \] and now the trick is noticing something and other thing and using Cauchy-Scharwz we have that \begin{multline*}
\int_K (f + Δu_h)(u - u_h - v_h) - \int_{∂K} ∇u_h · \vn (u - u_h - v_h) ≤ \\ ≤ \norm{f + Δu_h}_{L^2(K)} \norm{u - u_h - v_h}_{L^2(K)} + \frac{1}{2} \norm{[∇u_h · \vn]}_{L^2(∂K)} \end{multline*}

We use the Clément interpolant so that $v_h = R_h(u - u_h)$, which will gives us $\norm{v - R_h v}_{L^2(Ω)} ≤ Ch \norm{∇v}_{L^2(Ω)}$ which does not hold for the Lagrange interpolant in dimensions greater than 1.

The only issue here is that we have a function $v ∈ H_0^1$ so we don't have a point value for the interpolant $v(p_j)$. Luckily, we can average so that \[ R_h v(p_j) = \sum_{\substack{K ∈ \mesh \\ p_j ∈ K}} \fint_K v \]

Then the interpolation result is \[ \frac{1}{h_k^2} \norm{v - R_hv}^2_{L^2(K)} + \norm{∇(v - R_hv)}_{L^2(K)} + \frac{1}{h_K} \norm{v - R_hv}_{L^2(∂K)} ≤ \norm{∇v}_{L^2(ΔK)} \]

\proofpart{\eqref{eq:PDE:APostError}}

\end{proof}

% TODO
Some explanation of $η_K$ and more things.

\subsection{Adaptive algorithm}



\ifincludefirstsemester% -*- root: ../NumericalApproximationofPDEs.tex -*-

\section{Poisson problem}

We start by considering a simple Poisson equation, given $Ω ⊂ ℝ^d$ a bounded open domain with Lipschitz boundary $∂Ω$. There, we set the problem \[ \begin{cases}
-Δu = f & \text{ in } Ω \\
u =  0 & \text{ in } ∂Ω
\end{cases} \]

In \fref{sec:Theory:WeakFormulationPDE} we saw that the weak abstract form was  \[ \int_Ω \grad u · \grad v = \int_Ω f v \]

It is easy to see that we can apply the \nref{thm:Theory:LaxMilgram} to our problem: if $v ∈ H^1_0(Ω)$, then its $L^2$-norm is bounded so \[ \abs{\int_Ω fv} ≤ \norm{f}_{L^2} \norm{v}_{L^2} ≤ \norm{f}_{L^2} \norm{v}_{H^1} < ∞ \]

Same happens with $a$: the boundedness is directly obtained from the fact that $u, v ∈ H^1_0(Ω)$. Coercivity can be a little bit more complicated. For that, we must use the \nref{thm:Fund:PoincareInequality} which tells us that there is a constant $C_p > 0$ such that $\int v^2 ≤ C_p \int \abs{\grad u}^2$ for all $u ∈ H_0^1$. So, in this case we have that \[
\norm{u}^2_{H^1} = \int v^2 + \int \abs{\grad v}^2 ≤ (1+C_p^2) \int \abs{\grad u}^2
\] so our coercivity constant is $\frac{1}{1 + C_p^2}$.

This means that, applying the \nref{thm:Theory:LaxMilgram}, we have a unique solution $u ∈ V$ such that \[ \norm{u}_{H^1_0} ≤ (1+C_p^2) \norm{F}_{H^1_0} \]

\subsection{Poisson problems with mixed boundary conditions}

We will want to know what happens when we have mixed boundary conditions, that is, \nref{def:Theory:DirichletBoundary} and \nref{def:Theory:NeumannBoundary}. We will study the problem \( \begin{cases}
-Δu =f & \text{in }Ω \\
u = g & \text{on }Γ_D \\
∂_\vn u = h & \text{on }Γ_N \end{cases} \) with $Γ_D ∪ Γ_N = ∂Ω$.

We can start with the weak formulation as previously \[ \int_Ω fv = \int_Ω -Δu v = \int_Ω \grad u · \grad v - \int_{∂Ω} ∂_\vn u v = \int_Ω \grad u \grad v - \int_{Γ_N} ∂_\vn v - \int_{Γ_D} ∂_n u v\]

The problem here is how to deal with the boundary term on $Γ_D$. As we did in the previous section, we can select $\restr{v}{Γ_D} = 0$. Thus, our weak formulation is \( \int_Ω \grad u · \grad v = \int_Ω f v + \int_{Γ_N} h v \label{eq:Elliptic:WeakFormulationMixedBoundary} \) with $v ∈ H^1_{Γ_D}$, where we can define \( H^1_{Γ_D} = \set{ v ∈ H^1 \tq \restr{v}{Γ_D} = 0 } \)

Our solution should however live in $V_g = \set{v ∈ H^1 \tq \restr{v}{Γ_D} = g}$, which is not linear but only an affine subspace.

If we have $g = 0$, we still can use a \nlref{def:Theory:WeakAbstractFormulation} with $F(v) = \int fv + \int_{Γ_N} hv$ to find a solution $u ∈ H^1_{Γ_D} = V_0$. With $g ≠ 0$ things become a little bit more difficult.

\subsubsection{Lifting technique}

If the datum $g$ is not null, we need to find a solution $u ∈ V_g$, and as we said before this is not a linear subspace and, even worse, it is different to the test space $V_0$ so we cannot apply Lax-Milgram.


However, if $g ∈ H^{\sfrac{1}{2}}(∂Ω)$, we can apply the \nref{thm:Fund:Trace} and show that it is the trace of some function $G ∈ H^1(Ω)$ such that $\restr{G}{∂Ω} = g$ and $\norm{G}_{H^1(Ω)} ≤ γ\norm{γ}_{H^{\frac{1}{2}}(∂Ω)}$. We can then write $u = u_0 + G$ with $u_0 ∈ V_0$, and as $a$ is a linear form the weak abstract form changes: \begin{align*}
a(u, v) &= F(v) \\
a(u_0 + G, v) &= F(v) \\
a(u_0, v) &= \underbracket{F(v) - a(G, v)}_{\tilde{F}(v)}
\end{align*} which is a problem for which we can apply the \nref{thm:Theory:LaxMilgram} and solve for $u_0$, and then reconstruct the solution as $u = u_0 + G$.

\subsection{Derivation of the weak formulation based on calculus of variations}

We can try to study other approach to this problem with an example, which is the deformation $u$ of a membrane $Ω$ under a certain force $f$. In that case, we try to find a solution that minimizes the elastic energy $E = \int_Ω \frac{1}{2} κ \norm{\grad u}^2$. Supposing $κ = 1$, our energy functional to minimize is \( J(u) = \frac{1}{2} \int_Ω \norm{\grad u}^2 - \int_Ω f u - \int_{Γ_N} h u \label{eq:Elliptic:EnergyFunctional} \) so our solution should be \[ u = \argmin_{\substack{v ∈ H^1(Ω) \\ v = \restr{g}{Γ_D}}}  J(v) \] where we search for the function in $H^1(Ω)$ because we need for the gradient to be square integrable. We need also $f ∈ L^2$, and $h ∈ H^{-\sfrac{1}{2}}$ (the topological dual space of $H^{\sfrac{1}{2}}$) to be able to integrate that last term.

So, how do we do this? If the argument was real, we could find the point with gradient $0$ (all directional derivatives are null). But the functional $J$ from \eqref{eq:Elliptic:EnergyFunctional} is an application $\appl{J}{H^1}{ℝ}$, so we need something different. However, we can still translate the concept of ``gradient $0$''. If $J$ were a real variable function, we could do \[ \grad J (u) = 0 \iff \grad J(u) · \vA = 0\; ∀\vA ∈ ℝ^N \iff \lim_{ε \to 0} \frac{J(u + ε\vA) - J(u)}{ε} = 0 \]

That last notion is the one we can translate to the functional case. If we consider $u$ to be the equilibrum, we can study any variation of $u ∈ V_g$\footnote{Remember from the previous section that $V_g = \set{v ∈ H^1 \tq \restr{v}{Γ_D} = g}$.} such that $u + ε v ∈ V_g$ (that is, respecting the boundary conditions), with $v ∈ V_g$ forcibly.

Knowing this, we can try to calculate the ``derivative'', where some linear terms will be canceled but we will have to deal with the quadratic ones:
\begin{align*}
\Dif_v J(u) &= \lim_{ε \to 0} \frac{J(u + εv) - J(u)}{ε} = \\
	&= \lim_{ε \to 0} \frac{1}{ε} \left[ \frac{1}{2} \int_Ω \grad(u+εv)· \grad(u + εv) - \int_Ωf(u + εv) - \int_{Γ_N} h · (u + εv)\right. \\
	&\qquad \left.- \frac{1}{2}\int_Ω \grad u \grad u + \int_Ω fu + \int_{Γ_N} h u \right] = \\
	&= \lim_{ε \to 0}\frac{1}{ε} \left[ \frac{1}{2}\left( \int_Ω \grad(u+εv) \grad (u + εv) - \grad u \grad u \right) - ε \int_Ω fv - ε \int_{Γ_N} h v \right] = \\
	&= \lim_{ε \to 0}\frac{1}{ε} \left[ \frac{1}{2} \left( \grad u \grad u + ε \grad u \grad v + ε \grad v \grad u + ε^2 \grad v \grad u - \grad u \grad u\right) - ε \int_Ω fv - ε \int_{Γ_N} h v \right] = \\
	&= \int_Ω \grad u \grad v - \int_Ω fv - \int_{Γ_N} h v
\end{align*} which is the same weak formulation of the problem we had previously in \eqref{eq:Elliptic:WeakFormulationMixedBoundary}.

\subsection{Regularity of the solution}

We may have proved that the solution exists, but we will also be interested in knowing the regularity of the function in order to be able to prove rates of convergence, for example. That will be given in the following theorem.

\begin{theorem}[Shift theorem][Theorem!shift] \label{thm:PDE:Shift} Consider the PDE problem \[
-Δu =f \qquad \text{in }Ω \] with $f ∈ H^m(Ω)$, $Ω$ a smooth domain ($∂Ω ∈ C^{m+2}$, that is, we can parametrize the boundary as a $C^{m+2}$ manifold) and with the following restrictions depending on the boundary conditions:
\begin{itemize}
	\item Full Dirichlet conditions $\restr{u}{Γ_D} = g ∈ H^{m + \sfrac{3}{2}}(Ω)$.
	\item Full Neumann conditions $∂_\vn u = h ∈ H^{m + \sfrac{1}{2}}(Ω)$.
\end{itemize}

Under those conditions, $u ∈ H^{m+2}(Ω)$.
\end{theorem}

\subsubsection{Corner singularities}

\begin{wrapfigure}{L}{0.3\textwidth}
\centering
\inputtikz{CornerSingularity}
\caption{Corner singularity in a domain.}
\label{fig:Elliptic:CornerSingularity}
\end{wrapfigure}

One could try to see what happens if we have corner singularities, for example, in a square domain. In a set inside of the square we will have perfect smoothness, but the corners may present problems.

Suppose we want to solve $- Δ u = 0$ in a corner such as \fref{fig:Elliptic:CornerSingularity}. In that case, we will work in polar coordinates and the basis of our solution will be \[φ_k(r,θ) = r^{\frac{kπ}{ω}} \sin \frac{kπθ}{ω} \]

The worst case would be $k = 1$, which would leave us in a case of $u ∈ H^s$ with $s < 1 + \sfrac{π}{ω}$. If $ω > π$ (the case of the square), we have $s < 2$, that is, we only have $H^1$ regularity.

We would have the same situation with Neumann conditions. However, mixed boundary conditions (Neumann on one side, Dirichlet on the other) are more problematic: the solutions are \[ φ_k(r,θ) = r^\frac{(k + \sfrac{1}{2})π}{ω} \sin \frac{(k + \sfrac{1}{2})πθ}{ω} \] and, in the worst case ($k = 0$) we have singularities even in the flat case ($ω = π$) which was not a problem in full Neumann or Dirichlet conditions.

\section{Advection-diffusion-reaction}

A more complex form of elliptic problems are advection-diffusion-reaction, where the differential operator is \[ Lu = - \sum_{i,j=1}^d \dpd{}{x_i} \left(a_{ij}\dpd{u}{x_j}\right) + \sum_{i=1}^d b_i \dpd{u}{x_i} + c u = - \dv (A(x) \grad u) + \vb(x) \grad u + c u\] with $A(x) ∈ ℝ^{d×d}, \vb(x) ∈ ℝ^d, c ∈ ℝ$. Our problem, as usual, is $Lu = f$.

These three terms model respectively diffusion, with $A$ being the matrix of coefficients that depend on the axis (the material is not uniform), the transport term along the vector field, and the reaction (for example, a chemical that reacts and its concentration decreases)

In order for the problem to remain elliptic, we require $A(x)$ to be positive definite for all $x ∈ ℝ^d$.

As in previous cases, we can have Dirichlet boundary conditions $\restr{u}{Γ_D} = g$ and Neumann boundary conditions (which are a little bit more complicated): \[ A \grad u · \vn - (b \vn) u = h \quad\text{ on } Γ_N \]

We will want to do a weak formulation where the boundary terms of the problem appear naturally. As always, we multiply by a test function and integrate by parts the divergence and possibly the $b \grad u$ depending on the boundary conditions. Without doing the computations, the weak formulation will end up being \( \int_Ω A \grad u \grad v + b \grad u v + c u v = \int_Ω fv + \int_{Γ_N} h v \label{eq:Elliptic:ADRProblemWeak} \) with $u ∈ H^1$, $\restr{u}{Γ_D} = g$.

This has the same structure as in previous cases, so we are in the conditions of the \nref{thm:Theory:LaxMilgram} and there exists a unique solution. Coercivity would need a little bit of work, and may present problems with mixed boundary conditions. Enforcing Neumann conditions on boundaries with incoming flow may cause problems with coercivity.

\section{Linear elasticity}

\begin{figure}[hbtp]
\inputtikz{ElasticDeformation}
\caption{Elastic deformation of a something.}
\label{fig:Elliptic:ElasticDeformation}
\end{figure}

In this problem, we start with an undeformed configuration $Ω ⊂ ℝ^d$, and we want to study the displacement $\appl{\vu}{Ω}{ℝ^d}$. The involved terms are the strain measure \[ ε (\vu) = \frac{\grad \vu + \trans{(\grad \vu)}}{2}\], the stress tensor $σ = σ(ε)$ given by \[ σ_{ij} = \sum_{k,l=1}^d c_{ijkl} ε_{kl} \], which usually can be expressed as \[ σ(ε) = 2με + λ\tr(ε) I \] with $μ,λ$ the Lamé constants.

With all of this, our balance equation is \( \begin{cases} - \dv σ(ε(\vu)) = \vec{f} & \text{in } Ω \\
\vu = \vec{g} & \text{on } Γ_D \\
σ(u) · \vn = \vd & \text{on } Γ_N \end{cases} \)

We may want to write now our weak formulation and integrate by parts, caring a little bit about what is a tensor and what is a vector
\begin{align*}
0
	&= \int_Ω \left[- \dv (σ(ε(\vu))) - \vec{f} \right] · \vv = \\
	&= \int_Ω σ(ε(\vu)) \grad \vv - \int_{∂Ω} (σ · \vn) · \vv - \int_Ω \vec{f} \vv \\
\int_Ω σ(ε(\vu)) \colon \vv &= \int_Ω \vec{f}\vv + \int_{Γ_N} \vd \vv
\end{align*}

I'm a little bit lost but \[ \int_Ω - \dv (σ) \vv = \int_Ω - \sum_i \sum_j ∂_j σ_{ij}v_i = \sum_{ij} \int_Ω σ_{ij}∂_jv_i - \int_{∂Ω}σ_{ij}n_j v_i \]

We can rewrite the first term because \begin{align*}
\int_Ω σ (ε(\vu)) \colon \vv
	&= \int_Ω σ(ε(\vu))\colon ε(\vv) = \\
	&= \int_Ω(2με(\vu) + λ\tr(ε(\vu))I) \colon ε(\vv) = \\
	&= \int_Ω 2με(\vu) \colon ε(\vv) + λ\tr(ε(\vu))I \colon ε(\vv) = \\
	&= \int_Ω 2μ \frac{\grad \vu + \trans{(\grad \vu)}}{2} \colon \frac{\grad \vv + \trans{(\grad \vv)}}{2} + λ\dv \vu \dv \vv = \\
	&= a(\vu, \vv)
\end{align*}

So we have a bilinear form again and whatever.

\section{Error estimates}

Along this section, we will see examples of elliptic problems and applying to them the error estimates devised in \fref{sec:Theory:ApproxResults}.

\subsection{Error estimates for finite element approximations}

Again, we discuss the estimates for the model problem and then we will see that the procedures generalize. This problem will be the Poisson equations with boundary conditions:
\[ \begin{cases}
-Δu =f & \text{in } Ω \\
∂_\vn u = d & \text{on } Γ_N \\
u= g & \text{on } Γ_D
\end{cases}\]

Its abstract weak form is the problem of finding a function $u ∈ V_g$ such that $a(u,v) = F(v)$ for any $v ∈ V_0$, with
\begin{align*}
V_g &= \set{v ∈ H^1(Ω) \tq \restr{v}{Γ_D} = g} \\
a(u,v) &= \int_Ω ∇u∇v \\
F(v) &= \int_Ω fv + \int_{Γ_N} d v
\end{align*} and assuming that $a$ is continuous ($a(u,v) ≤ M \norm{u}_V \norm{v}_V$) and coercive with coercivity coefficient $α = \frac{1}{1 + c_p^2}$, and $F$ a bounded linear operator. For the spaces $V_0, V_g$ we pick the $H^1$ norm.

Our finite element space will be $X_h^r$, the space of $P_r$ finite elements on the mesh $\mesh$. We can construct also the finite element space vanishing on the boundary as $V_{0,h} = X_h^r ∩ H_{Γ_D}^1$. The finite element space for $V_g$ is a little bit more tricky, but we can simply try to approximate the boundary value with the interpolation operator so $V_{h,g} = \set{ v_h ∈ X_h^r \tq \restr{v_h}{Γ_D} = I_h^r g}$.

If he have homogeneous Dirichlet boundary conditions (that is, $g \equiv 0$) we have the \nref{lem:Theory:Cea} which says that \[ \norm{u - u_h}_V ≤ \frac{M}{α} \inf_{v_h ∈ V_h} \norm{u - v_h}_V \]

With that, we can apply the error estimates for interpolation so that $\inf_{v_h ∈ V_{h,0}} \norm{u -v_h}_{H^1} ≤ \norm{u - I_h^r u}_{H^1} ≤ \abs{u}_{H^η}$ and our estimate is \[ \norm{u-u_h}_{H^1} ≤ C h^{η - 1} \abs{u}_{H^η} \] with $η = \min \set{r + 1, s}$ and $u ∈ H^s(Ω)$.

\subsubsection{Error estimate in $L^2$}

We will use the Aubin-Nitsche trick and we will use duality. In order to estimate the error, we define $e_h = u - u_h$ and the adjoint problem of searching for a $φ ∈ V$ such that \[ a(v,φ) = \int_{Ω} e_h v \qquad ∀v ∈ V\]

Thus, in order to estimate the error in $L^2$ we can do the following: \[ \norm{u-u_h}^2_{L^2} = \norm{e_h}_{L^2}^2 = \int_Ω e_h e_h = a(e_h, φ) = a(u - u_h, φ) \]

By the Galerkin orthogonality condition, we know that $a(u - u_h, v_h) = 0$ for any $v_h ∈ V_h$. That allows to substract any $w_h ∈ V_h$ to φ, and then \[ a(u-u_h, φ) = a(u - u_h, φ - w_h) ≤ M \norm{u - u_h}_V \inf_{w_h ∈ V_h} \norm{φ-w_h}_V \]

We must know now which is the regularity of this function φ. To do that, we must notice that φ is the solution to the porblem \[
\begin{cases} - Δφ = e_h & \text{in } Ω \\
∂_\vn φ = 0 & \text{on } Γ_N \\
φ = 0 & \text{on } Γ_D
\end{cases} \] which in turn depends on the smoothness of $e_h$. We will use however only that $e_h ∈ L^2$, if Ω is a convex polygon. So, assuming $φ ∈ H^2$ we can bound the error in $L^2$, getting one more order of convergence: \[ \norm{u - u_h}_{L^2} ≤ C h \norm{u - u_h}_{H^1} \]
\fi
% -*- root: ../NumericalApproximationofPDEs.tex -*-
\section{Non linear problems}

Our model non-linear problem will the the following: find $\appl{u}{Ω ⊂ ℝ^2}{ℝ}$ such that \(
\begin{cases}
- Δu + u^3 = f & \text{in } Ω \\
u = 0 & \text{on } ∂Ω
\end{cases} \label{eq:PDE:NonlinearProblem}
\)

The cube does not have any specific physical significance, but it will be interesting for the mathematics.

The weak formulation of this problem is finding $u ∈ H_0^1(Ω)$ such that \( \int_Ω ∇u ∇v + u^3 v = \int_Ω fv \qquad ∀ v ∈ H_0^1(Ω) \label{eq:PDE:WeakNonlinearProblem} \)

We could have some problems with that term $u^3 v$: we need to ensure that it is bounded. Indeed, it is. We have that $\int_Ω u^3 v ≤ \norm[0]{u^3}_2 \norm{v}_2$. We know that $\norm{v} < ∞$ as $v ∈ H_0^1(Ω)$. For the other norm, we can see that $\norm[0]{u^3}_2 = \norm[0]{u}_6^3$ and by the \nref{thm:SobolevEmbedding}, as we are working in $Ω ⊂ ℝ^2$ and $u ∈ H^1(Ω)$, there is an embedding $H^1(Ω) ⊂ L^q(Ω)$ for $2 ≤ q < ∞$ and in particular $\norm{u}_6 < ∞$.

Thus we can talk about uniqueness and existence.

\begin{prop}[Existence and uniqueness\IS for a non-linear elliptic PDE] \label{prop:NonLinearProblemExistenceUniqueness} For the non-linear problem, if we assume $f ∈ L^2(Ω)$ and $∂Ω ∈ C^2$ (or $Ω$ is a convex polygon), then there exists a unique $u ∈ H^1_0(Ω)$ solution of \eqref{eq:PDE:WeakNonlinearProblem} and $C > 0 $ depending on the domain such that \( \norm{∇u}_{L^2(Ω)} ≤ C\norm{f}_{L^2(Ω)} \label{eq:PDE:NonLinearBound} \)
\end{prop}

\begin{proof} Uniqueness and the bound are not difficult. For the existence problem we will need an extra problem.

\proofpart{Bound \eqref{eq:PDE:NonLinearBound}}

Assume that there exists a solution $u$ of the problem. Then \[ \int_Ω \abs{∇u}^2 ≤ \int_Ω \abs{∇u}^2 + \underbracket{\int_Ω u^4}_{ > 0} = \int_Ω fu ≤ C_p \norm{f}_{L^2(Ω)} \norm{∇u}_{L^2(Ω)} \]

\proofpart{Uniqueness}

Let $u_1, u_2 ∈ H_0^1$ be solutions of the problem. Then,
\[ \int_Ω ∇(u_1 - u_2) ∇v + \int_Ω(u_1^3 - u_2^3) v = 0\]

By using the mean value theorem, we know that $φ(a) - φ(b) = φ'(θa + (1 - θ)b) (a - b)$ for some $θ ∈ (0,1)$. Choosing $φ(u) = u^3$, we can get rid of that substraction $u_1^3 - u_2^3$ and then \[
\int_Ω ∇(u_1 - u_2) ∇v + \int_Ω 3(θu_1 + (1 - θ)u_2)^2 (u_1 - u_2) v = 0
\]

Now we take $v = u_1 - u_2$ and we have
\[ \int_Ω \abs{∇(u_1 - u_2)}^2 + \underbracket{\int_Ω 3(θu_1 + (1 - θ) u_2)^2 (u_1 - u_2)^2}_{ ≥ 0} = 0\] so by some reason I'm not really sure of turns out $u_1 - u_2 = 0$.

\proofpart{Existence}

For this part of the proof we will use the \nref{thm:Fund:SchoeferFixedPoint}, with our operator \begin{align*}
\appl{G}{H_0^1(Ω)&}{H_0^1(Ω)} \\
u &\longmapsto w \st \int_{Ω} ∇w ∇v = \int_Ω (f - u^3)v \quad ∀v ∈ H_0^1(Ω)
\end{align*}

That is, $G$ maps $u$ to the weak solution of the problem $- Δw = (f - u^3)$, which is a well posed problem. In particular, if $u$ is a solution we have $G(u) = u$, so the fixed point is a solution of the problem.

We have to prove that $G$ is continuous and compact to apply the theorem. For continuity, let $w_1 = G(u_1)$ and $w_2 = G(u_2)$. We need to prove that $\norm{w_1 - w_2} \to 0$ when $\norm{u_1 - u_2} \to 0$. We can see that \[
\int_Ω ∇(w_1 - w_2) ∇v = \int_Ω - (u_1^3 -u_2^3)v \\
	= - \int_Ω (u_1 - u_2)(u_1^2 + u_1 u_2 + u_2^2)v
\]

Now we can bound the product of three elements by the product of the norms in $L^2$ for one factor and $L^4$ for the others\footnote{If $u,v,w ∈ H_0^1(Ω)$, then $\int uvw ≤ \norm{u}_2 \norm{vw}_2 ≤ \norm{u}_2 \norm{v}_4 \norm{w}_4$}. We apply that to the previous equation and assuming $v = w_1 - w_2$ we haves
\begin{align*}
\norm{∇(w_1 - w_2)}^2 &≤ \norm{u_1^2 + u_1u_2 + u_2^2}_{L^2} \norm{u_1 - u_2}_{L^4} \norm{w_1 - w_2}_{L^4} \\
\norm{∇(w_1 - w_2)} &≤ C_1 \norm{u_1^2 + u_1u_2 + u_2^2}_{L^2} \norm{∇(u_1 - u_2)}_{L^2} \\
	&≤ C_2 \left(\norm{∇u_1}^2_{L^2} + \norm{∇u_2}^2_{L^2}\right) \norm{∇(u_1 - u_2)}_{L^2}
\end{align*} so all the factors on the right-hand side are bounded and $\norm{∇(u_1 - u_2)} \to 0$ and $G$ is continuous.

Now we need to prove that $G$ is a \nref{def:Fund:CompactAppSubseq}. We take then $\set{u_n}_{n = 1}^∞  ⊂ H_0^1(Ω)$ a bounded sequence, and we want to prove that for a certain $u_{n_j}$ subsequence then $\set{w_{n_j}}_{j = 1}^∞$ converges, with $w_{n_j} = G(u_{n_j})$.

Since $∂Ω ∈ C^2$, then $w ∈ H^2(Ω) ∩ H_0^1(Ω)$ and indeed $f - u^3 ∈ L^2(Ω)$. We need to use the fact that the immersion from $H^2(Ω)$ to $H^1(Ω)$ is compact (this is also true for $H^1$ to $L^2$). So, if we take a bounded sequence $\set{w_n}$ in $H^2(Ω)$, then there exists a subsequence $\set[0]{w_{n_j}}$ convergent in $H^1(Ω)$. As $G$ is continuous, if $\set{u_n}$ is bounded then $\set{w_n}$ will be bounded too.

Finally, we need the ``a priori estimate'' of the fixed point theorem. But that part is easy: let $u ∈ H_0^1(Ω)$ be such that $u = λG(u) = λw$ with $0 ≤ λ ≤ 1$. In that case $u$, is solution of the problem $- Δu = λ(f - u^3)$ which is a well posed problem and $u$ is bounded.

Therefore, we are in the conditions of the \nref{thm:Fund:SchoeferFixedPoint}, there is a fixed point $u = G(u)$ and our problem has a solution.

\end{proof}

\subsection{Finite element approximation}

As usual, for $h > 0$ we will have our mesh \mesh of $Ω$, which will be a convex polygon domain. Then, we consider our finite element space $V_h = \spn \set{φ_1, \dotsc, φ_N}$ with $φ_i$ the shape functions corresponding to the internal vertices (we want the function to be zero in the boundary).

The Galerkin approximation given by finding $u_h ∈ V_h$ such that \[ \int_Ω (∇u_h ∇v_h + u_h^3 v_h) = \int_Ω f v_h \qquad ∀v_h ∈ V_h\]

Existence and unicity comes directly from the \nref{prop:NonLinearProblemExistenceUniqueness}, so we will not worry about that.

Now the question is how to compute $u_h$. The answer will be to use the Newton method. We want to find the solution $u_h ∈V_h$ such that \[ \dualp{F(u_h), v_h} ≝ \int_Ω ∇u_h ∇v_h + u_h^3 v_h - fv_h = 0\] so we can try to use the Newton method for solving nonlinear equations. Thus, for each step we will get the next iteration of the solution by solving \[ ∇F(u_h^n) (u_h^n - u_h^{n+1}) F(u_h^n)\] for $u_h^{n+1}$ which seems easy enough except for the fact that we do not know exactly what is the gradient of a functional.

For that, we will introduce the Fréchet derivative.

\begin{defn}[Fréchet derivative][Derivative\IS Fréchet] Let $\appl{F}{X}{X^*}$ be an application from a Banach space to its dual. It will be Fréchet differentiable at $u ∈ X$ if there exists a bounded linear operator $\appl{l_u}{X}{X^*}$ such that \[ \lim_{w \to 0} \frac{\norm{F(u + w) - F(u) - l_u(w)}_{X^*}}{\norm{w}_X} = 0\]

We then write $\Dif F(u) (w) = l_u (w)$ as the Fréchet derivative..
\end{defn}

We can now compute the Fréchet differential of $F$. Let $u,v,w ∈ H_0^1(Ω)$, so we can compute the duality product to see what's linear on $w$: \begin{align*}
\dualp{F(u+w) - F(u), v}
	&≝ \int_Ω ∇(u+w) ∇v + (u+w)^3 v - ∇u ∇v - u^3 v \\
	&= \underbracket{\int_Ω ∇w∇v + 3u^2wv}_{\dualp{l_u(w), v}} + \int_Ω 3uw^2v + w^3v
\end{align*} and we have a candidate for our differential. We have to check that the limit works.

First, recall that the norm of the dual is defined as $\norm{T}_{X^*} = \sup_{v ∈ X \setminus\set{0}} \frac{\dualp{X,v}}{\norm{v}} $ so that \begin{align*}
\norm{F(u + w) - F(u) - l_u(w)}_{(H_0^1(Ω))^*}
	&= \sup_{\substack{v ∈ H_0^1(Ω) \\ v ≠ 0}} \frac{\dualp{F(u+w) - F(u) - l_u(w), w}}{\norm{∇v}_{L^2(Ω)}} = \\
	&= \sup_{\substack{v ∈ H_0^1(Ω) \\ v ≠ 0}} \frac{\int_Ω 3uw^2v + w^3 v}{\norm{∇v}_{L^2(Ω)}}  \\
	&≤ \sup_{\substack{v ∈ H_0^1(Ω) \\ v ≠ 0}} \frac{C(3\norm{∇u} \norm{∇w}^2 + \norm{∇w}^3) \norm{∇v}}{\norm{∇v}} \\
	&≤ C(3\norm{∇u} \norm{∇w}^2 + \norm{∇w}^3) \convs[][w][0] 0
\end{align*} and therefore our Fréchet derivative is given by $\dualp{l_u(w), v} = \dualp{\Dif F(u)w, v} = \int_Ω ∇w ∇v + 3u^2 w v $.

Back to Newton now, for each step we will solve \[ \dualp{\Dif F(u_h^n) w_h, v_h} = \dualp{F(u_h^n), v_h} \] for $w_h^n$ and then update $u_h^{n+1} = u_h^n - w_h$. Practically, this means computing the linear problem \[ \int_Ω ∇w_h ∇u_h + 3u_h^2 w_h v_h = \int_Ω  ∇u_h^n ∇v_h (u_h^n)^3 v_h - f v_h\]

It can be seen that the left hand side is a coercive bilinear form and therfore the underlying matrix of coefficients of shape functions will be symmetric and positive definite, and this linear problem will be solved.

In practice, this Newton method is quadratic and the error at each stage is bounded by $e_{n+1} ≤ Ce_n^2$. The disadvantage is that the method only converges if the initial guess $u_h^0$is sufficiently close to $u_h$, which is an unknown solution. For this non linear problem it will usually work, but if we were to solve other, more complex nonlinear problems (such as Navier-Stokes with high Reynolds number) we could find that the method diverges or oscillates around a non-solution.

\subsection{A posteriori error estimates}

\begin{prop} Consider the non-linear problem \eqref{eq:PDE:NonlinearProblem} with the domain $Ω$ a convex polygon, $f ∈ L^2(Ω)$ and a regular mesh $\mesh$ (bounded aspect ratio, $\sfrac{h_K}{ρ_K} ≤ c$). Then, there exists a constant $C >0$ independent  of $h$, $f$ and $u$ but depending on the mesh aspect ratio such that \[ \norm{∇(u - u_h)}^2_{L^2(Ω)} ≤ C \sum_{K ∈ \mesh} η_K^2 \] with \[ η_K = h_K \norm{f + Δu_h - u_h^3}_{L^2(K)} + \frac{1}{2} \sqrt{h_K} \norm{[∇u_h · \vn]}_{L^2(∂K)} \]
\end{prop}

\begin{proof} As always, we start from the norm of the error and then add any $v_h ∈ V_h$ by the Galerkin orthogonality \begin{align*}
\int_Ω \abs{∇(u - u_h)}^2
	&= \int_Ω f(u - u_h) - u^3(u - u_h) - ∇u_h ∇(u - u_h) \\
	&= \int_Ω (f - u_h^3)(u - u_h) - ∇u_h ∇(u - u_h) - (u^3 - u_h^3) (u - u_h) \\
	&= \int_Ω (f - u_h^3)(u - u_h - v_h) - ∇u_h ∇(u - u_h - v_h) - \underbracket{\int_Ω (u^3 - u_h^3) (u - u_h)}_{≥ 0} \\
	&≤ \sum_{K ∈ \mesh} \left( \int_K (f - u_h^3)(u - u_h -v_h) + \frac{1}{2} \int_{∂K} [∇u_h · \vn] (u - u_h - v_h)\right)
\end{align*}
where we already saw the positiveness of $\int_Ω (u^3 - u_h^3)(u - u_h)$ in the uniqueness proof of \fref{prop:NonLinearProblemExistenceUniqueness}, so we can get rid of it and continue as usual.

Now we choose $v_h = R_h(u - u_h)$ the Clément interpolant and we get as always the result.
\end{proof}



\chapter{Mixed problems}
\ifincludefirstsemester
\section{Introduction and examples}

Until now, we have studied problems with one differential operator of the form $L u = f$ with some boundary conditions. However, some PDE problems require more operators. One example that we will see with more detail later is the Darcy equation, which models the flow of a viscous fluid in a porous medium.

The main equation is given by conservation of mass: $\dv \vu = f$, where $\vu$ is the flow and $f$ a function giving sources and sinks in the system. However, we have also a constraint on the flow: it must be proportional to the gradient of a pressure function $p$, so $\vu = -k ∇ p$.

As we will see, these kinds of equations can be adapted to the following weak abstract formulation.

\begin{defn}[Weak\IS abstract formulation of a mixed PDE problem] The weak abstract formulation of a PDE problem is given by two Hilbert spaces $V,Q$ and two bilinear forms \begin{align*}
\appl{a}{V×V&}{ℝ} \\
\appl{b}{V×Q&}{ℝ}
\end{align*}

The problem is then finding $(u,p) ∈ V×Q$ such that \(
\begin{aligned}
a(u,v) + b(v,p) &= F(v) \quad ∀v ∈ V\\
b(u,q) &= G(q) \quad ∀q ∈ Q
\end{aligned} \label{eq:MixedProblemFormulation}
\)
\end{defn}

If we have the condition $G = 0$, we can define the space \[ V_0 = \set{v ∈ V \tq b(v,q) = 0 \; ∀q ∈ Q} \] which translates the problem to the usual abstract formulation of finding a $u ∈ V_0$ such that $a(u,v) = F(v)$ for any $v ∈ V_0$.

This type of construction will be useful when studying elliptic equations with constraints, such the incompressible elasticity equation. The problem will come on how to construct the approximation space to $V_0$. That is why sometimes we will nevertheless work on the unconstrained problem.

\begin{example}[Incompressible linear elasticity] In this case, we have a displacement field $\appl{\vu}{Ω}{ℝ^d}$. The strain tensor is the symmetric part of the gradient, so that \[ ε(\vu) = \frac{∇\vu + \trans{∇\vu}}{2} \] and the stress tensor will be just linear on the strain if we assume linear elasticity: \[ σ_{ij}(\vu) = C_{ijkl} ε_{kl} \]

For an isotropic equation, this tells us that the stress tensor is \[ σ(\vu) = 2με(\vu) + λ\tr(ε(\vu)) I\]

We have the following balance equations for the boundary conditions:
\begin{align*}
\dv σ(\vu) &= \vec{f} \quad\text{in }Ω \\
\vu &= \vec{g} \quad\text{on } Γ_D \\
σ(\vu) \vn &= \vd
\end{align*}

The weak formulation requires the definition of the space $V_g = \set{\vv ∈ (H^1)^d \tq \restr{\vv}{Γ_D} = \vec{g}}$, and thus the problem is finding a $\vu ∈ V_g$ such that \[ \int_Ω 2μ\left(ε(\vu) \colon ε(\vv)\right) + λ \dv \vu \dv \vv = \int_Ω \vec{f} \vv + \int_{Γ_N} \vd \vv \qquad ∀\vv ∈ V_0\]

The question is what happens if we have an incompressible material, that is, with $λ \to ∞$ (λ is the bulk modulus). What we do is formulate a variational problem: we define the energy of a solution as \[ J_λ(\vu) = \frac{1}{2} \int_{Ω} 2μ ε(\vu) \colon ε(\vu) + λ(\dv \vu)^2 - \int_Ω \vec{f} \vu - \int_{Γ_N} \vd \vu \]

Solving the minimization problem of $J_λ$ (that is, finding $\argmin_{\vu ∈ V_g} J_λ(\vu)$) is equivalent to the weak formulation presented above.

In that case, if $λ \to ∞$ the only way to bound the energy functional is to have $\dv \vu = 0$, so we will search then for the minimization function $\vu$ of the functional $J_0$ with $\vu ∈ V_g$ \textit{and} $\dv \vu = 0$, with \[ J_\text{inc}(\vu) = \frac{1}{2} \int_{Ω} 2μ ε(\vu) \colon ε(\vu) - \int_Ω \vec{f} \vu - \int_{Γ_N} \vd \vu  \]

To enforce the constraint, we can use Lagrange multipliers. The lagrangian will be then \[ \mathcal{L}(\vu,p) = J_0(\vu) - \int_Ω \vec{p} \dv \vu  \] with $\vec{p}$ being the Lagrange multiplier, and thus the problem is finding \[ \argmin_{\vu ∈ V_g} \max_{\vec{p} ∈ ?} \mathcal{L}(\vu, p) \] with the space of the Lagrange multiplier pending to define. To solve this, we write the ``derivatives'' and make them equal to 0 \begin{align*}
\dpd{\mathcal{L}}{\vu}(\vv) &= \lim_{ε \to 0} \frac{\mathcal{L}(\vu + ε \vv, p) - \mathcal{L}(\vu, p)}{ε} = \\ &= \int_Ω 2μ ε(\vu) \colon ε(\vv) - \int_Ω p \dv \vv - \int_Ω \vec{f} \vv - \int_{Γ_N} \vd \vv = 0 \\
\dpd{\mathcal{L}}{p}(q) &= \lim_{ε \to 0} \frac{\mathcal{L}(\vu, p  + ε q) - \mathcal{L}(\vu, p)}{ε} = \\ &= \int_Ω q \dv \vu = 0
\end{align*}

In the first case, the variation should be $0$ on the boundary so that $v ∈ V_0$, and in the second case we will need only $q ∈ L^2(Ω) = Q$ given that $\dv \vu$  is square-integrable.

So, finally, the constrained problem is finding $(u,p) ∈ V_g × Q$ such that \begin{align*}
\int_Ω 2μ ε(\vu) \colon ε(\vv) - \int_Ω p \dv \vv &= \int_Ω \vec{f} \vv + \int_{Γ_N} \vd \vv & ∀v ∈ V_0 \\
\int_Ω q \dv \vu &= 0 & ∀q ∈ Q
\end{align*}

THe trick is that we can adapt this to the mixed formulation of \eqref{eq:MixedProblemFormulation}, defining $a(\vu, \vv) = \int_Ω 2μ ε(\vu) \colon ε(\vv)$ and $b(\vv, p) = \int_Ω p \dv \vv$ (we only have to change the sign on the second equation but that does not change the equation).
\end{example}

\begin{example}[Nearly incompressible elasticity] Now on to the partial case: what happens if we have an object which is nearly incompressible? The issue is that if we have a very large value of λ (but not ∞) the error estimate constants blow up completely and we have a poor estimate. What we do in that case is to constrain $p = -λ \dv \vu$ and then the problem becomes \begin{align*}
\int_Ω 2μ ε(\vu) \colon ε(\vv) - \int_Ω p \dv \vv &= \int_Ω \vec{f} \vv + \int_{Γ_N} \vd \vv & ∀v ∈ V_0 \\
- \int_Ω q \dv \vu  -\underbracket{\frac{1}{λ} \int_{Ω} pq}_{c(p,q)} &= 0 & ∀q ∈ Q
\end{align*} which is the same problem as before but adding another term $c(p,q)$, which disappears when $λ \to ∞$, and so the solution will be robust when λ is very large.
\end{example}

\begin{example}[Flows in porous media] \label{exm:PDE:Darcy} To define the equations, we use Darcy's law: $\vu = -k ∇p$ with $p$ the pressure and $k$ the permeability. Also, fluid cannot disappear, so we have to force that, for any small domain ω, we have $\int_{∂ω} \vu \vn = \int_ω f$ where the left hand side is the fluid that gets in/out of the domain, and the right hand side is the sources or sinks in that domain. This gives us the following model:
\( \begin{aligned}
\dv \vu &= f \\
\vu &= -k∇p
\end{aligned} \label{eq:DarcyLaw} \)

We will also force that on the impervious boundaries $Γ_D$ we have no fluid transfer, that is, $\vu \vn = 0$ on $Γ_D$.

A first approximation would be to replace the velocity field and set $-\dv (k∇p) = f$ and the other boundary conditions, but that approach has issues. Mainly, that if we solve for $p$, the gradient will be a worse approximation and, specifically, if we have $f = 0$  (conservation of mass) then the approximate solution will not exactly conserve mass, which is definitely not very good. So, we need another approach deriving directly the weak formulation of \ref{eq:DarcyLaw}. For the divergence, we get directly by integration and multiplying by a test function $q$ so that \[ \int_Ω q \dv \vu = \int_Ω fq \]

In the other equation $\vu = -k∇p$, we use a test vector function $\appl{\vv}{Ω}{ℝ^d}$ and then integrate by parts:
\begin{align*}
\frac{1}{k} \vu &= - ∇p \\
\int_{Ω} \frac{1}{k} \vu \vv &= - \int_Ω ∇p \vv \\
\int_{Ω} \frac{1}{k} \vu \vv &= - \int_{Ω} p \dv \vv - \underbracket{\cancelto{0}{\int_{Γ_D} p \vv \vn}}_{\vv \vn = 0 \text{ on } Γ_D} - \int_{Γ_N} \vv \vn
\end{align*}

With this, we end with the mixed problem formulation given by
\(
\begin{aligned}
\underbracket{\int_{Ω} \frac{1}{k} \vu \vv }_{a(\vu, \vv)} + \underbracket{\left(- \int_Ω p \dv \vv \right)}_{b(\vv, p)} &= \underbracket{\int_{Γ_N} d \vv \vn}_{F(\vv)} \\
\underbracket{-\int_Ω q \dv \vu}_{b(\vu,q)} &= \underbracket{- \int_Ω fq}_{G(q)}
\end{aligned} \label{eq:DarcyLawWeakForm}
\)

As always, we need to know the spaces for $\vu, \vv$. We only need control on the divergence, so we can define \[ H(\dv, Ω) = \set{\appl{\vv}{Ω}{ℝ^d} \tq \vv ∈ (L^2(Ω))^d,\, \dv \vv ∈ L^2(Ω)} ⊃ (H^1(Ω))^d \] which is a smaller space than $(H^1(Ω))^d$ and with norm \( \norm{\vv}_{H(\dv, Ω)} = \norm{\vv}^2_{L^2(Ω)} + \norm{\dv \vu}^2_{L^2(Ω)} \label{eq:PDE:HDivNorm} \)

And, luckily for us, in this space we can define the trace operator as \begin{align*}
\appl{\tr}{H(\dv, Ω)&}{\left(H^{\sfrac{1}{2}}(∂Ω)\right)' = H^{-\sfrac{1}{2}}(∂Ω)} \\
\vv &\longmapsto \restr{\vv · \vn}{∂Ω}
\end{align*} with $H^{-\sfrac{1}{2}}(∂Ω)$ the dual space of $H^{\sfrac{1}{2}}(∂Ω)$. The trace operator is bounded \[ \norm{\vv · \vn}_{H^{-\sfrac{1}{2}}(∂Ω)} ≤ C \norm{\vv}_{H(\dv, Ω)} \]

This allows us to define the test function space $V_g = \set{\vv ∈ H(\dv, Ω) \tq \restr{\vv·\vn}{Γ_D} = g}$, and then we need to find $u ∈ V_g$ and $p ∈ L^2(Ω)$ to solve \eqref{eq:DarcyLawWeakForm}.
\end{example}

\section{Theoretical framework for the mixed problem}

So, after the examples, we will define the theorem that gives us the sufficient conditions for well-posedness of the mixed problem.

\begin{theorem}[Theorem\IS of existence and uniqueness for the mixed PDE problem][Existence and uniqueness\IS for mixed PDEs] \label{thm:PDE:WellPosednessMixedProb} Let $V,Q$ be two Hilbert spaces and $V_0 = \set{v ∈ V \tq b(v,q) = 0 \; ∀q ∈ Q}$. Assume that:
\begin{itemize}
\item $\appl{a}{V×V}{ℝ}$ is a continuous (continuity bounding constant $M$) and coercive on $V_0$ ($∃ α > 0$ s.t. $a(u, u) ≥ α \norm{u}_V^2$ for any $v ∈ V_0$) bilinear form.
\item $\appl{b}{V×Q}{ℝ}$ is a continuous (continuity bounding constant $γ$) bilinear form and with an inf-sup condition: $∃β > 0$ such that \( \inf_{q∈Q}\sup_{v∈V} \frac{b(v,q)}{\norm{v}_V \norm{q}_Q} ≥ β > 0 \label{eq:PDE:InfSup} \) which is equivalent to saying that for any $q ∈ Q$ we have \[ \norm{q}_Q ≤ \frac{1}{β} \sup_{v ∈ V} \frac{b(v,q)}{\norm{\vv}_V} \]
\item $\appl{F}{V}{ℝ}$ and $\appl{G}{Q}{ℝ}$ are linear and bounded forms.
\end{itemize}

Under these conditions, there exists an unique solution $(u,p) ∈ V × Q$ for the problem \( \begin{aligned}
a(u,v) + b(v,p) &= F(v) \quad ∀v ∈ V\\
b(u,q) &= G(q) \quad ∀q ∈ Q
\end{aligned} \label{eq:PDE:MixedProblem}\) and there is a constant $C > 0$ such that \[ \norm{u}_V + \norm{p}_Q ≤ C\left(\norm{F}_{V'} + \norm{G}_{Q'}\right)\]
\end{theorem}

The inf-sup condition is actually very interesting, because it gives us the first introduction to the proof by forcing some properties on the following operator:
\( \begin{aligned}
\appl{\trans{B}}{Q&}{V'} \\
q &\longmapsto \pesc{\trans{B}q, ·} = b(·, q)
\end{aligned} \label{eq:PDE:MixedBTOperator} \)

The main property it forces is bijectivity with the image of its operator. This is important because for the proof we will try to get information on $u$ from $b(u,q) = G(q)$ and use that to solve $a(u,v) + b(v,p) = F(v)$, so it is important to have some kind of notion of the ``inverse'' of that operator $B$.

\begin{lemma} \label{lem:PDE:BijectivityOpBT} Let $\appl{\trans{B}}{Q}{V'}$ as defined on \eqref{eq:PDE:MixedBTOperator} and satisfying the inf-sup condition \eqref{eq:PDE:InfSup} . Then, $\trans{B}$ is bijective with its image, or equivalently:
\begin{enumerate}
	\item $\ker \trans{B} = \set{0}$ (or $\trans{B}$ is injective).
	\item $\img \trans{B}$ is closed.
\end{enumerate}
\end{lemma}

\begin{proof} Proofs are mostly straightforward for the two claims.

\proofpart{$\ker\trans{B} = \set{0}$}

As $b$ is continuous and fulfills the inf-sup condition, we can write \( β\norm{q} ≤ \norm{\trans{B}q}_{V'} ≤ γ \norm{q} \qquad ∀q ∈ Q  \) for two constants $β,γ > 0$. Now, let $q ∈ \ker \trans{B}$. In that case, $β\norm{q} ≤ \norm{\trans{B}q}_{V'} = 0$ so $\norm{q} = 0$ and $q = 0$.

\proofpart{$\img \trans{B}$ is closed.}

Take a Cauchy sequence in the image, then there is a corresponding Cauchy sequence of the preimages which has a limit because $Q$ is Banach. As $\trans{B}$ is continuous, the image of that limit is the limit of the Cauchy sequence in the image.
\end{proof}

However, $\trans{B}$ is not exactly the operator we want: we want its transpose. So let's define the operator \( \begin{aligned}
\appl{B}{V&}{Q'} \\
v &\longmapsto \pesc{Bv, ·} = b(v, ·) \end{aligned} \label{eq:PDE:MixedBOperator} \) which is continuous because $b$ is continuous.

What's the usefulness of starting by the transpose? Now we can apply the \nref{thm:Fund:ClosedRange}. We have that $\img \trans{B}$ is closed and as $\ker \trans{B} = \set{0}$, we have by the theorem $(\ker \trans{B})^\perp = Q' = \img B$ and $\left(\ker B\right)^\perp = \img \trans{B}$. This allows us to enunciate the following lemma:

\begin{lemma} \label{lem:PDE:BijectivityOpB} Let $\appl{B}{V}{Q'}$ as defined in \eqref{eq:PDE:MixedBOperator}. $B$ is surjective, $\restr{B}{\left(\ker B\right)^\perp}$ is bijective and there exists a constant $β > 0$ such that \[ \norm{Bu}_{Q'} ≥ β \norm{u}_V \qquad ∀u ∈ \left(\ker B\right)^\perp  \]

Moreover, there is a continuous inverse $\appl{\inv{B}}{Q'}{\left(\ker B\right)^\perp}$ with $\norm{\inv{B}} ≤ \sfrac{1}{β}$.
\end{lemma}

With this, we can prove the existence theorem.

\begin{proof}[\Fref{thm:PDE:WellPosednessMixedProb}]  As discussed previously, we will first try to invert $b(u,q) = G(q)$ to solve the first condition, and then we will retrieve $p$ from that solution.

\proofpart{Existence and uniqueness of $u$}

As $G ∈ Q'$, there exists its inverse via $B$ as defined in \eqref{eq:PDE:MixedBOperator}. Let $u_G = \inv{B} G ∈ V$. By \fref{lem:PDE:BijectivityOpB}, we know that $\norm{u_G} ≤ \frac{1}{β} \norm{G}$. Define $u_0 = u - u_G$. It is clear that $u_0 ∈ V^0$, a space we defined in the theorem as $V^0 = \set{v ∈ V \tq b(v,q) = 0 \; ∀q ∈ Q}$. We can then reformulate the first equation of \eqref{eq:PDE:MixedProblem} as \[ a(u_0, v) + b(v,p) = F(v) - a(u_G, v) \quad ∀v ∈ V \]

This is an elliptic coercive problem if we test for $v ∈ V^0$, so by \nref{thm:Theory:LaxMilgram} it has an unique solution with bound \[ \norm{u_0} ≤ \frac{1}{α} \sup_{v ∈ V^0} \frac{F(v) - a(u_G, v)}{\norm{v}} ≤ \frac{1}{α} \left(\norm{F} + M \norm{u_G} \right) \] and the solution $u$ is then bounded.

\proofpart{Recovery of $p$}

Now that we know $u$, we can recover $p$ which must fulfill \[ b(v,p) = F(v) - a(u,v) ≝ \tilde{F}(v) \quad ∀ v ∈ V \] or equivalently $\trans{B}p = \tilde{F}$.

If $v ∈ V^0$, then clearly $\tilde{F}(v) = 0$ so everything holds.\footnote{I mixed a little bit spaces and their orthogonal but everything is ok.}

\end{proof}

\subsection{General problem on Banach spaces}

So far, we have looked at equations $a(u,v) = F(v)$ with $u,v$ in Hilbert spaces. However, the inf-sup condition is much more well suited to define the well-posedness of the problem. Thus, we can work with $u,v$ in only Banach spaces.

We can write this in operator form. If we `freeze' $u ∈ V$ and let $v ∈ W$ vary, then $a(u,v) = \pesc{Au,v} $ with $\appl{A}{V}{W'}$. In the other direction, we have $\pesc{u, \trans{A}v} = a(u,v)$ with $\appl{\trans{A}}{W}{V'}$ the adjoint operator. In this expression, the inf-sup conditions imply the injectivity and surjectivity of the operator $A$ and its adjoint, and then the problem becomes finding an equality of functionals $A u = F ∈ W'$: if we have \[ \inf_{v∈W}\sup_{u∈V} \frac{a(u,v)}{\norm{u}_V \norm{v}_W} ≥ α > 0\] then we have $\ker \trans{A} = \set{0}$, $\img \trans{A}$ closed and $A$ surjective. If we additionally ask for the inverse inf-sup condition \[ \inf_{u∈V}\sup_{v ∈ W} \frac{a(u,v)}{\norm{u}_V \norm{v}_W} ≥ α > 0\] then we have unique solution $u$.

Even though these conditions are more general than the ones in the Lax-Milgram lemma, the problem is that they are far more difficult to prove than coercivity.

In the symmetric setting ($V = W$, $A = \trans{A}$) then only one check is necessary.

\begin{example} We can try to apply this to the Stokes elasticity problem. In the weak form, its equations were \( \label{eq:StokesElasticity} \begin{cases} \int ∇u ∇v - \int p \dv v = F(v) \\ \int q \dv u + ε \int pq = G(q) \end{cases} \)
, which can be simplified as a single bilinear form \[ A\left((u,p), (v,q)\right) = \int ∇u ∇v -\int p \dv v + \int q \dv u + ε \int pq \] and then check here the inf-sup condition, setting the norm of the tuple as $\norm{(u,p)}^2 = \norm{u}_V^2 + \norm{p}_Q^2$.

Given that $A$ is symmetric, we only have to prove that for any $(v,q)$ we have \[ \sup_{(u,p)} \frac{A\left((u,p), (v,q)\right) }{\norm{(u,p)}} ≥ α \norm{(v,q)} \], and for that it suffices to find a single $(u,p)$ for any $(v,q)$ for which $\frac{A\left((u,p), (v,q)\right) }{\norm{(u,p)}} ≥ α \norm{(v,q)}$.

Here, we would want to use $(u,p) = (v,q)$ which would give us however a little bit of bad control when $ε \to 0$, although then we can control it by something of the laplacian that was in the exercise and $q$ was like the gradient and something like that.
\end{example}

\section{Galerkin approximation}

Once we have the theoretical framework for the mixed problems, we will start working with the finite approximation and numerical solution model. Recall the general problem: find $(u,p) ∈ V×Q$ such that \begin{align*}
a(u,v) + b(v,p) &= F(v) \quad ∀v ∈ V \\
b(u,q) &= G(q) \quad ∀q ∈ Q
\end{align*}

To apply here the Galerkin approximation, we will define finite subspaces $V_h⊂V$ and $Q_h ⊂ Q$ with the approximation property that when $h \to 0$ they become ``dense'' in their respective parent spaces. More formally, we want for any $v ∈ V$ to have \[ \lim_{h \to 0} \inf_{v_h ∈ V_h} \norm{v - v_h} = 0\]  and the same for $Q$.

Once the spaces are defined, the Galerkin approximation problem will be to find $(u_h, v_h) ∈ V_h × Q_h$ such that \( \begin{aligned}
a(u_h,v_h) + b(v_h,p_h) &= F(v_h) \quad ∀v_h ∈ V_h \\
b(u_h,q_h) &= G(q_h) \quad ∀q_h ∈ Q_h
\end{aligned} \label{eq:GalerkinMixedProblem}\)

From here, we have to questions to answer: whether this approximated problem is well posed and how does it converge to the real solution when $h \to 0$.

\subsection{Well-posedness of the Galerkin approximation}

We will start, of course, by assuming that the original problem is well-posed and that $a,b$ fulfill the conditions of \fref{thm:PDE:WellPosednessMixedProb}. We will not have any problem with linearity and continuity as we are working on linear, closed subspaces. However, the coercivity of $a$ on $V_0$ and the inf-sup condition of $b$ may not necessarily hold.

We start by the inf-sup condition, which told us that \[ ∀q ∈ Q \quad \sup_{v ∈ V} \frac{b(v,q)}{\norm{v}_V} ≥ β \norm{q}_Q \]

It is easy to see that restricting $q$ to $Q_h$ maintains this relationship. However, if we restrict $v$ to $V_h$ we might lose that supremum and lose then the property. We will have to choose $V_h$ and $Q_h$ carefully so that restricting on $Q_h$ removes possible problems on that inequality. The discrete condition is the following \( ∀q_h ∈ Q_h \quad \sup_{v_h ∈ V_h} \frac{b(v_h, q_h)}{\norm{v_h}_{V_h}} ≥ β_h \norm{q_h}_{Q_h} \quad ∀ h > 0 \)

In particular, we will not be able to choose small spaces $V_h$ as in those we will have less ``options'' to find some $v$ that fulfills the condition.

What happens, however, if the condition is not satisfied? Then the operator $\trans{B}$ induced by $b$ is not injective, i.e. there exists a $q_h^*$ such that $b(v_h, q_h^*) = 0$ and $\ker \trans{B} ≠ \set{0}$. But then, if $(u_h, p_h)$ is a solution to \eqref{eq:GalerkinMixedProblem} then $(u_h, p_h + αq_h^*)$ is also a solution. This means we might have non-uniqueness or even non-existence if something happens with the second equation and the image of $G$ that completely passed over my head.

This problems tends to happen a lot in reality. Solutions like $q_h^*$ are called spurious solutions that are usually oscillatory. Thus, we will need to explicitly require the inf-sup condition at the discrete level.

The other problem was the coercivity on $V_0$ of $a$. However, if we have coercivity on the full space $V$, we will have coercivity too on any subspaces including $V_h$ and $V_{h,0} = \set{v_h ∈ V_h \tq b(v_h,q_h) = 0 \; ∀q_h ∈ Q_h}$ (the analogous space for the discrete setting).

For some PDE problems, such as the Stokes elasticity problem, coercivity on the full space $V$ will not be too much to ask and will come for free. However, in other problems such as the Darcy law, we only have control of the divergence on the restricted space and then we can't prove coercivity on the full space.

This prompts the question of what happens when $a$ is only coercive on $V_0$. We might be able to prove that $V_{h,0} ⊂ V_0$ so we have no problems, or maybe we are not able and then we have to prove specifically coercivity on the restricted discrete subspace. That depends heavily on the type of discretization used.

With all of this discussion, we can prepare the theorem for well-posedness\footnote{Frigging love existence and uniqueness theorems.}:

\begin{theorem} \label{thm:WellPosednessMixedGalerkin} Assume we start from a well-posed mixed PDE problem. Then, consider the Galerkin approximating problem of finding $(u_h, p_h) ∈ V_h × Q_h$ such that \[ \begin{aligned}
a(u_h,v_h) + b(v_h,p_h) &= F(v_h) \quad ∀v_h ∈ V_h \\
b(u_h,q_h) &= G(q_h) \quad ∀q_h ∈ Q_h
\end{aligned} \]

Assume that $a$ is coercive on $V_{h,0}$ \[ ∃ α_h > 0 \; a(v_h, v_h) ≥ α_h \norm{v_h}_V^2 \; ∀v_h ∈ V_{h,0} \] and $b$ satisfies the discrete inf-sup condition \[ ∃β_h > 0 \; \inf_{q_h ∈Q_h} \sup_{v_h ∈ V_h} \frac{b(v_h, q_h)}{\norm{v_h}_V \norm{q_h}_Q} ≥ β_h\]

Then, there exists an unique solution $(u_h, p_h) ∈ V_h × Q_h$ of the approximating problem and $\norm{u_h}_V + \norm{p_h}_Q ≤ C(\norm{F}_{V'} + \norm{G}_{Q'})$.
\end{theorem}

The inf-sup condition is considerably important as it gives us the uniqueness of the solution. If it does not hold, then there exists a $0 ≠ q_h^* ∈ Q_h$ such that $b(v_h, q_h^*) = 0\;∀v_h ∈ V_h$; so if $(u_h, p_h) ∈ V_h × Q_h$ is a solution then $(u_h, p_h + q_h^*) ∈ V_h × Q_h$ is another solution too.

\subsection{Convergence of the solution}

Next step, as always, is ensuring that we have good convergence conditions on the solution when we approximate well with finite element spaces.

\begin{theorem} \label{thm:WellPosednessMixedGalerkinConvergence} In the same conditions of \fref{thm:WellPosednessMixedGalerkin}, we can show that \[ \norm{u-u_h}_V + \norm{p - p_h} ≤ C\left(\inf_{v_h ∈ V_h} \norm{u - v_h}_V + \inf_{q_h ∈ Q_h} \norm{p - q_h}_Q \right) \] with $(u,p)$ a solution of the non-approximate problem.
\end{theorem}

\begin{proof} To prove this theorem, we will try to bound the $\norm{u_h - w_h}$ and $\norm{p_h - π_h}$ and then conclude by triangular inequality.

For the first one, let $w_h ∈ V_h$ such that $b(w_h, q_h) = G(q_h)$ for any $q_h ∈ Q_h$. In that case, we have $u_h - w_h ∈ V_{h,0}$ and there we have coercivity, so we can write \[ \norm{u_h - w_h}^2 ≤ \frac{1}{α_h}a(u_h - w_h, u_h - w_h)\]

Adding and substracting $u$ in the arguments of $a$; we have that
\begin{multline*} \frac{1}{α_h}a(u_h - w_h, u_h - w_h) = \frac{1}{α_h}\left(a(u_h - u, u_h - w_h) + a(u -w_h, u - w_h)\right) = \\ = \frac{1}{α_h} \left(F(u_h, w_h) - b(u_h - w_h, p_h) - F(u_h - w_h) + b(u_h - w_h, p) + a(u - w_h, u_h - w_h)\right)\end{multline*}

However, given that $u_h - w_h ∈ V_{h,0}$, we have $b(u_h - w_h, p_h) = 0 = b(u_h - w_h, π_h)$ so the equation becomes \begin{align*}
\norm{u_h - w_h}^2 &= \frac{1}{α_h} \left(F(u_h, w_h) - b(u_h - w_h, p_h) - F(u_h - w_h) + b(u_h - w_h, p) + a(u - w_h, u_h - w_h)\right) \\
	&= \frac{1}{α_h}\left(b(u_h - w_h, p - π_h) + a(u-w_h, u_h - w_h)\right) \\
	&≤ \frac{1}{α_h}\left(γ \norm{p - π_h}_Q \norm{u_h - w_h}_V + M \norm{u -w_h}\norm{u_h - w_h}\right) \\
\norm{u - u_h}_V &≤ \frac{γ}{α_h} \norm{p - π_h}_Q + \frac{M}{α_h}\norm{u -w_h}
\end{align*}

With this, we can bound $\norm{u - u_h}$ as $w_h, π_h$ are arbitrary so that \begin{align*}
\norm{u -u_h}_V &≤ \norm{u - w_h} + \norm{u_h - w_h} \\
&≤ \left(1 + \frac{M}{α_h}\right) \norm{u - w_h} + \frac{γ}{α_h} \norm{p - π_h} \\
\norm{u - u_v}_V &≤ \left(1 + \frac{M}{α_h}\right)\inf_{w_h ∈ V_h} \norm{u - w_h} + \frac{γ}{α_h} \inf_{π_h ∈ Q_h} \norm{p - π_h}
\end{align*} so we are done.

Now we need to recover the bound on $p$ so we have \begin{align*}
\norm{p - π_h} &≤ \frac{1}{β_h} \sup_{v_h ∈ V_h} \frac{b(v_h, p_h - π_h)}{\norm{v_h}} \\
&≤ \frac{1}{β_h} \sup\frac{b(v_h, p_h - p)}{\norm{v_h}} + \frac{1}{β_h} \underbracket{\sup \frac{b(v_h, p - π_h)}{\norm{v_h}}}_{≤ γ\norm{p - π_h}} \\
&≤ \frac{1}{β_h} \sup \frac{a(u - u_h,v_h)}{\norm{v_h}} + \frac{γ}{β} \norm{p - π_h} \\
&≤ \frac{M}{β_h} \norm{u - u_h} + \frac{γ}{β_h} \norm{p - π_h}
\end{align*} and then we have the triangular inequality \[ \norm{p - p_h} ≤ \norm{p - π_h} + \norm{p_h - π_h} ≤ \dotsb ≤ C\left(\inf_{w_h} \norm{u - w_h} + \inf_{π_h ∈ Q_h} \norm{p-π_h} \right)\]

We need to take care however that $w_h$ fulfills that $b(w_h, q_h) = G(q_h)$, which is a little bit annoying because we can't take the interpolant to bound $\inf_{w_h} \norm{u - w_h}$. So, let $V_h^G = \set{w_h ∈ V_h \tq b(w_h, q_h) = G(q_h) \; ∀q_h ∈ Q_h}$, and the missing step is to prove that if $b$ satisfies the discrete inf-sup condition then $\inf_{w_h ∈ V_h^G} \norm{u - w_h} ≤ C \inf_{v_h ∈ V_h} \norm{u - v_h}$.
\end{proof}

\section{Linear elasticity problem}

We are going to apply now the previous theory to the linear elasticity theory. We will have our domain $Ω ⊂ ℝ^d$, and our two Hilbert spaces $V = \left[H_{Γ_D}^1(Ω)\right]^d$ ($d$-dimensional $H^1$ functions on $Ω^d$ that are $0$ on the Dirichlet boundary $Γ_D ⊂ ∂Ω$) and $Q = L^2(Ω)$. The bilinear forms are \begin{align*}
a(u,v) &= 2 \int_Ω με(\vu) \colon ε(\vv) \dif x \\
b(v,p) &= \int_Ω p \dv \vv \dif x \\
F(v) &= \int_Ω fv \dif x + \int_{Γ_N} \vd · \vn \dif x \\
G(q) &= 0
\end{align*} where \[ ε(u) = \frac{∇\vu + \trans{∇\vu}}{2} \qquad A \colon B = \tr(A \trans{B}) \]

The untrusting reader should check that $a$ is continuous and coercive, $b$ is continuous and $F$ is bounded. These are not difficult and can be seen directly. However, the proof of the inf-sup condition is a little bit more ``interesting''. To make our lifes easier, we will assume that we don't have any boundary conditions, and that our domain is convex and with Lipschitz boundary.

Now, for any $p ∈ L^2(Ω)$, let $\appl{ψ}{Ω}{ℝ}$ of the following PDE: \( \begin{cases} Δψ = p & \text{ in } Ω \\ ψ = 0 & \text{ on } ∂Ω \end{cases} \label{eq:Mixed:AuxPDEPressure} \)

We know that \eqref{eq:Mixed:AuxPDEPressure} has an unique solution of regularity $H^2$ by the \nref{thm:PDE:Shift} and bounded with $\norm{ψ}_{H^2(Ω)} ≤ C^* \norm{p}_{L^2(Ω)}$. Set now $\vv ≝ ∇ψ ∈ V$, so that $\dv \vv = p$ with bound $\norm{\vv}_V ≤ C^* \norm{p}_Q$.

With this, we can start prove the inf-sup condition: \[ \frac{b(\vv,p)}{\norm{\vv}_V} = \frac{\int_Ω \dv \vv p \dif x }{\norm{\vv}_V} = \frac{\norm{p}^2_Q}{\norm{v}_Q} ≥ \frac{1}{C^*} \norm{p}_Q\]

\subsection{Numerical approximation}
\label{sec:PDE:NumericalApproximationMixedProblems}

Once we know that the continuous problem has an unique solution, we look at the Galerkin approximation problem of finding $(u_h, p_h) ∈ V_h × Q_h$, with $V_h, Q_h$ finite approximation spaces, such that \[ \begin{aligned} a(u_h, v_h) + b(v_h, p_h) &= F(v_h) & ∀ v_h ∈ V_h \\ b(u_h, q_h) &= 0 & ∀q_h ∈ Q_h \end{aligned} \]

We will need to define the space of functions that fulfill the second equation: \[ V_h^0 = \set{v_h ∈ V_h \tq b(v_h, q_h) = 0 \; ∀q_h ∈ Q_h} \]

For the finite element spaces, we can choose piecewise continuous polynomials of degree $r$ (space $X_h^r$) to approximate $V$, and piecewise non-necessarily continuous polynomials of degree $r$ (space $Y_h^r$) to approximate $Q$ (it's just $L^2$ so we don't need regularity).

However, choosing the domains is not trivial. $X_h^0, Y_h^1$ do not satisfy the inf-sup condition, and in general $X_h^r, Y_h^r$ do not satisfy it because of the same degree.

One polynomial space that will be useful for constructing approximation spaces with the inf-sup condition will be bubble spaces.

\begin{defn}[Bubble space][Space!bubble] Given a triangular mesh element $K ⊂ ℝ^d$, we can define a bubble space by defining $φ_K ∈ \pspace[d + 1](K)$ such that $\restr{φ_K}{∂K} = 0$. Then, adding this element to the base of $\pspace[1](K)$ gives $\pspace[1]^b(K)$, the bubble space.

This space allows the recovery of the inf-sup condition but does not improve accuracy, which is only first order accurate.
\end{defn}

In general, however, the combination of spaces $\pspace/\pspace[r -1]$ gives the inf-sup condition and a $r$\textsuperscript{th} order approximation.
\fi
% -*- root: ../NumericalApproximationofPDEs.tex -*-
\section{Stokes problem}

Once we know the theory, we are going to apply it to the study of the Stokes problem and its error estimates. Recall that the Stokes problem is the following:
\(
\begin{cases}
-Δ \vu + \grad p = \vf & \text{in } Ω \\
\dv \vu = 0 & \text{in } Ω \\
\vu = 0 & \text{on } ∂Ω
\end{cases} \label{eq:PDE:StokesProblem} \)

The weak formulation of this is finding $u ∈ H_0^1(Ω)^2$, $p ∈ L_0^2(Ω)$ such that
\( \int_Ω ∇\vu : ∇\vv - \int_Ω p \dv \vu - \int_Ω \dv \vu q  = \int_Ω \vf \vv \) for all functions $\vv ∈ H_0^1(Ω)^2$ and $q ∈ L_0^2(Ω)$ and we can prove the inf-sup condition \eqref{eq:PDE:InfSup} using the trick of that for any $p ∈ L^2(Ω)$ there is a $\vw ∈ H_0^1(Ω)^2$ such that $\norm{p}^{2}_{L^2(Ω)} = \int_Ω p \dv \vw $ with $\norm{∇w} ≤ C \norm{p}$.

\subsection{Finite element approximation}

As discussed in \fref{sec:PDE:NumericalApproximationMixedProblems}, we need to tweak a little bit the numerical approximation by finite elements to get a stable method, adding the term \( - \sum_{K ∈ \mesh} α h_k^2 \int_K (- Δ\vu_h + ∇p_h -f )(-Δ\vv_h + ∇q_h) \) with $α > 0$. We use piecewise linear gradient element so the gradient is not continuous. We can prove that the scheme is stable by setting $v_h = u_h$, $q_h = - p_h$ so that \[ \norm{∇u_h}_{L^2(Ω)}^2 + \sum_{K ∈ \mesh} α h_K^2 \norm{∇p_h}^2_{L^2(K)} = 0 \] when $f = 0$ so that we need $∇u_h = ∇p_h = 0$. Then they are constant and because they are zero on the boundary they are zero everywhere, and therefore the kernel of the matrix is zero. If we hadn't added the last term, we wouldn't have restrictions on $p_h$ and the matrix would not be invertible.

\subsection{A priori error estimates}

The a priori estimate is \[ \norm{u,p} ≤ C \sup \frac{a(u,p; v,q)}{\norm{v,q}} \]

\subsection{A posteriori error estimates}

\begin{prop} \label{eq:PDE:APosterioriErrorStokes} For the Stokes problem \eqref{eq:PDE:StokesProblem}, there exists a constant $C > 0$ independent of $f, \vu, p, h$ but dependendent on the mesh aspect ratio such that \( \norm{∇(u - u_h)}^2_{L^2(Ω)} + \norm{p - p_h}_{L^2(Ω)} ≤ C \sum_{K ∈ \mesh} η_K^2 \) for the solutions $u_h, p_h$ of the Galerkin problem, with \[ η_K^2 = \left(h_k \norm{f + Δu_h - ∇p_h}_{L^2(K)} + \frac{1}{2} h_K^{\sfrac{1}{2}} · \norm{[∇\vu_h · \vn]}_{L^2(∂K)}\right)^2 + \norm{\dv u_h}^2_{L^2(K)}\]
\end{prop}

\begin{proof} We know from a proposition that I did not copy from the previous lecture, that there exists a constant $C$ such that \[ \norm{\vu - \vu_h, p - p_h} ≤ C \sup_{(v,q) ∈ H} \frac{a(u - u_h, p - p_h, v, q)}{\norm{v, q}} \] with $H = V × Q$

We want to prove that $∀(v,q) ∈ H$, $a(u - u_h, p - p_h, v, q) ≤ C_2 \norm{v,q} \left(\sum_{K ∈ \mesh} η_K^2\right)^{\sfrac{1}{2}}$. We will try to do it by relation to the residual.

We have that \begin{align*}
a(u - u_h, p - p_h, v, q)
	&= F(v,q) - a(u_h, p_h, v, q) = \\
	&= F(v,q) - a(u_h, p_h, v, q) + \\
	&\quad + (F - F_h)(v_h, q_h) - (a - a_h)(u_h, p_h, v, q) = \\
	&= \int_Ω \left[\vf (\vv - \vv_h) - ∇\vu : ∇(\vv - \vv_h) \right. \\
	&\qquad \left. + p_h \dv (\vv - \vv_h) + \dv \vu_h (q - q_h)\right] + \\
	&\quad + \sum_{K ∈ \mesh} α h_k^2 (f + Δ\vu_h - ∇p_h) (-∇\vv_h + ∇q_h)
\end{align*}

What we have done is that we have used the inf-sup condition to relate the error to the residual, and now we will evaluate it to get to the strong form of the equation by integrating by parts over all elements of the mesh.
\begin{align*}
a(u - u_h, p - p_h, v, q) &= \sum_{K ∈ \mesh} \int_K \left[(f + Δ\vu_h - ∇p_h)(v - v_h) + \dv \vu_h (q - q_h)\right] + \\
	&\quad + \frac{1}{2} \int_{∂K} [∇\vu_h · \vn] (\vv - \vv_h) + \underbracket{[p_h (\vv - \vv_h) · \vn]}_{ = 0} + \\
	&\quad+ αh_K^2(f + Δ\vu_h - ∇p_h)∇q_h
\end{align*}

Now we choose $\vv_h = R_h \vv$, the Clément interpolant. For $q_h$ we have to choose $q_h = 0$, and to justify this we would need to prove that the error is bounded above and below by the error estimator we are proving. The proof of that fact is lengthy and technical so we will not do it.

Once the functions are chosen, we use Cauchy-Schwartz so that \begin{align*}
a(u - u_h, p - p_h, v, q)
	&≤ C_3 \sum_{K ∈ \mesh} \left(h_K \norm{\vf + Δ\vu_h - ∇p_h}_{L^2(K)} \norm{∇\vv}_{L^2(ΔK)} + \right. \\
	&\quad+ \left. \frac{1}{2} h_K^{\sfrac{1}{2}} \norm{[∇\vu_h · \vn]}_{L^2(∂K)} + \norm{\dv \vu_h}_{L^2(K)}\norm{q}_{L^2(K)} \right) \\
	&≤ C_3 \left(\sum_{K ∈ \mesh} \left(h_K \norm{\vf + Δ\vu_h - ∇p_h} + \frac{1}{2}h_K^{\sfrac{1}{2}} \norm{[∇\vu_h · \vn]}\right)^2 \right)^{\sfrac{1}{2}} + \\
	&\quad
	+ \left(\sum_{K ∈ \mesh} \norm{∇\vv}_{L^2(ΔK)}^2\right)^{\sfrac{1}{2}}
	+ \left(\sum_{ K ∈ \mesh} \norm{\dv \vu_h}^2\right)^{\sfrac{1}{2}}
	· \left(\sum_{K ∈ \mesh} \norm{q}_{L^2(K)}^2\right)^{\sfrac{1}{2}}
\end{align*} where $ΔK$ is the neighbouring triangles of $K$ and $\dv \vu_h \to 0$ as $h \to 0$.

So it works by using Cauchy-Schwartz once again.
\end{proof}

\section{Optimal control}

\begin{wrapfigure}{R}{0.4\textwidth}
\centering
\inputtikz{Glacier}
\caption{Model of the movement of a glacier with three different boundaries.}
\label{fig:PDE:Glacier}
\end{wrapfigure}

One example of a problem on optimal control is the movement $u$ of a glacier, so that it follows the equation $-Δu = f$. We have two sections of the bedrock: $Γ_1$ where we have no slip (so $u = 0$) and a section $Γ_2$ with a certain sliding $∂_\vn u = q$, with $q$ unknown.

We also have the border $Γ_3$ of the ice with the air, with observations $u_0$ of the displacement and with no force of the air onto the ice (so $∂_\vn u = 0$ there).

We will want to minimize the integral \[ \int_{Γ_3} (u - u_0)^2 \] (that is, we want our model to approximate the observations we have) constrained to $u = 0$ on $Γ_1$ and being a solution of the weak problem \[ \int_Ω ∇u ∇v - fv = \int_Ω q v \qquad ∀v \st \restr{v}{Γ_1} = 0\]

This problem is, however, ill-posed. We will have to actually minimize \[ \frac{1}{2} \int_{Γ_3} (u - u_0)^2 + \frac{α}{2} \int_{Γ_2} q^2\] to have well-posedness.

We will study first a simpler model problem: we have a primal variable $u$ and a control variable $q$ such that $-Δu = f + q$ on $Ω ⊂ ℝ^n$, with observations of $u$ in $Ω_0 ⊂ Ω$ and with $u = 0$ on the boundary $∂Ω$.

The problem will be to find $(u,q) ∈ H_0^1(Ω) × L_2(Ω)$ such that the quantity \[ \frac{1}{2}\int_{Ω_0} (u - u_0)^2 + \frac{α}{2} \int_Ω q^2 \] is minimized under the constraint of the weak formulation of the problem, which is \[ \int_Ω ∇u ∇v = \int_Ω (f + q)v \qquad ∀v ∈ H_0^1(Ω)\]

A small parenthesis: if we were solving this in $ℝ^n$ instead of in a functional environment, we would have to solve \begin{align*} \mA \vu &= \vf \\ \mM_0 \vu &= \mM_0 \vu_0 \end{align*} with $\vu_0$ the vector of observations and $\mM_0$ a matrix selecting the coordinates on which the observations are defined.

In this case, we would define the functional $J(\vv) = \frac{α}{2} \norm{\mA \vv - \vf} + \frac{1}{2} \norm{\mM_0(\vv - \vu_0)}$ (that is, discrepancy with the model and with the observations) and then try to find a $\vv ∈ ℝ^n$ that minimizes it. The solution is given by $J'(\vu) = \vec{0}$, which computed would give \[ J'(\vu) = \trans{\mM_0} \mM_0 (\vu - \vu_0) + α \mA^T (\mA \vu - \vf) = \vec{0} \] which can be written as a system if we define $\vq = \mA \vu - \vf$.

Now we have to translate that to the PDE setting. We introduce the functional to minimize, the Lagrangian \( \linop (u, λ, q) = \frac{1}{2} \int_{Ω_0} (u - u_0)^2 + \frac{α}{2} \int_Ω q^2 + \int_Ω \left(∇u ∇λ - (f + q)λ\right) \label{eq:PDE:Lagrangian} \) with $λ ∈ H_0^1(Ω)$ the Lagrange multiplier corresponding to the constraint, $u ∈ H_0^1(Ω)$ and $q ∈ L^2(Ω)$.

The solution satisfies \[ \dpd{\linop(u,λ,q)}{u} = \dpd{\linop(u,λ,q)}{λ} = \dpd{\linop(u,λ,q)}{q} = 0\] with the derivative understood in the sense of the Gateaux derivative.

\begin{defn}[Gateaux derivative][Derivative!Gateaux] Given a function $\appl{F}{X}{Y}$ between two Banach spaces, we define the Gateaux derivative at $v ∈ X$ in the direction $u ∈ X$ as \( \dualp{\dif F(v), u} = \lim_{ε \to 0} \frac{F(v + εu) - F(v)}{ε} \) where we consider $\dif F(v)  ∈ X'$ as an operator in the dual space.
\end{defn}

Thus, in this case we can define the derivative of the Lagrangian as \[ \dualp{∂_u \linop(u, λ, q), v} = \lim_{ε \to 0} \frac{\linop(u + εv, λ, q) - \linop(u, λ,q)}{ε} \] and we claim that actually \[ \dualp{∂_u \linop(u, λ, q), v} = \int_{Ω_0} (u - u_0) v + \int_Ω ∇v ∇λ\quad ∀v ∈ H_0^1(Ω) \]

Indeed, computing we have that \begin{align*}
\linop(u + εv, λ, q) - \linop(u, λ, q)
	&= \frac{1}{2} \int_{Ω_0} \left((u + εv - u_0) ^ 2 - (u - u_0)^2\right) + \int_Ω ∇(u + εv - u) ∇λ = \\
	&= \int_{Ω_0} εv(u - u_0) + ε^2 v^2 + \int_Ω ε∇v∇λ
\end{align*} which when divided by $ε$ and with $ε \to 0$ works as we said. Similarly, we will have that \begin{align*}
\dualp{∂_λ \linop(u, λ, q), μ} &= \int_Ω ∇u ∇μ - (f+q)μ = 0  & ∀μ ∈ H_0^1(Ω) \\
\dualp{∂_q \linop(u, λ, q), r} &= α \int_Ω qr - \int_Ω rλ  = 0 & ∀r ∈ L^2(Ω)
\end{align*}

We can write strong forms of those equations, which gives in order
\begin{align*}
-Δλ + \ind_{Ω_0}(u - u_0) &= 0 &\text{(dual problem)}\\
-Δu = (f + q) &= 0 & \text{(primal problem)} \\
αq - λ &= 0
\end{align*}

Sometimes λ is called the dual variable.

With this system we can eliminate $q$ so we get our final problem to solve. We will need to find $(u, λ) ∈ H_0^1(Ω) × H_0^1(Ω)$ such that \( \begin{cases} \int_Ω ∇u ∇μ - \left(f + \frac{1}{α} λ \right)μ = 0 & ∀μ ∈ H_0^1(Ω) \\
\int_Ω ∇λ ∇v + \int_{Ω_0} (u - u_0) v = 0 & ∀v ∈ H_0^1(Ω) \end{cases} \)

We can formulate this in weak form by calling \begin{align*}
a(u,λ; μ, v) &= \int_Ω \left(∇u ∇μ - \frac{1}{α}λμ\right) + \int_Ω ∇λ ∇v + \int_{Ω_0} uv \\
F(μ, v) &=
\end{align*}

Well-posedness of this problem is a consequence of the inf-sup condition with the space $H ≝ H_0^1(Ω) × H_0^1(Ω)$ and $\norm{u, λ}_H^2 = \norm{∇u}_{L^2} + \norm{∇λ}_{L^2}$.

\begin{prop} \label{prop:PDE:OptimalControlInfsup} For the optimal control problem, for any $α >0$ there exists a constant $C > 0$ such that \[\norm{u, λ}_H ≤ C \sup_{\substack{(μ, v) ∈ H \\ (μ, v) ≠ 0}} \frac{a(u,λ;μ,v)}{\norm{μ, v}_H} \]
\end{prop}

\begin{proof} As we did for the Stokes problem it will suffice to prove that there exists two constants $C_1, C_2$ such that for any $(u, λ) ∈ H$ there will exist $(μ, v) ∈ H$ such that \begin{align*}
a(u,λ;μ,v) &≥C_1\norm{u,λ}^2 \\
\norm{μ, v} &≤ C_2\norm{u, λ}
\end{align*}

\proofpart{Lower bound}

We will start selecting $(μ, v) = (u, λ)$ to have some ``good terms'' with the squares of the gradients. Thus, we would have
\[
a(u,λ; u, λ) = \int_Ω \abs{∇u}^2 + \abs{∇λ}^2 - \frac{1}{α}λu + \ind_{Ω_0} u λ
\]

We still have some ``bad terms'' that we cannot bound, so we will compute \[ a(u,λ; -λ, u) = \int_Ω \frac{1}{α} λ^2 + \ind_{Ω_0} u^2 \]
and then try to substract the second equation weighted by some coefficent $β$ from the first one. Thus, we are choosing $(μ, v) = (u - βλ, λ + βu)$ and
\[
a(u,λ; u - βλ, λ + βu) = \int_Ω \abs{∇u}^2 + \abs{∇λ}^2 + \frac{β}{α} λ^2 + β \ind_{Ω_0} u^2- \frac{1}{α} λu + \ind_{Ω_0} uλ
\]

Now we can start bounding different terms. For $\frac{1}{α} λu$ we use Cauchy-Scharwz so that \[ \int_Ω \frac{1}{α} λu ≤ \frac{1}{α} \norm{λ}_2 \norm{u}_u ≤ \frac{β}{2α} \int_Ω λ^2 + \frac{1}{2αβ} \int_Ω u^2 \] and similarly \[ \int_{Ω_0} uλ ≤ \frac{β}{2} \int_{Ω_0} u^2 + \frac{1}{2β} \int_Ω λ^2 \]

Introducing those in the equation, we have
\[
a(u,λ; u - βλ, λ + βu) = \int_Ω \left(\abs{∇u}^2 + \abs{∇λ}^2\right) + \underbracket{\frac{β}{2α} \int_Ω λ^2 + \frac{β}{2} \int_{Ω_0} u^2}_{≥ 0} - \frac{1}{2β} \int_Ω λ^2 - \frac{1}{2αβ} \int_Ω u^2
\]

We will throw out the positive terms (we want to bound $a$ from below), and using \nref{thm:Fund:PoincareInequality}, we can bound the negative terms:
\begin{align*}
-\int_Ω λ^2 &≥ - C_p^2 \int_Ω \abs{∇λ}^2 \\
-\int_Ω u^2 &≥ - C_p^2 \int_Ω \abs{∇u}^2 \\
\end{align*} and so
\[ a(u,λ; u - βλ, λ + βu) ≥ \left(1 - \frac{C_p^2}{2αβ}\right) \int_Ω \abs{∇u}^2 + \left(1 - \frac{C_p^2}{2β}\right) \int_Ω \abs{∇λ}^2 \]

That quantity is greater than $\frac{1}{2} \norm{u,λ}^2$ if we have
\[ 1 - \frac{C_p^2}{2αβ} ≥ \frac{1}{2} \quad \text{and} \quad 1 - \frac{C_p^2}{2β} ≥ \frac{1}{2} \] so that our $β$ is defined as \[ β = C_p^2 \max \set{1, \frac{1}{α}} \]

\proofpart{Lower bound}

With the election of $(μ, v)$ from the previous part of the proof we have directly the lower bound $\norm{μ, v} ≤ C_2\norm{u, λ}$.
\end{proof}

This proposition gives us existence and unicity of the solution for the problem.

\subsection{Finite element approximation}

For a given mesh size $h > 0$, we will construct the mesh \mesh of triangles or tetrahedrons $K$ with $\mathop{diam}\, h_K ≤ h$, with $P_1, \dotsc, P_N$ the internal vertices and $φ_1, \dotsc, φ_N$ the shape functions associated to those vertices.

Thus, our space of finite elements is $V_h = \spn \set{φ_1, \dotsc, φ_n} ⊂ H_0^1(Ω)$ and we will search for $(u_h, λ_h) ∈ V_h^2$ such that \[ a(u_h, v_h; μ_h, v_h) = F(μ_h, v_h) \quad ∀ (μ_h, v_h) ∈ V_h^2\]

We could say that existence and uniqueness of the finite element approximation comes from the existence and uniqueness of the original problem, but we will see it using linear algebra and trying to invert the matrix of the system. Our system will be \[ \begin{pmatrix} \mM_0 & \mA \\ \mA & - \frac{1}{α} \mM \end{pmatrix} \begin{pmatrix} \vu \\ \vec{λ} \end{pmatrix} = \begin{pmatrix} \mM_0 \vu_0 \\ \vf \end{pmatrix} \]

We can try to eliminate $\vec{λ}$ from that equation, so that $\vec{λ = α \inv{\mM} (\vf - \mA \vu} $ and then \[ (\mM_0 + α \mA \inv{\mM} \mA) \vu = \]

That matrix is symmetric and positive definite and so invertible. Thus, we can start with the error estimates.

\subsection{A priori error estimates}

\begin{prop} There exists a constant $C > 0$ indepedent of $f, u_0, h$ but dependent on the mesh aspect ratio such that \[ \norm{u - u_h, λ - λ_h}^2 ≤ C \sum_{K ∈ \mesh} η_K^2 \] with
\begin{multline*}
η_K^2 =
	\left(
		h_K \norm{f + Δu_h + \frac{1}{α} λ_h}_{L^2(K)}
		+ \frac{1}{2} h_K^{\sfrac{1}{2}} \norm{[∇u_h · \vn]}_{L^2(∂K)}
	\right)^2 + \\
	+ \left(
		h_K \norm{\ind_{Ω_0} (u_0 - u_h) + Δλ_h}_{L^2(K)}
		+ \frac{1}{2} h_k^{\sfrac{1}{2}} \norm{[λ_h · \vn]}_{L^2(∂K)}
	\right)^2
\end{multline*}
\end{prop}

\begin{proof} By using the inf-sup condition from \fref{prop:PDE:OptimalControlInfsup}, we have a constant $C_1 > 0$ independent of $h$ such that \[ \norm{u - u_h, λ - λ_h} ≤ C_1 \sup _{(μ,v) ∈ H} \frac{a(u - u_h, λ - λ_h; μ,v)}{\norm{μ,v}}\]

In non-mixed problems our strategy to prove these bounds was to use coercivity and then relate it to the residual. Here we will to the same but with the inf-sup condition, using that
\[ \norm{u - u_h, λ - λ_h} ≤ C_1  \sup _{(μ,v) ∈ H} \frac{F(μ,v) - a(u_h, λ_h; μ,v)}{\norm{μ,v}} \]

We want to prove that for any $(μ,v) ∈ H$ there exists another constant $C_2 > 0$ independent of $f, u_0$ and $h$  such that \[ F(u,v) - a(u_h, λ_h; μ,v) ≤ C_2 \left(\sum_{K ∈ \mesh} η_K\right)^{\sfrac{1}{2}} \norm{μ,v} \]

Indeed,
\begin{align*}
& F(μ, v) - a(u_h, λ_h; μ,v)= \\
	&= \int_Ω \left(\left(f + \frac{1}{α}λ_h\right) μ - ∇u_h ∇μ\right)
	+ \int_Ω\left(\ind_{Ω_0}(u - u_0) v - ∇λ_h ∇v \right) = \\
	&=\int_Ω \left(f + \frac{1}{α}λ_h\right) (μ - μ_h) - ∇u_h ∇(μ - μ_h) + \\
	&\qquad + \int_Ω \ind_{Ω_0} (u_0 - u_h)(v - v_h) - ∇λ_h ∇(v - v_h) = \\
	&= \sum_{K ∈ \mesh} \int_K (f + \frac{1}{α}λ_h + Δu_h)(μ - μ_h) + \frac{1}{2} \int_{∂K} [∇u_h · \vn](μ - μh) + \\
	&\qquad + \int_K \ind_{Ω_0} (u_0 - u_h) + Δλ_h)(v - v_h) + \frac{1}{2} \int_{∂K} [∇λ_h · \vn](v - v_h)
\end{align*} and then we use Cauchy-Schwarz inequality to boud by the norms of everything.

We choose the Clément interpolant for the finite element approximations so that $μ_h = R_h μ$ and $v_h = R_h v$ and we have the bounds \begin{align*}
\norm{μ - R_h μ}_{L^2(K)} &≤ C_3 h_k \norm{∇v}_{L^2(ΔK)} \\
\norm{μ - R_h μ}_{L^2(∂K)} &≤ C_3 h_k^{\sfrac{1}{2}} \norm{∇v}_{L^2(ΔK)}
\end{align*} so that we can continue:
\begin{gather*}
F(μ, v) - a(u_h, λ_h; μ,v) ≤ \\
	≤ C_3 \sum_{K ∈ \mesh} \left(h_K \norm{f +  \frac{1}{α}λ_h + Δ_h}_{L^2(K)} + \frac{1}{2} h_K^{\sfrac{1}{2}} \norm{[∇u_h · \vn]}_{L^2(∂K)}\right) \norm{∇μ}_{L^2(ΔK)} + \\
	+\quad \sum_{K ∈ \mesh} \left(h_K \norm{\ind_{Ω_0} (u_0 - u_h) + Δλ_h}_{L^2(K)} + \frac{1}{2} h_K^{\sfrac{1}{2}} \norm{[∇λ_h · \vn]}_{L^2(∂K)}\right) \norm{∇λ}_{L^2(ΔK)}
\end{gather*}

We get the final bound by using the discrete Cauchy-Schwarz inequality\footnote{For any vector $\pesc{\va, \vb} ≤ \norm{\va} + \norm{\vb}$ so in particular $a_1 b_1 + a_2 b_2 ≤ \sqrt{a_1^2 +a_2^2} + \sqrt{b_1^2 + b_2^2}$}, that will give us a third constant that depends on the angle of the mesh elements and in some way ``measures'' how many times are we counting some terms over the same boundary.

\end{proof}


\chapter{Parabolic problems}
% -*- root: ../NumericalApproximationofPDEs.tex -*-


\chapter{Hyperbolic problems}
% -*- root: ../NumericalApproximationofPDEs.tex -*-

\section{Transport equation}

In this chapter we will start studying equations such as the transport equation:
\( \begin{cases}
\dpd{u}{t} + \vb ∇u = f & Ω × (0, T) \\
u(0) = u_0 \\
u(x,t) = 0 & 0 ≤ t ≤ T, \, x ∈ Γ ≝ \set{ x ∈ ∂Ω \st \vb \vn < 0 }
\end{cases} \label{eq:PDE:Transport} \)

Then, we will search for a finite element approximation $u_h ∈ V_h = \spn \set{φ_1, \dotsc, φ_N}$ where the vertices are on $Γ_-$ excluded.

The weak formulation will be given by \( \int_Ω \left(\dpd{u_h}{t} + \vb ∇u_h - f\right)(v_h + δ_h \vb ∇v_h) = 0 \quad ∀v_h ∈ V_h\label{eq:PDE:TransportWeakStab} \) where we have added a stabilization term such that $v_h = u_h + δ_h \dpd{u_h}{t}$ to have stability, with $δ_h$ constant.

\begin{prop}[A priori error\IS for the transport equation] With the transport equation \eqref{eq:PDE:Transport} with $\dv \vb = 0$, $\vb(x) ≠ 0$ and $δ_h = \frac{h}{2\norm{\vb}_{L^∞(Ω)}}$, we have that $u, \pd{u}{t} ∈ L^2(0, T; H^2(Ω))$ and there exists a constant $C > 0$ independent of $h$ such that \[ \norm{u(T) - u_h(T)}^2_{L^2(Ω)} ≤ Ch^3 \]
\end{prop}

\begin{proof}
start from the \[ \int \dpd{}{t} (u - u_h)(u - u_h) + \int \vb ∇(u-u_h) (u - u_h) = \int_Ω (f - \dpd{u_h}{t} - \vb ∇u_h) (u - u_h) \] and then the extra stabilization terms come into play
\end{proof}

\begin{prop}[A posteriori error\IS for the transport equation] With the transport equation \eqref{eq:PDE:Transport} with $\vb ∈ C^1(Ω)$ and $\dv \vb = 0$, using \[ \restr{δ_h}{K} = \frac{h_K}{2 \norm{\vb}_{L^∞(K)}} \quad ∀ K ∈ \mesh \] if $\norm{\vb}_{L^∞(K)} ≠ 0$ (if it is zero, we have $\restr{δ_h}{K} = 0$ as the transport term is zero and we don't need a stabilization term).

Then, if\footnote{Note that if the initial data at $t = 0$ is not smooth, the solution will not be smooth either: it is a transport equation.} $u ∈ L^2(0,T; H^1(Ω))$ there exists a constant $C > 0$ dependent only on the mesh aspect ratio such that \( \norm{u(T) - u_h(T)}_{L^2(Ω)}^2 ≤ \norm{u(0) - u_h(0)}^2_{L^2(Ω)} + C \int_0^T \sum_{K ∈ \mesh} η_K^2 \label{eq:PDE:TransportAPosteriori} \) with \[ η_K^2 ≝ h_K \norm{f - \dpd{u_h}{t} - \vb ∇u_h}_{L^2(K)} \norm{∇(u - u_h)}_{L^2(ΔK)} \]

We will see later how to get someting computable from there because we still have $u$ in the $η_K$ definition.
\end{prop}

\begin{proof} Set $e = u - u_h$. We already know that \[ \int_Ω (\vb ∇v) v = \int_Ω \dv \left(\vb \frac{v^2}{2}\right) = \int_{∂Ω} \vb \vn \frac{v^2}{2} ≥ 0 \quad ∀v ∈ H^1_{Γ_-}(Ω)\] which corresponds to setting $v = 0$ at the inflow $Γ_-$.

Then we can integrate \begin{multline}
\frac{1}{2} \dod{}{t} \int_Ω e^2 = \int_Ω \dpd{e}{t} e ≤ \int_Ω \left(\dpd{e}{t} e + (\vb ∇e)e\right) = \\
= \int_Ω \left(\dpd{}{t}(u - u_h) + \vb ∇(u - u_h)\right)e = \int_Ω \left(f - \dpd{u_h}{t} - \vb ∇u_h\right)e \label{eq:PDE:ProofAPostTransport1}
\end{multline}

This is valid for any $u_h$ coming off of any numerical scheme, so now we will use that to get an optimal estimate. Using the orthogonality condition, we can subtract $v_h - δ_h \vb ∇u_h$ from $e$ to have
\[ \int_Ω \left(f - \dpd{u_h}{t} - \vb ∇u_h\right)e = \int_Ω \left(f - \dpd{u_h}{t} - \vb ∇u_h\right)\left(e - v_h - δ_h \vb ∇v_h\right)\] for any $v_h ∈ V_h$. Using $v_h = R_h e$ the Clément interpolant, we can estimate separately \[ \norm{e - v_h}_{L^2(K)} ≤ Ch_K \norm{∇e}_{L^2(ΔK)} \] and that \[ \norm{δ_h \vb ∇R_h e}_{L^2(K)} ≤ \frac{h_K}{2\norm{\vb}_{L^∞(K)}} \norm{\vb}_{L^∞(K)} \norm{∇R_h e}_{L^2(K)} \] and by knowing the estimate $\norm{∇R_h e}_{L^2(K)} ≤ C \norm{∇e}_{L^2(ΔK)}$ we finish using the Cauchy-Schwartz inequality so that \[ \frac{1}{2} \dod{}{t} \int_Ω e^2 ≤ C\sum_{K ∈ \mesh} \norm{f - \dpd{u_h}{t} - \vb ∇u_h}_{L^2(K)} h_K \norm{∇e}_{L^2(ΔK)}\] and then we just integrate on time on the left hand side to get to the estimate \eqref{eq:PDE:TransportAPosteriori}.

\end{proof}

We would like to remark that this is not a classical a posteriori error estimate since $u$ is in the error estimator. However, we can use $ZZ$ post-processing and use the post-processed gradient \eqref{eq:PDE:PostProcGrad} that we discussed in \fref{sec:PDE:GradientAPosterioriElliptic} such that $\norm{∇(u - u_h)} \approx \norm{\gradop u_h - ∇u_h}$ because $\norm{∇u - \gradop u_h}$ was super-convergent (\fref{prop:PostProcGradientConvergence}).

A second remark that we already made is that if $u$ is not smooth and we only have $u ∈ L^2(0,T; L^2(Ω))$, we can go back to \eqref{eq:PDE:ProofAPostTransport1} so that we would have the error estimator with just the residual \[ \norm{e(T)}_{L^2(Ω)} ≤ \norm{e(0)}_{L^2(Ω)} + \int_0^T \left(\sum_{K ∈ \mesh}  \norm{f - \dpd{u_h}{t} - \vb ∇u_h}^2_{L^2(K)} \right)^{\sfrac{1}{2}} \]

\section{Wave equation}

\begin{figure}[hbtp]
\inputtikz{WaveExample}
\caption{Example of a wave propagation solution.}
\label{fig:PDE:WaveExample}
\end{figure}

We will now study the wave equation given by \( \label{eq:PDE:WaveEq} \begin{cases}
\dpd[2]{u}{t}(x,t) - Δu(x,t) = f(x,t) & x ∈ Ω, 0 ≤ t ≤ T \\
u(x,t) = 0 & x ∈ ∂Ω, 0 ≤ t ≤ T \\
u(x,0) = u_0(x) & x ∈ Ω \\
\dpd{u}{t}(x,0) = v_0(x) & x ∈ Ω
\end{cases}\) where $u$ models the vertical deformation of a membrane Ω.

If we were in 1D with $Ω = [0,1]$ and $f = 0$, $v_0 = 0$, the solution to the problem is a traveling wave such as the one from \fref{fig:PDE:WaveExample} because of the D'Alembert solution \[ u(x,t) = \frac{1}{2}\left(u_0(x - t) + u_0(x + t)\right) \] with $u_0$ an odd, periodic extension of $u_0$ over $ℝ$.

We will get the a priori estimate by multiplying the wave equation by $\pd{u}{t}$ and integrating over $Ω$:
\begin{align*}
\int_Ω \dpd[2]{u}{t} \dpd{u}{t} + ∇u ∇\dpd{u}{t} &= \int_Ω f \dpd{u}{t} \\
\frac{1}{2} \dod{}{t} \int_Ω \abs{\dpd{u}{t}}^2 + \abs{∇u}^2 &= \int_Ω f \dpd{u}{t}
\end{align*} which is a conservation principle, which says that the change on kinetic energy plus deformation energy is the power of external forces, and as everything in the integral is positive and using Cauchy-Schwartz we get that
\begin{align*}
\frac{1}{2}\dod{}{t} \underbracket{\int_Ω \abs{\dpd{u}{t}}^2 + \abs{∇u}^2}_{y(t)} &≤ \norm{f} \norm{\dpd{u}{t}} \\
\frac{1}{2}\dod{}{t} y(t) &≤ \norm{f} \sqrt{y(t)} \\
\dod{}{t} \sqrt{y(t)} &≤ \norm{f} \\
\sqrt{y(T)} &≤ \sqrt{y(0)} + \int_0^T \norm{f} \dif t
\end{align*}
and if $f = 0$ we have \[ \int_Ω \left( \dpd{u}{t}(x,T)^2 + \abs{∇u(x,T)}^2 \right) = \int_Ω v_0(x)^2 +\abs{∇u_0(x)}^2 \]

\subsection{Space discretization}

As usual, we pick $Ω$ a polygon meshed with $Ω = \bigcup_{K ∈ \mesh} K$ regularly with $h_K ≤ h$, and we consider the finite element space $V_h = \spn \set{φ_1, \dotsc, φ_N}$ with corresponding internal vertices. Then, our approximation will be \[ u_h(x,t) = \sum_{j = 1}^N u_j(t) φ_j(x) ∈ V_h\] that satisfies the weak formulation \( \int_Ω \dpd[2]{u_h}{t} v_h + \int_Ω ∇u_h ∇v_h = \int_Ω f v_h \quad ∀v_h ∈ V_h \label{eq:PDE:WaveEqWeak} \) with $u_h(0)$ being prescribed, for instance $Π_h u_0$ the $L^2$ projection of $u_0$ onto $V_h$, and the same with $\pd{u_h}{t}(0)$.

Finding $u_h$ satisfying \eqref{eq:PDE:WaveEqWeak} is thus equivalent to solving a differential system \[ \mM \vu''(t) + \mA \vu(t) = \vf(t) \] with \[ M_{ij} = \int_Ω φ_j φ_i \dif x \qquad A_{ij} = \int_Ω ∇φ_j ∇φ_i \dif x \] with initial conditions $\vu(0) = \vu_0$ and $\vu'(0) = \vv_0$.

We can see that $u_h$ satisfies the a priori estimate satisfied by $u$ by choosing $v_h = \dpd{u_h}{t}$.

\subsection{Error estimates}

\begin{prop}[A priori error\IS for the wave equation] Assume that $\pd[2]{u}{t} ∈ L^2(0,T; H^2(Ω))$. Then there exists a constant $C > 0$ independent of $h$ such that \begin{align*}
\norm{\dpd{}{t}(u - u_h)(T)}_{L^2(Ω)} &≤ Ch^2 \\
 \norm{∇(u - u_h)(T)}_{L^2(Ω)} &≤ Ch
 \end{align*}
\end{prop}

Back to the functional setting and thanks to the a priori estimates, the following existence result can be proved:

\begin{prop}[Existence and uniqueness\IS for the wave equation] Assume that $u_0 ∈ H_0^1(Ω)$, $v_0 ∈ L^2(Ω)$ and $f ∈ L^2(0,T; L^2(Ω))$. Then there exists a unique $u ∈ C^1(0,T; L^2(Ω))$ and $u ∈ C^0(0,T; H_0^1(Ω))$ such that \[ \dod[2]{}{t} \int_Ω u(x,t) v(x) \dif x + \int_Ω ∇u∇v = \int_Ω fv \qquad ∀v ∈ H_0^1(Ω)\; \text{, a.e. } 0 ≤ t ≤ T\]

Recall the definition of spaces such as $C^0(0,T; L^2(Ω))$ we gave in \fref{sec:Fund:BochnerSpaces}. In this case, if $u ∈ C^1(0,T; L^2(Ω))$ we require the mapping $t \to \int_Ω \abs{∇u(x,t)}^2$ to be continuous and if $u ∈ C^0(0,T; H_0^1(Ω))$ we need $t \to \int_Ω \abs{\pd{u}{t}(x,t)}^2 $ to be continuous.
\end{prop}

\subsection{Time discretization}

If we choose a time distrtization $t_n = nτ$ with $n = 0,1,\dotsc, M$ and $τ = \sfrac{T}{M}$, then we can approximate $u_h^n \approx u(x,t_n)$ with $u_h^n = \sum_{j=1}^N u_j^n φ_j(x)$. The implict scheme will be \( \label{eq:PDE:WaveEqTimeDisc} \int_Ω \frac{u_h^{n+1} - 2u_h^n + u_h^{n-1}}{τ^2} v_h + \int_Ω \frac{∇u_h^{n+1} + 2u_h^n + u_h^{n-1}}{4} ∇v_h = \int_Ω f(t_n) v_h \qquad ∀v_h ∈ V_h \)

\begin{prop} If $f = 0$, then the implicit scheme \eqref{eq:PDE:WaveEqTimeDisc} is stable with \[ \int_Ω \abs{\frac{u_h^{n+1} - u_h^n}{τ}}^2 + \int_Ω \abs{\frac{∇u_h^n + u_h^{n+1}}{2}}^2 ≤ \int_Ω \abs{\frac{u_h^n - u_h^{n-1}}{τ}}^2 + \int_Ω \abs{\frac{∇u_h^n + u_h^{n-1}}{2}}^2  \]
\end{prop}

\begin{proof} Set $v_h = u_h^{n+1} - u_h^{n-1}$ in \eqref{eq:PDE:WaveEqTimeDisc} so that
\[ \int_Ω \frac{u_h^{n+1} - 2u_h^n + u_h^{n-1}}{τ^2} (u_h^{n+1} - u_h^{n-1}) + \int_Ω \frac{∇u_h^{n+1} + 2u_h^n + u_h^{n-1}}{4} ∇(u_h^{n+1} - u_h^{n-1}) = 0 \]

Adding and removing $u_h^n$ on the $(u_h^{n+1} - u_h^{n-1})$ terms, we can operate on the first term such that
\begin{multline*}
\int_Ω \frac{(u_h^{n+1} - u_h^{n}) - (u_h^{n} - u_h^{n-1})}{τ} \frac{u_h^{n+1} - u_h^n + u_h^n - u_h^{n-1}}{τ} = \\ = \int_Ω \abs{\frac{u_h^{n+1} - u_h^n}{τ}}^2 - \int_Ω \abs{\frac{u_h^n - u_h^{n-1}}{τ}}^2
\end{multline*} and on the second term
\begin{multline*}
\int_Ω \frac{∇(u_h^{n+1} + u_h^n + u_h^n + u_h^{n-1})}{2} \frac{∇(u_h^{n+1} + u_h^n - (u_h^n + u_h^{n-1}))}{2} = \\ =
\int_Ω \abs{\frac{∇(u_h^n + u_h^{n+1})}{2}}^2 - \int_Ω \abs{\frac{∇u_h^n + u_h^{n-1}}{2}}^2
\end{multline*} and putting everything together we have the estimate.
\end{proof}

The stability condition for this scheme is that $τ ≤ h$.


\chapter{Darcy problem - Approximation of vector valued functions}
\ifincludefirstsemester% -*- root: ../NumericalApproximationofPDEs.tex -*-

Along this chapter we will study the Darcy problem, which models the flow of a fluid in a saturated porous medium. Its equation are the following:
\( \label{eq:PDE:Darcy} \begin{cases}
\frac{1}{k} \vu + ∇p = 0 & \text{in } Ω \\
\dv \vu = f & \text{in } Ω \\
p = d & \text{on } ∂Ω
\end{cases}\) with $f ∈ L^2(Ω)$, $d ∈ H^{\sfrac{1}{2}}(∂Ω)$ and $0 < k_{\text{min}} ≤ k(x) ≤ k_{\text{max}} < ∞$ for all $x ∈ Ω$. For easiness of presentation we will use only Neumann boundary conditions, although it generalizes without difficulty with Dirichlet conditions.

In the \fref{exm:PDE:Darcy} we already derived an abstract weak formulation of the problem, which is that of finding $\vu ∈ H(\dv, Ω) ≝ V$ and $p ∈ L^2(Ω) ≝ Q$ such that \( \begin{cases}a(\vu, \vv) + b(\vv, p) = F(\vv) & ∀v ∈ V \\
b(\vu, q) = G(q) & ∀q ∈ Q
\end{cases} \label{eq:PDE:DarcyWeakAbs} \) with \begin{align*}
a(\vu, \vv) &= \int_Ω \frac{1}{k} \vu \vv & F(\vv) &= \int_{∂Ω} d · \vv · \vn_{∂Ω} \\
b(\vv, p) &= - \int_{Ω} p \dv \vv & G(q) &= - \int_{Ω} fq
\end{align*}

We need to check assumptions of \nref{thm:PDE:WellPosednessMixedProb} to ensure well-posedness of \eqref{eq:PDE:DarcyWeakAbs}. Continuity of $a, b, G$ is straightforward. For $F$, it comes from the \nref{thm:Fund:Trace}: \[ F(\vv) = \int_{∂Ω} d \vv · \vn_{∂Ω} ≤ \norm{d}_{H^{\sfrac{1}{2}}(∂Ω)} \norm{\vv · \vn_{∂Ω}}_{H^{-\sfrac{1}{2}} (∂Ω)} ≤ γ \norm{d}_{H^{\sfrac{1}{2}}(∂Ω)} \norm{\vv}_{H(\dv, Ω)} \]

For the coercivity of $a$ in $V_0 = \ker B = \set{\vv ∈ H(\dv, Ω) \tq \dv \vv = 0}$, we have it immediately because of the definition\footnote{As defined in \eqref{eq:PDE:HDivNorm}, we have $\norm{\vu}_{H(\dv, Ω)}^2 = \norm{\vu}_{L^2(Ω)}^2 + \norm{\dv \vu}_{L^2(Ω)}^2$.}
of the norm in $H(\dv, Ω)$:
\[
	a(\vu, \vu) = \int_{Ω} \frac{1}{k} \abs{\vu}^2 ≥ \frac{1}{k_{\text{max}}} \norm{\vu}^2_{L^2(Ω)} = \frac{1}{k_{\text{max}}} \norm{\vu}^2_{H(\dv,Ω)}
\]

However, $a$ would not be coercive on the full space $V$.

For the inf-sup condition on $b$, for any $p ∈ L^2(Ω)$ we can define $ψ ∈ H^1(Ω)$ as the solution to the Dirichlet problem $Δψ = p$ in $Ω$ with $\restr{ψ}{∂Ω} = 0$, and let $\vv_p = ∇ψ$. Clearly $v_p ∈ H(\dv, Ω)$ and it's bounded by $\norm{\vv_p}_{H(\dv, Ω)} ≤ C\norm{p}_{L^2(Ω)}$ so that \[ \sup_{\vv ∈ V} \frac{b(\vv, p)}{\norm{\vv}} ≥ \frac{b(\vv_p, p)}{\norm{\vv_p}} = \frac{\norm{p}^2}{\norm{\vv_p}} ≥ \frac{1}{C} \norm{p} \]

Therefore all the assumptions of the \nref{thm:PDE:WellPosednessMixedProb} are fulfilled and the problem is well-posed. However, we do not know exactly how to approximate velocity fields $\vu ∈ H(\dv, Ω)$ with finite elements, so we will go and see that.

\section{Raviart-Thomas finite element spaces}

We also need a basis of functions for a vector valued functions. This will be the Raviart-Thomas spaces, that we will construct on a triangular mesh \mesh.

\begin{defn}[Raviart-Thomas finite element space][Finite elements space!Raviart-Thomas] Given an element $κ ∈ ℝ^d$ of a triangular mesh, the Raviart-Thomas $\mop{RT}_r$ space of degree $r$ is given by functions of the form $\vv(\vx) = \vec{w} + \vx η$ where $\vec{w} ∈ \left(\mathbb{P}_r(κ)\right)^d$ and $η ∈ \mathbb{P}_r(κ)$.
\end{defn}

This space has several nice properties.

\begin{prop} Let $\mop{RT}_r(κ)$ be the Raviart-Thomas finite element space. Then the following properties hold:
\begin{enumerate}
	\item $\left(\mathbb{P}_r(κ)\right)^d ⊂ \mop{RT}_r(κ) ⊂ \left(\mathbb{P}_{r+1}(κ)\right)^d$.
	\item If $\vv ∈ \mop{RT}_r(κ)$ and $e ⊂ ∂κ$ is an edge of the mesh, then $\vv · \vn ∈ \mathbb{P}_r(e)$.
\end{enumerate}
\end{prop}

These basis functions allows us to define the $\mop{RT}_r$ global space as \( \label{eq:Mixed:RTSpace} W_h^r = \set{\appl{\vv}{Ω}{ℝ^d} \tq \restr{\vv}{κ} ∈ \mop{RT}_r(κ), \, \restr{\vv · \vn}{e} = 0 \, ∀e ⊂ E(∂κ)\, ∀κ ∈ \mesh} \) where $E(∂κ)$ is the set of edges of the element κ.

To prove that this space is useful, we will need the definition of the distributional divergence, done in the usual fashion: $\pesc{\dv \vv, φ} = - \pesc{\vv, \dv φ}$.

\begin{lemma} $W_h^r ⊂ H(\dv, Ω)$
\end{lemma}

\begin{proof} Clearly, if $\vv ∈ W_h^r$ then $\vv ∈ \left(L^2(Ω)\right)^d$. The question is if the distributional derivative is in $L^2(Ω)$ too.
\end{proof}

\begin{lemma}[Interpolation error\IS for Raviart-Thomas finite elements] Given a family of regular triangulations $\set{\mesh}_{h > 0}$ of a polygonal domain $Ω ⊂ ℝ^d$ with $d ≤ 3$ and the space $W_h^r$ of Raviart-Thomas finite elements of degree $r ≥ 0$, it holds \( \label{eq:PDE:RTApprox} \norm{\vv - I_{h}^{\mop{RT}_r} \vv}_{H(\dv, Ω)} ≤ C^{r+1} \left(\abs{\vv}_{H^{r+1}(Ω)} + \abs{\dv \vv}_{H^{r+1}(Ω)}\right) \qquad ∀\vv ∈ \left[H^{r+1}(Ω)\right]^d\) with a constant $C > 0$ independent of $h$.
\end{lemma}
\fi

\appendix
\part{Exercises}

\chapter{Analysis fundamentals}
% -*- root: ../NumericalApproximationofPDEs.tex -*-
\section{Functional Analysis}

\begin{problem} \textbf{Norms and Banach spaces} Consider the space $V = C([-1, 1])$, consisting of functions that are continuous on the interval $[-1, 1]$.

\ppart Show that the following object defines a norm for $V$: \[ \norm{f}_V = \int_{-1}^1 \abs{f(x)} \dif x \]

\ppart Show that $V$ is not Banach with respect to this norm.

\solution

\spart

It is obviously always positive, zero only when $f \equiv 0$, homogeneous and with the triangular identity because the integral is linear and absolute value has the triangular identity.

\spart

Let $f_n$ be a sequence of continuous functions defined in the following way: \[ f_n(x) = \begin{cases}
0 & \abs{x} > \frac{1}{n} \\
n^2 x + n & - \frac{1}{n} ≥ x > 0 \\
-n^2 x + n & 0 > x > \frac{1}{n}
\end{cases}
\]

This is a triangle, always continuous, always with the same integral ($2 · \sfrac{1}{n} · n = 2$) so it obviously converges in the given norm. However, the limit function is the Dirac delta which is not even a function.

\end{problem}


\begin{problem} \textbf{Linear operators} Consider the space $V = C([-1, 1])$ endowed with the supremum norm. Let $\appl{\mathcal{L}}{V}{ℝ}$ be defined as \[ \mathcal{L}(f) = \int_{-1}^1 f(x) \dif x \]

Show that $\mathcal{L}$ is a linear, bounded and Lipschitz continuous operator.

\solution

Obviously linear and bounded (functions are continuous, thus bounded, therefore integral is bounded on a finite interval). For Lipschitz continuity, we want to show that, given $f,g ∈ V$, then \[ \frac{\abs{\mathcal{L}(f) - \mathcal{L}(g)}}{\norm{f - g}} ≤ K \] for some $K ∈ ℝ^+$.

In this case we can do straightforward calculations to do it\[  \frac{\abs{\mathcal{L}(f) - \mathcal{L}(g)}}{\norm{f - g}} = \frac{\abs{\int_{-1}^1 (f - g) \dif x}}{\abs{\sup f - g}} ≤ \frac{\int_{-1}^1 \sup \abs{f - g} \dif x}{\sup \abs{f - g}} ≤ 2\]

\end{problem}

\begin{problem}[4] \textbf{Distributional derivative} Show that the distributional derivative of the signum function \[ \sign (x) = \begin{cases} 1 & x > 0 \\ 0 & x = 0 \\ -1 & x < 0 \end{cases} \] is given by $\sign'(x) = 2 δ_0$ where $δ_0$ is the Dirac delta located at $x = 0$.

\solution

Taking the definition of the distributional derivative, we know that $\sign'$ is a distributional derivative of $\sign$ if, for every test function φ in some test space\footnote{I don't really know which space to use but whatever.} we have that \[ \int_{ℝ} \sign' φ = - \int_{ℝ} \sign φ' \]

In this case, we can operate and \[ - \int_{ℝ} \sign φ' = - \int_{ℝ^+} φ' + \int_{ℝ^-} φ' = - \left(\lim_{x \to ∞} φ(x) - φ(0)\right) + \left(φ(0) - \lim_{x \to -∞} φ(0)\right) = 2φ(0) \] using the Fundamental Theorem of Calculus and the fact that test functions have compact support and thus are 0 at $\pm ∞$. This is exactly the same value we would obtain with $\int_ℝ 2δ_0 φ$.

\end{problem}


\begin{problem}[7] \textbf{A simplified Poincaré inequality}. let $Ω = (0,1) × (0,1)$ and $Γ = [0,1] × \set{0}$. Determine a constant $C_Ω^* > 0 $ such that \[ \norm{v}_{L^2(Ω)} ≤ C_Ω^*\abs{v}_{H^1(Ω)}\qquad ∀v ∈ C_Γ^∞(Ω) \], where $\abs{v}_{H^1(Ω)} = \norm{\nabla v}_{L^2(Ω)}$ and $C_Γ^∞ (Ω) = \set {v ∈ C^∞(Ω) \tq \restr{v}{Γ} = 0}$.

\solution

Following the hint, we can use the fundamental theorem of calculus on the second coordinate of $v$ to obtain \[ v(x,y) = \int_0^y \pd{v}{s}(x,s) \dif s + c \], where the constant $c$ must be 0 to verify the boundary condition.

\end{problem}


\chapter{Numerical approximation of ODEs}
% -*- root: ../NumericalApproximationofPDEs.tex -*-
\section{Weak formulation}

\begin{problem} Consider the following boundary value problem: \[ \begin{cases}
- (αu')' + βu' + γu = f & x ∈ Ω = (a,b) \\
u(a) = u(b) = 0
\end{cases}\] with $a, β, γ ∈ ℝ^+$ and $\appl{f}{(a,b)}{ℝ}$ a sufficiently regular function. Choose a suitable test function space $V$ and write the weak form of the problem.

\solution

We multiply by a test function $v ∈ V$, and we want to get rid of the second derivative in the term $-(αu')'$. To do that, we integrate by parts:
\[ - \int_Ω α u'' v \dif x = - \eval[1]{αu'v}_{x = a}^b + \int_Ω αu'v' \dif x \] and we choose $V = H_0^1(Ω)$ to make the first term go to zero. The weak form has then the following forms:
\begin{align*}
a(u,v) &= \int_Ω αu'v' + βu'v + γuv \dif x \\
F(v) &= \int_Ω f v \dif x
\end{align*}
\end{problem}

\begin{problem} \label{ex:ODE:WeakForm2} Choose a suitable test function space and write the weak formulation of the following problems in a weak formulation. In every case, $α, β, γ ∈ ℝ^+$.

\ppart \[ \begin{cases}
- (αu')' + βu' + γu = f & x ∈ (0,1) \\
u(0) = 0 \\
u'(1) = 3
\end{cases}\]

\ppart \[ \begin{cases}
- (αu')' + βu' + γu = f & x ∈ (0,1) \\
u(0) = 0 \\
c_1 u'(1) + c_2 u(1) = 3
\end{cases}\] with $c_1, c_2$ non-null constants.

\solution

\spart

We use the space $V = \set{v ∈ H^1(0,1) \tq v(0) = 0}$. As in the previous exercise, to remove the second derivative in the first term of the integral we have
\[ - \int_0^1 α u'' v \dif x = - \eval[1]{αu'v}_{x = 0}^1 + \int_0^1 αu'v' \dif x = - 3 α v(1) + \int_0^1 αu'v' \] and so the forms are
\begin{align*}
a(u,v) &= \int_0^1 αu'v' + βu'v + γuv \dif x \\
F(v) &= \int_0^1 f v \dif x + 3 α v(1)
\end{align*}

\spart

We use again the space $V = \set{v ∈ H^1(0,1) \tq v(0) = 0}$ which comes restricted by the Dirichlet condition on 0. Then, we see what happens in the first term:
\[ - \int_0^1 α u'' v \dif x = - \eval[1]{αu'v}_{x = 0}^1 + \int_0^1 αu'v' \dif x = - α \frac{1 - c_2 u(1)}{c_1} v(1) + \int_0^1 αu'v' \] and so the forms are
\begin{align*}
a(u,v) &= - α \frac{c_2 u(1)}{c_1} v(1) + \int_0^1 αu'v' + βu'v + γuv \dif x \\
F(v) &= \int_0^1 f v \dif x + α \frac{1}{c_1} v(1)
\end{align*}

\end{problem}

\begin{problem} Consider the following boundary value problem, with non-homogeneous boundary conditions: \[ \begin{cases}
-(αu')' = f & x ∈ (0,1) \\
u(0) = 1 \\
u(1) = 2
\end{cases} \] with $α ∈ ℝ^+$.

Use the lifting technique to write a weak formulation of this problem.

\solution

By what we have seen in the previous exercises, the space of test functions is $H_0^1$ and the forms are \begin{align*}
a(u,v) &= \int_0^1 α u' v' \dif x \\
F(v) &= \int_0^1 f v \dif x
\end{align*} although the issue here is that $u ∉ H_0^1$. However, we can decompose $u = \tilde{u} + R$ where $\tilde{u} ∈ H_0^1$ and $R(0) = u(0)$, $R(1) = u(1)$ and in that case the weak formulation changes to \begin{align*}
a(u,v) &= F(v) \\
a(\tilde{u} + R, v) &= F(v) \\
a(\tilde{u}, v) &= F(v) - a(R, v) ≝ \tilde{F}(v)
\end{align*}

Here, choosing $R = 1 + x$ is enough and the modified form $\tilde{F}$ is \[ \tilde{F}(v) = F(v) - a(R, v) = \int_0^1 fv \dif x - \int_0^1 αv' \dif x \]
\end{problem}

\section{Lax-Milgram theorem}

\begin{problem} Consider the following boundary value problem \[
\begin{cases}
-(μu')' +βu' + γu = f &\text{in } Ω = (a,b) \\
u(a) = 0 \\
u'(b) = g_b
\end{cases}\] with $μ, β, γ ∈ L^∞(Ω)$, $β' ∈ L^∞(Ω)$, $g_b ∈ ℝ$ and $f ∈ L^2$.

\ppart Write the problem in weak formulation.
\ppart Give sufficient conditions on $μ, β, γ$ such that the problem has an unique solution by the \nref{thm:Theory:LaxMilgram}.

\solution

\spart

Recalling \fref{ex:ODE:WeakForm2}, the weak form with mixed Neumann-Dirichlet boundary conditions is \begin{align*}
a(u,v) &= \int_0^1 μu'v' + βu'v + γuv \dif x \\
F(v) &= \int_0^1 f v \dif x + g_b  μ(1) v(1)
\end{align*} with $V = \set{v ∈ H^1(Ω) \tq v(0) = 0}$.

\spart

Let's prove continuity of $a, F$ and coercivity of $F$. For continuity of $a$, assuming $M_μ = \essup μ$ and similarly for $β,γ$ we have
\begin{align*}
\abs{a(u,v)} ≤ \abs{\int_0^1 M_μ u'v' + M_β + M_γ uv \dif x} ≤ M_μ \abs{u}_{H^1}  \abs{v}_{H^1} + β \abs{u}_{H^1}\norm{v}_{L^2} + M_γ \norm{u}_{L^2} \norm{v}_{L^2}
\end{align*} which is bounded by the correct norms when applying \nref{thm:Fund:PoincareInequality}.

For continuity of $F$, we see that $g_b μ(1) v(1) = g_b μ(1) \left(\int_{∂Ω} v^2\right)^{\sfrac{1}{2}}$ and so we can bound it by the \nref{thm:Fund:Trace} \[ F(v) = \int_0^1 fv \dif x + g_b μ(1) v(1) ≤ \left(\norm{f}_L^2 + C_T g_b μ(1)\right) \norm{v}_{V}\]

For coercivity of $a$, we need $m_μ = \esinf μ > 0$ and we have it.

\end{problem}

\begin{problem} The temperature $T$ of a bar of length $L$ and constant section $A$ satisfies, for $x ∈ (0,L)$, the following boundary value problem:
\[
\begin{cases}
- k AT'' + σpT = 0 & x ∈ (0,L) \\
T(0) = T_0 \\
T'(L) = 0
\end{cases}\] with $p$ the perimeter of the section $A$, $k$ the termal conductivity, $σ$ the convection coefficient and $T_0$ is a given value.

\ppart Compute the analytical solution of the problem.
\ppart Write the weak formulation of the problem.

\solution

\spart

The analytical solution must fulfill \[ \frac{T''}{T} = \frac{σp}{kA} ≝ λ^2 > 0 \] so that its general solution is given by $T(x) = C_1 e^{λx} + C_2 e^{-λx}$. By replacing on the boundary conditions, we get the constants:
\begin{align*}
T'(L) &= 0 \\
λ(C_1 e^{λL} - C_2 e^{-λL}) &= 0 \\
C_1 e^{λL} &= C_2 e^{-λL} \\
C_1 &= C_2 e^{-2λL} \\
\end{align*} and
\begin{align*}
T(0) &= T_0 \\
C_1 + C_2 &= T_0 \\
C_2(e^{-2λL} + 1) &= T_0 \\
C_2 &= \frac{T_0}{e^{-2λL} + 1} \\
C_1 &= \frac{T_0 e^{-2λL}}{e^{-2λL} + 1}
\end{align*}

\spart

Done already. Use a lifting function.

\end{problem}

\section{GLS Stabilization}

\begin{problem} Consider the following advection-diffusion-reaction problem in non-conservative form.
\[
\begin{cases}
Lu ≝ -(μu')' + bu' + γu = 0 & \text{in } Ω = (0,1) \\
u(0) = 0 \\
(μu')(1) = 0
\end{cases}\] with $μ,b, γ ∈ L^∞(Ω)$ and $f ∈ L^2(Ω)$.

\ppart Write the weak formulation of the problem.
\ppart Show that the problem has a unique solution.
\ppart Write the Galerkin formulation of the problem with GLS stabilization.
\ppart Show that under the same hypothesis of the second part of the problem, the following inequality holds for the GLS stabilized problem: \[ \norm{u_h}_{GLS} ≤ C_0 \norm{f}_{L^2} \] where $C_0$ is a constant and \[ \norm{v}_{\text{GLS}}^2 = \norm{\sqrt{μ}v'}^2_{L^2} + \norm{\sqrt{\bar{γ}} v}^2_{L^2} + \sum_{K ∈ \mesh} \pesc{Lv, τ_K Lv} \] with \[ \bar{γ} = -\frac{1}{2} b'(x) + γ(x) \]

\solution

\spart

As usual, multiply by a test function $v ∈ V = \set{v ∈ H^1(Ω) \tq v(0) = 0}$ and integrate by parts to get rid of the second derivative:
\[ - \int_Ω (μu')' v \dif x = - \eval[1]{v (μu') \dif x}_{x = 0}^1+ \int_Ω μu'v' \dif x = \int_Ω μu'v' \dif x \]

Thus, the weak formulation is given by the forms
\begin{align*}
a(u,v) &= \int_Ω μu'v' + bu' + γu \dif x \\
F(v) &= \int_Ω fv \dif x
\end{align*}

\spart

I am not going to prove yet another time again the hypothesis of Lax-Milgram on these forms. Do it yourself.

\spart

The problem under GLS stabilization\footnote{See \fref{sec:ODEs:StronglyConsistentMethods} for more details.} implies finding $u_h ∈ V_h$ such that for any $v_h ∈ V_h$ we have \[ a(u_h, v_h) + \linop(u_h, v_h) = F(v_h) + ψ(v_h)\] with \begin{align*}
\linop(u_h, v_h) &= \sum_{K ∈ \mesh} \pesc{Lu_h, τ_K Lv_h} \\
Ψ(v_h) &= \sum_{K ∈ \mesh} \pesc{f, τ_K Lv_h}
\end{align*} and $τ_K$ a function given by \[ τ_K(x) = δ \frac{h_K}{\abs{b}(x)} \qquad δ > 0 \]

\end{problem}

\section{Parabolic problems}

\begin{problem} Consider the following parabolic problem in $Ω = (0,1) ⊂ ℝ$: Find $\appl{u}{Ω ×ℝ^+}{ℝ}$ such that \[\begin{cases} \dpd{u}{t} - \dpd{}{x} \left(μ \dpd{u}{x} - βu\right) + γ u = 0 & x ∈ Ω × ℝ^+ \\
u(x,0) = u_0(x) & x ∈ Ω\\
u(0,t) = η & t > 0 \\
μ(1) \dpd{u}{x} (1,t) - β(1) u(1,t) = -δu(1,t) & t > 0
\end{cases} \] where $μ, β, β', γ ∈ L^∞(0,1)$, $μ(x) ≥ μ_0 > 0$, $u_0 \equiv u_0(x) ∈ L^2(Ω)$ are given functions, and $η ∈ ℝ$ and $δ ∈ ℝ^+$ are constants.

\ppart Write the weak formulation of the problem, study coercivity and continuity of the bilinear form $a(·, ·)$.

\ppart Show that the Galerkin formulation is stable

\solution

\spart

As usual, multiply by test function $v ∈ V = \set{v ∈ H^1(Ω) \st v(0) = 0}$ and integrate by parts:
\begin{align*}
\dpd{u}{t} - \dpd{}{x} \left(μ \dpd{u}{x} - βu\right) + γ u &= \pesc{∂_t u, v} - \int_Ω \dpd{}{x} \left(μ \dpd{u}{x} - βu\right)v + γ u v \dif x = \\
&= \pesc{∂_t u, v} - \int_{∂Ω} \left(μ \dpd{u}{x} - βu\right)v \dif x + \int_Ω \left(μ \dpd{u}{x} - βu\right) v_x \dif x + \int_Ω γ u v \dif x \\
&= \pesc{∂_t u, v} + \underbracket{δu(1,t) v(1,t) + \int_Ω \left(μ \dpd{u}{x} - βu\right) v_x \dif x + \int_Ω γ u v \dif x}_{a(u,v)}
\end{align*}

We need a lifting function, so we say $u = \tilde{u} + R$ with $\tilde{u} ∈ V$, so $F(v) = - a(R, v)$. To fulfill boundary conditions, we say $R = η(1 - x)$ and then \[ F(v) = -a(R,v) = η \int_Ω \left(-μ - β(1-x)\right)v_x + γ(1-x)v \dif x \]

Continuity of $a$:
\begin{align*}
\abs{a(u,v)} &≤ δ \abs{u(1,t) v(1,t)} + \abs{\int_Ω μ u_x v_x \dif x} + \abs{\int_Ω βuv_x \dif x} + \abs{\int_Ω γuv \dif x} \\
	&≤ δC_{T,u} C_{T,v} \norm{u}_{V} \norm{v}_{V} + M_μ C_{P,u} C_{P,v} \norm{u}_V \norm{v}_V + M_β C_{P,v} \norm{u}_V \norm{v}_{V} + M_γ \norm{u}_V \norm{v}_V = \\
	&= (∂C_T + M_μC_P + M_βC_P +M_γ)\norm{u}_V \norm{v}_V
\end{align*} where $M_μ = \essup \abs{μ} = \norm{μ}_{∞} < ∞$ because  $μ ∈ L^∞$, similarly with others, and we used the \nref{thm:Fund:Trace} and \nref{thm:Fund:PoincareInequality} for inequalities on the norms of the border and seminorms.

For coercivity, using that $uu_x = \frac{1}{2} ∂_x (u^2)$ and noting $\bar{γ} = \sfrac{β'}{2} + γ$:
\begin{align*}
a(u,u) &= δ u(1,t)^2 + \int_Ω μu_x^2 \dif x - \int_{Ω} β uu_x \dif x + \int_Ω γu^2 \dif x \\
	&≥ δ u(1,t)^2 + μ_0 \abs{u}_V^2 - \frac{1}{2}\int_Ω β ∂_x (u^2)+ β'u^2 \dif x  + \frac{1}{2}\int_{Ω} β' u^2 \dif x + \int_Ω γu^2 \dif x \\
	&= δ u(1,t)^2 + μ_0 \abs{u}_V^2 - \frac{1}{2} \int_Ω ∂_x(βu^2) \dif x + \int_Ω \left(\frac{β'}{2} + γ\right)u^2 \dif x \\
	&≥ δu(1,t)^2 + μ_0 \abs{u}_V^2 - \frac{1}{2} \eval[1]{βu^2}_{x = 0}^1 + m_{\bar{γ}} \norm{u}^2_{L^2} \\
	&≥ \left(δ - \frac{β(1)}{2}\right)u(1,t)^2 + μ_0\abs{u}^2_V+ m_{\bar{γ}} \norm{u}_{L^2}^2
\end{align*} so if $δ - \sfrac{β(1)}{2} > 0$ and $m_{\bar{γ}} = \esinf \bar{γ} ≥ 0$ we have coercivity with $α = μ_0$.

\spart

Say $v_h = u_h$, so then \begin{align*}
\pesc{∂_tu_h, u_h} + a(u_h, u_h) &= F(u_h) \\
\int_Ω \dpd{u_h}{t} u_h + μ_0 \norm{u_h}^2_{V} &≥ C \norm{u_h}_{V} \\
\frac{1}{2}\int_Ω \dpd{u_h^2}{t} + μ_0 \norm{u_h}^2_V &≥ C \norm{u_h}_V \\
\frac{1}{2} \dpd{}{t} \norm{u_h}^2_{L^2} + μ_0 \norm{u_h}^2_V &≥ C \norm{u_h}_V \\
\frac{1}{2} \norm{u_h(t)}_{L^2}^2 - \frac{1}{2}\norm{u_{h0}}_{L^2}^2 + μ_0 \norm{u_h}^2_{V} &≥ C \norm{u_h}_{L^2}
\end{align*}

\end{problem}


\chapter{Numerical approximation of PDEs II}
% -*- root: ../NumericalApproximationofPDEs.tex -*-
\section{Session 1 - The Poisson problem / Laplace equation}

\begin{problem}
Let $Ω ⊂ ℝ^2$ be an open bounded set with $C^1$ boundary $∂Ω$ and $f ∈ L^2(Ω)$. Given $\vb ∈ ℝ^2$, we define $Γ_- = \set{x ∈ ∂Ω \tq \vb · \vn < 0}$ where $\vn$ is the outwards unit normal vector to $∂Ω$. We assume that $Γ_- ≠ ∅$ and we consider the problem of finding a function $\appl{u}{Ω}{ℝ}$ such that \[ \begin{cases}
- \dv (\mB ∇u) + \vb · ∇u + cu = f & \text{in }Ω \\
u = 0 & \text{on }Γ_- \\
\mB ∇u · \vn = 0 & \text{on } Γ_+ ≝ ∂Ω \setminus Γ_-
\end{cases}\] where $\mB ∈ ℝ^{2×2}$ is a symmetric positive definite matrix and $c ∈ ℝ^+$.

\ppart Write the weak form of the problem.
\ppart Show that the problem is well-posed.
\ppart Consider the finite element method presented in the course and the resulting linear system $\mA \vu = \vf$. Prove that the matrix $\mA$ is invertible.
\solution

\spart

As always, we multiply by a test function $v ∈ H_{Γ_-}^1 ≝ \set{v ∈ H^1(Ω) \tq \restr{v}{Γ_-} = 0}$  and integrate on the whole domain:
\[ -\int_Ω \dv (\mB ∇u)v + (\vb · ∇u) v + cuv = \int_Ω fv  \]

We have to integrate by parts in the first term to get rid of the second derivative:
\begin{align*}
-\int_Ω \dv (\mB∇u) v
	&= -\int_{∂Ω}(\mB∇u) v + \int_Ω  ∇v (\mB ∇u) = \\
	&= - \cancelwhy{\int_{Γ_+} (\mB∇u) v}{\restr{B∇u · \vn}{Γ_+} = 0} - \cancelwhy{\int_{Γ_-} (\mB∇u) v}{\restr{v}{Γ_-} = 0} + \int_{Ω} ∇v (\mB ∇u)  = \\
	&= \int_Ω \trans{∇v} {\mB} ∇u
\end{align*}

Thus, the weak form is given by the two following forms:
\begin{align*}
a(u,v) &= \int_{Ω} \trans{∇v} \mB ∇u + \int_Ω (\vb · ∇u)v + \int_Ω cuv \\
F(v) &= \int_Ω fv
\end{align*}

\spart

To show well-posedness we need to prove the hypothesis of the \nref{thm:Theory:LaxMilgram}. Continuity of $F$ is trivial as $f ∈ L^2$. For continuity of $a$, we bound each integrand separately. For the first, as $\mB$ is symmetric and positive definite, it has a finite number of eigenvalues and all positive, so we consider $λ_\text{max}$ its maximum one and then\footnote{Ignoring the absolute value because the integrand is always positive.} \[ \int_Ω \trans{∇v} \mB ∇u ≤ \int_Ω λ_\text{max} \abs{∇v} \abs{∇u} ≤ λ_\text{max} \abs{v}_{H^1} \abs{u}_{H^1} ≤ λ_\text{max} \norm{v}_{H^1} \norm{u}_{H^1} \]

For the second one, we have that \[ \int_Ω \abs{(\vb · ∇u) v } ≤ \int_Ω \abs{\vb} \abs{∇u} \abs{v} ≤ \abs{\vb} \norm{v}_{H^1} \norm{u}_{H^1} \]

The third one is trivial, and so we have continuity.

We study coercivity again by each integrand. For the first one, we consider $λ_\text{min} > 0$  the minimum eigenvalue of $\mB$ and so \[ \int_Ω \trans{∇v} \mB ∇v ≥ \int_Ω λ_\text{min} \abs{∇v}^2 = λ_\text{min} \abs{v}_{H^1}^2 ≥ \frac{λ_\text{min}}{1 + C_p} \norm{v}_{H^2}^2 \] using also \nref{thm:Fund:PoincareInequality}\footnote{More specifically, its \fref{crl:Fund:PoincareReverse}.}. For the second, we see that $∇(v^2) = 2 v ∇v$ so we can replace and use integration by parts \begin{align*}
\int_Ω (\vb ∇v)v
	&= \frac{1}{2} \int_Ω \vb · ∇(v^2)
	= - \frac{1}{2} \int_Ω v^2 \underbracket{\dv \vb}_{= 0} + \frac{1}{2} \int_{∂Ω} \vb v^2 = \\
	&= \frac{1}{2} \underbracket{\int_{Γ_+} \vb v^2}_{>0\; (\restr{\vb · \vn}{Γ_+} > 0)} + \frac{1}{2} \underbracket{\int_{Γ_-} \vb v^2}_{= 0\; (\restr{v}{ΦΓ_-} = 0)} ≥ 0
\end{align*}

The third one is again trivially greater than zero, so we have coercivity of $a$ and the form is well-posed.

\spart

The finite elements method implies searching in a finite space $V_h$, so that $a(u_h, v_h)$ can be expressed as $\trans{\vv} \mA \vv = \trans{\vv} \vf$. As $a$ is coercive, we have that for any non-null $\vu ∈ V_h$ we have $\mA \vu ≠ 0$ so that $\ker \mA = 0$ and $\mA$ is invertible.

\end{problem}


\begin{problem} Let $Ω = (0,1)$ and consider the nodes $x_i = ih$ for $i = 0, \dotsc, N + 1$ with $h = \frac{1}{N+1}$. Let $V_h$ be the finite dimensional space $V_h = \spn\set{φ_i}_{i=1}^N$ spanned by the $N$ affine shape functions defined by \( φ_i(x) ≝ \begin{cases}
\frac{x - x_{i-1}}{x_i - x_{i-1}} & x ∈ [x_{i-1}, x_i] \\
\frac{x_{i+1} - x}{x_{i+1} - x_{i}} & x ∈ [x_{i}, x_{i + 1}] \\
0 & \text{otherwise}
\end{cases} \label{eq:ExPDEII:ShapeFuncs}\)

\ppart Show that there exists a constant $C > 0$ such that for all $h > 0$, for all $v_h ∈ V_h$ and for all $i = 0, \dotsc, N$, we have \( \int_{x_i}^{x_{i+1}} (v_h'(x))^2 \dif x ≤ Ch^{-2} \int_{x_i}^{x_{i+1}} v_h^2(x) \dif x \)

\ppart Show that there exists two constants $C_1, C_2 > 0$ such that for all $h > 0$ and for all $v ∈ H^2(0,1)$, $i = 0, \dotsc, N$ we have \begin{align*}
\int_{x_i}^{x_{i+1}} (v'(x) - (r_hv)' (x))^2 \dif x &≤ C_1 h^2 \int_{x_i}^{x_{i+1}} (v''(x))^2 \dif x \\
\int_{x_i}^{x_{i+1}} (v(x) - r_hv (x))^2 \dif x &≤ C_2 h^2 \int_{x_i}^{x_{i+1}} (v''(x))^2 \dif x
\end{align*} where $r_h v ≝ \sum_{i = 1}^N v(x_i) φ_i$ is the Lagrange interpolant of $v$.

\solution

\spart

First, we see that the derivative of $φ_i$ is defined by \[ φ_i'(x) =\begin{cases}
\frac{1}{x_i - x_{i-1}} & x ∈ [x_{i-1}, x_i] \\
\frac{-1}{x_{i+1} - x_{i}} & x ∈ [x_{i}, x_{i + 1}] \\
0 & \text{otherwise} \end{cases} \] and then we can compute both integrals:
\begin{align*}
\int_{x_i}^{x_{i+1}} (v_h'(x))^2 \dif x &= \int_{x_i}^{x_{i+1}} (v_i φ_i' + v_{i+1}φ_{i+1}')^2 \dif x = \\
&= \frac{(v_{i+1} - v_i)^2}{h} \\
\int_{x_i}^{x_{i+1}} v_h^2(x) \dif x &= \int_{x_i}^{x_{i+1}} (v_i φ_i + v_{i+1}φ_{i+1})^2 \dif x = \\
&=\frac{h}{6}(v_i^2 + (v_{i} + v_{i+1})^2 + v_{i+1}^2)
\end{align*}

Then we apply Cauchy-Schwarz and we have it.

\spart

It suffices to prove the results for $v ∈ C^2([0,1])$. Let $w = v - r_h v$. As for any vertex $x_i$ we have $w(x_i) = 0$, then by Rolle's theorem there exists some $ξ_i ∈ (x_i, x_{i+1})$ such that $w'(ξ_i) = 0$ so that we can write $w'(x) = \int_{ξ_i}^x w''(s) \dif s$. Additionally, as $r_hv$ is a polynomial of degree one, we have that \[ w'(x) = \int_{ξ_i}^{x} w''(s) \dif s = \int_{ξ_i}^x v''(s) \dif s \]

Taking absolute values and applying the Cauchy-Schwarz inequality\footnote{Dirty trick: $\int f = \pesc{1, f} ≤ \norm{1} \norm{f}$.} we have \[ \abs{w'(x)} ≤ \int_{x_i}^{x_{i+1}} \abs{v''(s)} \dif s ≤ h^{\sfrac{1}{2}} \left(\int_{x_i}^{x_{i+1}} \abs{v''(s)}^2 \dif s \right)^{\sfrac{1}{2}} \]

Integrating over the interval $[x_i, x_{i+1}]$ yields the first inequality. For the other one, simply say that $\abs{w} ≤ \int_{x_i}^{x_{i+1}} \abs{w'}$ and integrate.

\end{problem}

\section{Session 2 - Interpolation error}

\begin{problem} Let $Ω = (0,1)$, let $N ∈ ℕ^+$ and consider the nodes $x_i = ih$ for $i = , \dotsc, N + 1$. Let $V_h$ be the finite dimensional space generated by the usual basis functions $\set{φ_i}_{i=1}^N$ (see \eqref{eq:ExPDEII:ShapeFuncs}). Show that there exists a constant $C > 0$ such that for all $h >0$ and for all $v ∈ H^1(0,1)$ and for all $i = 0, \dotsc, N$ we have \[ \norm{v - r_hv}_{L^2(x_i, x_{i+1}} ≤ Ch \norm{v'}_{L^2(x_i, x_{i+1})} \] where $r_h v$ is the Lagrange interpolant of $v$ defined by $r_h v ≝ \sum_{i = 1}^N v(x_i) φ_i$.

\solution

By density, it suffices to prove this for $v ∈ C^1([0,1])$. Let us define $w = v - r_h v$, it is clear that \[ w(x) = \int_{x_i}^x w'(s) \dif s = \int_{x_i}^x v'(s) - (r_hv)'(s) \dif s\]

Now, as the Lagrange interpolant is linear we have that $(r_hv)'(s) = \frac{v(x_{i+1}) - v(x_i)}{h}$ when restricted to $[x_i, x_{i+1}]$. Using that $v(x_{i+1}) = v(x_i) + \int_{x_i}^{x_{i+1}} v' $ we have \begin{multline*} w(x) = \int_{x_i}^x v'(s) - \frac{v(x_{i+1}) - v(x_i)}{h} \dif s = \int_{x_i}^x \left(v'(s) - \frac{1}{h} \int_{x_i}^{x_{i+1}} v'(t) \dif t \right) = \\ = \int_{x_i}^x v'(s) \dif s - \frac{x - x_i}{h} \int_{x_i}^{x_{i+1}} v'(t) \dif t \end{multline*}

Now we can take the absolute value of $w$ and by using Cauchy-Schwarz we get \[ \abs{w(x)} ≤ \underbracket{\int_{x_i}^x \abs{v'(s)} \dif s}_{≤ \int_{x_i}^{x_{i+1}} \dotsb} + \underbracket{\frac{x - x_i}{h}}_{≤ 1} \int_{x_i}{x_{i+1}} \abs{v'(s)} \dif s ≤ 2 \int_{x_i}{x_{i+1}} \abs{v'(s)} \dif s ≤ \]

\end{problem}

\begin{problem} Let $f ∈ L^2(0,1)$ and let $ε > 0$. We consider the problem of finding $\appl{u}{(0,1)}{ℝ}$ such that \[ \begin{cases} - εu'' + u' + u = f & 0 < x < 1 \\ u(0) = u(1) = 0 \end{cases}\]

\ppart Write the weak form of the problem.
\ppart (A priori error estimates) Let $0 = x_0 < x_1 < \dotsb < x_N < x_{N+1} = 1$ be a discretization of $[0,1]$ and let $h_i = x_{i+1} - x_i$, $h ≝ \max_i h_i$. Show that if $u ∈ H^2(0,1)$ then there exists a constant $C > 0$ independent of $h$ such that \[ \norm{u - u_h}_{L^2(0,1)} ≤ Ch \]

\ppart (A posteriori error estimates) Show that \[ \norm{u' - u_h'}_{L^2(0,1)} ≤ C\left(\sum_{i=0}^N η_i^2\right)^{\sfrac{1}{2}}\] with \[ η_i^2 = h_i^2 \norm{\frac{1}{ε} f + u_h'' - \frac{1}{ε}u_h' - \frac{1}{ε}u_h}^2_{L^2(x_i, x_{i+1})}\] where $C$ is the constant previously derived. Use the Lagrange interpolant and the result of the previous exercise.

\solution

\spart

As usual, multiply by test function, integrate and apply divergence theorems
\begin{align*}
-ε \int_Ω u''v + \int_Ω u'v + \int_Ω uv &= \int_Ω fv \\
-ε \left(\eval[1]{u'v}_{∂Ω} - \int_Ω u'v'\right) + \int_Ω u'v + \int_Ω uv &= \int_Ω fv \\
\underbracket{\int_Ω εu'v' + \int_Ω u'v + \int_Ω uv}_{a(u,v)} &= \underbracket{\int_Ω fv}_{F(v)}
\end{align*}


\spart

We will, in order apply Galerkin orthogonality for some $v_h ∈ V_h$, coercivity of $a$ and continuity
\begin{align*}
a(u - u_h, u - u_h) &= a(u - u_h, u - v_h + \underbracket{v_h - u_h}_{∈ V_h}) \\
ε \norm{u - u_h}^2_V &≤ M \norm{u - u_h}_V \norm{u - v_h}_V
\end{align*}

Now choose $v_h = r_hv$ so we have the estimation of the previous exercise $\norm{v - r_hv}_V ≤ Ch \norm{v'}_V$ and we have it.

\spart

For any $u_h ∈ V_h$ and $v ∈ V$, we define the residual as \[ R_{u_h} (v) = F(v) - a(u_h, v)\] (we might omit the $u_h$ subscript) which corresponds to the error made when we insert $u_h$ instead of $u$. Note that $R(v_h) = 0$ for any $v_h ∈ V_h$. Thanks to coercivity of $a$ we have that \begin{align*}
ε\norm{u' - u_h'}^2_{L^2(0,1)} & ≤ a(u - u_h, u - u_h) \\
	&= a(u, u - u_h) - a(u_h, u - u_h) \\
	&= R(u - u_h)
\end{align*}

Then we proceed to estimate the residual. For any $v ∈V$ we have that
\begin{multline*}
R(v) = R(v - v_h) = F(v - v_h) - a(u_h, v - v_h)
\end{multline*}

\end{problem}

\section{Session 3 - Error bounds and numerical tests}

\begin{problem} Let $V$ be a Hilbert space with norm $\norm{·}$, $\appl{a}{V×V}{ℝ}$ a bilinear form and $\appl{F}{V}{ℝ}$ a bilinear form. Assume that $a$ is coercive with constant $α$ and continuous with constant $β$.

Let $u ∈ V$ be such that $a(u,v) = F(v)$ for any $v ∈ V$. Let $V_h ⊂ V$ be a finite dimensional space and $u_h ∈ V_h$ such that $a(u_h, v_h) = F(v_h)$ for any $v_h ∈ V_h$. Prove the following estimates:

\ppart A priori: $α \norm{u - u_h} ≤ β \norm{u - v_h}$ for any $v_h ∈ V_h$.
\ppart A posteriori: $α \norm{u - u_h} ≤ \sup_{v ∈ V\minuszero} \frac{F(v) - a(u_h, v)}{\norm{v}}$.

\solution

\spart

By coercivity, $α\norm{u - u_h}^2 ≤ a(u - u_h, u - u_h)$. Now we add and substract $v_h ∈ V_h$ in the second argument and using bilinearity we have \begin{align*}
α\norm{u - u_h}^2 &≤ a(u - u_h, u - u_h) = a(u - u_h, u - v_h + v_h - u_h) \\
&= a(u - u_h, u - v_h) + a(u - u_h, v_h - u_h) = \\
&= a(u - u_h, u - v_h) + \underbracket{a(u, v_h - u_h)}_{= 0} - \underbracket{a(u_h, v_h - u_h)}_{= 0}  \\
&= a(u - u_h, u - v_h) \\
α\norm{u - u_h}^2 &= \abs{a(u - u_h, u - v_h)} ≤ β \norm{u - u_h} \norm{ u - v_h} \\
α\norm{u - u_h} &≤ β \norm{u - v_h}
\end{align*}

\spart

Again applying coercivity
\begin{align*}
α \norm{u - u_h}^2 &≤ a(u - u_h, u - u_h) \\
	&= a(u - u_h, u - v + v - u_h) = a(u - u_h, v) + a(u - u_h, u - v - u_h) \\
	&= F(u - u_h) - a(u_h, u - u_h) = \resid[u_h](u - u_h)
\end{align*}

Maybe. IDK.

\end{problem}

\section{Session 4 - Goal oriented errors, superconvergence and adaptive mesh algorithm}

\begin{problem} Let $ε > 0$. Consider the problem of finding $\appl{u}{(0,1)}{ℝ}$ such that \[ \begin{cases}
-ε u'' + u' = 0 & 0 < x < 1 \\
u(0) = u(1) = 1
\end{cases}\]

Let $ρ ∈ L^2(0,1)$ be given. Consider a discretization $0 = x_0 < x_1 < \dotsb < x_N < x_{N+1} = 1$ of $[0,1]$ with $h_i = x_{i+1} - x_i$ and $I_i = [x_{i}, x_{i+1}]$. Show that there exists a constant $C$ independent of $h$ and $u$ such that \[ \int_0^1 ρ · (u - u_h) \dif x ≤ C\sum_{i = 1}^N h_i \norm{1 + εu_h'' - u_h'}_{L^2(I_i)} \norm{φ' - φ_h'}_{L^2(I_i)}\] where $u_h$ is the finite element approximation of $u$ and φ and $φ_h$ are the solution and its approximation of the dual problem \[
\begin{cases}
- ε φ'' - φ' = ρ & 0 < x < 1\\
φ(0) = φ(1)  = 0
\end{cases}\]

\solution

\end{problem}

\section{Session 5 - Stokes problem}

\begin{problem} \label{ex:PDE2:Stokes} Let $Ω ⊂ ℝ^2$ be aconvex polygonal domain. We consider the Stokes problems of finding $\appl{\vu}{Ω}{ℝ^2}$ and $\appl{p}{Ω}{ℝ}$ such that
\[ \begin{cases}
- Δ\vu + ∇p = \vf & \text{in } Ω \\
\dv \vu = 0 & \text{in } Ω \\
\vu = 0 & \text{on } Ω
\end{cases}\]

\ppart Show that the weak form of the problem is finding $(\vu, p) ∈ V× Q$ such that \[ a(\vu, p; \vv, q) = F(\vv, q)\qquad ∀(\vv, q) ∈ V×Q\] with \begin{align*}
a(\vu, p; \vv, q) &= \int_Ω ∇\vu : ∇\vv - \int_Ω p \dv \vv - \int_Ω q \dv \vu \\
F(\vv, q) &= \int_Ω \vf · \vv
\end{align*} where $V, Q$ are spaces to be specified.

\ppart For any $h > 0$ let \mesh be a conformal mesh of triangles $K$ with diameter $h_K ≤ h$ and consider $V_h ⊂ V$, $Q_h ⊂ Q$ two finite dimensional spaces consisting of affine elements $\pspace[1]$. The solution of the weak form is approximated by solving the associated Galerkin problem with stabilization term:
\begin{align*}
a_h(\vu_h, p_h;\vv_h, q_h) &≝ a(\vu_h, p_h; \vv_h, q_h) - α \sum_{K ∈ \mesh} h_K^2 \int_K ( - Δ\vu_h + ∇p_h) · (- Δ\vv_h + ∇q_h) \\
F_h(\vv_h, q_h) &≝ F(\vv_h, q_h) - α \sum_{K ∈ \mesh} h_K^2 \int_K \vf · (- Δ\vv_h + ∇q_h) \\
\end{align*} with $α ≥ 0$ a stability parameter. Show that there exists a constant $C > 0$ independent of $h, \vu$ and $p$ such that \[ \norm{\vu - \vu_h, p - p_h}^2 ≤ Ch^2\left(\abs{\vu}_{H^2(Ω)}^2 + \abs{p}_{H^1(Ω)}^2\right)\] where \[ \norm{\vv, q}^2 ≝ \norm{∇\vv}^2_{L^2(Ω)} + α \sum_{K ∈ \mesh} h_K^2 \norm{∇q}^2_{L^2(K)}\]

\solution

\spart

As usual, multiply by test function $\vv$ and integrate:
\begin{align*}
- Δ\vu + ∇p &= \vf \\
- \int_Ω Δ\vu · \vv + \int_Ω ∇p · \vv &= \int_Ω \vf \vv
\end{align*}

Now we operate with details on the vector laplacian, and using integration by parts:
\begin{align*}
\int_Ω Δ\vu · \vv &= \int_Ω \sum_{i = 1}^N Δu_i v_i = \int_Ω \sum_{i,j = 1}^N \dpd[2]{u_i}{x_j} v_i \\
&= \sum_{i,j=1}^N \left(\int_{∂Ω} v_i \dpd{u_i}{x_j} - \int_Ω \dpd{u_i}{x_j} \dpd{v_i}{x_j} \right) ≝ - \int_Ω ∇\vu : ∇\vv
\end{align*}

For $∇p · \vv$, recall the divergence of the product: $\dv (p \vv) = p \dv \vv + ∇p · \vv$. By Gauss theorem, $\int_Ω \dv \vv = \int_{∂Ω} \vv$, so forcing $\vv$ to be zero on the boundary we have \[ \int_Ω ∇p · \vv = - \int_Ω p \dv \vv\]

Now we only have to multiply by $q$ in the continuity equation:
\[
\dv \vu = 0 \implies
\int_Ω q \dv \vu = 0
\]
and as it is zero, we can sum it in the previous one to have
\[ \underbracket{\int_Ω ∇\vu:∇\vv - \int_Ω p \dv \vv - \int_Ω q \dv \vv}_{a(u, p; v, q)} = \underbracket{\int_Ω \vf \vv}_{F(v)} \]

Regarding the spaces, we need $Q = L^2(Ω)$ and $V = [H_0^1(Ω)]^N$.

\spart

For this proof we are going to do the interpolation: we first go from $\vu$ to the interpolant (which we know how to bound) and then from there to the finite element approximation. By the triangular inequality, we have that
\[ \norm{\vu - \vu_h, p - p_h}^2 ≤ \norm{\vu - r_h \vu, p - Π_h p}^2 + \norm{r_h \vu - \vu_h, Π_h p - p_h} \] where $r_h$ is the Clément interpolant of $p$ and $Π_h$ the Lagrange interpolant of $\vu$, so that we directly have the following estimations
\begin{align*}
\norm{\vu - r_h \vu} &≤ Ch^2\abs{\vu}_{H^2(Ω)} \\
\norm{p - Π_hp} &≤ Ch \abs{p}_{H^1(Ω)}
\end{align*}


\end{problem}

\section{Session 6 - Convection-diffusion problem with stabilization: Well-posedness and a posteriori errors}

\begin{problem} Let $Ω ⊂ ℝ^2$ be a polygonal domain. We consider the convection-diffusion problem of finding a $\appl{u}{Ω}{ℝ}$ such that \[ \begin{cases}
-εΔu + \vb ∇u = f & \text{in } Ω \\
u = 0 & \text{on } ∂Ω
\end{cases}\] where $ε > 0$ is a constant and $\vb ∈ ℝ^2$. For any $h > 0$, let \mesh be a conformal mesh of Ω into triangles $K$ with diameter $h_K ≤ h$ and consider the finite dimensional space
\[ V_h = \set{v_h ∈ C^0(\adh{Ω}) \st \restr{v_h}{K} ∈ \pspace[1] \; ∀K ∈ \mesh} ∩ H_0^1(Ω) \]

We seek an $u_h ∈ V_h$ that verifies \[ \int_Ω \left(ε ∇u_h ∇v_h + (\vb ∇u_h)v_h - fv_h\right) + \sum_{K ∈ \mesh} α_K \int_K (-εΔu_h + \vb ∇\vu_h - f)(-εΔv_h + \vb∇v_h) = 0\] for all $v_h ∈ V_h$, where $α_K = \frac{h_K}{\norm{\vb}_{L^∞(K)}}$ if $\frac{h_K\norm{\vb}_{L^∞(K)}}{ε} ≥ 1$ and $α_K = \frac{h_K^2}{ε}$ otherwise.

\ppart Prove tha

\solution

\end{problem}


\backmatter

\chapter{Final remarks and quick summaries}
% -*- root: ../NumericalApproximationofPDEs.tex -*-

\section{Numerical approximation of PDEs II}

\subsection{Laplace equation}

The Laplace equation is the problem of finding $u$ such that \[ \begin{cases}
-Δu =f & \text{in } Ω \\
u = 0 & \text{on } ∂Ω
\end{cases}\]

We can prove regularity (\fref{thm:PDE:Shift}), a priori error estimates \[ \norm{u - u_h}_{L^2} + h \norm{∇(u - u_h)}_{L^2} ≤ Ch^2 \norm{∇^2 u}_{L^2} \] from \fref{prop:PDE:APrioriLaplace} and a Posteriori error estimates (\fref{prop:PDE:APostLaplace}) \[ \norm{∇(u - u_h)}_{L^2(Ω)}^2 ≤ C_1 \sum_{K ∈ \mesh} \left(h_K \norm{Δu_h + f}_{L^2(K)} + h_K^{\sfrac{1}{2}} \norm{[∇u_h]}_{L^2(∂K)}\right)^2 \]

For the first proof, the idea is to use Galerkin orthogonality to bound $\norm{u - u_h}$ by $\norm{u - v_h}$ for any $v_h$, and then choose the correct interpolant (Lagrange in this case). For the second, we translate the thing to bound to the residual, and then choose the correct interpolant for a $v_h$. In the case of the bound for $\norm{u - u_h}$ (not the gradient) note that we will need to find a function $φ$ such that $Δφ = u - u_h$ to perform the proof.

We can also build a posteriori gradient (\fref{prop:PostProcGradientConvergence}) which converges with $h^2$ if $u ∈ H^3$ and we have a parallel mesh, or $h^β, β ≤ 2$ if we just have structured meshes. We can also perform goal-oriented estimations (\fref{sec:PDE:GoalOrientedEstimation}), which has a posteriori error of $\oof{h^2}$.

% Witht the Laplace equation, we proved the regularity, a priori for the solution and gradient. A posteriori error estimates, goal oriented estimation and post processed ZZ gradient. THe proofs too. Know the error estimators too.
% Conditions on structured meshes and cancelation terms.

\subsection{Stokes problem}

The Stokes problem is defined as in \eqref{eq:PDE:StokesProblem}, with a finite element approximation in which we need to add a stabilization term to make the bilinear form injective. We can check that the inf-sup condition holds (\fref{sec:PDE:APrioriStokes}). As usual, the a posteriori error estimate (\fref{prop:PDE:APosterioriErrorStokes}) is done by means of the residual, which gives the bound
\[ \norm{∇(u - u_h)}^2_{L^2(Ω)} + \norm{p - p_h}_{L^2(Ω)}^2 ≤ C \sum_{K ∈ \mesh} η_K^2 \] for the solutions $u_h, p_h$ of the Galerkin problem, with \[ η_K^2 = \left(h_k \norm{f + Δu_h - ∇p_h}_{L^2(K)} + \frac{1}{2} h_K^{\sfrac{1}{2}} · \norm{[∇\vu_h · \vn]}_{L^2(∂K)}\right)^2 + \norm{\dv u_h}^2_{L^2(K)}\]

\subsection{Optimal control problem}

The optimal control problem is similar to the Stokes problem. The infsup condition (\fref{prop:PDE:OptimalControlInfsup}) is a bit tricky as it needs to check with $a(u,λ; u,λ) - βa(u,λ; -λ, u)$ the first bound. That bound gives directly the election for the second bound. The a posteriori error estimate (\fref{prop:PDE:APostOptimalControl}) \[ \norm{u - u_h, λ - λ_h}^2 ≤ C \sum_{K ∈ \mesh} η_K^2 \] with
\begin{multline*}
η_K^2 =
	\left(
		h_K \norm{f + Δu_h + \frac{1}{α} λ_h}_{L^2(K)}
		+ \frac{1}{2} h_K^{\sfrac{1}{2}} \norm{[∇u_h · \vn]}_{L^2(∂K)}
	\right)^2 + \\
	+ \left(
		h_K \norm{\ind_{Ω_0} (u_0 - u_h) + Δλ_h}_{L^2(K)}
		+ \frac{1}{2} h_k^{\sfrac{1}{2}} \norm{[λ_h · \vn]}_{L^2(∂K)}
	\right)^2
\end{multline*} can be proven by using the inf-sup condition: that relates the residual to the error we want to bound, and then we operate and use Cauchy-Schwartz inequalities to finish with the Clément interpolant.

\subsection{Nonlinear problems}

Adding non-linear terms (\fref{sec:PDE:NonLinearProbs}) to the equations can increase the complexity. A priori error and uniqueness of the solution are easy (\fref{prop:NonLinearProblemExistenceUniqueness}), although for the existence proof a fixed point theorem is required. The Galerkin approximation is computed by the use of the Newton method.

A posteriori error estimates (\fref{prop:PDE:APostNonlinearProblem}) can be obtained, as always, by using coercivity and then the residual. Specifically, the bound is
\[ \norm{∇(u - u_h)}^2_{L^2(Ω)} ≤ C \sum_{K ∈ \mesh} η_K^2 \] with \[ η_K = h_K \norm{f + Δu_h - u_h^3}_{L^2(K)} + \frac{1}{2} \sqrt{h_K} \norm{[∇u_h · \vn]}_{L^2(∂K)} \]

\subsection{Heat equation}

The usual heat equation is also studied. An \nref{prop:PDE:APrioriHeatEq} can be obtained just multiplying the equation by the solution and operating, which in turn gives uniqueness. An \nref{prop:PDE:APrioriHeatSpace} can also be obtained, which is
 \[ \int_0^T \int_Ω \abs{∇(u - u_h)}^2 ≤  C \int_0^T \sum_{K ∈ \mesh} η_K^2 + \text{higher order terms}  \] with \[ η_K^2 = \left(h^2_K\norm{f+ Δu_h - \pd{u_h}{t}} + \frac{1}{2} \sqrt{h_K} \norm{[∇u_h · \vn]}_{L^2(∂K)} \right)^2 \]

\subsection{Hyperbolic equations}

In this case we need to add a stabilization term such that $v_h = u_h + δ_h \pd{u_h}{t}$ to have stability. That gives us an \nref{prop:PDE:APrioriTransport} of \[ \norm{u(T) - u_h(T)}_{L^2(Ω)}^2 ≤ Ch^3 \] and an \nref{prop:PDE:APosterioriTransport} of \[ \norm{u(T) - u_h(T)}_{L^2(Ω)}^2 ≤ \norm{u(0) - u_h(0)}^2_{L^2(Ω)} + C \int_0^T \sum_{K ∈ \mesh} η_K^2 \label{eq:PDE:TransportAPosteriori} \] with \[ η_K^2 ≝ h_K \norm{f - \dpd{u_h}{t} - \vb ∇u_h}_{L^2(K)} \norm{∇(u - u_h)}_{L^2(ΔK)} \]

\subsection{Some tricks}

\begin{lemma}[Clément\IS interpolant bounds]\label{lem:PDE:ClementBounds} Let $\mesh$ be a regular mesh of a domain $Ω$, $h_K$ the size of each mesh element $K ∈ \mesh$ and $R_h$ the Clément interpolation operator. Then, for any $v ∈ H_0^1(Ω)$ we have the following estimators:
\begin{align*}
\norm{v - R_h v}_{L^2(K)} &≤ C h_K \abs{v}_{H^1(ΔK)} \\
\norm{∇v - ∇R_h v}_{L^2(K)} &≤ C\abs{v}_{H^1(ΔK)} \\
\norm{v - R_h v}_{L^2(∂K)} &≤ Ch_K^{\sfrac{1}{2}} \abs{v}_{H^1(ΔK)}
\end{align*} where $ΔK$ is the set of all neighbouring triangles of $K$.
\end{lemma}



\bibliography{../EPFLNotes.bib}


\printindex
\printtheorems
\end{document}

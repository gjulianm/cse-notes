\documentclass[palatino]{epflnotes}

\title{Differential Geometry of Framed Curves}
\author{Guillermo Julián Moreno}
\date{16/17 - Fall semester}

% Additional packages
\usepackage{tikztools}
% --------------------

\begin{document}
\frontmatter
\pagestyle{plain}
\maketitle

\tableofcontents
\mainmatter
% Content

\chapter{Framed curves or curves in $SE(3)$}

\begin{figure}[hbtp]
\inputtikz{FramedCurve}
\caption{The object of our study is the study of framed curves in the space of rigid body motions. In this space, we represent the position of an object (a displacement $\vr$) but also its rotation in space, represented by the transformation of the frame of reference (in purple).}
\label{fig:FramedCurve}
\end{figure}

During this course, we will study curves in the special euclidean group $SE(3)$, the Lie group of matrix representing rigid body motions. It is called like this because, as explained in \fref{fig:FramedCurve}, we want to have a displacement and also a transformation of the frame of reference.

This has a direct translation to the matrix representation of $SE(3)$: \(
\left(\begin{array}{ccc|c}
 & & &  \\
 & R & & \vec{r} \\
 & & &  \\ \hline
 & \vec{0} & & 1
\end{array}\right) \label{eq:SE3MatrixRepr} \), where $R ∈ SO(3)$ is a matrix representing the rotation and $\vr ∈ ℝ^3$ is a vector representing the movement or position in space.

The fact that $R ∈ SO(3)$ (special orthogonal group) gives us the necessary restrictions on allowable rotations. Rotations of the frame of reference must be isometries, and must preserve orientations. Thus, $\det R = 1$ for all $R ∈ SO(3)$.


\section{The living space $SE(3)$}

Once we now the reason of working in this space, we can go for the formal definitions and see how do operations behave in here.

\begin{defn}[Special orthogonal group $SO(N)$] The special ortogonal group $SO(N)$ of dimension $N$ is the group of isometries of $ℝ^N$ that maintain the origin as a fixed point and that preserve orientation.

Elements of this space can be represented an $N × N$ matrix with determinant $1$ (that is, $N$ vectors representing the frame of reference which are orthogonal and preserve orientation). These matrixes have the property that, for $R ∈ SO(N)$, we have that $\trans{R} = \inv{R}$.
\end{defn}

\begin{defn}[Special Euclidean group $SE(N)$] The special euclidean group $SE(N)$ of dimension $N$ is the group of all isometries of $ℝ^N$ that preserve orientations. These isometries can be thought of as a rotation of $SO(N)$ and a displacement given by a vector $ℝ^N$, and with the matrix representation from \eqref{eq:SE3MatrixRepr}.
\end{defn}

So, now on for computations, we can see how multiplication works \( \label{eq:SE3Mult} \begin{pmatrix} Q_1 & q_1 \\ 0 & 1 \end{pmatrix} ·  \begin{pmatrix} Q_2 & q_2 \\ 0 & 2 \end{pmatrix} = \begin{pmatrix} Q_1 Q_2 & Q_1q_2 + q_1 \\ 0 & 1 \end{pmatrix} \) and also the inverse operation \( \begin{pmatrix} Q & q \\ 0 & 1 \end{pmatrix}^{-1} = \begin{pmatrix} \trans{Q} & - \trans{Q} q \\ 0 & 1 \end{pmatrix} \)

Something about relative rotations that I will copy later.

\section{Curves in $SE(3)$}

We are not only interested in the static picture of $SE(3)$: we actually want to study uniparametric families of these matrixes or, in other words, curves in $SE(3)$.

What we will study will be configurations of a ``tube'', ``rod'' or filaments, which actually will be considered as curves $X(τ) ∈ SE(3)$ for $τ ∈ (a,b)$.

Our first example is a curve in $ℝ^3$ with its Frenet frame of reference. Let's define both things:

\begin{defn}[Frenet\IS frame of reference] \label{def:FrenetFrame}
\end{defn}

\begin{defn}[Frenet\IS curve] A Frenet curve $\mathcal{F}(τ)$ in $SE(3)$ is a regular curve $\vr(τ)$ in $ℝ^3$ together with its \nref{def:FrenetFrame}: \[ \mathcal{F}(τ) = \begin{pmatrix} F(τ) & \vr(τ) \\ 0 & 1 \end{pmatrix} \]
\end{defn}

\section{Curves in $SO(N)$}

We are studying curves $R(t) ∈ SO(N)$ with $t ∈ (a,b)$, so $\trans{R}(t)R(t) = \mathrm{Id}$, which are one-parameter families of matrices. We are going to assume that the family is smooth so we can differentiate. In that case, we can obtain some properties: \begin{align*}
\trans{R} R &= \mathrm{Id}\\
\trans{R}R' + \trans{R'}R &= 0 \\
(\trans{R}R') + (\trans{R}R')^{\mathrm{T}} &= 0
\end{align*}

So, if we say $S = \trans{R}R'$, $S$ is a skew-symmetric matrix ($S = - \trans{S}$). These matrixes have $\frac{1}{2}n(n+1) - n$ independent entries, so in case $n = 3$ it means that there are $3$ independent entries and thus there is a mapping between skew-symmetric $3 × 3$ and vectors in $ℝ^3$ given by \[ [u_1, u_2, u_3] = \vu^{×} = \begin{pmatrix}0 & -u_3 & u_2 \\ u_3 & 0 & -u_1 \\ -u_2 & u_1 & 0 \end{pmatrix} \]

The nice thing about this is that $\vu × \vv = \vu^× \vv$.

So now we are searching for something similar to the Frenet frame. This takes us to the \concept{Basic kinematic equation} for a curve in $SO(N)$: \( R' = \underbracket{R\trans{R}}_{\mathrm{Id}} R' = R S \) with $S$ a skew-symmetric matrix. We can also have coefficients on the right \( R' = \tilde{S} R \) with $\tilde{S} = R S \trans{R}$.

In the case $N = 3$, we have $S = \vu^×$ and $\tilde{S} = \tilde{\vu}^×$. The question is what is the relation between the triple $\vu$ and the triple $\tilde{\vu}$. For the easy case $R\vu$ we compute $\vec{w} (Q\vu)^× \vv$ for $Q ∈ SO(N)$ and after some horrible matricial computations we end with \( (Q\vu)^× = Q\vu^×\trans{Q} \) and finally  $\tilde{\vu}^× = (R\vu)^×$ so $\tilde{\vu} = Ru$, which is the change of bases formula. So if $\vu$ has components in the frame $R$, then $\tilde{\vu}$ has components in the canonical bases. So finally again u twiddle down (I don't know how to write undertwiddle in LaTeX?) is the Darboux vector for the curve $R(t) ∈ SO(3)$. So the down twiddle means real vector.

We also identify the columns of any $R ∈ SO(3)$ with an orthonormal bases $\set{\vec{d}_1, \vec{d}_2, \vec{d}_3}$. So the equation $R' = \tilde{u}^× R$ is equivalent to $\vd_i' = \vu × \vd_i$.

Question now is question 1 of series 2, given a curve $\vr(t) ∈ ℝ^3$, find the Darboux vector $\vu$ for any adapted framing $R(t) ∈ SO(3)$ such that $\vd_3(t) = \vt(t) = \frac{\vec{τ}'(t)}{\norm{\vr'(t)}}$.

Aside: integrals of systems of first order ordinary differential equations. For an initial value problem and under some hypothesis there exists a unique solution. The system is autonomous if there are not explicit dependences on $t$; that is $y' = f(y(t))$. Then a first integral is some function $\appl{G}{ℝ^N}{R}$ such that $\od{}{t}G(y(t)) = 0$ along solutions of the function. We will be focusing for reasons that right now escape my comprehension on linear systems $y' = Ay$, which we will be able to re-arrange as linear matrix equations. So going back to our something $\vd_i' = \vu^× \vd_i$. The claim is that these equations have the integrals $\vd_i · \vd_j$ for $i,j = 1,2,3$. So in other words \[ \od{}{t} (\vd_i · \vd_j) = 0\qquad ∀t ∈ (a,b) \] and then \[ \vd_i \vd_j' + \vd_i' \vd_j = \vd_i (\vu × \vd_j) + (\vu × \vd_i) \vd_j = \vu (\vd_j × \vd_i + \vd_i × \vd_j) = \vu · 0 = 0\] so if $\vd_i(0) · \vd_j(o) = δ_{ij}$ then $\vd_i(t) · \vd_j(t) = δ_{ij}$ (a frame that starts orthonormal ends orthonormal).

So question 1 series 2 (this is still an exercise?) is that $\vu = \vr' × \vr'' + g(t) \vr'$ with $g(t)$ an arbitrary function. Knowing that the Frenet frame $F$ is an adapted frame we know that $F' = F \vu^× = \tilde{\vu}^× F$ and there's a single Darboux vector $\vu$ such that $\vd_i' = \vu × \vd_i$. So yeah, it exists and it is unique.

On the other hand, from last week we remember that \[ F' = F · \begin{pmatrix} 0 & -τ & κ \\ τ & 0 & 0 \\ κ & 0 & 0 \end{pmatrix} \] so $\vu(s) = 0 \vn + κ \vb + τ \vt$.

Overview/summary: Any curve (one-parameter family of matrices $SO(3)$) has a Darboux vector, which has components with respect to fixed bases $\tilde{u}$ and the frame $R(t)$ (moving frame something Cartan) and $R' = \tilde{vu}^× R = R \vu$. Notice no mentions of any curve in $ℝ^3$.

On the other hand if $\vr(s) ∈ ℝ^3$ is an arc-length framed curve is $C^3$ and $\norm{\vr''} ≠ 0$ (that is $κ(s) > 0$) then the Frenet Frame $F(s)$ is a curve in $SO(3)$ and then the associated Darboux vector is $\vu = \begin{pmatrix} 0 & κ & τ \end{pmatrix}^T$. The Frenet frame is adapted to the underlying curve $\vr(s)$ because somethin.

The Frenet frame is intrinsic (knowledge of $r(s)$ and derivatives implies $F(s)$). Question 1 series 2 implies that non-intrinsic adapted framings of $r(t)$ incolve an additional function $g(s)$.

\section{Curves in $SE(3)$}

Further clarify the connections between cures in $ℝ^3$ and curves in $SO(3)$. Our curves $X(t) ∈ SE(3)$ with \[ X(t) = \begin{pmatrix} Q(t) & q(t) \\ 0 & 1 \end{pmatrix} \qquad Q(t) ∈ SO(3),\, q(t) ∈ ℝ^3\]

We want to ask the question: what's the analog of $R' = RS$ with $S$ skew-symmetric for $R ∈ SO(3)$? There's no exact analog in $SE(3)$ to $R\trans{R} = \mathrm{Id}$ so we don't have the same starting point. Luckily, we know that \[ \inv{X} = \begin{pmatrix} \trans{Q} & - \trans{Q} q \\ 0 & 1 \end{pmatrix}\qquad V' = \begin{pmatrix}QS & q' \\ 0 & 0 \end{pmatrix}\] and computing $X' = X\inv{X} X'$ we can compute $\inv{X}X'$ and then \[ \inv{X}X' = \begin{pmatrix} S & \trans{Q}q' \\ 0 & 0 \end{pmatrix} \]

Again, our ODE would be $X' = X\begin{pmatrix}\vu^× & \vv \\ 0 & 0 \end{pmatrix}$ where $\vu$ compoennts of Darboux vector in the $Q$ frame and the $\vv = \trans{Q}q'$. But $q'$ is the tangent vector to $q(t)$ with respect to the canonical basis so $\vv$ is actually the tangent vector with respect to the $Q$ frame.

Another question: when is a curve in $SE(3)$ equivalent to an adapted framing. The answer is that these curves naturally combine curves in $ℝ^3$ and curves in $SO(3)$. In general, $Q$ will not be adapted to $q$. When is $Q$ adapted to $q$, which must be the case sometimes, e.g. Frenet frame)? the key is to look at the matricial equations of before and see that $q' = Q\vv$ so $q'$ is $\vd_3$ if an only if $\vv = (0,0,1)^T$ so we have to take this specific case of a differential equation in order to get an adapted frame.

Something something $D' = DU$ where $U$ is skew and exists a vector $u$ such that $u^× = U$ and then $\vu = D u$.

\chapter{Linking theorem}

\begin{defn}[Link of a curve] Given a curve $\appl{x}{[a,b]}{ℝ^3}$ with adapted framing $R(t) = (\vd_1(t), \vd_2(t), \vd_3(t))$, we can define an adjacent curve such as $y(t) = x(t) + ε \vd_1(t)$. Then, the linking number is \[ \lk (x,y) = \frac{1}{4π} \int_a^b \int_a^b \frac{\left(y(t_2) - x(t_1)\right)  ·\left((y'(t_1) ∧ x'(t_1)\right)}{\norm{y(t_2) - x(t_1)}^3} \dif t_1 \dif t_2 \]
\end{defn}

We can define other two things

\[ \mop{Tw}(x,R) = \frac{1}{2π} \int_a^b u_3(t) \dif t\]

\[ \mop{Wr}(x) = \frac{1}{4π} \int_a^b \int_a^b \frac{\left(x(t_2) - x(t_1)\right)  ·\left((x'(t_1) ∧ x'(t_1)\right)}{\norm{x(t_2) - x(t_1)}^3} \dif t_1 \dif t_2 \]


\begin{theorem}[Horrendousnameguy theorem] If $x$ is regular and smoothly closed ($x(b) = x(a) $ and $x'(b) = x'(a)$) and the same with $R$, we then define $z(t,η) = x(t) + η\vd_1(t)$ with $η∈(0,ε)$ smoothly closed $x ∩z = \emptyset$ for all $n$? The heck is $n$? Actually we want to have curves $z$ not intersecting $x$.

Then \[ \lk  = \mop{Tw} + \mop{Wr}\]
\end{theorem}

The linking number can be defines for any two curves closed.

\section{Linking number}

Several properties.

\begin{prop} Let $x(t) ∈ SE(3)$ be a curve with adapted frame, and then let $x_s(t) = x(s(t))$ be a reparametrization of $x$, preserving orientation. Then, $\lk (x,y) = \lk (x_s,y)$.

In other words, the linking number is invariant over orientation-preserving reparametrizations.

If the reparametrization changes the orientation, the linking number changes sign.
\end{prop}

\begin{prop} The link number is independent of dilations ($\lk (x,y) = \lk (λx, λy)$ for $λ > 0$), of translations ($\lk (x,y) = \lk (x + a, y + a)$ for $a ∈ ℝ^3$) and of rotations ($\lk (x,y) = \lk (Rx,Ry)$ for $R ∈ SO(3)$).

\end{prop}

\begin{prop} The link number is invariant by homotopy.
\end{prop}

\begin{prop}[Signed crossing]
Some weird rule for computing the number based on crossings.
\end{prop}


Something infinity (physically unlinked curves have link number 0). The converse is not true.

\begin{prop} $\lk(C_1 ∪ C_2, C_3) = \lk(C_1, C_3) + \lk (C_2, C_3)$.
\end{prop}

Define \( e(s,σ) = \frac{y(s) - x(σ)}{\norm{y(s) - x(σ)}} \label{eq:LinkNormal} \) then \[ e · \left(∂_σ e ∧ ∂_s e\right) = \frac{\left(y(s) - x(σ)\right)  ·\left((y'(s) ∧ x'(σ)\right)}{\norm{y(s) - x(σ)}^3}   \] so we can replace that in the equation for the linking number.

Thenif we define a 2-manifold α orientable and regular ($\norm{∂_s α ∧ ∂_σα} ≠ 0$) and then the linking number is the signed area of something.

Something something curves in $SE(3)$. Then  \[ χ' = χa \qquad\text{ with } a =  \begin{pmatrix} u^× & v \\ 0 & 0\end{pmatrix} ∈ SO(3) \] and $u$ is the components in the $X$ basis of the Darboux vector and $u = X u$ for some combination of twiddles and bars for $u$.

Something something $u(t), v(t)$ as arbitrary functions of $t$ plus an initial condition we can use existence and uniqueness and then define uniquely χ up to a rigid body motion. If $u,v$ are constant we have a helix.

$t$ is arc-length if $\norm{x'} = 1$, and provided that $\norm{x'} ≠ 0$ we can always find an arc-length parametrization.

Something something adapted framing if $x'(s) = d_3(s)$ and Darboux vector of any arbitrary function frame $X(s)$ adapted to $x(s)$ is \[ u(s) = x' ∧ x'' + u_3(s) x'\]

Most common is Frenet Frame $F(s) = \begin{pmatrix}\vn(s) & \vb(s) & \vt(s)\end{pmatrix}$

We are interested in adapted framings, some intrinsic (e.g. the Frenet frame) and some extrinsic.

Jump to CFW. Then $y(σ) = x(σ) + ε \vd_(σ)$ so an adapted framing is \[ Y(σ) = X(σ) \begin{pmatrix}  & ε \\ Q(σ) & 0 \\ & 0 \\ 0 & 1 \end{pmatrix} \] and a drawing of a thing.

Remark if you can give $κ$ and τ as functions of arc-length you can recover $x(s)$ and $F(s)$.

Curves on surfaces: Nappe of a surface $\appl{α}{Ω⊂ ℝ^2}{ℝ^3}$ and now let's consider a curve $w(σ)$ with σ arc-length parameter. THen we consider $α(s(σ), t(σ))$ for some functions $\appl{s,t}{ℝ}{ℝ}$. Differentiating we get \[ \dot{w} = α_s \dot{s} + α_t \dot{σ} \] so $\set{α_s, α_t}$ is a basis for the tangent plane to the surface $S$, so $\dot{w}$ is tangent to $S$ and so we can define \[ N(s,t) = \frac{α_s ∧ α_t}{\norm{α_s ∧ α_t}}  \] is a unit surface normal well-defined if $α_s ∧ α_t ≠ 0$. We have only two choices of orientation.

If $w(σ)$ is regular then $\dot{w} ≠ 0$ so $\vt(σ) = \frac{\dot{w}}{\norm{\dot{w}}}$ is well-defined and $\set{N, B, t}$ is an adapted extrinsic framing of $w(σ)$ where $B = \vt ∧ N$.

Tantrix says that $x(s) = t(s)$ is a unit vector field. But that does not let you recover the curver $x(s)$ because you need a tantrix + mapping $σ$ in Tantrix to s in $x(s)$, because σ may not be arc-length in $t(s)$ which is the curve on the unit ball.

If we get two closed curves, then something.

\section{Writhe}

We will be working with a single closed curve $\appl{\vx}{[0,L]}{ℝ^3}$ with $\vx(0)= \vx(L)$, and ``injective'' in the sense that $\vx(s) = \vx(σ)$ if and only if $s = σ + kL$, with $k ∈ ℕ$.

\begin{defn}[Writhe]
\( \mop{Wr}(\vx) = \frac{1}{4π} \int\limits_{C_1}\int\limits_{C_1}\frac{(\vx(σ) - \vx(s)) · (\vx'(σ) ∧ \vx'(s))}{\norm{\vx(σ) - \vx(s)}^3} \dif s \dif σ \label{eq:Writhe} \) with the denominator not vanishing because of injectivity.
\end{defn}

In the linking number, many of the properties came from the unit vector \eqref{eq:LinkNormal}, which is well-defined everywhere. We can still introduce the analogous vector given by \( \ve(σ, s) = \frac{\vx(σ) - \vx(s)}{\norm{\vx(σ) - \vx(s)}} \) which is well defined away from diagonal but presents a problem, which is that $\lim_{σ \to s} = \pm x'(s)$ depending on whether we approach from above from or from below.

Result: integral in \eqref{eq:Writhe} is convergent. Taylor and prove that denominator goes to zero faster. I'm not copying a proof by Taylor expansion.
Without loss of generality, assume that σ and $s$ are both arc-length.\footnote{Exercises: Writhe is invariant under reparametrization}.

Soooo... $\mop{Wr}(x)$ is an appropriate signed area of $\ve(σ, s)$ on the unit sphere. But for discontinuity in $\ve(σ, s)$ on diagonal invalidates many key arguments done with $\mop{Lk}$. So writhe is not invariation under deformation in $x$. And there are results relating area enclosed by the Tantrix of $\vx(s)$ to $\mop{Wr(\vx)}$.

Something something signed crossing. For link, pick a projection and count signed corssing. For Wrthe, sum of signed crossing depends on projection so not invariant under projection changes.

Global Radius curvature and self-avoiding tubes and optimal packings. Any three points in space define an unique circle in space. And then we ask which is the smallest possible circle. Second order tangential contact so circle only defined by two points, being twangent to the curve at one of them. Some extra calcs. Take two points by angle θ and \[ \sin θ = \frac{\norm{\vy - \vx}}{R}\] and then we can define \[ \mop{Wr}(\vx) = \frac{1}{4π} \iint\limits_{C_1×C_1} \frac{F(s,σ)}{d(x(σ), x(s)) · d(x(s), x(σ))} \dif s \dif σ \] with \[ F(s,σ) = \frac{\ve (\vx'(σ) ∧ \vx'(s))}{4 \sin θ(s) \sin θ(σ)}, \quad \sin θ(s) = \abs{\ve ∧ \vx'(s)} \]

\seprule
\clearpage

Some classes after... final class!

\appendix
\chapter{Exercises}
% -*- root: ../DifferentialGeometryofFramedCurves.tex -*-


\backmatter
\printindex
\end{document}

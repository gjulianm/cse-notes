\documentclass[palatino]{epflnotes}

\title{Differential Geometry of Framed Curves}
\author{Guillermo Julián Moreno}
\date{16/17 - Fall semester}

% Additional packages
\usepackage{tikztools}
% --------------------

\begin{document}
\frontmatter
\pagestyle{plain}
\maketitle

\tableofcontents
\mainmatter
% Content

\chapter{Framed curves or curves in $SE(3)$}

\begin{figure}[hbtp]
\inputtikz{FramedCurve}
\caption{The object of our study is the study of framed curves in the space of rigid body motions. In this space, we represent the position of an object (a displacement $\vr$) but also its rotation in space, represented by the transformation of the frame of reference (in purple).}
\label{fig:FramedCurve}
\end{figure}

During this course, we will study curves in the special euclidean group $SE(3)$, the Lie group of matrix representing rigid body motions. It is called like this because, as explained in \fref{fig:FramedCurve}, we want to have a displacement and also a transformation of the frame of reference.

This has a direct translation to the matrix representation of $SE(3)$: \(
\left(\begin{array}{ccc|c}
 & & &  \\
 & R & & \vec{r} \\
 & & &  \\ \hline
 & \vec{0} & & 1
\end{array}\right) \label{eq:SE3MatrixRepr} \), where $R ∈ SO(3)$ is a matrix representing the rotation and $\vr ∈ ℝ^3$ is a vector representing the movement or position in space.

The fact that $R ∈ SO(3)$ (special orthogonal group) gives us the necessary restrictions on allowable rotations. Rotations of the frame of reference must be isometries, and must preserve orientations. Thus, $\det R = 1$ for all $R ∈ SO(3)$.


\section{The living space $SE(3)$}

Once we now the reason of working in this space, we can go for the formal definitions and see how do operations behave in here.

\begin{defn}[Special orthogonal group $SO(N)$] The special ortogonal group $SO(N)$ of dimension $N$ is the group of isometries of $ℝ^N$ that maintain the origin as a fixed point and that preserve orientation.

Elements of this space can be represented an $N × N$ matrix with determinant $1$ (that is, $N$ vectors representing the frame of reference which are orthogonal and preserve orientation). These matrixes have the property that, for $R ∈ SO(N)$, we have that $\trans{R} = \inv{R}$.
\end{defn}

\begin{defn}[Special Euclidean group $SE(N)$] The special euclidean group $SE(N)$ of dimension $N$ is the group of all isometries of $ℝ^N$ that preserve orientations. These isometries can be thought of as a rotation of $SO(N)$ and a displacement given by a vector $ℝ^N$, and with the matrix representation from \eqref{eq:SE3MatrixRepr}.
\end{defn}

So, now on for computations, we can see how multiplication works \( \label{eq:SE3Mult} \begin{pmatrix} Q_1 & q_1 \\ 0 & 1 \end{pmatrix} ·  \begin{pmatrix} Q_2 & q_2 \\ 0 & 2 \end{pmatrix} = \begin{pmatrix} Q_1 Q_2 & Q_1q_2 + q_1 \\ 0 & 1 \end{pmatrix} \) and also the inverse operation \( \begin{pmatrix} Q & q \\ 0 & 1 \end{pmatrix}^{-1} = \begin{pmatrix} \trans{Q} & - \trans{Q} q \\ 0 & 1 \end{pmatrix} \)

Something about relative rotations that I will copy later.

\section{Curves in $SE(3)$}

We are not only interested in the static picture of $SE(3)$: we actually want to study uniparametric families of these matrixes or, in other words, curves in $SE(3)$.

What we will study will be configurations of a ``tube'', ``rod'' or filaments, which actually will be considered as curves $X(τ) ∈ SE(3)$ for $τ ∈ (a,b)$.

Our first example is a curve in $ℝ^3$ with its Frenet frame of reference. Let's define both things:

\begin{defn}[Frenet\IS frame of reference] \label{def:FrenetFrame}
\end{defn}

\begin{defn}[Frenet\IS curve] A Frenet curve $\mathcal{F}(τ)$ in $SE(3)$ is a regular curve $\vr(τ)$ in $ℝ^3$ together with its \nref{def:FrenetFrame}: \[ \mathcal{F}(τ) = \begin{pmatrix} F(τ) & \vr(τ) \\ 0 & 1 \end{pmatrix} \]
\end{defn}

\appendix
\chapter{Exercises}
% -*- root: ../DifferentialGeometryofFramedCurves.tex -*-
\section{Series 1}

\section{Series 2}

\begin{problem} \textbf{General form of the Darboux vector of an adapted framing of a given curve.}

Given a smooth curve $\vr(s)$ and a function $u_3(s)$ where $s$ is the arclength parameter, we will show that \( \vec{ξ}' = (u_3\vr' + \vr' ×\vr'')× \vec{ξ} \) with initial condition \( \vec{ξ}(0) · \vr'(0) = 0\qquad \norm{\vec{ξ}(0)}^2 = 1 \label{eq:Series2:Ex2:InitialCond} \) generates an orthonormal framing $(\vec{ξ}, \vr' × \vec{ξ}, \vr')$ of $\vr(s)$.

Verify and calculate

\ppart That if $\norm{\vec{ξ}(0)}^2 = 1$, then $\norm{\vec{ξ}(s)}^2 = 1\;∀s$.
\ppart That if $\vec{ξ}(0) · \vr'(0) = 0$ then $\vec{ξ}(s) · \vr'(s) = 0$.
\ppart Now, by picking an initial value of $\vec{ξ}(0)$ satisfying \eqref{eq:Series2:Ex2:InitialCond} we have an orthonormal frame $(\vec{ξ}, \vr' × \vec{ξ}, \vr')$ of $\vr(s)$. What is the Darboux vector?

\solution

\spart

To do that, we compute the derivative of the modulus and see what happens:
\[ \od{}{s}\norm{\vec{ξ}}^2 = \od{}{s}(\vec{ξ}·\vec{ξ}) = 2\vec{ξ}'\vec{ξ} = 2 · (\dotsb) × \vec{ξ} · \vec{ξ} = 0\]

So the modulus is constant, and given the initial condition $\norm{\vec{ξ}(0)}^2 = 1$ then $\norm{\vec{ξ}(s)}^2 = 1$.

\spart

Again, same idea. \begin{align*}
\od{}{s} (\vec{ξ}(s) · \vr'(s))
	&= \vec{ξ}' \vr' + \vec{ξ} · \vr''
	 = \left((u_3\vr' + \vr' ×\vr'') × \vec{ξ}\right) · \vr' + \vec{ξ} · \vr'' = \\
	&= \underbracket{(u_3 \vr' × \vec{ξ}) · \vr'}_{=0} + (\vr' × \vr'') × \vec{ξ} · \vr' + \vec{ξ} · \vr'' = \\
	&= -((\vr'' · \vr') \vec{ξ} - (\vec{ξ} · \vr') \vr'') · \vr' + \vec{ξ} \vr'' = \\
	&= \dots
\end{align*} it's 0 so cool.

\spart

The Darboux vector is $(u_3\vr' + \vr' ×\vr'')$. Computations to see that it is the case that $(u_3\vr' + \vr' ×\vr'') × \vd_i = \vd_i'$.

\end{problem}


\backmatter
\printindex
\end{document}

\documentclass[palatino]{epflnotes}

\title{Harmonic Analysis}
\author{Guillermo Julián Moreno}
\date{16/17 - Spring semester}

% Additional packages

\precompileTikz
% --------------------

\begin{document}
\frontmatter
\pagestyle{plain}
\maketitle

\tableofcontents
\mainmatter
% Content

\chapter{Fourier analysis}

\section{Basic structures in Banach and Hilbert spaces}

Fourier analysis is the study of how Fourier series and Fourier transforms work, and these are esentially a ``change of basis'' in the space of functions. In order to study them correctly, we will first review some facts of the structure of functions in Banach and Hilbert spaces. Just remember that the $L_p$ spaces defined by \[ L_p(Ω) = \set{\appl{f}{Ω}{ℝ} \st \norm{f}_{L^p} ≝ \left(\int_{Ω} \abs{f}^{p} \dif μ\right)^{\sfrac{1}{p}} < ∞ } \] for $1≤ p ≤ ∞$, $Ω ⊂ ℝ^d$ and μ a corresponding measure for Ω, are Banach spaces. In the particlar case of $p = 2$ we have a Hilbert space with the scalar product defined as \[ \pesc{f,g} = \int_Ω f \conj{g} \dif μ \]

In these spaces we can define an operation, the convolution, which will become useful in the analysis of Fourier series.

\subsection{Convolution and approximate identities}

\begin{defn}[Convolution] \label{def:Convolution} Given two functions $f ∈ L^p(Ω)$ and $g ∈ L^q(Ω)$, the convolution of both is defined as \[ (f \ast g) (x) ≝ \int_Ω f(x-y) g(y) \dif y\]
\end{defn}

The convolution is an operation that ``plays nicely''. It has the follwing result giving a bound on the norm of the convolution.

\begin{prop} \label{prop:YoungInequality} Let $1 ≤ p,q,r ≤ ∞$, and let $f ∈ L^p(Ω)$, $g ∈ L^q(Ω)$. Then, \[ \norm{f*g}_r≤ \norm{f}_p \norm{g}_q \] and so $f \ast g ∈ L^r(Ω)$ when $1 + \frac{1}{r} = \frac{1}{p} + \frac{1}{q}$.

In particular, if $g ∈ L^1$ and $f ∈ L^p$, then $f * g ∈ L^p$.
\end{prop}

It also goes well with continuous functions.

\begin{prop} \label{prop:ConvolutionContinuous} Let $f ∈ C^0(Ω)$ and $g ∈ L^1(Ω)$. Then $f * g ∈ C^0(Ω)$.
\end{prop}

\begin{proof} Let $y ∈ Ω$ such that $\abs{x - y} < δ$. We can try and bound so that \begin{align*}
\abs{(f*g)(x) - (f*g)(y)}
	&= \abs{\int_Ω f(x - t) g(t) \dif t - \int_Ω f(y - t) g(t) \dif t} = \\
	&= \abs{\int_Ω f(x - t) - f(y - t) g(t) \dif t } ≤ \\
	&≤ \int_Ω \abs{f(x - t) - f(y - t)} \abs{g(t)} \dif t ≤ \\
	&≤ ε \int_Ω \abs{g(t)} \dif t ≤ ε \norm{g(t)}
\end{align*} and we have it.
\end{proof}

The convolution does not have an identity in the space of functions, as there is no function $φ$ such that $φ * f = f$. However, we can construct approximations:

\begin{defn}[Approximate\IS identity] \label{def:ApproximateIdentity} A family of functions $\set{φ_N}_{N ≥ 0} ⊂ L^∞(Ω)$ is called a family of approximate identities if and only if:
\begin{enumerate}
	\item $\int_Ω φ_N = 1$.
	\item $\sup_{N ≥ 0} \norm{φ_N}_{L^1} < ∞$.
	\item Their support goes to zero. That is, for any $δ > 0$ the following holds: \[ \lim_{N \to ∞} \int_{\abs{x} > δ} \abs{φ_N} = 0\]
\end{enumerate}
\end{defn}

The main property of these functions is that they are indeed approximations of the identity element for the convolution:

\begin{prop} \label{prop:ApproximateIdentity} Let $\set{φ_N}_{N ≥ 0}$ be a family of approximate identities, and let $f ∈ L^p(Ω)$ for $1 ≤ p < ∞$. Then, \[ \lim_{N \to ∞} \norm{φ_N * f - f}_p = 0 \]
\end{prop}

\begin{proof}

\proofpart{$p = 1$}

We prove it first for $p = 1$. There, operating we have
\begin{align*}
\norm{φ_N * f - f}_1
	&= \int_Ω \abs{(φ_N * f)(x) - f(x)} \dif x = \\
	&= \int_Ω \abs{\int_Ω φ_N(y) f(x - y) \dif y - f(x) \underbracket{\norm{φ_N}_1}_{=1}} \dif x ≤ \\
	&≤ \int_Ω \int_Ω \abs{φ_N(y)} \abs{f(x-y) - f(x)} \dif y \dif x = \\
	&= \int_Ω \abs{φ_N(y)} \int_Ω \abs{f(x-y) - f(x)} \dif x \dif y = \\
	&= \int_Ω \abs{φ_N(y)} \norm[0]{T_yf - f}_1 \dif y
\end{align*} where $T_y f(x) ≝ f(x - y)$ is the translation operator, and where we have used Fubini's theorem to interchange the order of the integrals.

Now we will use the ``support goes to zero'' property of the approximate identity to separate the integral in two parts: one where the approximate identity goes to zero ($\abs{y} > δ$ for some $δ > 0$) and other where we will use properties of the translation operator.

So, for the first part we have \begin{align*}
\int_{\abs{y} > δ} \abs{φ_N(y)} \norm[0]{T_yf - T_0 f}_1 \dif y
	&≤ \int_{\abs{y} > δ} \abs{φ_N(y)} \left(\norm[0]{T_y f}_1 + \norm[0]{T_0 f}_1\right) \dif y \\
	&≤ 2\norm{f}_1 \int_{\abs{y} > δ} \abs{φ_N(y)} \dif y \convs[][N][∞] 0
\end{align*}

The second is a little bit more complex. As $C^∞(Ω) \densein L^p(Ω)$, choose a $g ∈ C^∞(Ω)$ such that $\norm{f - g}_1 < ε$ for any $ε < 0$. Now, adding and substracting $T_yg$ and $g$ we have that \begin{align*}
\norm[0]{T_y f - f}_1 &= \norm[0]{T_y f - T_y g + T_yg - g + g - f} ≤ \\
	&≤ \norm[0]{T_y f - T_y g}_1 + \norm{T_y g - g}_1 + \norm{g - f} ≤ \\
	&≤ (\norm[0]{T_y} + 1)\norm{f - g} + \norm[0]{T_y g - g}_1 \convs[][y][0] 0
\end{align*} as $\norm[0]{T_y g - g} = \int_Ω g(x - y) - g(x) \dif x$ goes to zero because we chose $g ∈ C^∞$. With that, we complete the proof for $p = 1$.

\proofpart{$1 < p < ∞$}

In \fref{prop:YoungInequality} we proved that if $φ_N ∈ L^1$ and $f ∈ L^p$, then $φ_N * f ∈ L^p$. Then, it makes sense to compute the $p$-norm of $φ_N * f - f$ as that is in $L^p$, and we can bound it as $\norm{φ_N * f - f}_p ≤ \norm{φ_N * f - f}_1$.

\end{proof}

If we ask for a little bit more of regularity, we have even uniform convergence.

\begin{prop} \label{prop:ApproximateIdentityUnif} Let $f ∈ C_0(Ω)$ and $\set{φ_N}_{N > 0}$ a family of approximate identities. Then, $φ_N * f \to f$ uniformly.
\end{prop}

\begin{proof} We will proceed to prove convergence in $L^∞(Ω)$, and that together with \fref{prop:ConvolutionContinuous} (the convolution of $L^1$ functions with a continuous function is continuous) will prove uniform convergence.

The proof is not difficult as it is just repeating the previous arguments, interchanging integrals and then using ``support goes to zero'' to bound for $\abs{y} > δ$ and continuity of $f$ when $\abs{y} < δ$.
\end{proof}

\section{Fourier series}

The first tool in harmonic analysis will be the Fourier series, which allows representation of a function as a trigonometric polynomial.

\begin{defn}[Fourier\IS series] Given $f ∈ C^0(\crc[1])$, we can define its Fourier series with coefficients \[ \hat{f}(n) = \int_0^1 e^{-2π\imath n x} f(x) \dif x \qquad n ∈ ℤ \] as \( \sum_{n ∈ ℤ} \hat{f}(n) e^{2π\imath n x} \label{eq:FourierSeries} \)
\end{defn}


In our case, we will consider $\crc[1]$ as the space of periodic functions, so equivalently $\crc[1] \cong \quot{ℝ}{ℤ}$. One interesting property of the Fourier series is its relation to the convolution.


With this, one can see the relationship between the convolution and the Fourier transform.

\begin{prop} Consider $f, g ∈ C^0(\crc[1])$. Then, the Fourier series of the convolution is the same that the product of the Fourier series: \[ \widehat{f\ast g} (n) = \hat{f}(n) · \hat{g}(n)\]
\end{prop}

\begin{proof} % TODO
\end{proof}

\subsection{Convergence for continuous functions}
\label{sec:Fourier:ConvergenceContinuous}

\begin{figure}[hbtp]
\centering
\inputtikz{DirichletKernel}
\caption{Dirichlet kernel for different values of $N$.}
\label{fig:DirichletKernel}
\end{figure}

We will start off trying to prove convergence in ``easy'' situations, that is, those with continuous functions. We will have to turn to the \concept[Kernel\IS Dirichlet]{Dirichlet kernel}, defined as \( \label{eq:DirichletKernel} D_N(x) ≝ \sum_{\abs{n} ≤ N} e^{2π\imath n x} = \frac{\sin \left(2πt (N + \sfrac{1}{2})\right)}{\sin πt} \), mainly because it has the interesting property that $S_Nf = D_N \ast f$, which allows a better treatment of the Fourier series. With a little bit of calculations (see \cite[Exercise 2.4]{ApuntesAnalisisFunc}) one can also show that \( \label{eq:DirichletKernelNormL1} \norm{D_N}_{L^1(\crc[1])} = C \log N + \mathcal{O}(1) \) so that the Dirichlet kernel is unbounded.

This allows us to prove the following proposition of non-convergence of the Fourier series for continuous functions.

\begin{prop} \label{prop:ConvergenceFourierCont} There exists at least a function $f ∈ C^0(\crc[1])$ for which the Fourier series does not converge pointwise. That is, such that \[ \lim_{N \to ∞} \abs{S_Nf(0) - f(0)} = ∞ \]
\end{prop}

\begin{proof}
The proof here is a summary of \cite[Exercise 2.4]{ApuntesAnalisisFunc}. Define the operator $λ_{D_N}(f) = S_Nf(0)$ on the space of continuous functions $X =(C^0(\crc[1]), \norm{·}_∞)$, which is Banach. Each of the operators are bounded with $\norm{λ_{D_N}}_{\linapp[X][ℝ]} = \norm{D_N}_{L^1(\crc[1])}$. However, this family of operators is not uniformly bounded so the \nref{thm:BanachSteinhaus} says that there exists a set $B$ dense in $X$ such that \[ \sup_{N ≥ 1} \norm{λ_{D_N} (f)}_{ℝ} = \sup_{N ≥ 1} \norm{S_N f(0)}_{ℝ} = ∞ \quad ∀f ∈ B\] and thus $\norm{S_Nf(0)}_{ℝ}$ is divergent for those $f ∈ B$.
\end{proof}

So, in general, the Fourier series does not necessarily converge pointwise for $C^0$ functions. We will be able, however, to prove pointwise convergence asking just for a little bit of regularity.

\subsubsection{Pointwise convergence for Hölder functions}

Luckily for us, we can have pointwise convergence just by asking a little bit of regularity to the function being transformed. This will be Hölder continuity:

\begin{defn}[Hölder\IS continuous] A function $\appl{f}{X}{Y}$ between two metric spaces is Hölder continuous if there exists two constants $C > 0$, $α ≥ 0$ such that for any $x, y ∈ X$, the following holds: \[ \norm{f(x) - f(y)} ≤ C \norm{x-  y}^α\]
\end{defn}

Obviously, if $α = 0$ we have boundedness, for $α > 0$ continuity and for $α = 1$

\begin{prop} \label{prop:FourierConvHolder} If $f ∈ C^0(\crc[1])$ is Hölder continuous, then the Fourier series of a function is equal to the same function.
\end{prop}

\begin{proof} We want to prove that, $∀x ∈ \crc$, the following limit is null: \[ \lim_{N \to ∞} S_Nf(x) -f(x) = 0\]

In order to do that, we will put the $f(x)$ inside of the integral and separate it in two intervals: one in which we will use the regularity of $f$ to compensate for the growth of the Dirichlet kernel and another one where we will use the oscillations of the kernel to bound the integral. Making use of the symmetry of $D_N$ we have that \begin{align*}
S_Nf(x) - f(x) &= \int_0^1 \left(f(x-y) - f(x)\right) D_N(y) \dif y = \int_{-\sfrac{1}{2}}^{\sfrac{1}{2}} \left(f(x-y) - f(x)\right) D_N(y) \dif y = \\
&= \underbracket{\int_{\abs{y} < δ} \left(f(x-y) - f(x)\right) D_N(y) \dif y}_{A} +
	\underbracket{\int_{\sfrac{1}{2} > \abs{y} > δ} \left(f(x-y) - f(x)\right) D_N(y) \dif y}_{B}
\end{align*}

For the estimate of $A$ we use the Hölder continuity:
\begin{align*}
\abs{A}
	&≤ \int_{\abs{y} < δ} \abs{\left(f(x-y) - f(x)\right) D_N(y)} \dif y  \\
	&≤ \int_{\abs{y} < δ} C \norm{x-y - x}^α \abs{D_N(y)} \dif y  \\
	&≤ Cδ^α \int_{\abs{y} < δ} \abs{D_N(y)} \dif y \\
	&\eqexpl[≤]{\eqref{eq:DirichletKernelNormL1}} Cδ^α \log N
\end{align*}

% TODO: Fix and end this proof
\end{proof}

\subsubsection{Césaro sums - Pointwise convergence for $C^0$ functions}

In the previous sections we have shown that, while the Fourier series is convergent pointwise for Hölder functions, it can even be divergent for general $C^0$ functions. However, we can recover pointwise values using averages over the coefficients, by means of the Césaro sum.

\begin{defn}[Césaro\IS means] Given $f ∈ C^0(\crc)$, we define the Césaro mean as \[ σ_N f(x) ≝ \frac{1}{N} \sum_{n = 0}^{N-1} S_n f(x) = \frac{1}{N} \int_0^1 \underbracket{\left(\sum_{n = - (N-1)}^{N-1} (N - \abs{n}) e^{2π\imath nx}\right)}_{K_N(x)} f(y) \dif y \]
\end{defn}

That expression $K_N$ is what we will call the \concept[Kernel\IS Féjer]{Féjer kernel}, and has the following expression: \( \label{eq:FejerKernel} K_N(x) = \frac{1}{N} \left(\frac{\sin πNx}{\sin πx}\right)^2 \)

\begin{figure}[hbtp]
\centering
\inputtikz{FejerKernel}
\caption{Fejer kernel for several values of $N$.}
\label{fig:FejerKernel}
\end{figure}

As opposed to the Dirichlet kernel, the Féjer kernel is positive and bounded in norm, forming what we call an \nlref{def:ApproximateIdentity}\footnote{Verification left to the untrusting reader.}. As it can be seen in \fref{fig:FejerKernel}, its supports tends to be smaller and smaller while maintaining area, which is the main property of these approximations.


With this proof we can show that the Césaro means actually converge to the value of the function.

\begin{corol} Let $f ∈ C^0(\crc)$. Then, the Césaro means (that is, the averages of its Fourier coefficients) converge pointwise to the function: \[ \lim_{N \to ∞} σ_N f(x) = f(x)\]
\end{corol}

\begin{proof} As we saw previously, the Césaro means can be expressed as $σ_N f = K_N * f$, and as $K_N$ is an \nlref{def:ApproximateIdentity} we can apply \fref{prop:ApproximateIdentityUnif} to have the result.
\end{proof}

\subsection{General convergence results}

In the more general framework of $L^p$ functions we can still prove interesting results about the Fourier series. The most important is the fact that the functions $\set{e^{2π\imath nx}}$ are dense in $L^p(\crc)$ for $1 ≤ p < ∞$. In the specific case of $p = 2$, which is a Hilbert space, that will imply that they are a basis of the space, thanks to the fact that the functions are orthonormal\footnote{Recall the inner product is defined as $\pesc{f,g} = \int_0^1 f \conj{g}$.}.

\begin{prop} \label{prop:TrigoPolyDense} The trigonometric family of functions $\set{e^{2π \imath nx}}_{n ∈ ℤ}$ is dense in $L^p(\crc)$ for $1 ≤ p < ∞$.
\end{prop}

\begin{proof} Using the Féjer kernel, as $K_N * f$ is a trigonometric polynomial we can just apply \fref{prop:ApproximateIdentity}.
\end{proof}

This allows us to prove in a very straightforward way the following theorem:

\begin{theorem}[Riesz–Fischer\IS theorem]\label{thm:RieszFischer} Let $f ∈ L^2(\crc[1])$ and define its partial Fourier series as \( \label{eq:PartialFourier} S_Nf(x) = \sum_{ \abs{n} ≤ N} \hat{f}(n) e^{2π\imath nx} \)

Then, $S_Nf$ converges to $f$ in the $L^2$ norm: \[ \norm{S_Nf - f}_{L^2(\crc[1])} = 0 \]
\end{theorem}

\begin{proof} Just apply \fref{prop:TrigoPolyDense}.
\end{proof}

The question is what happens in the general case $1 ≤ p ≤ ∞$. We have the following proposition which answer it partially:

\begin{prop} \label{prop:FourierSeriesConvLp} Let $1 ≤ p ≤ ∞$. Then we have convergence of the Fourier series in $L^p$ norm if and only if $\norm{S_N}_{\linapp[L^p(\crc)][L^p(\crc)]}$ is uniformly bounded. That is,
\[ \lim_{N \to ∞} \norm{S_N f - f}_{L^p(\crc)} = 0 \iff \sup_{N > 0} \norm{S_N}_{\linapp[L^p(\crc)][L^p(\crc)]} < ∞ \] for all $f ∈ L^p(\crc)$ and where, as a reminder, we define the operator norm as \[ \norm{S_N}_{\linapp[L^p(\crc)][L^p(\crc)]} ≝ \sup_{\substack{f ∈ L^p(\crc) \\ \norm{f} ≤ 1}} \norm{S_N f}_{L^p(\crc)}\]
\end{prop}

\begin{proof} We will separate the proofs in the if and only if parts. For convenience, we will drop the subscripts for the norms where its meaning is clear.

\proofpart{If: $\sup_{N > 0} \norm{S_N} <∞ \implies \lim_{N \to ∞} \norm{S_N f - f} = 0$}

Say $M ≝ \sup_{N > 0} \norm{S_N}$. As $C^∞(\crc) \densein L^p(Ω)$, we can choose a $g ∈ C^∞(\crc[1])$ such that $\norm{f - g} < ε$. Adding and substracting it in the norm, we can bound it:
\begin{align*}
\norm{S_Nf - f}
	&= \norm{S_Nf - f + g - g + S_N g - S_N g} = \\
	&= \norm{S_N(f - g) + (g - f) + (S_N g - g)} ≤ \\
	&≤ \norm{S_N(f -g)} + \norm{g -f} + \norm{S_N g - g} ≤ \\
	&≤ M ε + ε + ε
\end{align*} where we use, in order, the fact that $S_N$ is uniformly bounded, the approximation $g \to f$ and the fact that $S_N g \to g$ for smooth functions (\fref{prop:FourierConvHolder}, smooth functions are Hölder continuous).

\proofpart{Only if: $\lim_{N \to ∞} \norm{S_N f - f} = 0 \implies \sup_{N > 0} \norm{S_N} <∞$}

We will do this proof by contradiction: assume that $\lim_{N \to ∞} \norm{S_N f - f} = 0$ but $S_N$ are not uniformly bounded. By the \nref{thm:BanachSteinhaus}, we have that there exists a $G_δ$ set $B \densein L^p(\crc[1])$ such that $\sup_{N > 0} \norm{S_Nf} = ∞$ for all $f ∈ B$, and as $\norm{S_Nf -f } ≥ \norm{S_N f} - \norm{f}$ we have the contradiction.

\end{proof}

This gives us the following result for $p = 1$ and $p = ∞$:

\begin{prop} For $p = 1, ∞$, there exists at least a $f ∈ \crc$ such that \[ \lim_{N \to ∞} \norm{S_N f - f}_{L^p(\crc)} ≠ 0 \]
\end{prop}

\begin{proof} Simply use \fref{prop:FourierSeriesConvLp} knowing that $S_N f = D_N * f$ and that $\norm{D_N}_p \convs[][N][∞] ∞$ for $p = 1, ∞$.
\end{proof}


So, we proved non-convergence for $p = 1, ∞$ (in fact we even proved non-pointwise convergence in the case of continuous functions in \fref{prop:ConvergenceFourierCont}) and convergence for $p = 2$ (\fref{thm:RieszFischer}). For the other cases we will have convergence although we will see it later with a little bit more of work.

\subsection{Regularity and decay of Fourier coefficients}

One of the most useful aspects of the Fourier transform is the translation of derivation in the function space to multiplication in the Fourier coefficients. Indeed, one can compute and see that \[ \widehat{f'}(n) = \int_0^1 f'(x) e^{-2π\imath n x} \dif x = \eval[1]{f(x) e^{2π\imath nx}}_{x = 0}^1 + \int_0^1 2π\imath n f(x) e^{-2π\imath nx} \dif x = 2π\imath n \hat{f} (n) \]

Furthermore, if we have a previous bound on $\abs{\widehat{f'}(n)} < C$, then we can bound $\hat{f}(n)$ which will decay as \[ \abs{\hat{f}(n)} ≤ \frac{C}{2πn} \]

We will also be able to do the converse: control in the decay of the Fourier coeffiencts gives us control in the smoothness of $f$.

% TODO: there is a theorem here but it is sincerely shitty so I am not going to copy it until we need it.

% I also did not want to copy Lecture3.pdf.

\section{Fourier transform}

\chapter{Hilbert transform}

\chapter{Calderon-Zygmund operator}

\chapter{BMO functions}

\chapter{Fourier multipliers}

\appendix

\chapter{Functional analysis theorems}

\section{Banach spaces and linear applications}

\begin{theorem}[Banach-Steinhaus\IS uniform boundedness theorem] \citep[Theorem II.8]{ApuntesAnalisisFunc} \label{thm:BanachSteinhaus} Let $\set{T_α}_{α ∈ A}$ a subset of the space of continuous linear applications $\linapp$ between $X$ and $Y$, where $X$ is a Banach space and $Y$ is a normed space. Then, either the linear applications $T_α$ are uniformly bounded (that is, $∃ C < ∞$ such that $\norm{T_α}_{\linapp} ≤ C$ for all $α ∈ A$) or either there is a certain $G_δ$ set\footnote{Countable intersection of open set} dense in $X$, denoted by $B$, such that \[ \sup_{α ∈ A} \norm{T_α(x)}_Y = ∞ \quad ∀x ∈ B\]
\end{theorem}


\chapter{Exercises}
% -*- root: ../HarmonicAnalysis.tex -*-

\backmatter

\nocite{muscalu2013classical}
\bibliography{../EPFLNotes.bib}

\printindex
\end{document}

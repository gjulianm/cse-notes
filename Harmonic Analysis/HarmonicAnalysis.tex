\documentclass[palatino]{epflnotes}

\title{Harmonic Analysis}
\author{}
\date{16/17 - Spring semester}

% Additional packages

\precompileTikz
% --------------------

\begin{document}
\frontmatter
\pagestyle{plain}
\maketitle

\tableofcontents
\mainmatter
% Content

\chapter{Fourier analysis}

\section{Fourier series}

The first tool in harmonic analysis will be the Fourier series, which allows representation of a function as a trigonometric polynomial.

\begin{defn}[Fourier\IS series] Given $f ∈ C^0(\crc[1])$, we can define its Fourier series with coefficients \[ \hat{f}(n) = \int_0^1 e^{-2π\imath n x} f(x) \dif x \qquad n ∈ ℤ \] as \( \sum_{n ∈ ℤ} \hat{f}(n) e^{2π\imath n x} \label{eq:FourierSeries} \)
\end{defn}


In our case, we will consider $\crc[1]$ as the space of periodic functions, so equivalently $\crc[1] \cong \quot{ℝ}{ℤ}$. One interesting property of the Fourier series is its relation to the convolution.

\begin{defn}[Convolution] Given two functions $f, g ∈ L^1(Ω)$, the convolution of both is defined as \[ (f \ast g) (x) ≝ \int_Ω f(x-y) g(y) \dif y\] which is itself an $L^1(Ω)$ function.
\end{defn}

With this, one can see the relationship between the convolution and the Fourier transform.

\begin{prop} Consider $f, g ∈ C^0(\crc[1])$. Then, the Fourier series of the convolution is the same that the product of the Fourier series: \[ \widehat{f\ast g} (n) = \hat{f}(n) · \hat{g}(n)\]
\end{prop}

\begin{proof} % TODO
\end{proof}

\subsection{Convergence for continuous functions}

The first real interesting question in this topic is, of course, when is the serie defined in \eqref{eq:FourierSeries} is convergent and, moreover, when does it converge to the original function. The first one is easy and direct computation:

\begin{prop}[Parseval\IS identity] Let $f ∈ L^2(\crc[1])$. Then, its Fourier series $\{\hat{f}(n)\}$ is absolutely summable (that is, $\{\hat{f}(n)\} ∈ \ell^2(ℤ)$) and $\norm{\hat{f}}_{\ell^2(ℤ)} = \norm{f}_{L^2(\crc[1])}$.
\end{prop}

\begin{proof} % TODO
\end{proof}

Now, whether the Fourier series converges to its original function is another question, fairly interesting. A fact that should assureus that we are not working in some crazy world is that we have at least convergence in the $L^2$ norm.

\begin{theorem}[Riesz–Fischer\IS theorem] Let $f ∈ L^2(\crc[1])$ and define its partial Fourier series as \( \label{eq:PartialFourier} S_Nf(x) = \sum_{ \abs{n} ≤ N} \hat{f}(n) e^{2π\imath nx} \)

Then, $S_Nf$ converges to $f$ in the $L^2$ norm: \[ \norm{S_Nf - f}_{L^2(\crc[1])} = 0 \]
\end{theorem}

\begin{proof} % TODO
\end{proof}

\begin{figure}[hbtp]
\centering
\inputtikz{DirichletKernel}
\caption{Dirichlet kernel for different values of $N$.}
\label{fig:DirichletKernel}
\end{figure}

For a deeper study of the Fourier series, we have to turn to the \concept[Kernel\IS Dirichlet]{Dirichlet kernel}, defined as \( \label{eq:DirichletKernel} D_N(x) ≝ \sum_{\abs{n} ≤ N} e^{2π\imath n x} = \frac{\sin \left(2πt (N + \sfrac{1}{2})\right)}{\sin πt} \), mainly because it has the interesting property that $S_Nf = D_N \ast f$, which allows a better treatment of the Fourier series. With a little bit of calculations (see \cite[Exercise 2.4]{ApuntesAnalisisFunc}) one can also show that \( \label{eq:DirichletKernelNormL1} \norm{D_N}_{L^1(\crc[1])} = C \log N + \mathcal{O}(1) \) so that the Dirichlet kernel is unbounded.

This allows\footnote{This is a summary of \cite[Exercise 2.4]{ApuntesAnalisisFunc}.} us to define the operator $λ_{D_N}(f) = S_Nf(0)$ on the space of continuous functions $X =(C^0(\crc[1]), \norm{·}_∞)$, which is Banach. Each of the operators are bounded with $\norm{λ_{D_N}}_{\linapp[X][ℝ]} = \norm{D_N}_{L^1(\crc[1])}$. However, this family of operators is not uniformly bounded so the Banach-Steinhaus uniform boundedness theorem \citep[Theorem II.8]{ApuntesAnalisisFunc} says that there exists a set $B$ dense in $X$ such that \[ \sup_{N ≥ 1} \norm{λ_{D_N} (f)}_{ℝ} = \sup_{N ≥ 1} \norm{S_N f(0)}_{ℝ} = ∞ \quad ∀f ∈ B\] and thus $\norm{S_Nf(0)}_{ℝ}$ is divergent for those $f ∈ B$.

So, in general, the Fourier series does not necessarily converge pointwise for $C^0$ functions. We will be able, however, to prove pointwise convergence asking just for a little bit of regularity.

\subsubsection{Pointwise convergence for Hölder functions}

Luckily for us, we can have pointwise convergence just by asking a little bit of regularity to the function being transformed. This will be Hölder continuity:

\begin{defn}[Hölder\IS continuous] A function $\appl{f}{X}{Y}$ between two metric spaces is Hölder continuous if there exists two constants $C > 0$, $α ≥ 0$ such that for any $x, y ∈ X$, the following holds: \[ \norm{f(x) - f(y)} ≤ C \norm{x-  y}^α\]
\end{defn}

Obviously, if $α = 0$ we have boundedness, for $α > 0$ continuity and for $α = 1$

\begin{prop} If $f ∈ C^0(\crc[1])$ is Hölder continuous, then the Fourier series of a function is equal to the same function.
\end{prop}

\begin{proof} We want to prove that, $∀x ∈ \crc$, the following limit is null: \[ \lim_{N \to ∞} S_Nf(x) -f(x) = 0\]

In order to do that, we will put the $f(x)$ inside of the integral and separate it in two intervals: one in which we will use the regularity of $f$ to compensate for the growth of the Dirichlet kernel and another one where we will use the oscillations of the kernel to bound the integral. Making use of the symmetry of $D_N$ we have that \begin{align*}
S_Nf(x) - f(x) &= \int_0^1 \left(f(x-y) - f(x)\right) D_N(y) \dif y = \int_{-\sfrac{1}{2}}^{\sfrac{1}{2}} \left(f(x-y) - f(x)\right) D_N(y) \dif y = \\
&= \underbracket{\int_{\abs{y} < δ} \left(f(x-y) - f(x)\right) D_N(y) \dif y}_{A} +
	\underbracket{\int_{\sfrac{1}{2} > \abs{y} > δ} \left(f(x-y) - f(x)\right) D_N(y) \dif y}_{B}
\end{align*}

For the estimate of $A$ we use the Hölder continuity:
\begin{align*}
\abs{A}
	&≤ \int_{\abs{y} < δ} \abs{\left(f(x-y) - f(x)\right) D_N(y)} \dif y  \\
	&≤ \int_{\abs{y} < δ} C \norm{x-y - x}^α \abs{D_N(y)} \dif y  \\
	&≤ Cδ^α \int_{\abs{y} < δ} \abs{D_N(y)} \dif y \\
	&\eqexpl[≤]{\eqref{eq:DirichletKernelNormL1}} Cδ^α \log N
\end{align*}

% TODO: Fix and end this proof
\end{proof}

\subsubsection{Césaro sums - Pointwise convergence for $C^0$ functions}

In the previous sections we have shown that, while the Fourier series is convergent pointwise for Hölder functions, it can even be divergent for general $C^0$ functions. However, we can recover pointwise values using averages over the coefficients, by means of the Césaro sum.

\begin{defn}[Césaro\IS means] Given $f ∈ C^0(\crc)$, we define the Césaro mean as \[ σ_N f(x) ≝ \frac{1}{N} \sum_{n = 0}^{N-1} S_n f(x) = \frac{1}{N} \int_0^1 \underbracket{\left(\sum_{n = - (N-1)}^{N-1} (N - \abs{n}) e^{2π\imath nx}\right)}_{K_N(x)} f(y) \dif y \]
\end{defn}

That expression $K_N$ is what we will call the \concept[Kernel\IS Féjer]{Féjer kernel}, and has the following expression: \[ K_N(x) = \frac{1}{N} \left(\frac{\sin πNx}{\sin πx}\right)^2\]

\begin{figure}[hbtp]
\centering
\inputtikz{FejerKernel}
\caption{Fejer kernel for several values of $N$.}
\label{fig:FejerKernel}
\end{figure}

As opposed to the Dirichlet kernel, the Féjer kernel is positive and bounded in norm, forming what we call an approximate identity\footnote{Verification left to the untrusting reader.}. As it can be seen in \fref{fig:FejerKernel}, its supports tends to be smaller and smaller while maintaining area, which is the main property of these approximations.

\begin{defn}[Approximate\IS identity] \label{def:ApproximateIdentity} A family of functions $\set{φ_N}_{N ≥ 0} ⊂ L^∞(Ω)$ is called a family of approximate identities if and only if:
\begin{enumerate}
	\item $\int_Ω φ_N = 1$.
	\item $\sup_{N ≥ 0} \norm{φ_N}_{L^1} < ∞$.
	\item Their support goes to zero. That is, for any $δ > 0$ the following holds: \[ \lim_{N \to ∞} \int_{\abs{x} > δ} \abs{φ_N} = 0\]
\end{enumerate}
\end{defn}

The name ``approximate identity'' is not casual, as these functions are approximations for the identity element in the convolution. We will prove it now for $C^0(\crc)$ functions although a more general result holds for $L^p$ functions.

\begin{prop} \label{prop:ApproximateIdentity} Let $\set{φ_N}_{N ≥ 0}$ be a family of approximate identities, and let $f ∈ C^0(\crc)$. Then, \[ \lim_{N \to ∞} (φ_N \ast f)(x) = f(x) \]
\end{prop}

\begin{proof} We proceed to estimate $(φ_N * f)(x) - f(x)$, separating it in two integrals:
\begin{align*}
(φ_N * f)(x) - f(x)
	&= \int_0^1 φ_N(x-y) (f(y) - f(x)) \dif y = \\
	&= \underbracket{\int\limits_{\abs{x-y} < δ} φ_N(x-y) (f(y) - f(x)) \dif y}_{A}
	 + \underbracket{\int\limits_{\abs{x-y} > δ} φ_N(x-y) (f(y) - f(x)) \dif y}_{B}
\end{align*} for some $δ > 0$.

For the first integral, say $\sup_{N ≥ 0} \norm{φ_N}_{L^1} = M < ∞$. Fixing $ε > 0$ and by continuity of $f$, we know that we can select a $δ > 0$ such that $\abs{f(x) -f(y)} < \sfrac{ε}{2M}$, so that \begin{align*}
\abs{A}
	&≤ \sup_{N > 0} \int_{\abs{x-y} < δ} φ_N(x-y) (f(y) - f(x)) \dif y ≤ \\
	&≤ \frac{ε}{2M} \int_{\abs{x-y} < δ} φ_N(x-y) \dif y ≤ \frac{ε}{2}
\end{align*}

For $B$, we know that $\abs{f(y) - f(x)} < m$ by continuity, so applying the ``support goes to zero'' property of the approximate identities, we have that \begin{align*}
\lim_{N \to ∞} \abs{B}
	&≤ \lim_{N \to ∞} \int_{\abs{x-y} > δ} \abs{φ_N(x-y)} \abs{f(y) - f(x)} \dif y ≤ \\
	&≤ m  \lim_{N \to ∞} \int_{\abs{x-y} > δ} \abs{φ_N(x-y)} \dif y = 0
\end{align*} which completes the proof.
\end{proof}

With this proof we can show that the Césaro means actually converge to the value of the function.

\begin{corol} Let $f ∈ C^0(\crc)$. Then, the Césaro means (that is, the averages of its Fourier coefficients) converge pointwise to the function: \[ \lim_{N \to ∞} σ_N f(x) = f(x)\]
\end{corol}

\begin{proof} As we saw previously, the Césaro means can be expressed as $σ_N f = K_N * f$, and as $K_N$ is an \nlref{def:ApproximateIdentity} we can apply \fref{prop:ApproximateIdentity} to have the result.
\end{proof}

\subsection{General convergence results}

In the more general framework of $L^p$ functions we can still prove interesting results about the Fourier series. The most important is the fact that the functions $\set{e^{2π\imath nx}}$ are dense in $L^p(\crc)$ for $1 ≤ p < ∞$. In the specific case of $p = 2$, which is a Hilbert space, that will imply that they are a basis of the space, thanks to the fact that the functions are orthonormal\footnote{Recall the inner product is defined as $\pesc{f,g} = \int_0^1 f \conj{g}$.}.

\begin{prop} The trigonometric family of functions $\set{e^{2π \imath nx}}_{n ∈ ℤ}$ is dense in $L^p(\crc)$ for $1 ≤ p < ∞$.
\end{prop}

\section{Fourier transform}

\chapter{Hilbert transform}

\chapter{Calderon-Zygmund operator}

\chapter{BMO functions}

\chapter{Fourier multipliers}

\appendix

\chapter{Exercises}
\input{tex/HarmonicAnalysis_Exerc.tex}
\backmatter

\nocite{muscalu2013classical}
\bibliography{../EPFLNotes.bib}

\printindex
\end{document}

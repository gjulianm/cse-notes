\documentclass[palatino]{epflnotes}

\title{Harmonic Analysis}
\author{}
\date{16/17 - Spring semester}

% Additional packages

% --------------------

\begin{document}
\frontmatter
\pagestyle{plain}
\maketitle

\tableofcontents
\mainmatter
% Content

\chapter{Fourier analysis}

\section{Fourier series}

\begin{defn}[Fourier\IS series] Given $f ∈ C^0(\crc[1])$, we can define its fourier series with coefficients \[ \hat{f}(n) = \int_0^1 e^{-2π\imath n x} f(x) \dif x \qquad n ∈ ℤ \] as \[ \sum_{n ∈ ℤ} \hat{f}(n) e^{2π\imath n x}\]
\end{defn}

The first interesting fact is that functions in $C^0$ are not necessarily equal to their Fourier series. We only have that with Hölder functions.

\begin{defn}[Hölder\IS continuous] A function $\appl{f}{X}{Y}$ between two metric spaces is Hölder continuous if there exists two constants $C > 0$, $α ≥ 0$ such that for any $x, y ∈ X$, the following holds: \[ \norm{f(x) - f(y)} ≤ C \norm{x-  y}^α\]
\end{defn}

Obviously, if $α = 0$ we have boundedness, for $α > 0$ continuity and for $α = 1$

\begin{prop} If $f ∈ C^0(\crc[1])$ is Hölder continuous, then the Fourier series of a function is equal to the same function.
\end{prop}

\begin{proof}

\end{proof}


\begin{defn}[Convolution] Given two functions $f, g ∈ L^1(Ω)$, the convolution of both is defined as \[ (f \ast g) (x) ≝ \int_Ω f(x-y) g(y) \dif y\]
\end{defn}

\begin{prop} Consider $f, g ∈ C^0(\crc[1])$. Then, the Fourier series of the convolution is the same that the product of the Fourier series: \[ \widehat{f\ast g} (n) = \hat{f}(n) · \hat{g}(n)\]
\end{prop}

Dirichlet kernel. Trigonometric thing and you end up with only some terms.


\section{Fourier transform}

\chapter{Hilbert transform}

\chapter{Calderon-Zygmund operator}

\chapter{BMO functions}

\chapter{Fourier multipliers}

\appendix

\chapter{Exercises}
\input{tex/HarmonicAnalysis_Exerc.tex}
\backmatter

\nocite{muscalu2013classical}
\bibliography{../EPFLNotes.bib}

\printindex
\end{document}
